\subsection{Future Work} \label{sec:futurework}

\subsubsection{``Hatching'' new designs: Which came first?}

Whereas van Andel speaks of ``patterns of serendipity'' in a
relatively informal way, this paper will rely on the somewhat more
formal theory of \emph{design patterns} \cite{alexander1999origins},
to which it makes several additions and alterations.  This theory is
by no means limited to computing, and indeed, has its origins in
architecture and urban planning.  Our approach to ``designing for
serendipity'' \cite{andre2009discovery} centres on the use of design
patterns to capture the dynamic aspects of serendipitous situations.

The typical use of design patterns, since they were introduced by
Christopher Alexander
\cite{alexander1979timeless,alexander1977pattern}, is to prescribe as
well as to describe.  Design patterns provide models \emph{for} as
well as models \emph{of} \cite<cf.>[p. 93]{geertz1973interpretation}.
Thus, when Alexander describes the pattern \emph{A place to wait}, he
is telling readers that it is a good idea to consider building such
places when designing living spaces.  In connection with our
understanding of serendipity as closely associated with deviations
from familiar patterns, the central concern in this paper is the way
in which \emph{new} patterns are formed.

For example, when Poincar\'e \citeyear{poincare1910creation} describes his
discovery of the existence of Fuchsian functions, he includes the
detail: ``contrary to my habit I took black coffee, I could not
sleep.''  This is much more interesting as part of a story about an
exceptional case of productive insomnia than it is as the broad
characterisation of a typical nightly sleep schedule.  It might best
be described as a part of a ``situational pattern,'' with a title like
\emph{Change of pace}, rather than a ``behaviour pattern''; indeed, at
the level of behaviour, a \emph{Change of pace} is the exception to a
pattern!  Nevertheless, along with Poincar\'e, we can recognize a
pattern at another level.

\subsubsection{Computational poetry}

Naturally, we hope to realise the Writers Workshop in one or more
suitable formats.  Initial experiments with {\sf FloWr} are underway.
We believe that this project forms a critical but useful challenge for
the computational creativity community as a whole, and we expect to
balance research with outreach.

Within the context of the ongoing COINVENT project, we are interested
in using design patterns together with computational blending theory
to realise certain aspects of this model in a stand-alone
architecture.
%
It will be useful to consider how we can take both the \emph{discovery
  step}, which combines a serendipity trigger $T$, and prior
preparation $p$ and produces a classification $T^{\star}$ -- and the
\emph{invention step}, which combines the classified trigger
$T^{\star}$, and preparations $p^{\prime}$, and produces a novel
result $R$ -- to be \emph{blends} in the sense of Joseph Goguen
\cite{goguen1999introduction}.  The epistemological framework of
discovery gives some important clues about how to compute a common
base between $T$ and $p$.  Although $T$ was previously uninteresting,
it will have attributes or attribute-types that match the patterns
recognised by $p$ (e.g. \emph{One surprising observation}).  In the
invention step, reasoning, experimentation, social interaction
strategies rely on $p^{\prime}$, which might include familiarity with
patterns like \emph{Watch out for hidden symmetries} or
\emph{Successful error}, in order to extract a fruitful result from
$T^{\star}$.  Here, an important guidepost for implementation is that
many outcomes will result in new patterns of behaviour that the system
can draw on in subsequent interactions.

