\section{Future work}

Naturally, we hope to realise the Writers Workshop in one or more
suitable formats.  Initial experiments with {\sf FloWr} are underway.
We believe that this project forms a critical but useful challenge for
the computational creativity community as a whole, and we expect to
balance research with outreach.

Within the context of the ongoing COINVENT project, we are interested
in using design patterns together with computational blending theory
to realise certain aspects of this model in a stand-alone
architecture.
%
It will be useful to consider how we can take both the \emph{discovery
  step}, which combines a serendipity trigger $T$, and prior
preparation $p$ and produces a classification $T^{\star}$ -- and the
\emph{invention step}, which combines the classified trigger
$T^{\star}$, and preparations $p^{\prime}$, and produces a novel
result $R$ -- to be \emph{blends} in the sense of Joseph Goguen
\cite{goguen1999introduction}.  The epistemological framework of
discovery gives some important clues about how to compute a common
base between $T$ and $p$.  Although $T$ was previously uninteresting,
it will have attributes or attribute-types that match the patterns
recognised by $p$ (e.g. \emph{One surprising observation}).  In the
invention step, reasoning, experimentation, social interaction
strategies rely on $p^{\prime}$, which might include familiarity with
patterns like \emph{Watch out for hidden symmetries} or
\emph{Successful error}, in order to extract a fruitful result from
$T^{\star}$.  Here, an important guidepost for implementation is that
many outcomes will result in new patterns of behaviour that the system
can draw on in subsequent interactions.
