\subsection{Future Work} \label{sec:futurework} \label{sec:hatching}

Within the context of the ongoing COINVENT project \cite{coinvent14},
we are interested in using computational theory blending to realise
certain aspects of this model in a stand-alone architecture.
%
It will be useful to consider how we can take both the \emph{discovery
  step}, which combines a serendipity trigger $T$, and prior
preparation $p$, to produce a classification $T^{\star}$ -- and the
\emph{invention step}, which combines the classified trigger
$T^{\star}$, and preparations $p^{\prime}$, and produces a novel
result $R$ -- to be \emph{blends} in the sense of Joseph Goguen
\citeyear{goguen1999introduction}.

The epistemological framework of discovery gives some important clues
about how to compute a common base between $T$ and $p$, a key step for
blending, since these common features will typically be preserved in
the blend.  Although $T$ was previously uninteresting, it will have
attributes or attribute-types that match the patterns recognised by
$p$ (e.g. van Andel's \citeyear{van1994anatomy} \emph{One surprising
  observation}).
%
In the invention step, reasoning, experimentation, social interaction
strategies rely on $p^{\prime}$, which might draw on patterns like van
Andel's \emph{Successful error} in order to pinpoint the seeds of a useful result
within $T^{\star}$.  One important guidepost for implementation is
the theory-building orientation that says that outcomes may include
new patterns of behaviour that the system can draw on in subsequent interactions.

What is particularly needed is an approach to encoding patterns and
methods for pattern discovery in a computationally accessible manner.
Here we are drawn to the approach taken by the \emph{design pattern}
community \cite{alexander1999origins}, although we recognize that we
would be using design patterns in rather nonstandard way:
\begin{itemize}
\item[(1)] We want to encode our design patterns directly in runnable
  programs, not just give them to programmers as heuristic guidance.
\item[(2)] We want the (automated) programmer to generate new design
  patterns, not just apply or adapt old ones.
\item[(3)] We want our design patterns to help describe new problems,
  not just capture the solutions to existing problems.
\end{itemize}

\citeA{meszaros1998pattern} describe the typical scenario for design
pattern writers: ``You are an experienced practitioner in your
field. You have noticed that you keep using a certain solution to a
commonly occurring problem. You would like to share your experience
with others.''  They also remark that ``What sets patterns apart is
their ability to explain the rationale for using the solution (the
`why') in addition describing the solution (the `how').''  Regarding
the criteria that pattern writers seek to address, they write: ``The
most appropriate solution to a problem in a context is the one that
best resolves the highest priority forces as determined by the
particular context.''  Their article describes a number of criteria
relevant at the meta-level of pattern writing, e.g. \emph{Clear target
  audience}, \emph{Visible forces}, and \emph{Relationship to other
  patterns}.  A good pattern describes the resolution of forces, but
it also resolves certain forces itself.  In terms of our now-familiar
diagram:

\begin{center}
\begingroup
\tikzset{
block/.style = {draw, fill=white, rectangle, minimum height=3em, minimum width=3em},
tmp/.style  = {coordinate}, 
sum/.style= {draw, fill=white, circle, node distance=1cm},
input/.style = {coordinate},
output/.style= {coordinate},
pinstyle/.style = {pin edge={to-,thin,black}}
}

\begin{tikzpicture}[auto, node distance=2cm,>=latex']
    \node [sum] (sum1) {};
    \node [input, name=pinput, above left=.7cm and .7cm of sum1] (pinput) {};
    \node [input, name=tinput, left=2cm of sum1] (tinput) {};
    \node [input, name=minput, below left of=sum1] (minput) {};
    \node [input, name=minput, right of=sum1] (moutput) {};
    \draw [->] (tinput) -- node{\vphantom{{\tiny g}}{\tiny context}} (sum1);
    \draw [->] (pinput) -- node{{\tiny problem}} (sum1);
    \draw [->] (sum1) -- node{\vphantom{{\tiny g}}{\tiny solution}}  (moutput);
\end{tikzpicture}
\hspace{1cm}
\begin{tikzpicture}[auto, node distance=2cm,>=latex']
    \node [sum] (sum1) {};
    \node [input, name=pinput, above left=.7cm and .7cm of sum1] (pinput) {};
    \node [input, name=tinput, left of=sum1] (tinput) {};
    \node [input, name=minput, below left of=sum1] (minput) {};
    \node [sum, right=1.5cm of sum1] (sum2) {};
    \node [input, name=minput, right of=sum2] (moutput) {};
    \draw [->] (tinput) -- node{\vphantom{{\tiny g}}{\tiny solution}} (sum1);
    \draw [->] (pinput) -- node{{\tiny rationale}} (sum1);
    \draw [->] (sum1) -- node{\vphantom{{\tiny g}}{\tiny pattern}} (sum2);
    \draw [->] (sum2) -- node[text width=1.5cm,execute at begin node=\setlength{\baselineskip}{.3ex}]{\tiny \emph{resolution\\~of forces}}  (moutput);
\end{tikzpicture}
\endgroup
\end{center}


This diagram does not suggest that every instance of ``a solution to a
problem in a context'' is serendipity -- on the contrary, that is just
the discovery step.  To van Andel's assertion that ``The very moment I
can plan or programme `serendipity' it cannot be called serendipity
anymore'' -- of course, if the context is fully determined in advance,
and if the solution is completely replicable, then some of the
fundamental conditions for serendipity are not met.  However, we can
also describe patterns with built-in indeterminacy:

\begin{mdframed}
\vspace{-.35cm}
\paragraph{Successful error}
\emph{Van Andel's example} -- Post-it\texttrademark\ Notes\\[.05cm]
\begin{description}
\item[{\tt context}] -- You run a creative organisation with several different divisions and many contributors with different expertise.  Unexpected discoveries are often made.
\item[{\tt problem}] -- One of the members of your organisation
  discovers something with interesting properties, but that no one
  knows how to turn into a product with industrial or commercial application.
\item[{\tt solution}] -- You create a space for sharing and discussing
  interesting ideas on an ongoing basis (perhaps a Writers Workshop).
\item[{\tt rationale}] -- You suspect it's possible that one of the
  other members of the firm will come up with an idea about an
  application; you know that if a potential application is found, it
  may not be directly marketable, but at least there will be a
  prototype that can be concretely discussed.
\item[{\tt resolution}] -- Writing down and promulgating the
  \emph{Successful error} pattern using this template gives one such
  prototype.
\end{description}
\end{mdframed}
