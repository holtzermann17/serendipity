\subsection{Future Work} \label{sec:futurework} \label{sec:hatching}

Within the context of the ongoing COINVENT project \cite{coinvent14},
we are interested in using computational theory blending to realise
certain aspects of this model in a stand-alone architecture.
%
It will be useful to consider how we can take both the \emph{discovery
  step}, which combines a serendipity trigger $T$, and prior
preparation $p$, to produce a classification $T^{\star}$ -- and the
\emph{invention step}, which combines the classified trigger
$T^{\star}$, and preparations $p^{\prime}$, and produces a novel
result $R$ -- to be \emph{blends} in the sense of Joseph Goguen
\citeyear{goguen1999introduction}.  

The epistemological framework of discovery gives some important clues
about how to compute a common base between $T$ and $p$, a key step for
blending.  Although $T$ was previously uninteresting, it will have
attributes or attribute-types that match the patterns recognised by
$p$ (e.g. van Andel's \citeyear{van1994anatomy} ``\emph{One surprising
  observation}'').
%
In the invention step, reasoning, experimentation, social interaction
strategies rely on $p^{\prime}$, which might draw on patterns like van
Andel's \emph{Successful error} in order to pinpoint the seeds of useful result
inside $T^{\star}$.  One important guidepost for implementation is
the theory-building orientation that says that outcomes should result in
new patterns of behaviour that the system can draw on in subsequent interactions.

What is particularly needed is an approach to encoding patterns and
methods for pattern discovery in a computationally accessible manner.
Here we are drawn to the approach taken by the \emph{design pattern}
community \cite{alexander1999origins}, although we recognize that we
would be using design patterns in rather nonstandard way:
\begin{itemize}
\item[(1)] We want to encode our design patterns directly in runnable
  programs, not just give them to programmers as heuristic guidance.
\item[(2)] We want the automated programming system to generate new
  patterns, not just apply or adapt old ones.
\end{itemize}

