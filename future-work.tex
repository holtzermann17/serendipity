\subsection{Future Work} \label{sec:futurework}

\subsubsection{``Hatching'' new designs: Which came first?}

Merton \cite{merton1948bearing} \cite<cited in>[pp. 195--196]{merton} refers to a generalised ``serendipity pattern''
and its constituent parts:

\begin{quote}
\emph{The serendipity pattern refers to the fairly common experience of observing an \emph{unanticipated}, \emph{anomalous} \emph{and strategic} datum which becomes the occasion for developing a new theory or for extending an existing theory.}~\cite[p. 506]{merton1948bearing} (original emphasis)
    %% The datum [that exerts a pressure for initiating theory] is, first of all, unanticipated. A research directed toward the test of one hypothesis yields a fortuitous by-product, an unexpected observation which bears upon theories not in question when the research was begun.
    %% Secondly, the observation is anomalous, surprising, either because it seems inconsistent with prevailing theory or with other established facts. In either case, the seeming inconsistency provokes curiosity; it stimulates the investigator to "make sense of the datum," to fit it into a broader frame of knowledge....
    %% And thirdly, in noting that the unexpected fact must be "strategic," i. e., that it must permit of implications which bear upon generalized theory, we are, of course, referring rather to what the observer brings to the datum than to the datum itself. For it obviously requires a theoretically sensitized observer to detect the universal in the particular. 
\end{quote}

These features match our earlier description quite well: $T$ is the
unexpected observation; $T^\star$ classifies this as interesting; and
as we noted in Section \ref{sec:connections-to-formal-definition}, the
result $R$ may include updates to $p$ or $p^{\prime}$ that inform
further phases of research.
%
When van Andel \citeyear{van1994anatomy} speaks of ``patterns of
serendipity'' in a somewhat informal way, they are often instances of
this broader pattern.

%% \begin{quote}
%% ``Given research set up for a certain purpose, some unexpected, puzzling data, and a scientist capable of being puzzled -- given all of these, an accidental discovery will occur, because the relationship between fact and theory in science is such that it must occur.''
%% \end{quote}


We now ask whether it is possible to use the somewhat more formal
theory of \emph{design patterns} \cite{alexander1999origins} to design
for serendipity \cite{andre2009discovery}.  Alexandrian design
patterns are by no means limited to computing, and indeed, the
approach has its origins in architecture and urban planning
\cite{alexander1979timeless,alexander1977pattern}.

Design patterns prescribe and describe: they provide models \emph{for}
as well as models \emph{of}
\cite<cf.>[p. 93]{geertz1973interpretation}.  Thus, when Alexander
describes the pattern \emph{A place to wait}, he is also telling
readers that it may be a good idea to consider build a place to wait
when designing a living environment.
%% Architecture always contains
%% indeterminacy:

%% \begin{quote}
%% \emph{A bench is an object that provides a means for unforeseeable
%%   behaviour; depending on its location, it can be a seat, a bed or a
%%   sanctuary where conscious individuals may share a common durational
%%   experience.}~\cite[p. 26]{mckay-serendipity}
%% \end{quote}

%% In connection with our understanding of serendipity as closely
%% associated with deviations from familiar patterns, the central
%% concern in this paper is the way in which \emph{new} patterns are
%% formed.

A question we would like to consider in future work is where new
design patterns come from, and how they could be generated
computationally.  This is challenging because noticing a new pattern
means noticing something that deviates from what's known already.
Creating new design patterns is almost the antithesis of ``pattern
recognition'' in the usual computing sense.

%% For example, when Poincar\'e
%% \citeyear{poincare1910creation} describes his discovery of the
%% existence of Fuchsian functions, he includes the detail: ``contrary to
%% my habit I took black coffee, I could not sleep.''  This is much more
%% interesting as part of a story about an exceptional case of productive
%% insomnia than it is as the broad characterisation of a typical nightly
%% sleep schedule.  It might best be described as a part of a
%% ``situational pattern,'' with a title like \emph{Change of pace},
%% rather than a ``behaviour pattern''; indeed, at the level of
%% behaviour, a \emph{Change of pace} is the exception to a pattern!
%% Nevertheless, along with Poincar\'e, we can recognize a pattern at
%% another level.

\textbf{[Tentatively\ldots]}

\emph{Problem setting} is a fundamental issue for the field of
computational creativity that will only be given due attention when
the research culture is ready to fully embrace serendipity.

\begin{quote}
``[S]\emph{ocial cybernetics must be a second-order cybernetics--a
    cybernetics of cybernetics--in order that the observer who enters
    the system shall be allowed to stipulate his own purpose: he is
    autonomous.}'' \cite[p. 286]{von2003essays}
\end{quote}

\subsubsection{Computational poetry}

Naturally, we hope to realise the Writers Workshop in one or more
suitable formats.  Initial experiments with {\sf FloWr} are underway.
We believe that this project forms a critical but useful challenge for
the computational creativity community as a whole, and we expect to
balance research with outreach.

Within the context of the ongoing COINVENT project, we are interested
in using design patterns together with computational blending theory
to realise certain aspects of this model in a stand-alone
architecture.
%
It will be useful to consider how we can take both the \emph{discovery
  step}, which combines a serendipity trigger $T$, and prior
preparation $p$ and produces a classification $T^{\star}$ -- and the
\emph{invention step}, which combines the classified trigger
$T^{\star}$, and preparations $p^{\prime}$, and produces a novel
result $R$ -- to be \emph{blends} in the sense of Joseph Goguen
\cite{goguen1999introduction}.  The epistemological framework of
discovery gives some important clues about how to compute a common
base between $T$ and $p$.  Although $T$ was previously uninteresting,
it will have attributes or attribute-types that match the patterns
recognised by $p$ (e.g. \emph{One surprising observation}).  In the
invention step, reasoning, experimentation, social interaction
strategies rely on $p^{\prime}$, which might include familiarity with
patterns like \emph{Watch out for hidden symmetries} or
\emph{Successful error}, in order to extract a fruitful result from
$T^{\star}$.  Here, an important guidepost for implementation is that
many outcomes will result in new patterns of behaviour that the system
can draw on in subsequent interactions.

