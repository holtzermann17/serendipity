The connection to the key condition and components of serendipity
introduced in our literature survey are as follows:
%
The \textbf{serendipity trigger} is denoted by $T$.
%
The \textbf{focus shift} takes place with the identification of
$T^\star$, which is common to both the discovery and the invention
phase.  If the process operates in an ``online'' manner, $T^\star$ may
be an evolving vector of interesting possibilities.
%
The \textbf{prepared mind} corresponds to the prior training $p$ and
$p^{\prime}$ in our diagram.
%
%
The \textbf{bridge} is comprised of the actions based on $p^{\prime}$
that are taken on $T^\star$ leading to the \textbf{result} $R$, which is ultimately given a positive evaluation.

%% Here, $T$ is the trigger and $p$ denotes those preparations that afford the
%% classification $T^\star$, indicating $T$ to be of interest, while
%% $p^{\prime}$ denotes the preparations that facilitate the creation of a
%% bridge to a result $R$, which is ultimately given a positive
%% evaluation.
