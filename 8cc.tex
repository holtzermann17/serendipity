\section{Serendipity in computational systems} \label{sec:computational-serendipity}

The 13 facets of serendipity from Section \ref{sec:literature-review} specify the
conditions and preconditions that are conducive to serendipitous
discovery.  Section \ref{sec:our-model} distilled these elements into a computational model,
culminating in a method for evaluating computational serendipity in Section \ref{specs-overview}.
%
Along with clear criteria, it is important to clearly delineate the
scope of the system being evaluated.  For example, a standard
spell-checking program might suggest a substitution that the user
deems especially fortuitous; and we might agree that serendipity has
occurred, but we would not attribute serendipity to the spell-checker.

\citeA{pease2013discussion} used an earlier variant the SPECS criteria
to analyse three examples of potentially serendipitous behaviour:
dynamic investigation problems, model generation, and poetry
flowcharts.  Using our updated criteria, we discuss two new examples
below, and revisit poetry flowcharts in our third example, reporting
on recent work.  As Campbell \citeyear{campbell2005serendipity}
writes, ``serendipity presupposes a smart mind,'' and these examples
suggest potential directions for further work in computational
intelligence.


%% If the system learns an $N$th fact or
%% If applied to a system which could be described as minimally
%% serendipitous at best, and perhaps not at all serendipitous, does our
%% model identify the lack or presence of serendipity?  
%% %% As example, a spellchecker
%% %% program identifies spelling errors in text input and optionally can
%% %% correct spelling automatically. The only situation we can conceive of
%% %% where serendipity could possibly occur is tenuous; perhaps a suggested
%% %% correction may be incorrect, but may lead the user to interpret the
%% %% correction in an unexpected way. In all other aspects that we have
%% %% considered, spellchecker software would be a decidedly unlikely
%% %% candidate for harbouring serendipitous opportunities.  
%% Traditional spellchecker programs could be said to have a
%% \textbf{prepared mind}, in that they are constructed with internal
%% dictionaries with which to check spelling and ways of deciding what a
%% misspelled word might be.  Given our above discussion of how the
%% system might be serendipitous, the \textbf{serendipity trigger} could
%% be seen as the user misspelling a word and the system suggesting
%% alternative possibilities that the user had not previously conceived.
%% However, the \textbf{bridge} from trigger to serendipitous result (if
%% any) would have been built by the user, not by the system.  With
%% adaptive context-aware text completion tools, we can imagine a
%% ``Cyrano de Bergero'' effect in which the machine finds a
%% serendipitous bridge and offers the \textbf{result} to the user.
%% However, the current generation of text completion tools are known
%% more for infelicities than for exceptional wit.

\subsection{Case Study: Evolutionary music improvisation} \label{sec:evomusic}

\citeA{jordanous10} reported a computational jazz improvisation system
(later given the name {\sf GAmprovising} \cite{jordanous:12}) that
uses genetic algorithms.  Reevaluating {\sf GAmprovising} can shed
light on the degree to which evolutionary computing can encourage
computational serendipity.

{\sf GAmprovising} uses genetic algorithms to evolve a population of \emph{Improvisors}. Each Improvisor is able to randomly generate music based on various parameters such as the range of notes to be used, preferred notes, rhythmic implications around note lengths and other musical parameters, see \cite{jordanous10}. These parameters are what define the Improvisor at any point in the system's evolution.  After a cycle of evolution, each Improvisor is evaluated using a fitness function based on Ritchie's \citeyear{ritchie07} formal criteria for creativity.  This model relies on user-supplied ratings of the novelty and appropriateness of the music produced by the Improvisor to calculate 18 metrics that collectively indicate how creative the system is.  The fittest Improvisors are used to seed a new generation of Improvisors, through crossover and mutation operations.

The {\sf GAmprovising} system can be said to have a \textbf{prepared mind} through its background knowledge of what musical concepts to embed in the Improvisors and the evolutionary abilities to evolve Improvisors. A potential \textbf{serendipity trigger} comes from the combination of previous mutation and crossover operations with current user input.  To be clear, in the current version of the system it is a human evaluator who is largely responsible for the system's \textbf{focus shift}, since the user tells the system which improvisations are most valuable.   \citeA{jordanous10}
notes that this ``introduces a fitness bottleneck.''  In future versions of the system, autonomous evaluation could potentially take over for the human evaluator.  Once the interesting samples have been collected (from whatever source), a \textbf{bridge} is then built to new results through the creation of new Improvisors.  The \textbf{results} are the various musical improvisations produced by the fittest Improvisors (as well as, perhaps, the parameters that have been considered fittest).

%% The likelihood of serendipitous evolution is greatly enhanced by the
%% use of random mutation and crossover operations within the genetic
%% algorithm, which increase the diversity of the search space covered by
%% the system during evolution.  
The probability of encountering any particular pair of Improvisor and
user evaluation is vanishingly low, given the massive dimensions of
this search space.  However, there will always be a highest-scoring
Improviser, whose parameters will be used to seed the next round.  Do
we estimate the \textbf{chance} of the trigger appearing according to
its uniqueness, or according to the system's attentive observation of
all triggers that cross its path?  Consider de Mestral's encounter
with burrs: the chance of encountering burrs while out walking was
high, and the details of the particular burrs that were encountered
effectively irrelevant.  The situation here is similar: despite their
uniqueness, the trigger appearing is ``high.''  The evolution of
Improvisors captures a sense of \textbf{curiosity} about how to
satisfy the musical tastes of a particular human user who identifies
certain Improvisors as interesting.  The \textbf{sagacity} of the
system corresponds to its methods for enhancing the likelihood that
the user will appreciate a given Improvisor's music (or similar music)
over time.  With little basis for comparison, we can only say that
these two dimensions are ``typical.''  The aim of the system is to
maximise the \textbf{value} of the generated results by employing a
fitness function, and indeed, the system:
\begin{quote}
``{[}W{]}\emph{as able to produce jazz
improvisations which slowly evolved from what was essentially random
noise, to become more pleasing and sound more like jazz to the human
evaluator's ears}'' \cite{jordanous10}.
\end{quote}
The very reliability of the
system ultimately bears against its overall serendipity.  Following
Step 2, Part B of the SPECS procedure, we find a likelihood measure of
$\mathit{high}\times\mathit{moderate}\times\mathit{moderate}$, with
outcomes of moderate value, so that the system as a whole is ``not
very serendipitous.''  Evaluating individual threads (as members
of a larger population) would yield varied results, which
emphasises the importance of system scoping, mentioned above.
However, it would be inaccurate to simply say that successful threads are serendipitous and unsuccessful threads are unserendipitous,  since that ignores components other than value.  At the moment, individual threads are effectively equivalent regarding chance,
curiosity, and sagacity; a thread-by-thread analysis should be deferred until there would be more to say.

This is related to other changes that would improve the global serendipity score, as the following qualitative factor analysis indicates.
%
The {\sf GAmprovising} system does operate in \textbf{dynamic world},
assuming that the user's tastes may change.  A more elaborate version
of the system that could cater to multiple users is not yet
implemented, but would be occupied with a considerably more complex
problem, spanning and integrating \textbf{multiple contexts}.  The
system clearly engages with \textbf{multiple tasks}, but these are
largely separate, for instance, one global fitness function is used,
rather than evolving a local fitness function for each user along with
their ratings.  \textbf{Multiple influences} are present but currently
only at compile time, in the design of the fitness function, and the
selection of musical parameters that can later be set.  Greater
dynamism in future versions of the system would be likely to increase
its potential for serendipity.

\subsection{Case Study: Next-generation recommender systems} \label{sec:nextgenrec}
% Stress distinction between serendipity on the system- vs. serendipity on the user's side.
As discussed in Section \ref{sec:related}, recommender systems are one
of the primary contexts in computing where serendipity is currently discussed.  Serendipity, for current recommender systems, means suggesting items to a user that will be likely to introduce new ideas that are unexpected, but close to what the user is already interested in.  As we noted, these systems mostly focus on supporting \emph{discovery} for the user -- but some architectures also seem to take account of \emph{invention} of new methods for making recommendations, e.g.~using Bayesian methods, as surveyed in \citeNP{shengbo-guo-thesis}.  In light of our working definition of serendipity, we need to distinguish serendipity on the user side from serendipity in the system itself.

Current recommendation techniques associate less popular items with high unexpectedness \cite{Herlocker2004,Lu2012}, and use clustering to discover latent structures in the search space, e.g., partitioning users into clusters of common interests, or clustering users and domain objects \cite{Kamahara2005,Onuma2009,Zhang2011}.  But even in the Bayesian case, the system has limited autonomy.  A case for giving more autonomy to recommender systems can be made, especially in complex and rapidly evolving domains where hand-tuning is cost-intensive or infeasible.

With this challenge in mind, we ask how serendipity could be achieved on the system side. In terms of our model, current systems have at least the makings of a \textbf{prepared mind}, comprising both a user- and a domain model, both of which can be updated dynamically.  User behaviour (e.g.~following certain recommendations) or changes to the domain (e.g.~adding a new product) may serve as a potential \textbf{trigger} that could ultimately cause the system to discover a new way to make recommendations in the future.  Note, however, that it is unexpected behaviour in aggregate, rather than a one-off event, that is likely to provide grounds for a \textbf{focus shift}.   A \textbf{bridge} to a new kind of recommendation could be created by looking at exceptional patterns as they appear over time.  For instance, new elements may have been introduced into the domain that do not cluster well, and clusters may appear in the user model that do not have obvious connections between them.  A new recommendation strategy serves the organisational mission would be a serendipitous \textbf{result} for the system.

The system has only imperfect knowledge of user preferences and
interests.  At least relative to current recommender systems, the
\textbf{chance} of noticing some particular pattern in user behaviour
seems quite low.  The urge to make recommendations specifically for
the purposes of finding out more about users could be described as
\textbf{curiosity}.  Such recommendations may work to the detriment of
other metrics over the short term.  In principle, the system's
curiosity could be set as a parameter, depending on how much coherence
is permitted to suffer for the sake of gaining new knowledge.
Measures of \textbf{sagacity} would relate to the system's ability to
develop useful experiments and draw sensible inferences from user
behaviour.  For example, the system would have to select the best time
to initiate an A/B test.  A significant amount programming would have
to be invested in order to make this sort of judgement call
autonomously, so such systems are understandably rare.  The
\textbf{value} of recommendation strategies can be measured in terms
of traditional business metrics or other organisational objectives.
In this case, we compute a likelihood measure of
$\mathit{low}\times\mathit{variable}\times\mathit{low}$, with outcomes
of potentially high value, so that such a system is ``potentially
highly serendipitous.''

Recommender systems have to cope with a \textbf{dynamic world} of changing user preferences and a changing collection of items to recommend.  A dynamic environment which exhibits some degree of regularity represents a precondition for useful A/B testing.  The system's \textbf{multiple contexts} include the user model, the domain model, as well as an evolving model of its own organisation.  A system matching the description here would have \textbf{multiple tasks}: making useful recommendations, generating new experiments to learn about users, and improving its models.  In order to make effective decisions, a system would have to avail itself of \textbf{multiple influences} related to experimental design, psychology, and domain understanding.

\subsection{Case Study: Automated flowchart assembly} \label{sec:flowchartassembly}

Here we consider the design of a contemporary experiment with the
{\sf FloWr} flowcharting framework \cite{colton-flowcharting}.  {\sf FloWr} is a
user interface for creating and runnable flowcharts, built of small
modules called ProcessNodes.  In day-to-day use, {\sf FloWr} can be viewed
as a visual programming environment.  However, it can also be invoked
programmatically, on the Java Virtual Machine, or with any language
using a new web API.  The goals of {\sf FloWr} are both to be a user
friendly tool for co-creativity, and to be an autonomous
\emph{Flowchart Writer}.  Our experiment targets the latter scenario,
assembling available ProcessNodes into flowcharts automatically.  This can be viewed as a simple example of automated programming.

In the backend, {\sf FloWr}'s flowcharts are stored as scripts.  These
detail the names of the involved nodes together with their (input)
parameters and (output) variable settings.  Connections between nodes
are established when one node's input parameter references the output
variable of another node.
%
Inputs and outputs have constraints.  For instance, the {\tt
  WordSenseCategoriser} node has a {\tt stringsToCategorise}
parameter, which is seeded with an ArrayList of strings.  The node
produces useful output only when these strings can be parsed as as a
space-separated list of words.  The node's {\tt requiredSense}
parameter needs to be seeded with a string that represents exactly one
of the 57 British National Corpus Part of Speech tags.  Given
constraints of this nature, the first challenge in automated flowchart
assembly is to match inputs to outputs correctly, and to make sure
that all required inputs are satisfied.

In our current experiment, the system's potential \textbf{triggers}
result from trial and error with flowchart assembly.  Some valid
combinations of nodes will produce results, and some will not.  Due to
the dynamically changing environment (e.g., updates to data sources
like Twitter) some flowcharts that did not produce results earlier may
unexpectedly begin to produce results.
%
The system's \textbf{prepared mind} lies in a distributed knowledge
base provided by ProcessNodes, showing the constraints on their inputs
and outputs, and in the global history of successful and unsuccessful
combinations.
%
The system will not try combinations that it knows cannot produce
results, but it will try novel combinations and may retry earlier
flowchart specimens that have the chance to become viable.  Turning a
collection of nodes for which no known working combination existed
into a working flowchart is an occasion for a \textbf{focus shift}:
what made this particular combination work?  Is there a pattern that
could be exploited in the future?  However, it may be that no
broader pattern can be found, other than the fact that the combination works.
%
Successful combinations and any further inferences are stored, and
referred to in future runs.  The \textbf{bridge} to the next set of
results is accordingly found by informed trial and error.
%
In these early experiments, the basic \textbf{result} the system is
aiming to achieve is simply to generate a new combination of nodes that can fit together and that generate non-empty output.  Subsequent versions of the system may have more detailed evaluation functions, setting a higher bar.  For example, a future version of the system could be tuned to search for flowcharts that generate poetry, as we discuss in \cite{corneli2015computational}.

The \textbf{chance} of finding a novel successful combination of nodes
is fairly low, as this depends on both the output from certain nodes,
and in terms the combinatorial search strategy itself.  Compared to
humans users of {\sf FloWr}, the seems exceptionally \textbf{curious}
about finding novel combinations of nodes.  Remembering viable
combinations and avoiding combinations that are known not to work
presents only a modest degree of \textbf{sagacity}.  At the moment,
the system's criterion for attributing \textbf{value} is simply that
the combination of nodes generates non-empty output; however an
external evaluator is not likely to judge these combinations as
useful.  The associated likelihood score is
$\mathit{low}\times\mathit{low}\times\mathit{high}$, which
is favourable, however, until there is a
more discriminating way to judge value, the
attribution of serendipity to any particular run may
be premature.  One fairly obvious route would be to attribute
value to explanatory heuristics, rather than generated texts;
this would require increased sagacity on the part of the
system as well.

The \textbf{dynamic world} the system operates in is dynamic in two
ways: first, in the straightforward sense that some of the input
sources, like Twitter, are changing; and also in the sense that the
system's knowledge of successful and unsuccessful node combinations
changes over time.  The current version of the system does not seem to
deal with \textbf{multiple contexts}; even though we have broken the
experiment into separate sub-populations to constrain the search,
these do not interact.  However, in a future version of the system,
interaction between different heuristically-driven search processes
would be possible, and could produce more unexpected results.  Along
these lines, as more goals are added, the system could more readily be
seen to have \textbf{multiple tasks}.  For instance, one search
process could look for narrative outlines to structure a poem with,
and another process could look for lines or stanzas to fill out that
outline.  As for \textbf{multiple influences}, the population of
ProcessNodes will constrain (and, as more nodes are added, extend) the
possible strategies for assembling flowcharts.

\afterpage{\clearpage}
\begin{table}[p]
{\centering \renewcommand{\arraystretch}{1.5}
\scriptsize
\begin{tabular}{p{1.5in}@{\hspace{.1in}}p{1.5in}@{\hspace{.1in}}p{1.5in}}
\multicolumn{1}{c}{\textbf{{\footnotesize Evolutionary music}}} & \multicolumn{1}{c}{\textbf{{\footnotesize Next-gen.~recommenders\hspace{.4cm}}}} & \multicolumn{1}{c}{\textbf{{\footnotesize Flowchart assembly}}} \\[.05in]
\multicolumn{3}{l}{\em {\textbf{Condition}}} \\
\cline{1-3}
\multicolumn{3}{l}{\em Focus shift} \\[-.1cm]
Driven by (currently, human) evaluation of samples
& Unexpected behaviour in the aggregate
& Find a pattern to explain a successful combination of nodes\\
\cline{1-3}
~\\[-.1cm]
\multicolumn{3}{l}{\em {\textbf{Components}}} \\
\cline{1-3}
\multicolumn{3}{l}{\em Trigger} \\[-.1cm]
% \textbf{Trigger}
Previous evolutionary steps, in combination with user input
& Input from user behaviour
& Trial and error in combinatorial search \\
% \cline{1-3}
\multicolumn{3}{l}{\em Prepared mind} \\[-.1cm]
% \textbf{Prepared mind}
Musical knowledge, evolution mechanisms
& Through user/domain model
& Constraints on node inputs and outputs; history of successes and failures\\
% \cline{1-3}
%\textbf{Bridge}
\multicolumn{3}{l}{\em Bridge} \\[-.1cm]
Newly-evolved Improvisors
& Elements identified outside clusters
& Try novel combinations \\
% \cline{1-3}
%\textbf{Result}
\multicolumn{3}{l}{\em Result} \\[-.1cm]
Music generated by the fittest Improvisors
& Dependent on organisation goals
& Non-empty or more highly qualified output \\ \cline{1-3}
~\\[-.1cm]
%%%%%%%%%%%%%%%%%%%%%%%%%%%%%%%%%%%%%%%%%%%%%%%%%%%%%%%%%%%%%%%%%%%%%%%%%%%%%%%%%%%%%%%%%%%%%%%%%%%%
\multicolumn{3}{l}{\em \textbf{Dimensions}}  \\
\cline{1-3}
%\textbf{Chance}
\multicolumn{3}{l}{\em Chance} \\[-.1cm]
Looking for rare gems in a huge search space
& Imperfect knowledge of user preferences and behaviour
& Changing state of the outside world; random selection of nodes to try \\
% \cline{1-3}
%\textbf{Curiosity}
\multicolumn{3}{l}{\em Curiosity} \\[-.1cm]
Aiming to have a particular user take note of an Improvisor
& Making unusual recommendations
& Search for novel combinations \\
% \cline{1-3}
%\textbf{Sagacity}
\multicolumn{3}{l}{\em Sagacity} \\[-.1cm]
Enhance user appreciation of Improvisor over time,
using a fitness function
& Update recommendation model after user behaviour 
& Don't try things known not to work; consider variations on successful patterns \\
% \cline{1-3}
%\textbf{Value} &
\multicolumn{3}{l}{\em Value} \\[-.1cm]
Via fitness function (as a proxy measure of creativity)
& Per business metrics/objectives
& Currently ``non-empty results''; more interesting evaluation functions possible \\
\cline{1-3}
%%%%%%%%%%%%%%%%%%%%%%%%%%%%%%%%%%%%%%%%%%%%%%%%%%%%%%%%%%%%%%%%%%%%%%%%%%%%%%%%%%%%%%%%%%%%%%%%%%%%
~\\[-.1cm]
\multicolumn{3}{l}{\em \textbf{Factors}} \\
\cline{1-3}
%\textbf{Dynamic world}
\multicolumn{3}{l}{\em Dynamic world} \\[-.1cm]
Changes in the user tastes
& As precondition for testing system's influences on user behaviour
& Changing data sources and growing domain knowledge \\
%\cline{1-3}
%\textbf{Multiple contexts}
\multicolumn{3}{l}{\em Multiple contexts} \\[-.1cm]
Multiple users' opinions would change what the system is curious about and require greater sagacity
& User model, domain model, model of its own behaviour
& Interaction between different heuristic search processes would increase unexpectedness \\
% \cline{1-3}
%\textbf{Multiple tasks}
\multicolumn{3}{l}{\em Multiple tasks} \\[-.1cm]
Evolve Improvisors, generate music, collect user input, carry out fitness calculations
& Make recommendations, learn from users, update models
& Generate new heuristics and new domain artefacts \\
% \cline{1-3}
%\textbf{Multiple influences}
\multicolumn{3}{l}{\em Multiple influences} \\[-.1cm]
Through programming of fitness function and musical parameter combinations
& Experimental design, psychology, domain understanding
& Learning to combine new kinds of ProcessNodes\\
\cline{1-3}
\end{tabular}
\par}
\normalsize
\bigskip

\caption{Summary: applying our computational serendipity model to three case studies\label{caseStudies}}
\end{table}

\subsection{Summary}

Table \ref{caseStudies} summarises how the condition, components,
dimensions and factors in our model of serendipity appear in an
evolutionary music system, in hypothetical ``next-generation''
recommender systems, and in our current work on a flowchart-assembly
system.  Each of the case studies shows clear potential for
serendipity.  There are also clear ways in which the measure of
serendipity could be enhanced.

\begin{enumerate}
\item A future version of the evolutionary music system would be more
  convincingly sagacious if it could evaluate works without user
  intervention.  It might also be able to tailor its fitness function
  to the individual user.  More broadly, interaction between the
  system's tasks and more dynamism in its influences would help
  differentiate individual threads or system runs, and some elements
  of this population might be more serendipitous than others.

\item The next-generation recommender systems we've envisioned need to
  be able to make inferences from aggregate user behaviour.  This
  points to long-term considerations that go beyond the unique
  serendipitous event.  How ``curious'' should these systems be?  One
  obvious criterion is that short-term value should be allowed to
  suffer as long as expected value is still higher.

\item The flowchart assembly process would need more stringent, and
  more meaningful, criteria for value before third-party observers
  would be likely to attribute serendipity to the system.  In addition
  to raising challenges for autonomous evaluation (as in the
  evolutionary music system case), this requirement would impose more
  sophisticated constaints on processing in earlier steps, which would
  require the system to be more sagacious.
\end{enumerate}

