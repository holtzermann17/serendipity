\subsection{Using SPECS to evaluate computational serendipity}\label{specs-overview}

In a 2012 special issue of the journal {\em Cognitive Computation}, on
``Computational Creativity, Intelligence and Autonomy'', Jordanous
analyses current evaluation procedures used in computational
creativity, and provides a much-needed set of customisable evaluation
guidelines, the \emph{Standardised Procedure for Evaluating Creative
  Systems} (SPECS) \cite{jordanous:12}. The three step process of SPECS requires the evaluator to define the concept(s) they are evaluating the system on (originally SPECS was designed to evaluate the concept of creativity). This definition is then converted into testable standards that can be used to evaluate individual systems, or comparatively evaluate multiple systems.

We give a slightly modified version of her earlier evaluation
guidelines, in that rather than attempt a definition and evaluation of
{\em creativity}, we follow the three steps for \emph{serendipity}. 

\newpage

\subsubsection*{Step 1: A computational definition of serendipity}
\begin{quote} {\em Identify a definition of serendipity that your
    system should satisfy to be considered serendipitous.}\end{quote}

\noindent Our computational definition of serendipity is as given in Section \ref{sec:our-model}.
%% This situation can be pictured schematically as follows:

