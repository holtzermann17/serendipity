\subsection{Using SPECS to evaluate computational serendipity}\label{specs-overview}

In this section, we use the elements of the conceptual framework
described in Section \ref{sec:by-example} help to flesh out this
definition, to develop quite detailed evaluation criteria.
We adapt the \emph{Standardised Procedure for Evaluating Creative Systems} (SPECS),
a high-level, customisable evaluation strategy that was devised to judge the creativity
of computational systems \cite{jordanous:12}.  

In the three step SPECS process, the evaluator defines the concepts
and behaviours that signal creativity, converts this definition into
clear standards, and then applies them to evaluate the target systems.
%
We follow a slightly modified version of Jordanous's earlier evaluation
guidelines, in that rather than attempt a definition and evaluation of
{\em creativity}, we follow the three steps for \emph{serendipity}.

%\vspace{-.3cm}
\subsubsection*{Step 1: Identify a definition of serendipity that your system should satisfy to be considered serendipitous.}
%~\\
%\vspace{-.1cm}

\noindent We adopt the definition of serendipity from Section
\ref{sec:our-model}.

%\vspace{-.3cm}
\subsubsection*{Step 2: Using Step 1, clearly state what standards you use to evaluate the serendipity of your system.}
%~\\
%\vspace{-.1cm}

\noindent With our definition and other features of the model in mind, we propose the following standards for evaluating serendipity in computational systems. These criteria allow the evaluator to assess the degree of seredipity that is present in a given system's operation.

%% Serendipity relies on a reassessment or reevaluation -- a \emph{focus shift} in which something that was previously uninteresting, of neutral, or even negative value, becomes interesting.

\begin{description}[itemsep=16pt]
\item[{(\textbf{A - Definitional characteristics})}] {The system can
  be said to have a {\textbf{prepared mind}}, consisting of previous
  experiences, background knowledge, a store of unsolved problems,
  skills, expectations, readiness to learn, and (optionally) a current
  focus or goal.  It then processes a {\textbf{trigger}} that is at
  least partially the result of factors outside of its control,
  including randomness or unexpected events.  It classifies this
  trigger as interesting, constituting a {\textbf{focus shift}}.  The
  system then uses reasoning techniques and/or social or otherwise
  externally enacted alternatives to create a {\textbf{bridge}} from
  the trigger to a result.  The {\textbf{result}} is evaluated as
  useful, by the system and/or by an external source.}  The evaluator
  should specify all of these aspects relative to the system under
  consideration at a sufficient degree of precision to show their
  processual interconnection.
%%%%%%%%%%%%%%%%%%%%%%%%%%%%%%%%%%%%%%%%%%%%%%%%%%%%%%%%%%%%%%%%%%%%%
\item[{(\textbf{B - Dimensions})}] {Serendipity, and its various
  dimensions, can be present to a greater or lesser degree.  If the
  criteria above have been met, we consider the system (and
  optionally, generate ratings as estimated probabilities) along
  several dimensions:
%
{($\mathbf{a}$: \textbf{chance})} how likely was this trigger to appear to
  the system?
%
{($\mathbf{b}$: \textbf{curiosity})} On a population basis, comparing
similar circumstances, how likely was the trigger to be identified as
interesting?
%
{($\mathbf{c}$: \textbf{sagacity})} On a population basis, comparing
similar circumstances, how likely was it that the trigger would be
turned into a result?
%
Finally, we ask, again, comparing similar results where possible:
{($\mathbf{d}$: \textbf{value})} How valuable is the result that
is ultimately produced?}
%
%Then combining $\mathbf{a}\times\mathbf{b}\times\mathbf{c}$ gives a
 % likelihood score: 
{Low but nonzero likelihood $\mathbf{a}\times\mathbf{b}\times\mathbf{c}$ 
 and high value $\mathbf{d}$ are the criteria we use to say that the event was ``highly serendipitous.''}
%%%%%%%%%%%%%%%%%%%%%%%%%%%%%%%%%%%%%%%%%%%%%%%%%%%%%%%%%%%%%%%%%%%%%
\item[{(\textbf{C - Factors})}] {Finally, if the criteria from Part A
  are met, and if the event is deemed sufficiently serendipitous to
  warrant further investigation according to the criteria in Part B,
  then in order to deepen our qualitative understanding of the
  serendipitous behaviour, we ask: To what extent does the system
  exist in a {\textbf{dynamic world}}, spanning {\textbf{multiple
      contexts}}, featuring {\textbf{multiple tasks}}, and
  incorporating {\textbf{multiple influences}}?}
\end{description}

%\vspace{-.3cm}
\subsubsection*{Step 3: Test your serendipitous system against the standards stated in Step 2 and report the results.}
%~\\
%\vspace{-.1cm}

\noindent In Section \ref{sec:computational-serendipity}, we will pilot our framework by examining the degree of serendipity of existing and hypothetical computational systems. 

\subsection{Heuristics}\label{specs-heuristics}

How can we we estimate the chance of the trigger appearing, if every
trigger is unique?  Consider de Mestral's encounter with burrs.
The probability of encountering burrs while out walking is high: many
people have had that experience.  The unique features of de Mestral's
experience are that he had the curiosity to investigate the burrs
under a microscope, and the sagacity (and tenacity) to turn what he
discovered into a successful product.  The details of the particular
burrs that were encountered effectively irrelevant.  In the genera case, we are not interested in the chance of encountering a particular object or set of data, which may be vanishingly small.  Rather, we are interested the chance of encountering a trigger that could precipitate an interested response.  The trigger itself may be a complex object or event that takes place over a period of time; in other words, it may be a pattern, rather than a fact.  Noticing that a pattern exists is a key aspect of sagacity.

Although it is in no way required by the SPECS methodology outlined
above, many systems (including all of the examples below) have an
iterative aspect.  This means that a result may serve as a trigger for
further discovery.  In such a case it may be important for further
indeterminacy to be introduced to the system, lest the results be
convergent, and therefor, non-fallible.  In applying the critera to
such systems, we consider long-term behaviour.


