% \section{Literature review} 

\subsection{Etymology and selected definitions} \label{sec:overview-serendipity}  \label{sec:literature-review}
The English term ``serendipity'' derives from the 1302 long poem \emph{Eight Paradises}, written in Persian by the Sufi poet Am\={\i}r Khusrow in Uttar Pradesh.\footnote{\url{http://en.wikipedia.org/wiki/Hasht-Bihisht}}  In the English-speaking world, its first chapter became known as ``The Three Princes of Serendip'', where ``Serendip'' represents the Old Tamil-Malayalam word for Sri Lanka (%{\tam சேரன்தீவு},
\emph{Cerantivu}), ``island of the Ceran kings.''

The term ``serendipity'' is first found in a 1757 letter by Horace Walpole to Horace Mann:
\begin{quote}
\emph{``This discovery is almost of that kind which I call serendipity, a very expressive
word} \ldots \emph{You will understand it better by the derivation than by the
definition. I once read a silly fairy tale, called The Three Princes of Serendip:
as their Highness travelled, they were always making discoveries, by accidents
\& sagacity, of things which they were not in quest of}[.]''~\cite[p. 633]{van1994anatomy}
\end{quote}
The term became more widely known in the 1940s through studies of serendipity as a factor in scientific discovery, surveyed by Robert Merton and Elinor Barber \citeyear{merton} in ``The Travels and Adventures of Serendipity, A Study in Historical Semantics and the Sociology of Sciences''.  Merton \citeyear{merton1948bearing} \cite<cited in>[pp. 195--196]{merton} describes a generalised ``serendipity pattern'' and its constituent parts:

\begin{quote}
``\emph{The serendipity pattern refers to the fairly common experience of observing an \emph{unanticipated}, \emph{anomalous} \emph{and strategic} datum which becomes the occasion for developing a new theory or for extending an existing theory.}''~\cite[p. 506]{merton1948bearing} (original emphasis)
    %% The datum [that exerts a pressure for initiating theory] is, first of all, unanticipated. A research directed toward the test of one hypothesis yields a fortuitous by-product, an unexpected observation which bears upon theories not in question when the research was begun.
    %% Secondly, the observation is anomalous, surprising, either because it seems inconsistent with prevailing theory or with other established facts. In either case, the seeming inconsistency provokes curiosity; it stimulates the investigator to "make sense of the datum," to fit it into a broader frame of knowledge....
    %% And thirdly, in noting that the unexpected fact must be "strategic," i. e., that it must permit of implications which bear upon generalized theory, we are, of course, referring rather to what the observer brings to the datum than to the datum itself. For it obviously requires a theoretically sensitized observer to detect the universal in the particular. 
\end{quote}

In 1986, Philippe Qu\'eau described serendipity as ``the art of
finding what we are not looking for by looking for what we are not
finding'' (\citeNP{eloge-de-la-simulation}, as quoted in
\citeNP[p. 121]{Campos2002}).  Pek van Andel
\citeyear[p. 631]{van1994anatomy} describes it simply as ``the art of
making an unsought finding''.


Roberts \citeyear[pp. 246--249]{roberts} records 30 entries for the term ``serendipity'' from English language dictionaries dating from 1909 to 1989.  
%
Classic definitions require the investigator not to be aware of the problem they serendipitously solve, but this criterion has largely dropped from dictionary definitions. Only 5 of Roberts' collected definitions explicitly say ``not sought for.''  Roberts characterises ``sought findings'' in which an accident leads to a discovery with the term \emph{pseudoserendipity} \cite{chumaceiro1995serendipity}.
%
While Walpole initially described serendipity as an event
(i.e., a kind of discovery), it has
since been reconceptualised as a psychological attribute, a matter of
sagacity on the part of the discoverer: a ``gift'' or ``faculty'' more
than a ``state of mind.''  Only one of the collected definitions, from
1952, defined it solely as an event, while five define it as both
event and attribute.

However, there are numerous examples that exhibit features of
serendipity which develop on a social scale rather than an individual
scale.  For instance, between Spencer Silver's creation of high-tack,
low-adhesion glue in 1968, the invention of a sticky bookmark in 1973,
and the eventual launch of the distinctive canary yellow re-stickable
notes in 1980, there were many opportunities for
Post-it\texttrademark\ Notes \emph{not} to have come to be
\cite{tce-postits}.  Merton and Barber argue that the
psychological perspective needs to be integrated with a
\emph{sociological} one.\footnote{ ``For if chance favours prepared
  minds, it particularly favours those at work in microenvironments
  that make for unanticipated sociocognitive interactions between
  those prepared minds. These may be described as serendipitous
  sociocognitive microenvironments'' \cite[p. 259--260]{merton}.}
Large-scale scientific and technical projects generally rely on the
``convergence of interests of several key actors''
\cite{companions-in-geography}, along with other supporting cultural
factors.  For example, Umberto Eco \citeyear{eco2013serendipities} focuses on the
historical role of serendipitous mistakes and falsehoods in the
production of knowledge.

It is important to note that serendipity is usually discussed within
the context of \emph{discovery}, rather than \emph{creativity},
although in typical parlance these terms are closely related
\cite{jordanous12jims}.  In our definition of serendipity, we have
made use of Henri Bergson's distinction:
\begin{quote}
``\emph{Discovery, or uncovering, has to do with what already exists,
    actually or virtually; it was therefore certain to happen sooner
    or later.  Invention gives being to what did not exist; it might
    never have happened.}''~\cite{bergson2010creative}
\end{quote}
As we have indicated, serendipity would seem to require features of
both; that is, the discovery of something unexpected and the invention
of an application for the same.  We must complement \emph{analysis}
with \emph{synthesis} \cite{delanda1993virtual}.  The balance between
these two features will differ from case to case.

\citeA{creativity-crisis} write that: ``To be creative requires
divergent thinking (generating many unique ideas) and then convergent
thinking (combining those ideas into the best result).''  This is
exemplified by Voltaire's \citeyear{zadig} character Zadig (a figure inspired
in part by the ``The Three Princes of Serendip'') who ``was capable of
discerning a Thousand Variations in visible Objects, that others, less
curious, imagin’d were all alike'' -- and in addition had the
``peculiar Talent to render Truth as obvious as possible: Whereas most
Men study to render it intricate and obscure.''
