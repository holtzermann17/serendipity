\subsection{Future Work} \label{sec:futurework} \label{sec:hatching}

In looking for ways to manage and encourage serendipity, we are drawn
to the approach taken by the \emph{design pattern} community
\cite{alexander1999origins}.
%% The essential features of this approach
%% are described below, but we point out straight away that we propose to
%% use design patterns in rather nonstandard fashion.  These adaptations
%% to the typical design pattern methodology are proposed to parallel the
%% four themes outlined above.
%% \begin{itemize}
%% \item[(1)] We want to encode our design patterns directly in runnable
%%   programs, not just give them to programmers as heuristic guidance.
%% \item[(2)] We want the (automated) programmer to generate new design
%%   patterns, not just apply or adapt old ones.
%% \item[(3)] We want our design patterns themselves, working in
%%   combination, to contribute to the discovery of new emergent problems
%%   and patterns, not just capture the solutions to existing known
%%   problems.
%% \item[(4)] We want our design patterns to play an overt role in the
%%   dynamical systems they describe.
%% \end{itemize}
%%
\citeA{meszaros1998pattern} describe the typical scenario for authors of design
patterns: 

\begin{figure}[!ht]
\begin{mdframed}
\paragraph{\textbf{Successful error}}~
\vskip -1\baselineskip
\begin{flushright}\emph{Van Andel's example} -- Post-it\texttrademark\ Notes
\end{flushright}
\vspace{-.15cm}
\begin{description}[itemsep=2pt]
\item[{\tt context}] -- You run a creative organisation with several different divisions and many contributors with different expertise.  
\item[{\tt problem}] -- One of the members of your organisation
  discovers something with interesting properties, but no one
  knows how to turn it into a product with industrial or commercial application.
\item[{\tt solution}] -- You create a space for sharing and discussing
  interesting ideas on an ongoing basis (perhaps a Writers Workshop).
\item[{\tt rationale}] -- You suspect it's possible that one of the
  other members of the firm will come up with an idea about an
  application; you know that if a potential application is found, it
  may not be directly marketable, but at least there will be a
  prototype that can be concretely discussed.
\item[{\tt resolution}] -- The \emph{Successful error} pattern
  rewritten using this template is an example of a similar
  prototype, showing that serendipity can be talked about in
  terms of design patterns.
\end{description}
\end{mdframed}
\caption{Our design pattern template applied to van Andel's \emph{Successful error} pattern\label{fig:va-pattern-figure}}
\end{figure}

\noindent ``You are an experienced practitioner in your
field. You have noticed that you keep using a certain solution to a
commonly occurring problem. You would like to share your experience
with others.''  There are many ways to describe a solution.
Meszaros and Doble remark, ``What sets patterns apart is their
ability to explain the rationale for using the solution (the `why') in
addition to describing the solution (the `how').''  Regarding the
criteria that pattern writers seek to address: ``The most appropriate
solution to a problem in a context is the one that best resolves the
highest priority forces as determined by the particular context.'' 
%
%% Their article describes a number of criteria relevant to writing
%% good design patterns, e.g. \emph{Clear target audience},
%% \emph{Visible forces}, and \emph{Relationship to other patterns}.
%
A good design pattern \emph{describes} the resolution of forces in the
target domain; in the setting we're interested in, creating a new
design pattern also \emph{effects} a resolution of forces directly.
The use case of design pattern development maps into our diagram of
the basic features of serendipity as follows:

\begin{center}
\begingroup
\tikzset{
block/.style = {draw, fill=white, rectangle, minimum height=3em, minimum width=3em},
tmp/.style  = {coordinate}, 
sum/.style= {draw, fill=white, circle, node distance=1cm},
input/.style = {coordinate},
output/.style= {coordinate},
pinstyle/.style = {pin edge={to-,thin,black}}
}

\begin{tikzpicture}[auto, node distance=2cm,>=latex']
    \node [sum] (sum1) {};
    \node [input, name=pinput, above left=.7cm and .7cm of sum1] (pinput) {};
    \node [input, name=tinput, left=2cm of sum1] (tinput) {};
    \node [input, name=minput, below left of=sum1] (minput) {};
    \node [input, name=minput, right of=sum1] (moutput) {};
    \draw [->] (tinput) -- node{\vphantom{{\tiny g}}{\tiny context}} (sum1);
    \draw [->] (pinput) -- node{{\tiny problem}} (sum1);
    \draw [->] (sum1) -- node{\vphantom{{\tiny g}}{\tiny solution}}  (moutput);
\end{tikzpicture}
\hspace{1cm}
\begin{tikzpicture}[auto, node distance=2cm,>=latex']
    \node [sum] (sum1) {};
    \node [input, name=pinput, above left=.7cm and .7cm of sum1] (pinput) {};
    \node [input, name=tinput, left of=sum1] (tinput) {};
    \node [input, name=minput, below left of=sum1] (minput) {};
    \node [sum, right=1.5cm of sum1] (sum2) {};
    \node [input, name=minput, right of=sum2] (moutput) {};
    \draw [->] (tinput) -- node{\vphantom{{\tiny g}}{\tiny solution}} (sum1);
    \draw [->] (pinput) -- node{{\tiny rationale}} (sum1);
    \draw [->] (sum1) -- node{\vphantom{{\tiny g}}{\tiny pattern}} (sum2);
    \draw [->] (sum2) -- node[text width=1.5cm,execute at begin node=\setlength{\baselineskip}{.3ex}]{\tiny \emph{resolution\\~of forces}}  (moutput);
\end{tikzpicture}
\endgroup
\end{center}


