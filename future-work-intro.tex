\subsection{Future Work} \label{sec:futurework} \label{sec:hatching}

As a framework for managing the twinned processes of discovery and invention,
we are drawn to the approach taken by the \emph{design pattern}
community \cite{alexander1999origins}, although we propose to use
design patterns in rather nonstandard way:
\begin{itemize}
\item[(1)] We want to encode our design patterns directly in runnable
  programs, not just give them to programmers as heuristic guidance.
\item[(2)] We want the (automated) programmer to generate new design
  patterns, not just apply or adapt old ones.
\item[(3)] We want our design patterns to help find new problems,
  not just capture the solutions to existing ones.
\end{itemize}

\citeA{meszaros1998pattern} describe the typical scenario for design
pattern writers: ``You are an experienced practitioner in your
field. You have noticed that you keep using a certain solution to a
commonly occurring problem. You would like to share your experience
with others.''  They remark, ``What sets patterns apart is their
ability to explain the rationale for using the solution (the `why') in
addition describing the solution (the `how').''  Regarding the
criteria that pattern writers seek to address: ``The most appropriate
solution to a problem in a context is the one that best resolves the
highest priority forces as determined by the particular context.'' 
%% Their article describes a number of criteria relevant to writing
%% good design patterns, e.g. \emph{Clear target audience},
%% \emph{Visible forces}, and \emph{Relationship to other patterns}.

A good design pattern \emph{describes} the resolution of forces in the
target domain; at least in the setting we're interested in, creating a new
design pattern also \emph{effects} a resolution of forces directly.
The use case of design pattern development maps into our diagram of
the basic features of serendipity as follows:

\begin{center}
\begingroup
\tikzset{
block/.style = {draw, fill=white, rectangle, minimum height=3em, minimum width=3em},
tmp/.style  = {coordinate}, 
sum/.style= {draw, fill=white, circle, node distance=1cm},
input/.style = {coordinate},
output/.style= {coordinate},
pinstyle/.style = {pin edge={to-,thin,black}}
}

\begin{tikzpicture}[auto, node distance=2cm,>=latex']
    \node [sum] (sum1) {};
    \node [input, name=pinput, above left=.7cm and .7cm of sum1] (pinput) {};
    \node [input, name=tinput, left=2cm of sum1] (tinput) {};
    \node [input, name=minput, below left of=sum1] (minput) {};
    \node [input, name=minput, right of=sum1] (moutput) {};
    \draw [->] (tinput) -- node{\vphantom{{\tiny g}}{\tiny context}} (sum1);
    \draw [->] (pinput) -- node{{\tiny problem}} (sum1);
    \draw [->] (sum1) -- node{\vphantom{{\tiny g}}{\tiny solution}}  (moutput);
\end{tikzpicture}
\hspace{1cm}
\begin{tikzpicture}[auto, node distance=2cm,>=latex']
    \node [sum] (sum1) {};
    \node [input, name=pinput, above left=.7cm and .7cm of sum1] (pinput) {};
    \node [input, name=tinput, left of=sum1] (tinput) {};
    \node [input, name=minput, below left of=sum1] (minput) {};
    \node [sum, right=1.5cm of sum1] (sum2) {};
    \node [input, name=minput, right of=sum2] (moutput) {};
    \draw [->] (tinput) -- node{\vphantom{{\tiny g}}{\tiny solution}} (sum1);
    \draw [->] (pinput) -- node{{\tiny rationale}} (sum1);
    \draw [->] (sum1) -- node{\vphantom{{\tiny g}}{\tiny pattern}} (sum2);
    \draw [->] (sum2) -- node[text width=1.5cm,execute at begin node=\setlength{\baselineskip}{.3ex}]{\tiny \emph{resolution\\~of forces}}  (moutput);
\end{tikzpicture}
\endgroup
\end{center}


