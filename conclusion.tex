\section{Conclusion} \label{sec:conclusion}

%% In Section \ref{sec:literature-review}, we survey the broad literature
%% on serendipity, and examine prior applications of the concept of
%% serendipity in a computing context.
%
We began by surveying ``serendipity'', developing a broad historical
view, and several criteria which are computationally feasible.  Along
similar lines to \citeA{andre2009discovery}, we propose a two-phase
model.
%
%% In Section \ref{sec:background} we present our formal definition of
%% serendipity, drawing connections with historical examples and
%% presenting standards for evaluation.
%
Adapting the ``Standardised Procedure for Evaluating Creative
Systems'' we developed a set of assessment standards for serendipity.
%
%% Section \ref{sec:computational-serendipity} presents computational
%% case studies and thought experiments in terms of this model.
%
We used this model to examine several partial examples of serendipity,
in recommender systems, computerised jazz, and computational concept
invention.  We then presented a thought experiment that exhibits all
of the features of the model.
%
%% Section \ref{sec:discussion} offers recommendations for researchers
%% working in computational creativity (a key research area concerned
%% with the computational modelling of serendipity), and describes our
%% own plans for future work.
We then extracted recommendations related to the themes of
\emph{autonomy}, \emph{learning}, \emph{sociality}, and \emph{embedded
  evaluation} which appear to be corollaries of serendipitous
computing.  

%% Section \ref{sec:conclusion} reviews the argument and summarises the
%% limitations of our analysis.

% What answers have we offered?
The ideas presented in this article outline several possible
directions for implementation, but in any case considerable concrete
work remains to be done in order to realise our model in code.  Even
our hand-picked examples of prior art pale in comparison to the
examples of serendipitous discovery and invention from human history.
It would seem that a fully-automated system that can realistically be
said to behave in a serendipitous manner has not yet been built.
% Further questions
Nevertheless, the theoretical work in this paper shows that it is
indeed possible to plan -- and program -- for serendipity.
