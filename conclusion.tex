\section{Conclusion} \label{sec:conclusion}

%
We began by surveying ``serendipity'', developing a broad historical
view, and several criteria which are computationally feasible.  Along
similar lines to \citeA{andre2009discovery}, we propose a two-part
definition of serendipity: \emph{discovery} followed by
\emph{invention}.
%
Adapting the ``Standardised Procedure for Evaluating Creative
Systems'' we developed a set of assessment standards for serendipity.
%
We used this model to examine several prior examples of serendipity,
and developed a thought experiment that exhibits ``high serendipity''
a novel and computationally feasible design.
%
We then extracted several corollaries of our definition, which outline
a programme for serendipitous computing in the pursuit of
\emph{autonomy}, \emph{learning}, \emph{sociality}, and \emph{embedded
  evaluation and ethics}.

The argument presented here is not supported by experimental results,
but focuses on clarifying conceptual issues and examining design
implications.
% 
We indicate several possible further directions for implementation
work in each of our case studies.  We have also drawn attention to
more theoretical questions related to doing program design for an
autonomous programming context.  Our examples show that serendipity is
not foreign to computing practice.  There are further gains to be had
for research in computing by planning -- and programming -- for
serendipity.
%

