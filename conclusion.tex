\section{Conclusion} \label{sec:conclusion}

%% In Section \ref{sec:literature-review}, we survey the broad literature
%% on serendipity, and examine prior applications of the concept of
%% serendipity in a computing context.
%
We began by surveying ``serendipity'', developing a broad historical
view, and several criteria which are computationally feasible.  Along
similar lines to \citeA{andre2009discovery}, we propose a two-phase
model of serendipity as both \emph{discovery} and \emph{invention}.
%
%% In Section \ref{sec:background} we present our formal definition of
%% serendipity, drawing connections with historical examples and
%% presenting standards for evaluation.
%
Adapting the ``Standardised Procedure for Evaluating Creative
Systems'' we developed a set of assessment standards for serendipity.
%
%% Section \ref{sec:computational-serendipity} presents computational
%% case studies and thought experiments in terms of this model.
%
We used this model to examine several prior examples of serendipity,
and developed thought experiment that exhibits these features in a
novel design.
%
%% Section \ref{sec:discussion} offers recommendations for researchers
%% working in computational creativity (a key research area concerned
%% with the computational modelling of serendipity), and describes our
%% own plans for future work.
We then extracted several corollaries of our definitions, which
outline a programme for serendipitous computing rooted in
\emph{autonomy}, \emph{learning}, \emph{sociality}, and \emph{embedded
  evaluation and ethics}.

%% Section \ref{sec:conclusion} reviews the argument and summarises the
%% limitations of our analysis.

% What answers have we offered?
The ideas presented in this article point to several possible
directions for implementation, and to further theoretical questions
about program design and dynamics.  Our examples show that serendipity
is not alien to computing.  There are further gains to be had by
planning -- and programming -- for serendipity.
