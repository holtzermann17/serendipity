The key idea in this paper is to computationally model situations
where emergence of this particular sort can happen.
%
It will take some work to get there, however.  Section
\ref{sec:literature-review} develops 13 key criteria for the
evaluation of serendipity based on a review of several well-known
examples of serendipitous discoveries from human history.  Section
\ref{sec:foundations} describes a working testbed for exploring
serendipitous computational discovery.  In Section
\ref{sec:patterns-of-serendipity}, we apply our 13 criteria to analyse
several narrative ``patterns of serendipity'' collected by van Andel
\cite{van1994anatomy}.  Section \ref{sec:patterns-of-serendipity} is
the theoretical core of the paper; here we give our interpretation of
the design pattern methodology.  In Section
\ref{sec:computational-serendipity}, we focus on serendipity in a
computational context, condensing our criteria into an operational
definition, making our treatment of design patterns more concrete, and
proposing an experimental setup that we think will exhibit many of the
relevant features.  In Section \ref{sec:related}, we examine related
work, and in Section \ref{sec:recommendations}, we advance our
recommendations for researchers working on computational creativity
(and serendipity).
