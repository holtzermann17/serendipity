We begin with some words of caution.
%
Note that the classic examples of human serendipity tend to focus on
ground-breaking discoveries.  In computational creativity, we have
learned that we must not aim to build systems which perform
domain-changing acts of creativity before we can build systems which
can perform everyday, mundane creativity (distinguished as ``big C''
and ``little c'' creativity.)  Similarly, we should be prepared to
model ``little s'' serendipity before we are able to model ``big S''
serendipity.  Furthermore, attempts to introduce serendipity into
computer systems may initially diminish artefact value.
%
A system which allowed itself to be derailed from a task at hand might
not achieve as much as one which maintains focus.  A system that uses
a random search or that has its behaviour determined by environmental
conditions may be deemed less intelligent than one which follows
detailed, explicit, pre-programming.
%
To such arguments, we would respond that serendipity is not ``mere
chance'' -- the axes of sagacity (skills) and useful results
(recognised as such at least by the discoverer) are equally important.
