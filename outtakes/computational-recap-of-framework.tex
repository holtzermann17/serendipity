% What is the goal of the computation (input and output)
% Why is it appropriate (formal spec e.g. considering externalities)
% what is the logic of the strategy by which it can be carried out.

\textbf{[Do we need to include the partial repetition below or is the
    above formal enough?  Could these bulleted ideas be condensed into
    one or two paragraphs]}
    
Following Section \ref{specs-overview}, the 13 criteria can be used to evaluate the serendipity potential of existing systems, which we discuss first. The 13 criteria could also be used to guide design of future systems to maximise potential for serendipity; we explore this in the thought experiment outlined in Section \ref{sec:ww}.

\subsubsection*{Key condition for serendipity}

\begin{itemize}
\item \textbf{Focus shift}: A focus shift is linked to re-evaluation
  of data, processes, or products.  It may precipitate changes in the
  entire framework of evaluation or its effects may be more contained.
  Such reevaluation could be modelled using a multi-agent
  architecture, in which each agent has a goal and evaluates generated
  products relative this goal, but in which agents also share their
  products with other, who then evaluate them against their own
  metrics.
\end{itemize}

\subsubsection*{Components of serendipity}

\begin{itemize}
\item \textbf{Prepared mind}: This comprises the background knowledge,
  unsolved problems, current goal, programming, and operating
  environment of a computational system.
%%
\item \textbf{Serendipity trigger}: The generation or observation of a
  potentially novel example, concept, or conjecture, etc., which
  precedes a discovery in a computational system.\footnote{Triggers
    are often examples without an explanation, rather than
    wholly-formed concepts.}  The trigger is outside of the direct
  control of the system components responsible for evaluations.
%%
\item \textbf{Bridge}: Reasoning and/or programmatic interaction
  brings about a focus shift at an opportune juncture, building on
  prior preparation and on the serendipity trigger.  The bridge may be
  constructed on the basis of logical methods, analogies, conceptual
  blending, evolutionary search, automated theory formation and may
  draw on interactions with other systems.
%%
\item \textbf{Result}: The discovery itself may be a new product,
  artefact, process, hypothesis, use for an object, etc., generated by
  computational means, which may influence the future operations of
  the system.
\end{itemize}

\subsubsection*{Dimensions of serendipity}

\begin{itemize}
\item \textbf{Chance}: Controlled randomness in AI systems is
  well-established, e.g. in Genetic Algorithms and search.  Chance
  also applies in connection with an under-determined outside world
  (see below).
%%
\item \textbf{Curiosity}: The system needs to expend discretionary
  computational effort on the serendipity trigger.  This may be
  accompanied by system features that an observer would describe as
  playfulness, inventiveness, and the drive to experiment or
  understand.
%%
\item \textbf{Sagacity}: Sagacity be modelled by employing reasoning
  over multiple application domains simultaneously; or, again, with a
  social analogue in cases where the system does not know, but ``knows
  who to ask.''
%%
\item \textbf{Value}: The result should be interesting or useful, as
  judged by the system, the programmer, the user, or another party
  (potentially another system).
\end{itemize}

\subsubsection*{Environmental factors}

\begin{itemize}
\item \textbf{Dynamic world}: Connections with other systems, data
  sources, or user input, e.g., via the web, which is highly dynamic --
  or in the context of a larger simulation.
%%
\item \textbf{Multiple contexts}: Reasoning which operates across
  domains, such as analogical reasoning, or that considers multiple
  perspectives, as in systems with social awareness.
%%
\item \textbf{Multiple tasks}: Multiple goals or targets that compete
  for resources.  The system may be implemented using a multithreaded,
  parallel processing design.
%%
\item \textbf{Multiple influences}: This may again be modelled as a
  multi-agent systems, as or multiple interacting systems, each with
  different knowledge and goals.  The source of unexpectedness may be
  arise on various levels, and a system may bring this to bear using
  techniques of reflection.
\end{itemize}
