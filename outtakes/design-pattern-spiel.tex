Van Andel's \citeyear{van1994anatomy} ``patterns of serendipity'' are
often instances of this broader pattern.

%% \begin{quote}
%% ``Given research set up for a certain purpose, some unexpected, puzzling data, and a scientist capable of being puzzled -- given all of these, an accidental discovery will occur, because the relationship between fact and theory in science is such that it must occur.''
%% \end{quote}


In future work, we would like to explore the usefulness of the
somewhat more formal theory of \emph{design patterns}
\cite{alexander1999origins} for designing with serendipity in mind.
Alexandrian design patterns are by no means limited to computing: the
approach has its origins in architecture and urban planning
\cite{alexander1979timeless,alexander1977pattern}.

Design patterns prescribe and describe: they provide models \emph{for}
as well as models \emph{of}
\cite<cf.>[p. 93]{geertz1973interpretation}.  Thus, when Alexander
describes the pattern \emph{A place to wait}, he is also telling
readers that it may be a good idea to consider building a place to
wait when designing a living environment.

In connection with our understanding of serendipity as closely
associated with deviations from familiar patterns (in the everyday
sense of the word), it is interesting to ask how it can play a role in
the creation of new design patterns and pattern languages.  Noticing
and describing a new pattern is almost the antithesis of ``pattern
recognition'' in the usual computing sense.

As a beginning, we examined the 14 ``patterns of serendipity''
selected and described by van Andel \citeyear{van1994anatomy}, using the criteria
described in our Section \ref{sec:connections-to-formal-definition}.
We found all of these patterns do indeed include a focus shift, a
prepared mind, a serendipity trigger, a bridge, and a result, although
two of the patterns raised questions:

\begin{itemize}
\item In the case of \emph{Testing popular belief}, van Andel focuses
  on an account of a medical practise that originated in a folk claim,
  namely cowpox-derived immunity to smallpox.  It is challenging in
  this case to identify one specific serendipity trigger -- although a
  curious chain of events connected Edward Jenner with the smallpox
  vaccine.  It may be most appropriate to think of Jenner himself as
  the serendipity trigger at the societal level: his ``relentless
  promotion and devoted research of vaccination \ldots changed the way
  medicine was practised'' \cite{riedel2005edward}.
%% This effect, for milkmaids, might
%% indeed be called serendipitous. Indeed, the medical use of cowpox has
%% been described as ``widely know'' \cite{riedel2005edward} prior to its
%% popularisation by Edward Jenner.  Nevertheless, Jenner's
\item \emph{Inversion} is closer to what is called an
  \emph{antipattern} in the design pattern literature
  \cite{brown1998antipatterns}.  Van Andel describes the story of a
  researcher observing an effect due to the anticoagulant heparine
  which was precisely the opposite of the one sought -- factors that
  \emph{cause} blood clotting -- and then failing to acknowledge that
  this observation was important for over 40 years.  The result was
  eventually seen to be of value, however, again, this may be a
  pattern of \emph{antiserendipity}.
\end{itemize}

Among the 14 patterns, four are cases of ``perfect'' serendipity from
the point of view of our extended set of criteria (i.e.~they included
all of the 13 components, dimensions, and environmental factors) --
these patterns were \emph{Successful error}, \emph{Side effect},
\emph{Wrong hypothesis}, and \emph{Outsider}.
%
We wondered whether these were patterns might be used to support
serendipity in other settings -- such as the Writers Workshop.  Table
\ref{tab:reinterpret} gives an initial sketch, and initial experiments
that will bring this material to computational life are underway.

\begin{table}[p]
\begin{tabular}{lp{.7\textwidth}}
{\bf\emph{Successful error}} & \\
\emph{Van Andel's example}: & Post-it\texttrademark\ notes \\[.2cm]
{\tt presentation}& Systems should be prepared to share interesting ideas even if they don't know directly how they will be useful. \\
{\tt listening} & Systems should listen with interest, too. \\
{\tt feedback} & Even interesting ideas may not be ``marketable.''\\
{\tt questions} & How is your suggestion useful? \\
{\tt reflections} & New combinations of ideas take a long time to realise, and many different ideas may need to be combined in order to come up with something useful.\\
\end{tabular}
\bigskip

\begin{tabular}{lp{.7\textwidth}}
{\bf\emph{Side effect}} & \\
\emph{Van Andel's example}: & Nicotinamide used to treat side-effects of radiation therapy proves efficacious against tuberculosis. \\[.2cm]
{\tt presentation}& Systems should use their presentation as an experiment. \\
{\tt listening} & Listeners should allow themselves to be affected by what they are hearing. \\
{\tt feedback} & Feedback should convey the nature of the effect.\\
{\tt questions} & The presenter may need to ask follow-up questions to gain insight. \\
{\tt reflections} & Form a new hypothesis before seeking a new audience. \\
\end{tabular}
\bigskip

\begin{tabular}{lp{.7\textwidth}}
{\bf\emph{Wrong hypothesis}} & \\
\emph{Van Andel's example}: & Lithium, used in a control study, had an unexpected calming effect. \\[.2cm]
{\tt presentation}& How is this presentation interpretable as a (``natural'') control study? \\
{\tt listening} & Listeners are ``guinea pigs''.\\
{\tt feedback} & Discuss side-effects that do not necessarily correspond to the author's perceived intent. \\
{\tt questions} & Zero in on the most interesting part of the conversation.\\
{\tt reflections} & Revise hypotheses to correspond to the most surprising feedback. \\
\end{tabular}
\bigskip

\begin{tabular}{lp{.7\textwidth}}
{\bf\emph{Outsider}} & \\
\emph{Van Andel's example}: & A mother suggests a new hypothesis to a doctor. \\[.2cm]
{\tt presentation}& The presenter is here to learn from the audience. \\
{\tt listening} & The audience is here to give help, but also to get help.\\
{\tt feedback} & Feedback will inevitably draw on previous experiences and ideas.\\
{\tt questions} & What is the basis for that remark?\\
{\tt reflections} & How can I implement the suggestions?\\
\end{tabular}

\caption{Reinterpreting patterns of serendipity for use in a computational poetry workshop\label{tab:reinterpret}}
\end{table}
