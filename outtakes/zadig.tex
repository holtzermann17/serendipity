In the next subsection we will review several historical examples.
First, one further point should be made with reference to the ``The
Three Princes of Serendip''.  Prior to Walpole's coinage, this story
had been adapted by Voltaire into an early chapter of \emph{Zadig},
and in turn ``the method of Zadig'' informed subsequent approaches
both to fiction writing and natural science.  This method is rooted
firstly in discovery:

\begin{quote}
``[Zadig] \emph{pry’d into the Nature and Properties of Animals and
    Plants, and soon, by his strict and repeated Enquiries, he was
    capable of discerning a Thousand Variations in visible Objects,
    that others, less curious, imagin’d were all
    alike.}''~\cite[pp. 21--22]{zadig}
\end{quote}

\noindent Secondly, from disparate observations, Zadig is often able
to assemble a coherent picture:
\begin{quote}
\emph{It was his peculiar Talent to render Truth as obvious as
  possible: Whereas most Men study to render it intricate and
  obscure.}~\cite[p. 54]{zadig}
\end{quote}
