This paper has developed a perspective on how to model serendipity in
a computational context.  We advanced 13 criteria which were developed
based on review of the prior literature on serendipitous discovery.
We piloted these criteria as an evaluation framework by examining 14
patterns of serendipity that had been previously identified by van
Andel.  We found our criteria to be well represented, but not
uniformly present, and the exceptions are interesting; for instance,
we observed that \emph{A good story is liable to change}.  We then
advanced a unified approach to modelling serendipity grounded in
Deleuze's philosophy of difference, with a debt to the dynamical
interpretation of this work due to DeLanda, drawing as well on the
technical strategies employed by the interdisciplinary design pattern
community.  This approach was developed further into a proposed
experimental platform for doing collaborative research in
computational creativity.  We showed how four of van Andel's patterns
could be relevant in this setting, and introduced a new pattern
template oriented toward facilitating the encounter of computational
systems.
%
Finally, we surveyed related work, and summarised how computational
serendipity can contribute to the field of computational creativity.
We suggest that more attention should be focused on the role of
creativity in problem-setting, and on creative computer systems as a
key stakeholder group in computational creativity.  
