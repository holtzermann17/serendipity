\section{Introduction}

Although computational creativity is well studied in both theory and
practice, the role of \emph{serendipity} has been largely neglected
in this field -- even though serendipity has played a well-documented
role in historical instances of scientific and technical creativity.
One reason for this omission may be that the field of computational
creativity has tended to focus on artistic creativity, conceptualised in such a way that creative outputs are largely under the direct control of the creative agent.  However,  serendipity is increasingly seen as relevant within the arts
\cite{mckay-serendipity} and other enterprises, where it is encouraged with methods drawn from fields ranging from architecture to data science \cite{kakko2009homo,engineering-serendipity}. 


An interdisciplinary perspective on the phenomenon of serendipity
promises further illumination.  Here, we consider the potential for
formalising this concept. 
  This paper follows and expands \citeA{pease2013discussion}, where many of the ideas that are developed here were first presented.  The current paper reassesses and updates this earlier work, developing it towards a computational characterisation of serendipity for computational modelling and evaluation. New claims are advanced, positioning serendipity as a fundamental concept in computational creativity, with exciting potential to play a key role in computational intelligence more broadly.  There is particularly interesting potential for serendipity within computational systems whose processes involve interaction with users.\footnote{It should not be assumed, of course, that a system that can accommodate user interaction would directly lead to serendipity; take for example the use of a calculator, where potential for serendipity through user interaction is (at the greatest stretch of the imagination) minimal at best.} 

Serendipity is itself centred on reevaluation.  For example, a
non-sticky ``superglue'' that no one was quite sure how to use turned
out to be just the right ingredient for 3M's
Post-it\texttrademark\ notes.
%
Serendipity is related, firstly, to deviations from expected or
familiar patterns, and secondly, to new insight.
%
When we consider the practical uses for weak glue, the possibility
that a life-saving antibiotic might be found growing on contaminated
petri dishes, and or the idea that burdock burrs could be anything but
annoying, we encounter radical changes in the evaluation of what's
interesting.  In the \emph{d\'enouement}, what was initially
unexpected is found to be both explicable and useful.

Van Andel \citeyear{van1994anatomy} -- echoing Poincar\'e's
\citeyear{poincare1910creation} (negative) reflections on the potential
for a purely computational approach to mathematics -- claimed that:
\begin{quote}
``\emph{Like all intuitive operating, pure serendipity is not amenable
    to generation by a computer.  The very moment I can plan or
    programme `serendipity' it cannot be called serendipity
    anymore}.'' \cite{van1994anatomy}
\end{quote}
We believe that serendipity is not so mystical as such statements
might seem to imply, and in Section \ref{sec:discussion} we indicate
that ``patterns of serendipity'' like those collected by van Andel
are likely to applicable in computational settings.

First, in
Section \ref{sec:literature-review}, we survey the broad literature on
serendipity including etymology of the term itself, and examine prior applications of the concept of serendipity in a computing context.  Then in Section \ref{sec:our-model} we present our formal
definition of serendipity, drawing connections with historical examples 
and presenting standards for evaluation.  We further develop our model towards evaluative standards in Section \ref{specs-overview}. Section
\ref{sec:computational-serendipity} applies our work to computational case studies and
to a thought experiment in computational serendipity.  Section
\ref{sec:discussion} offers recommendations for researchers working in the computational modelling of serendipity and related areas such as computational creativity, and describes our own plans for future
work.  Section \ref{sec:conclusion} reviews the contributions of this paper towards computational modelling and evaluation of serendipity. This section also clarifies the limitations of this work thus far and extracts key themes around which fascinating challenges are posed in future work on computational serendipity.


