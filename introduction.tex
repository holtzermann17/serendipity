\section{Introduction}

Materials, like gold, and processes, like metalurgy, have no value
without a context of application: decoration, trade, circuitry, and so
on.  In practice, we are likely to attribute \emph{value} to materials
that are useful, and \emph{creativity} to a person who puts materials
to use in a novel way.

Although computational creativity is well studied in both theory and
practice, the role of \emph{serendipity} has often not been discussed
in this field -- even though serendipity has played a well-documented
role in historical instances of scientific and technical creativity.
One reason for this omission may be that the field of computational
creativity has tended to focus on artistic creativity.  But
serendipity is increasingly seen as relevant within the arts
\cite{mckay-serendipity} and other creative enterprises
\cite{kakko2009homo,engineering-serendipity}: it is managed and
encouraged with methods ranging from architecture to data science.
%
An interdisciplinary perspective on the phenomenon of serendipity
promises further illumination.  Here, we consider the potential for
formalising this concept and investigate its utility as a new
framework for computational creativity.

Serendipity centres on reassessment.  For example, a non-sticky
``superglue'' that no one was quite sure how to use turned out to be
just the right ingredient for 3M's Post-it\texttrademark\ notes.
%
Serendipity is related, firstly, to deviations from expected or
familiar patterns, and secondly, to new insight.
%
When we consider the practical uses for weak glue, the possibility
that a life-saving antibiotic might be found growing on contaminated
petri dishes, and or the idea that cockle-burs could be anything but
annoying, we encounter radical changes in the evaluation of what's
interesting.  In the \emph{d\'enouement}, what was initially
unexpected is found to be both explicable and useful.

Van Andel \citeyear{van1994anatomy} -- echoing Poincar\'e's
\citeyear{poincare1910creation} (negative) reflections on the potential
for a purely computational approach to mathematics -- claimed that:
\begin{quote}
``\emph{Like all intuitive operating, pure serendipity is not amenable
    to generation by a computer.  The very moment I can plan or
    programme `serendipity' it cannot be called serendipity
    anymore}.'' \cite{van1994anatomy}
\end{quote}
We believe that serendipity is not so mystical as such statements
might seem to imply, and in Section \ref{sec:discussion} we indicate
van Andel's ``patterns of serendipity'' are likely to be highly
applicable in computational settings.

First, in Section \ref{sec:background} we present our formal
definition of serendipity, and examine related work that has applied
the concept of serendipity in a computational context.  Then in
Section \ref{sec:literature-review}, we survey the broad literature on
serendipity, making connections from historical examples of
serendipitous discovery and invention to our formal model.  Section
\ref{sec:computational-serendipity} then presents case studies and
thought experiments in terms of this model.  Section
\ref{sec:discussion} offers recommendations for researcher working in
the computational creativity, and describes our own plans for future
work.  Section \ref{sec:conclusion} reviews the argument and
summarises the limitations of our analysis.


