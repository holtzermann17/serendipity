\section{Introduction}

Materials, like gold, and processes, like metalurgy, have no value
without a context of application: decoration, trade, circuitry, and so
on.  In practice, we are likely to attribute \emph{value} to materials that
are useful, and \emph{creativity} to a person who puts materials to use in a
novel way.
%
Many instances of \emph{serendipity} centre on reevaluation.  For
example, a non-sticky ``superglue'' that no one was quite sure how to
use turned out to be just the right ingredient for 3M's
Post-it\texttrademark\ notes.
%
Serendipity is related, firstly, to deviations from familiar patterns,
and secondly, to new insight.
%
When we consider the practical uses for weak glue, the possibility
that a life-saving antibiotic might be found growing on contaminated
petri dishes, and or the idea that cockle-burs could be anything but
annoying, we encounter radical changes in the evaluation of what's
interesting.  In the \emph{d\'enouement}, what was initially
unexpected is found to be both explicable and useful.

Van Andel \citeyear{van1994anatomy} -- echoing Poincar\'e's
\citeyear{poincare1910creation} (negative) reflections on the potential
for a purely computational approach to mathematics -- claimed that:
\begin{quote}
``\emph{Like all intuitive operating, pure serendipity is not amenable
    to generation by a computer.  The very moment I can plan or
    programme `serendipity' it cannot be called serendipity
    anymore}.'' \cite{van1994anatomy}
\end{quote}
We believe that serendipity is not so mystical as such statements
might imply, and in Sections \ref{sec:patterns-of-serendipity} and
\ref{sec:computational-serendipity} we will show how it is possible to
reinterpret van Andel's ``patterns of serendipity'' in computational
settings.  

The real problem with computers is not that they only do what they're
told, but that the act of programming forces us to confront the
emergence of the new \cite{mead1932philosophy}.
%
Minsky \citeyear{minsky1967programming} suggests that in practice,
programmers write programs ``for the individuals of little societies''
precisely because we cannot envision in advance all of the details of
program interactions.
%
Indeterminacy forms an important part of any proposal for
``intelligent machines'', after Turing:
\begin{quote}
``\emph{They will make mistakes at times, and at times they may make
    new and very interesting statements, and on the whole the output
    of them will be worth attention to the same sort of extent as the
    output of a human mind}.''  \cite{turing-intelligent}
\end{quote}

Serendipity has played a role in the large-scale history of the
computing field \cite{de2013turing} and in artistic applications of
computer technology \cite{reichardt1969cybernetic}.  We aim to clarify
the role it has to play in the future development of computational
creativity.

Whereas van Andel speaks of ``patterns of serendipity'' in a
relatively informal way, this paper will rely on the somewhat more
formal theory of \emph{design patterns} \cite{alexander1999origins},
to which it makes several additions and alterations.  This theory is
by no means limited to computing, and indeed, has its origins in
architecture and urban planning.  Our approach to ``designing for
serendipity'' \cite{andre2009discovery} centres on the use of design
patterns to capture the dynamic aspects of serendipitous situations.

The typical use of design patterns, since they were introduced by
Christopher Alexander
\cite{alexander1979timeless,alexander1977pattern}, is to prescribe as
well as to describe.  Design patterns provide models \emph{for} as
well as models \emph{of} \cite<cf.>[p. 93]{geertz1973interpretation}.
Thus, when Alexander describes the pattern \emph{A place to wait}, he
is telling readers that it is a good idea to consider building such
places when designing living spaces.  In connection with our
understanding of serendipity as closely associated with deviations
from familiar patterns, the central concern in this paper is the way
in which \emph{new} patterns are formed.

For example, when Poincar\'e \citeyear{poincare1910creation} describes his
discovery of the existence of Fuchsian functions, he includes the
detail: ``contrary to my habit I took black coffee, I could not
sleep.''  This is much more interesting as part of a story about an
exceptional case of productive insomnia than it is as the broad
characterisation of a typical nightly sleep schedule.  It might best
be described as a part of a ``situational pattern,'' with a title like
\emph{Change of pace}, rather than a ``behaviour pattern''; indeed, at
the level of behaviour, a \emph{Change of pace} is the exception to a
pattern!  Nevertheless, along with Poincar\'e, we can recognize a
pattern at another level.

The key idea in this paper is to computationally model situations
where emergence of this particular sort can happen.
%
It will take some work to get there, however.  Section
\ref{sec:literature-review} develops 13 key criteria for the
evaluation of serendipity based on a review of several well-known
examples of serendipitous discoveries from human history.  Section
\ref{sec:foundations} describes a working testbed for exploring
serendipitous computational discovery.  In Section
\ref{sec:patterns-of-serendipity}, we apply our 13 criteria to analyse
several narrative ``patterns of serendipity'' collected by van Andel
\cite{van1994anatomy}.  Section \ref{sec:patterns-of-serendipity} is
the theoretical core of the paper; here we give our interpretation of
the design pattern methodology.  In Section
\ref{sec:computational-serendipity}, we focus on serendipity in a
computational context, condensing our criteria into an operational
definition, making our treatment of design patterns more concrete, and
proposing an experimental setup that we think will exhibit many of the
relevant features.  In Section \ref{sec:related}, we examine related
work, and in Section \ref{sec:recommendations}, we advance our
recommendations for researchers working on computational creativity
(and serendipity).
