\section{Introduction}

Materials, like gold, and processes, like metalurgy, have no value
without a context of application: decoration, trade, circuitry, and so
on.  In practice, we are likely to attribute \emph{value} to materials that
are useful, and \emph{creativity} to a person who puts materials to use in a
novel way.

Although computational creativity is well studied in both theory and
practice, the role of \emph{serendipity} has often not been discussed
in this field -- even though serendipity has often played a role in
historical instances of scientific and technical creativity.  Many
instances of serendipity centre on reevaluation.  For example, a
non-sticky ``superglue'' that no one was quite sure how to use turned
out to be just the right ingredient for 3M's
Post-it\texttrademark\ notes.
%
Serendipity is related, firstly, to deviations from expected or
familiar patterns, and secondly, to new insight.
%
When we consider the practical uses for weak glue, the possibility
that a life-saving antibiotic might be found growing on contaminated
petri dishes, and or the idea that cockle-burs could be anything but
annoying, we encounter radical changes in the evaluation of what's
interesting.  In the \emph{d\'enouement}, what was initially
unexpected is found to be both explicable and useful.

Van Andel \citeyear{van1994anatomy} -- echoing Poincar\'e's
\citeyear{poincare1910creation} (negative) reflections on the potential
for a purely computational approach to mathematics -- claimed that:
\begin{quote}
``\emph{Like all intuitive operating, pure serendipity is not amenable
    to generation by a computer.  The very moment I can plan or
    programme `serendipity' it cannot be called serendipity
    anymore}.'' \cite{van1994anatomy}
\end{quote}
We believe that serendipity is not so mystical as such statements
might imply, and in Section \ref{sec:computational-serendipity} we
will show how it is possible to reinterpret van Andel's ``patterns of
serendipity'' in computational settings.

The real problem with computers is not that they only do what they're
told, but that the act of programming forces us to confront the
emergence of the new \cite{mead1932philosophy}.
%
Indeterminacy forms an important part of any proposal for
``intelligent machines'', after Turing:
\begin{quote}
``\emph{They will make mistakes at times, and at times they may make
    new and very interesting statements, and on the whole the output
    of them will be worth attention to the same sort of extent as the
    output of a human mind}.''  \cite{turing-intelligent}
\end{quote}

Serendipity has played a role in the large-scale history of the
computing field \cite{de2013turing} and in artistic applications of
computer technology \cite{reichardt1969cybernetic}.  We aim to clarify
the role it has to play in the future development of computational
creativity.




