\section{Introduction}

Although computational creativity is well studied in both theory and
practice, the role of \emph{serendipity} has been largely neglected
in this field -- even though serendipity has played a well-documented
role in historical instances of scientific and technical creativity.
One reason for this omission may be that the field of computational
creativity has tended to focus on artistic creativity, which is modelled in such a way that creative outputs are largely under the direct control of the creative agent.  However,  
serendipity is increasingly seen as relevant within the arts
\cite{mckay-serendipity} and other creative enterprises
\cite{kakko2009homo,engineering-serendipity}, where it is managed and
encouraged with methods drawn from fields ranging from architecture to data science.
%
An interdisciplinary perspective on the phenomenon of serendipity
promises further illumination.  Here, we consider the potential for
formalising this concept.  This paper follows and expands \citeA{pease2013discussion}, which is a robust brief survey where many of the ideas developed here were first presented.  The current paper uses the opportunity of a fresh start and a wider canvas to advance some bold claims about the usefulness of serendipity as a new framework for computational creativity.

Serendipity itself is centred on reassessment.  For example, a non-sticky
``superglue'' that no one was quite sure how to use turned out to be
just the right ingredient for 3M's Post-it\texttrademark\ notes.
%
Serendipity is related, firstly, to deviations from expected or
familiar patterns, and secondly, to new insight.
%
When we consider the practical uses for weak glue, the possibility
that a life-saving antibiotic might be found growing on contaminated
petri dishes, and or the idea that cockle-burs could be anything but
annoying, we encounter radical changes in the evaluation of what's
interesting.  In the \emph{d\'enouement}, what was initially
unexpected is found to be both explicable and useful.

Van Andel \citeyear{van1994anatomy} -- echoing Poincar\'e's
\citeyear{poincare1910creation} (negative) reflections on the potential
for a purely computational approach to mathematics -- claimed that:
\begin{quote}
``\emph{Like all intuitive operating, pure serendipity is not amenable
    to generation by a computer.  The very moment I can plan or
    programme `serendipity' it cannot be called serendipity
    anymore}.'' \cite{van1994anatomy}
\end{quote}
We believe that serendipity is not so mystical as such statements
might seem to imply, and in Section \ref{sec:discussion} we indicate
that ``patterns of serendipity'' like those collected by van Andel
are likely to applicable in computational settings.

First, in
Section \ref{sec:literature-review}, we survey the broad literature on
serendipity, and examine prior applications of the concept of serendipity in a computing context.  Then in Section \ref{sec:background} we present our formal
definition of serendipity, drawing connections with historical examples 
and presenting standards for evaluation.  Section
\ref{sec:computational-serendipity} then presents case studies and
thought experiments in terms of this model.  Section
\ref{sec:discussion} offers recommendations for researchers working in
computational creativity (a key research area concerned with the computational modelling of serendipity), and describes our own plans for future
work.  Section \ref{sec:conclusion} reviews the argument and
summarises the limitations of our analysis.


