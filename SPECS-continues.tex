\subsubsection*{ Step 2: Evaluation standards for computational serendipity}
\begin{quote} {\em Using Step 1, clearly state what standards you use to evaluate the serendipity of your
    system. }\end{quote}

With our definition in mind, we propose the following standards for evaluating our definition of
serendipity:

\begin{quote}
\begin{description}
\item[\emph{Prepared mind}] \emph{The system can be said to have a
  prepared mind, consisting of previous experiences, background
  knowledge, a store of unsolved problems, skills, expectations, and
  (optionally) a current focus or goal.}
\item[\emph{Serendipity trigger}] \emph{The serendipity trigger is at
  least partially the result of factors outside the system's control.
  These may include randomness or simple unexpected events.  The
  trigger should be determined independently from the end result.}
\item[\emph{Bridge}] \emph{The system uses reasoning techniques
  that support a process of invention -- e.g.~abduction, analogy,
  conceptual blending -- and/or social or otherwise externally enacted
  alternatives -- to create a bridge from the trigger to a result.}
\item[\emph{Result}] \emph{A novel result is obtained, which is
  evaluated as useful, by the system and/or by an external source.}
\end{description}
\end{quote}

\subsubsection*{Step 3: Testing our serendipitous system}

\begin{quote} {\em Test your serendipitous system against the standards stated in Step 2 and report the
results.}\end{quote}

We will develop several examples of the application of this framework
in Section \ref{sec:computational-serendipity}.

%% In order to develop connections with our theoretical framework, and
%% because existing experiments have not been particularly strong, we
%% focus on a thought experiment in the
%% Section \ref{sec:computational-serendipity} detailing some of the
%% outcomes we would like to see, and some of the risks.
