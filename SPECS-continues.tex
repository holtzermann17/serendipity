\subsubsection*{Step 2: Evaluation standards for computational serendipity}
\begin{quote} {\em Using Step 1, clearly state what standards you use to evaluate the serendipity of your
    system. }\end{quote}

\noindent Here we need to identify testable standards from our definition of computational serendipity. in other words, we now state the key parts of our definition in a form that can be evaluated as to what degree they are or are not met. 

With our definition in mind, we propose the following standards for evaluating serendipity in computational systems:

%% Serendipity relies on a reassessment or reevaluation -- a \emph{focus shift} in which something that was previously uninteresting, of neutral, or even negative value, becomes interesting.

\begin{description}[itemsep=4pt]
\item[\emph{(\textbf{A - Definitional characteristics})}] \emph{The
  system can be said to have a \emph{\textbf{prepared mind}},
  consisting of previous experiences, background knowledge, a store of
  unsolved problems, skills, expectations, and (optionally) a current
  focus or goal.  It then processes a \emph{\textbf{serendipity
  trigger}} that is at least partially the result of factors outside
  of its control, including randomness or unexpected events.  The
  system then uses reasoning techniques and/or social or otherwise
  externally enacted alternatives to create a \emph{\textbf{bridge}}
  from the trigger to a result.  The \emph{\textbf{result}} is
  evaluated as useful, by the system and/or by an external source.}
\item[\emph{(\textbf{B - Dimensions})}] \emph{Serendipity, and its
  various dimensions, can be present to a greater or lesser degree.
  If the criteria above have been met, we consider the system (and optionally, generate ratings as
  estimated probabilities) along several dimensions:
%
\emph{($\mathbf{a}$ - \textbf{chance})} how likely was this trigger to appear to
  the system?
%
\emph{($\mathbf{b}$ - \textbf{curiosity})} On a population basis, comparing
  similar circumstances, how likely was the trigger to be identified
  as interesting?
%
\emph{($\mathbf{c}$ - \textbf{sagacity})} On a population basis, comparing
  similar circumstances, how likely was it that the trigger
  would be turned into a result?
%
Finally, we ask, again, comparing similar results where possible:
\emph{($\mathbf{d}$ - \textbf{value})} How valuable is the result that
is ultimately produced?}

\medskip

\emph{Then aggregating $\mathbf{a}\times\mathbf{b}\times\mathbf{c}$ gives a
  likelihood score: low likelihood $\mathbf{a}\times\mathbf{b}\times\mathbf{c}$ and high value $\mathbf{d}$ are the criteria we use to say that the event was ``highly serendipitous.''}

\item[\emph{(\textbf{C - Factors})}] \emph{Finally, if the criteria
  from Part A are met, and if the event is deemed ``highly
  serendipitous'' according to the criteria in Part B, then in order
  to deepen our qualitative understanding of the serendipitous
  behaviour, we ask: To what extent does the system exist in a
  \emph{\textbf{dynamic world}}, spanning \emph{\textbf{multiple
      contexts}}, featuring \emph{\textbf{multiple tasks}}, and
  incorporating \emph{\textbf{multiple influences}}?}
\end{description}

\subsubsection*{Step 3: Testing our serendipitous system}

\begin{quote} {\em Test your serendipitous system against the standards stated in Step 2 and report the
results.}\end{quote}

\noindent We will develop several examples of the application of this framework
to examples from computing in Section
\ref{sec:computational-serendipity}.  

\paragraph{Example.}
To get a feel for the likelihood score, briefly consider the case of
de Mestral's invention of Velcro\texttrademark.  There is a
\emph{high} chance of encountering burrs while out walking, but
\emph{very few} of de Mestral's contemporaries would have thought to
investigate them under a microscope, and furthermore even among the
scientifically minded \emph{very few} would conceive of a useful
application or have the perseverance required to carry out the
associated product development.  Hook-and-loop fasteners are now
widely used; and we can call the invention ``highly serendipitous.''

