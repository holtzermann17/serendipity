\subsection{Serendipity by example} \label{sec:by-example}

We adapt the conceptual framework for describing serendipity proposed
by \citeA{pease2013discussion}.  This section will briefly introduce
the relevant concepts, and illustrate them by means of historical
examples of serendipity.

\subsubsection*{Key condition for serendipity.}

Serendipity relies on a reassessment or reevaluation -- a \emph{focus shift} in which something that was previously uninteresting, of neutral, or even negative value, becomes interesting.

\begin{itemize}
\item \textbf{Focus shift}: George de Mestral, an electrical engineer
  by training, and an experienced inventor, returned from a hunting
  trip in the Alps.  He removed several burdock burrs from his clothes
  and his dog's fur and became curious about how they worked. After
  examining them under a microscope, he realised the possibility of
  creating a new kind of fastener that worked in a similar fashion,
  laying the foundations for the hook-and-loop mechanism in Velcro\texttrademark.
% \cite[p. x]{roberts}
\end{itemize}

\subsubsection*{Components of serendipity.}

A focus shift is brought about by the meeting of a \emph{serendipity trigger} and a \emph{prepared mind}.  The next step involves building a \emph{bridge} to a valuable \emph{result}.

\begin{itemize}
\item \textbf{Prepared mind}: 
Fleming's ``prepared mind'' included his focus
on carrying out experiments to investigate influenza as well as his
previous experience that showed that foreign substances in petri dishes can kill
bacteria.  He was concerned above all with the question ``Is there a
substance which is harmful to harmful bacteria but harmless to human
tissue?''  \cite[p. 161]{roberts}.
\end{itemize}

\begin{itemize}
\item \textbf{Serendipity trigger}: The trigger does not directly
  cause the outcome, but rather, inspires a new insight.  It was long
  known by Quechua medics that cinchona bark stops shivering.  In
  particular, it worked well to stop shivering in malaria patients, as
  was observed when malarial Europeans first arrived in Peru.  The
  joint appearance of shivering Europeans and a South American remedy
  was the trigger.  That an extract from cinchona bark can cure and
  can even prevent malaria was understood subsequently.
\end{itemize}

\begin{itemize}
\item \textbf{Bridge}: These include reasoning techniques, such as
  abductive inference (what might cause a clear patch in a petri
  dish?); analogical reasoning (de Mestral constructed a target domain
  from the source domain of burrs hooked onto fabric); and conceptual
  blending (Kekul\'e, discoverer of the benzene ring structure, blended his knowledge of molecule structure with
  his vision of a snake biting its tail).  The bridge may also rely on
  new social arrangements, such as the formation of cross-cultural
  research networks.
\end{itemize}

\begin{itemize}
\item \textbf{Result}: This may be a new product, artefact, process,
  hypothesis, a new use for a material substance, and so on.  The
  outcome may contribute evidence in support of a known hypothesis, or
  a solution to a known problem.  Alternatively, the result may itself
  {\em be} a new hypothesis or problem.  The result may be
  ``pseudoserendipitous'' in the sense that it was {\em sought}, while
  nevertheless arising from an unknown, unlikely, coincidental or
  unexpected source.  More classically, it is an \emph{unsought}
  finding, such as the discovery of the Rosetta stone.
\end{itemize}

\subsubsection*{Dimensions of serendipity.}

The four components described above have attributes that may be present to a greater or lesser degree.  These are: \emph{Chance} -- how likely was the trigger to appear?; \emph{Curiosity} -- how likely was this trigger to be identified as interesting?; \emph{Sagacity} -- how likely was it that the interesting trigger would be turned into a result?; -- and \emph{Value} (how valuable is the result that is ultimately produced?).

\begin{itemize}
\item \textbf{Chance}: Fleming \citeyear{fleming} noted: ``There are
  thousands of different moulds'' -- and ``that chance put the mould
  in the right spot at the right time was like winning the Irish
  sweep.''  It is important to notice that \emph{he} was in the right
  spot at the right time as well -- and that this was not a complete
  coincidence.  The chance events we're interested in always include
  at least one observer.
\end{itemize}

\begin{itemize}
\item \textbf{Curiosity}: Curiosity can dispose a creative person to
  begin or to continue a search into unfamiliar territory.  We use
  this word to describe both simple curiousity and related deeper
  drives.  Charles Goodyear \citeyear{goodyear1855gum} reflects on his
  own life experience as follows: ``from the time his attention was first given
  to the subject, a strong and abiding impression was made upon his
  mind, that an object so desirable and important, and so necessary to
  man's comfort, as the making of gum-elastic available to his use,
  was most certainly placed within his reach.  Having this
  presentiment, of which he could not divest himself, under the most
  trying adversity, he was stimulated with the hope of ultimately
  attaining this object.''
\end{itemize}

\begin{itemize}
\item \textbf{Sagacity}: This old-fashioned word is related to
  ``wisdom,'' ``insight,'' and especially to ``taste'' -- and
  describes the attributes, or skill, of the discoverer that
  contribute to forming the bridge between the trigger and the result.
  \citeA{merton1948bearing} writes: ``{[}M{]}en had for centuries
  noticed such `trivial' occurrences as slips of the tongue, slips of
  the pen, typographical errors, and lapses of memory, but it required
  the theoretic sensitivity of a Freud to see these as strategic data
  through which he could extend his theory of repression and
  symptomatic acts.''
\end{itemize}

%% Note that the chance ``discovery'' of, say, a \pounds 10 note may
%% be seen as happy by the person who finds it, whereas the loss of
%% the same note would generally be regarded as unhappy.

\begin{itemize}
\item \textbf{Value}: 
  Positive judgements of serendipity by a third party would be less
  likely in scenarios in which ``One man's loss is another man's
  gain'' than in scenarios where ``One man's trash is another man's
  treasure.''  One quite literal example is the Swiss
  company Freitag, started by design students who built a business
  around ``upcycling'' used truck tarpaulins into bags and backpacks.
  Thanks in part to clever marketing \cite[pp. 54--55,
    68--69,]{russo2010companies}, their product has sold well.
  Wherever possible, we prefer an independent judgement of value
  \cite{jordanous:12}.
\end{itemize}

\subsubsection*{Environmental factors.}

Finally, serendipity seems to be more likely for agents who experience and participate in a \emph{dynamic world}, who are active in \emph{multiple contexts}, occupied with \emph{multiple tasks}, and who avail themselves of \emph{multiple influences}.

\begin{itemize}
\item \textbf{Dynamic world}: Information about the world develops
  over time, and is not presented as a complete, consistent whole.  In
  particular, \emph{value} may come later.  Van Andel
  \citeyear[p. 643]{van1994anatomy} estimates that in twenty percent
  of innovations ``something was discovered before there was a demand
  for it.''  To illustrate the role of this factor, it may be most
  revealing to consider a ``counterexample,'' in which dynamics were
  not attended to carefully and the process suffers as a result.
  Cropley \citeyear{cropley2006praise} describes Eugen Semmer's
  failure to recognise the role of \emph{penicillium notatum} in
  restoring two unwell horses to health: ``Semmer saw the horses'
  return to good health as a problem that made it impossible for him
  to investigate the cause of their death, and reported \ldots\ on
  how he had succeeded in eliminating the mould from his laboratory!''
\end{itemize}

\begin{itemize}
\item \textbf{Multiple contexts}: One of the dynamical aspects at play
  may be the discoverer going back and forth between different
  contexts, with different stimuli.  3M employee Arthur Fry sang in a
  church choir and needed a good way to mark pages in his hymn book;
  he happened to have been attending seminars offered by his colleague
  Silver about restickable glue.
\end{itemize}

\begin{itemize}
\item \textbf{Multiple tasks}: Even within what would typically be
  seen as a single context, a discoverer may take on multiple tasks
  that segment the context into sub-contexts, or that cause the
  investigator to look in more than one direction.  The tasks may have
  an interesting \emph{overlap}, or they may point to a \emph{gap} in
  knowledge.  As an example of the latter, Penzias and Wilson used a
  large antenna to detect radio waves that were relayed by bouncing
  off of satellites.  After they had removed interference effects due
  to radar, radio, and heat, they found residual ambient noise that
  couldn't be eliminated.
\end{itemize}

\begin{itemize}
\item \textbf{Multiple influences}: The ``bridge'' from trigger to
  result is often found through a social network, thus, for instance
  Penzias and Wilson only understood the significance of their work
  after reading a preprint by Jim Peebles that hypothesised the
  possibility of measuring radiation released by the big bang.
\end{itemize}

\noindent We will show how the key condition, the components,
dimensions and environmental factors of serendipity can be modelled
and assessed in computational systems in Sections \ref{sec:our-model}
and \ref{sec:computational-serendipity}.
