\subsection{Related work} \label{sec:related}

An active research community investigating computational models of serendipity exists in the field of information retrieval, and specifically, in recommender systems \cite{Toms2000}. In this domain, \citeA{Herlocker2004} and \citeA{McNee2006} view serendipity as an important factor for user satisfaction, next to accuracy and diversity.  Serendipity in recommendations variously require the system to deliver an \emph{unexpected} and \emph{useful} \cite{Lu2012}, \emph{interesting} \cite{Herlocker2004}, \emph{attractive} or \emph{relevant} item \cite{Ge2010}. 
%% Recommendations are typically meant to help address the user's difficulty in finding items that meet his or her interests or demands within a large and potentially unobservable search space. The end user can also be passive, and items are suggested to support other stakeholder's goals, e.g. to increase sells. 
Definitions differ as to the requirement of \emph{novelty}; \citeA{Adamopoulos2011}, for example, describe systems that suggest items that may already be known, but are still unexpected in the current context.  While standardized measures such as the $F_1$-score or the (R)MSE are used to determine the accuracy of an evaluation in terms of preferred items in the user's history, there is no common agreement on a measure for serendipity yet, although there are several proposals \cite{Murakami2008, Adamopoulos2011, McCay-Peet2011}.
  In terms of our model, these systems focus mainly on producing a serendipity trigger, but they include aspects of user modeling which could bring other elements into play.

Paul Andr{\'e} et al.~\citeyear{andre2009discovery} have examined
serendipity from a design perspective.  These authors also propose a
two-part model, in which what we have called \emph{discovery} above
exposes the unexpected, while \emph{invention} is the responsibility
another subsystem that finds applications.  According to Andr\'e et
al., the first phase is the one that has most frequently been
automated -- but they suggest that computational systems should be
developed that support both aspects.  Their specific suggestions focus
on representational features: \emph{domain expertise} and a
\emph{common language model}.

Although tremendously useful when they are available, these features
are not always enough to account for serendipitious events.  Using the
terminology we introduced earlier, these features seem to exemplify
aspects of the \emph{prepared mind}.  However, as we mentioned above,
the \emph{bridge} is a distinct process that mental preparation can
support, but not always fully determine.  For example, participants in
a poetry workshop may possess a very limited understanding of each
other's aims or of the work they are critiquing, and may as a
consequence talk past one another to a greater or lesser degree --
while nevertheless finding the overall process of participating in the
workshop itself illuminating and rewarding (often precisely because
such misunderstandings elucidate poor communication choices!).
Various social strategies, ranging from Writers Workshops to open
source software, pair programming, and design charettes
\cite[p. 11]{gabriel2002writer} have been developed to exploit similar
emergent effects and to develop \emph{new} shared language.  In
\cite{poetry-workshop}, we investigate the feasibility of using
designs of this sort in multi-agent systems that learn by sharing and
discussing partial understandings.  This earlier paper remains broadly
indicative, however, and the ideas it describes can see considerable
benefit from the more formal thinking we develop in the current work.

\citeA{robot-rendezvous} develop a discussion of serendipitous
rendezvous in a multi-agent system for a graph exploration problem, in
which ``[h]aving more data about their colleagues, better decisions
are made about the potential serendipity path.''  This has some
similarity to the discursive scenario described above, and shows that
\emph{asymmetric partial knowledge} can support serendipitious
findings.  The distinction between knowledge of other actors and
knowledge about an underlying domain is useful.

The issue of designing for serendipity has been taken up recently by
Deborah Maxwell et al.~\citeyear{maxwell2012designing}, in their
description of a prototype of the {\sf SerenA} system.  This system is
designed to support serendipitous discovery for its (human) users
\cite{forth2013serena}.  The authors rely on a process-based model of
serendipity \cite{Makri2012,Makri2012a} that is derived from user
studies, including interviews with 28 researchers, looking for
instances of serendipity from both their personal and professional
lives.  This material was coded along three dimensions:
\emph{unexpectedness}, \emph{insightfulness}, and \emph{value}.  This
research aims to support the process of forming bridging connections
from unexpected encounter to a previously unanticipated but valuable
outcome.  They particularly focus on the acts of \emph{reflection}
that foment both the creation of a bridge and estimates of the
potential value of the result.
%
Although this description touches on all of the features of our model, {\sf
  SerenA} largely matches the description offered by Andr{\'e} et
al.~\citeyear{andre2009discovery} of discovery-focused systems, and a
case study of a recommender system focused on serendipity, in which
%%   Here, the
%% user is the primary agent with a prepared mind.  Accordingly it 
is user experiences an ``aha'' moment and takes the
creative steps to realise the result.  {\sf SerenA}'s primary computational method is to
search outside of the normal search parameters in order to engineer
potentially serendipitous (or at least pseudo-serendipitous)
encounters.
%% Another
%% earlier related example of this sort of system is {\sf Max}, created
%% by Figueiredo and Campos \citeyear{Campos2002}.  The user emailed {\sf
%%   Max} with an existing list of interests and {\sf Max} would return a
%% web page that might also be of interest.  Other systems with similar
%% support for serendipitous discovery involve searching for analogies
%% \cite{Donoghue2002,Donoghue2012}) as well as content \cite{Iaquinta2008}.

In recent joint work \cite{colton-assessingprogress}, we presented a
diagrammatic formalism for evaluating progress in computational
creativity.  It is useful to ask what serendipity would add to this
formalism, and how the result compares with other attempts to
formalise serendipity, notably Figueiredo and Campos's
\citeyear{Figueiredo2001} `Serendipity Equations'.  In this work,
Figueiredo and Campos describe serendipitous ``moves'' from one
problem to another, which transform a problem that cannot be solved
into one that can.  In our diagrammatic formalism, we spoke about
progress with \emph{systems} rather than with \emph{problems}.  It
would be a useful generalisation of the formalism -- and not just a
simple relabelling -- for it to be able to tackle problems as well.
However, it is important to notice that progress with problems does not always mean transforming a
problem that cannot be solved into one that can.  Progress may also
apply to growth in the ability to \emph{posit} problems.  In keeping
track of progress, it would be useful for system designers to record
(or get their systems to record) what problem a given system solves,
and the degree to which the computer was responsible for coming up
with this problem.

As \cite[p. 69]{pease2013discussion} remark, anomaly detection and
outlier analysis are part of the standard machine learning toolkit --
but recognising \emph{new} patterns and defining \emph{new} problems
is more ambitious.  Complex analogies between evolving problems and
solutions form one of the key strategies for teams of human designers
\cite{Analogical-problem-evolution-DCC}.  Kazjon Grace
\citeyear{kaz-thesis} presents a computational model of the creation
of new concepts and interpretations, but this work did include the
ability to create new higher order relationships necessary for complex
analogies.  New patterns and higher-order analogies were considered in
Hofstadter and Mitchell's {\sf Copycat} and the subsequent {\sf
  Metacat}, but these systems operated in a simple and fairly abstract
``microdomain''
\cite{hofstadter1994copycat,DBLP:journals/jetai/Marshall06}.  More
recent work in this tradition is surveyed in
\cite{eric-nichols-thesis}.

The relationship between serendipity and novel problems receives
considerable attention in the current work, since we want to
increasingly turn over responsibility for creating and maintaining a
prepared mind to the machine.
