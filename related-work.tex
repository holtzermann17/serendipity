\subsection{Related work} \label{sec:related}

\textbf{[Since we are now have the formal definition, let's be sure
    that we're sufficiently thorough in connecting this related work
    back to it -- and pointing out what still needs to be done.]}
\textbf{[AJ I would prefer this section and the Literature review goes before the formal definition. Then the definition section can draw from the related work and lit review (why are related work and lit review separate sections?) and it is a bit clearer where the definition work comes from.]}
Paul Andr{\'e} et al.~\citeyear{andre2009discovery} look at
serendipity from a design point of view.  These authors also propose a
two-part model, in which what we have called \emph{discovery} above
exposes the unexpected, while \emph{invention} is the responsibility
another subsystem that finds applications.  According to Andr\'e et
al., the first phase is the one that has most frequently been
automated, but they suggest that computational systems should be
developed that support both aspects.  Their specific suggestions focus
on representational features: \emph{domain expertise} and a
\emph{common language model}.

Although tremendously useful when they are available, these features
are not always enough to account for serendipitious events.  Using the
terminology we introduced above, these features seem to exemplify
aspects of the \emph{prepared mind}.  However, as we mentioned above,
the \emph{bridge} is a distinct process that mental preparation can
support, but not always fully determine.  For example, participants in
a Writers Workshop may a possess a very limited understanding of each
other's aims or of the work they are critiquing, and may as a
consequence talk past one another to a greater or lesser degree --
while nevertheless finding the overall process of participating in the
workshop itself illuminating and rewarding (often precisely because
such misunderstandings elucidate poor communication choices!).
Various social strategies, ranging from Writers Workshops to open
source software, pair programming, and design charettes
\cite[p. 11]{gabriel2002writer} have been developed to exploit similar
emergent effects to develop new insights, and to develop \emph{new}
shared language.  In \cite{poetry-workshop}, we investigate the
feasibility of using designs of this sort in multi-agent systems that
learn by sharing and discussing partial understandings.  This earlier
paper remains broadly indicative, however, and the ideas it describes
can see considerable benefit from the more formal thinking we develop
in the current work.

The issue of designing for serendipity has also been taken up recently
by Deborah Maxwell et al.~\citeyear{maxwell2012designing}, in their
description of a prototype of the {\sf SerenA} system.  This system is
designed to support serendipitous discovery for its (human) users
\cite{forth2013serena}.  The authors rely on a process-based model of
serendipity \cite{Makri2012,Makri2012a} that is derived from user
studies, including interviews with 28 researchers, looking for
instances of serendipity from both their personal and professional
lives.  This material was coded along three dimensions:
\emph{unexpectedness}, \emph{insightfulness}, and \emph{value}.  This
research aims to support the process of forming bridging connections
from unexpected encounter to a previously unanticipated but valuable
outcome.  They particularly focus on the acts of \emph{reflection}
that foment both the creation of a bridge and estimates of the
potential value of the result.

Although this touches on all of the features of our model, {\sf
  SerenA} nevertheless matches the description offered by Andr{\'e} et
al.~\citeyear{andre2009discovery} of discovery-focused systems: the
user is the primary agent with a prepared mind.  Accordingly it is the
user that undergoes an ``aha'' moment and takes the creative steps to
realise the result; the computer is mainly used to facilitate this.
The primary computational method is to search outside of the normal
search parameters in order to engineer potentially serendipitous (or
at least pseudo-serendipitous) encounters.  Another earlier related
example of this sort of system is {\sf Max}, created by Figueiredo and
Campos \citeyear{Campos2002}.  The user emailed {\sf Max} with a list
of interests and {\sf Max} would find a webpage that may be of
interest to the user.  Similar systems with support for serendipitous
discovery involve searching for analogies
\cite{Donoghue2002,Donoghue2012}) and content \cite{Iaquinta2008}.

In earlier joint work \cite{colton-assessingprogress}, we presented a
diagrammatic formalism for evaluating progress in computational
creativity.  It is useful to ask what serendipity would add to this
formalism, and how the result compares with other attempts to
formalise serendipity, notably Figueiredo and Campos's
\citeyear{Figueiredo2001} `Serendipity Equations'.  In this work,
Figueiredo and Campos describe serendipitous ``moves'' from one
problem to another, which transform a problem that cannot be solved
into one that can.  In our diagrammatic formalism, we spoke about
progress with \emph{systems} rather than with \emph{problems}.  It
would be a useful generalisation of the formalism -- and not just a
simple relabelling -- for it to be able to tackle problems as well.
However, progress with problems does not always mean transforming a
problem that cannot be solved into one that can.  Progress may also
apply to growth in the ability to \emph{posit} problems.  In keeping
track of progress, it would be useful for system designers to record
(or get their systems to record) what problem a given system solves,
and the degree to which the computer was responsible for coming up
with this problem.  The relationship between serendipity and novel
problems receives considerable attention here, since we want to
increasingly turn over responsibility for creating and maintaining a
prepared mind to the machine.
