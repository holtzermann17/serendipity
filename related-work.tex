\subsection{Related work} \label{sec:related}

Paul Andr{\'e} et al.~\citeyear{andre2009discovery} look at serendipity
from a design point of view.  They propose a two-part model, in which
what we might call chance+curiosity exposes the unexpected, and
sagacity+value is determined by another subsystem.  This corresponds
to Bergson's distinction between \emph{discovery} and \emph{invention}
(see Section \ref{sec:overview-serendipity}).  One survey related to
the first phase is \cite{foster2003serendipity}.  According to Andr\'e
et al., the first phase is the one that has most frequently been
automated, but they suggest that computational systems should be
developed that support both aspects.  Their specific suggestions focus
on representational features: \emph{domain expertise} and a
\emph{common language model}.  We've advocated for a more
experimentally-based approach that does not directly rely on shared
understandings.  For example, participants in a Writers Workshop in
poetry may not ``understand'' one another but can still find the
experience of participating in the workshop rewarding.

The issue of designing for serendipity has also been taken up by
Deborah Maxwell et al.~\cite{maxwell2012designing}, in their
description of a prototype of the {\sf SerenA} system.  This system is
designed to support serendipitous discovery for its \emph{users}
\cite{forth2013serena}.  The authors rely on a process-based model of
serendipity \cite{Makri2012,Makri2012a} that is derived from user
studies, including interviews with 28 researchers, looking for
instances of serendipity from both their personal and professional
lives.  This material was coded along three dimensions:
\emph{unexpectedness}, \emph{insightfulness}, and \emph{value}.  This
work aims to support the process of forming bridging connections that
eventuate in an unanticipated valuable outcome.  They particularly
focus on the acts of \emph{reflection} that foment both the bridge and
estimates of the potential value of the result.  Both pattern-building
activities and the practice of fomenting thought by structured
encounters in Writers Workshops can be understood to contribute to the
theory and practise of reflection\footnote{As with creativity and
  serendipity, in order to carry out concrete evaluations of automated
  reflection we may well ask ``what, exactly, are we looking for as
  evidence of reflection?'' \cite{rodgers2002defining}.  A detailed
  answer derived from the classic work of John Dewey
  \citeyear{dewey1997we} is explored in \cite{rodgers2002defining}.}

{\sf SerenA} is a system like the ones described by Andr{\'e} et
al.~\cite{andre2009discovery}, in which the user is expected to have
the ``aha'' moment, and take the creative steps.  The computer is
mainly used to facilitate this; and as indicated above this is usually
done by searching outside of the normal search parameters to engineer
potentially serendipitous (or at least pseudo-serendipitous)
encounters.  Another earlier example of this sort of system is {\sf
  Max}, created by Figueiredo and Campos \citeyear{Campos2002}.  The user
emailed {\sf Max} with a list of interests and {\sf Max} would find a
webpage that may be of interest to the user.  Other search-related
examples support searching for analogies (\cite{Donoghue2002} and
\cite{Donoghue2012}) and content \cite{Iaquinta2008}.

In earlier joint work \cite{colton-assessingprogress}, we presented a
diagrammatic formalism for evaluating progress in computational
creativity.  It is useful to ask what serendipity would add to this
formalism, and how the result compares with other attempts to
formalise serendipity, notably Figueiredo and Campos's
\citeyear{Figueiredo2001} `Serendipity Equations'.
%
In \cite{stakeholder-groups-bookchapter}, we advanced several
hypotheses related to the development of the computational creativity
field.  Again, we should ask here how serendipity contributes.  We
discuss these points in the following section.
