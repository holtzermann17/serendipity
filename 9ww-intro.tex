\subsection{Thought experiment:  Serendipity by design} \label{sec:ww}

To further evaluate our computational framework in usage, in this
section we develop a thought experiment in system design, based on a
novel computational scenario where there is high potential for
serendipity.  As discussed above, sociological factors can influence
serendipitous discoveries on a social scale.  In our two case studies,
user input played a significant role.  The exploitation of social
creativity and feedback can create scenarios where serendipity could
occur within a computer system as well.

In \cite{corneli2015computational}, we described the preliminary designs for
multi-agent systems that learn by sharing work in progress and
discussing partial understandings.  
%
Following \citeA{gabriel2002writer}
% we described a template for a pattern
% language for interactions in a computational poetry workshop, closely
we define a \emph{Workshop} to be an activity for two or more agents
consisting of the following steps:
%itemize?
{\tt presentation}, {\tt listening}, {\tt feedback}, {\tt questions},
and {\tt reflections}.  In general, the first and most important
feature of {\tt feedback} is for the listener to say what they heard;
in other words, what they find in the presented work.  In some
settings this is augmented with {\tt suggestions}.  After any {\tt
  questions} from the author, the commentators may make {\tt replies}
to offer clarification.

The key steps map quite conveniently into the schematic description of serendipity that we introduced in Section \ref{sec:our-model}:


