\begin{center}
\begingroup
\tikzset{
block/.style = {draw, fill=white, rectangle, minimum height=3em, minimum width=3em},
tmp/.style  = {coordinate}, 
sum/.style= {draw, fill=white, circle, node distance=1cm},
input/.style = {coordinate},
output/.style= {coordinate},
pinstyle/.style = {pin edge={to-,thin,black}}
}

\begin{tikzpicture}[auto, node distance=2cm,>=latex']
    \node [sum] (A-sum1) {};
    \node [input, name=pinput, above left=.7cm and .7cm of A-sum1] (A-pinput) {};
    \node [input, name=tinput, left=2.2cm of A-sum1] (A-tinput) {};
    \node [input, name=minput, below left of=A-sum1] (A-minput) {};
    \node [input, name=minput, right of=A-sum1] (A-moutput) {};

    \draw [->] (A-pinput) -- node{{\footnotesize problem}} (A-sum1);
    \draw [->] (A-tinput) -- node{\vphantom{{\footnotesize g}}{\footnotesize context~}} (A-sum1);
%    \draw [->] (A-sum1) -- node{\vphantom{{\footnotesize g}}$T^{\star}$}  (A-moutput);

    \node [sum, right=2.3cm of A-sum1] (B-sum1) {};
    \node [input, name=pinput, above left=.7cm and .7cm of B-sum1] (B-pinput) {};
    \node [input, name=tinput, left of=B-sum1] (B-tinput) {};
    \node [input, name=minput, below left of=B-sum1] (B-minput) {};
    \node [sum, right=2cm of B-sum1] (B-sum2) {};

    \node [input, name=minput, right of=B-sum2] (B-moutput) {};
    \draw [->] (B-pinput) -- node{{\footnotesize rationale}} (B-sum1);

    \draw [->] (A-sum1) -- node{\vphantom{{\footnotesize g}}{\footnotesize solution}} (B-sum1);

    \draw [->] (B-sum1) -- node{\vphantom{{\footnotesize g}}{\footnotesize pattern}} (B-sum2);
    \draw [->] (B-sum2) -- node[text width=1.5cm,execute at begin node=\setlength{\baselineskip}{.9ex}]{\footnotesize \emph{resolution\\~of forces}} (B-moutput);
\end{tikzpicture}

\endgroup
\end{center}
