
\section{Discussion} \label{sec:discussion}

In the preceding section, we applied our model to evaluate the serendipity of an evolutionary music improvisation system, a hypothetical class of next-generation recommender systems, and a system for assembling flowcharts.  The model has helped to highlight directions for development that would increase the potential for serendipity in existing systems, either incrementally or more transformatively.  Our model suggested case studies of systems that can observe events that would otherwise not be observed, take an interest in them, and transform the observations into artefacts with lasting value.  We will now discuss implications from our findings for future research, and outline potential next steps.

%\subsection{Challenges for future research} \label{sec:recommendations}

Viewing the concepts in Section \ref{sec:by-example} through the
practice scenarios we have discussed, we can describe the following
challenges for research in computational serendipity.

\begin{itemize}
\item \textbf{Autonomy}: Our case studies in Section \ref{sec:priorart} highlight the potential value of increased autonomy on the system side.
%% The thought experiment in Section
%% \ref{sec:ww} develops a design illustrating the relationship between
%% creativity at the level of artefacts (e.g.~new poems) and
%% creativity at the level of \emph{problem specification} (learning
%% new poetic concepts).
The search for connections that make raw data into ``strategic data''
is an appropriate theme for research in computational intelligence and
machine learning to grapple with.  In the standard cybernetic model,
we control computers, and we also control the computer's operating
context.  There is little room for serendipity if there is nothing
outside of our direct control.  In contrast with the mainstream model,
von Foerster \citeyear[p. 286]{von2003cybernetics} advocated a
\emph{second-order cybernetics} in which ``the observer who enters the
system shall be allowed to stipulate his own purpose.''  \emph{A
  primary challenge to the serendipitous operation of computers is
  developing computational agents that specify their own problems.}
\end{itemize}

\begin{itemize}
\item \textbf{Learning}: The Writers Workshop described in Section
  \ref{sec:ww} is one possible design for a system
  that can \emph{learn from experience}.  The Workshop model
  ``personifies'' the wider world in the form of one or several
  critics.  It is also possible for a lone creative agent to
  take its own critical approach in relationship to the world at
  large, using an experimental approach to generate feedback, and then
  looking for models to fit this feedback.   We are led to consider 
  computational agents that operate in our world rather
  than a circumscribed microdomain, and that are curious about this
  world.  \emph{A second challenge is for computational agents to
    learn more and more about the world we live in.}
\end{itemize}

\begin{itemize}
\item \textbf{Sociality}: We may be aided in our pursuit of the
  ``smart mind'' required for serendipity by recalling Turing's
  proposal that computers should ``be able to converse with each other
  to sharpen their wits'' \cite{turing-intelligent}.  Other fields,
  including computer Chess, Go, and argumentation have achieved this,
  and to good effect.  Turing recognised that computers would have to
  be coached in the direction of social learning, but that once they
  attain that standard they will learn much more quickly.  Deleuze
  \citeyear[p. 26]{deleuze1994difference} wrote: ``We learn nothing
  from those who say: `Do as I do'. Our only teachers are those who
  tell us to `do with me'[.]''  \emph{A third challenge is for
    computational agents to interact in a recognisably social way with
    us and with each other, resulting in emergent effects.}
\end{itemize}

\begin{itemize}
\item \textbf{Embedded evaluation}:
  \citeA{stakeholder-groups-bookchapter} outlined a general programme
  for computational creativity, and examined perceptions of creativity
  in computational systems found among members of the general public,
  Computational Creativity researchers, and existing creative
  communities.  We should now add a fourth important ``stakeholder''
  group in computational creativity research: computer systems
  themselves.  Creativity may look very different to this fourth
  stakeholder group than it looks to us.  It is our responsibility as
  system designers to teach our systems how to make
  evaluations in way that is both reasonable and ethical.  This is
  exemplified by the preference for a ``non-zero sum'' criterion for
  value suggested in our discussion of the dimensions of serendipity
  in Section \ref{sec:by-example}.  \emph{A fourth challenge is for
    computational agents to evaluate their own creative process and
    products.}
\end{itemize}


%% A survey of word occurrences from a recent special issue of
%% \emph{Cognitive Computation} on ``Computational Creativity, Intelligence and Autonomy'' \cite{bishop-erden-special-issue} shows that related themes are broadly
%% active in the research community.  Here
%% \emph{italics} indicates that the word stem accounted for 0.1\% of the
%% article or more; added \textbf{\emph{bold}} indicates that it
%% accounted for 1\% or more.\footnote{Articles were converted to text
%%   via {\tt pdftotext -layout}, individual counts found via {\tt tr
%%     \textquotesingle~\textquotesingle~\textquotesingle\textbackslash
%%     n\textquotesingle~< file.txt | grep -c "stem*"}, and total word counts
%%   via {\tt wc -w}.  The corresponding counts for the \emph{current}
%%   paper are 12, \emph{25}, \emph{16}, \emph{44} and 12.7K.}

%% \medskip

%% {\centering \setlength{\tabcolsep}{3pt} \footnotesize
\begin{tabular}{ccccccccccccccc}
paper \#
&1
&2
&3
&4
&5
&6
&7
&8
&9
&10
&11
&12
&13
&14
\\
\cline{2-15}
"autonom.*"
&0
&\textbf{\emph{32}}
&\emph{12}
&\emph{41}
&0
&1
&\emph{31}
&2
&1
&\emph{92}
&11
&2
&5
&\textbf{\emph{22}}
\\
"learn.*"
&6
&2
&2
&\emph{14}
&\emph{9}
&\textbf{\emph{118}}
&\emph{14}
&\emph{18}
&\emph{44}
&\emph{12}
&11
&\emph{42}
&\emph{44}
&2
\\
"social.*"
&0
&0
&\emph{23}
&\emph{25}
&0
&1
&2
&\emph{10}
&\emph{19}
&\emph{19}
&8
&\emph{21}
&13
&2
\\
"evaluat.*"
&0
&1
&\emph{11}
&\emph{20}
&0
&1
&3
&6
&4
&9
&8
&2
&\textbf{\emph{304}}
&0
\\
\cline{2-15}
total(K)
& 8.3  % &8337 (/ 6 8337.0)
& 2.2  % &2221 (/ 32 2221.0)  0.0135074290859973
& 7.5  % &7507  (/ 12 7507.0) 0.0015985080591447982 (/ 23 7507.0) 0.0026641800985746636 (/ 11 7507.0)0.001465299054216065
& 7.4  % &7453 (/ 41 7453.0) 0.004964443848114853 (/ 14 7453.0) 0.0009392191064001073 (/ 16 7453.0) 0.002146786528914531 (/ 19 7453.0) 0.0025493090030860054
& 8.6  % &8675 (/ 9 8675.0)
& 5.8  % & 5816 (/ 89 5816.0) 0.015302613480055021
&10.3 % &10341 (/ 30 10341.0) 0.002901073397156948
& 9.6  % &9632  (/ 18 9632.0) 0.0018687707641196014  (/ 10 9632.0)0.0010382059800664453
&10.8 % &10851 (/ 36 10851.0) 0.0033176665745092617
&11.6 % &11693 (/ 92 11693.0)0.007867955186863935 (/ 12 11693.0)
&14.4 % &14407 (/ 11 14407.0) 0.0007635177344346498
&10.8 % &10840 (/ 31 10840.0) 0.0028597785977859777
&25.3 % &25326 (/ 13  25326.0)  (/ 44  25326.0) 0.0008291873963515755 (/ 304 25326.0) 0.011174287293690278
& 1.6  % &1673 (/ 21 1673.0) 0.012552301255230125
\\
\end{tabular}
}


%% \bigskip

%% Paper 4, Rob Saunders's \citeyear{saunders2012towards} ``Towards
%% Autonomous Creative Systems: A Computational Approach'' was the only
%% contributed paper to emphasise all four of our themes according to the
%% metric above.  Saunders asks: ``What would it mean to produce an
%% autonomous creative system? How might we approach this task? And, how
%% would we know if we had succeeded?''  He argues for an approach ``that
%% models personal motivations, social interactions and the evolution of
%% domains.''  Paper 10, d'Inverno and Luck's \citeyear{d2012creativity}
%% ``Creativity Through Autonomy and Interaction'', also contains a
%% theoretical engagement with these themes, and presents a formalism for
%% multi-agent systems that could usefully be adapted to model
%% serendipitous encounters.  Both papers are particularly concerned with
%% \emph{motivation}, a topic that relates to both the prepared mind and
%% the theme of embedded evaluation.

%% We believe that our clarifications to the multifaceted concept of
%% serendipity will help encourage future computer-aided (and
%% computer-driven) investigations of the above themes and their
%% interrelationships.  Our extension of SPECS to cover serendipity will
%% be useful for evaluating progress.  We discuss some of our related
%% research plans below.

%\subsection{Future Work} \label{sec:futurework} \label{sec:hatching}

In looking for ways to manage and encourage serendipity, we are drawn
to the approach taken by the \emph{design pattern} community
\cite{alexander1999origins}.
%% The essential features of this approach
%% are described below, but we point out straight away that we propose to
%% use design patterns in rather nonstandard fashion.  These adaptations
%% to the typical design pattern methodology are proposed to parallel the
%% four themes outlined above.
%% \begin{itemize}
%% \item[(1)] We want to encode our design patterns directly in runnable
%%   programs, not just give them to programmers as heuristic guidance.
%% \item[(2)] We want the (automated) programmer to generate new design
%%   patterns, not just apply or adapt old ones.
%% \item[(3)] We want our design patterns themselves, working in
%%   combination, to contribute to the discovery of new emergent problems
%%   and patterns, not just capture the solutions to existing known
%%   problems.
%% \item[(4)] We want our design patterns to play an overt role in the
%%   dynamical systems they describe.
%% \end{itemize}
%%
\citeA{meszaros1998pattern} describe the typical scenario for authors of design
patterns: ``You are an experienced practitioner in your
field. You have noticed that you keep using a certain solution to a
commonly occurring problem. You would like to share your experience
with others.''  There are many ways to describe a solution.
Meszaros and Doble remark, ``What sets patterns apart is their
ability to explain the rationale for using the solution (the `why') in
addition to describing the solution (the `how').''  Regarding the
criteria that pattern writers seek to address: ``The most appropriate
solution to a problem in a context is the one that best resolves the
highest priority forces as determined by the particular context.'' 
%
%% Their article describes a number of criteria relevant to writing
%% good design patterns, e.g. \emph{Clear target audience},
%% \emph{Visible forces}, and \emph{Relationship to other patterns}.
%
A good design pattern \emph{describes} the resolution of forces in the
target domain; in the setting we're interested in, creating a new
design pattern also \emph{effects} a resolution of forces directly.
The use case of design pattern development maps into our diagram of
the basic features of serendipity as follows:

\begin{center}
\begingroup
\tikzset{
block/.style = {draw, fill=white, rectangle, minimum height=3em, minimum width=3em},
tmp/.style  = {coordinate}, 
sum/.style= {draw, fill=white, circle, node distance=1cm},
input/.style = {coordinate},
output/.style= {coordinate},
pinstyle/.style = {pin edge={to-,thin,black}}
}

\begin{tikzpicture}[auto, node distance=2cm,>=latex']
    \node [sum] (sum1) {};
    \node [input, name=pinput, above left=.7cm and .7cm of sum1] (pinput) {};
    \node [input, name=tinput, left=2cm of sum1] (tinput) {};
    \node [input, name=minput, below left of=sum1] (minput) {};
    \node [input, name=minput, right of=sum1] (moutput) {};
    \draw [->] (tinput) -- node{\vphantom{{\tiny g}}{\tiny context}} (sum1);
    \draw [->] (pinput) -- node{{\tiny problem}} (sum1);
    \draw [->] (sum1) -- node{\vphantom{{\tiny g}}{\tiny solution}}  (moutput);
\end{tikzpicture}
\hspace{1cm}
\begin{tikzpicture}[auto, node distance=2cm,>=latex']
    \node [sum] (sum1) {};
    \node [input, name=pinput, above left=.7cm and .7cm of sum1] (pinput) {};
    \node [input, name=tinput, left of=sum1] (tinput) {};
    \node [input, name=minput, below left of=sum1] (minput) {};
    \node [sum, right=1.5cm of sum1] (sum2) {};
    \node [input, name=minput, right of=sum2] (moutput) {};
    \draw [->] (tinput) -- node{\vphantom{{\tiny g}}{\tiny solution}} (sum1);
    \draw [->] (pinput) -- node{{\tiny rationale}} (sum1);
    \draw [->] (sum1) -- node{\vphantom{{\tiny g}}{\tiny pattern}} (sum2);
    \draw [->] (sum2) -- node[text width=1.5cm,execute at begin node=\setlength{\baselineskip}{.3ex}]{\tiny \emph{resolution\\~of forces}}  (moutput);
\end{tikzpicture}
\endgroup
\end{center}



%%% Here we do not mean to suggest that every instance of ``a solution to a
%% problem in a context'' is due to serendipity at work -- on the
%% contrary, that is just the discovery step.  Inventing a viable design pattern
%% only happens when the solution is found to be explicable and useful.

To van Andel's assertion that ``The very moment I can plan or
programme `serendipity' it cannot be called serendipity anymore,'' we
would reply that we can certainly describe patterns (and programs)
with built-in indeterminacy.  Figure \ref{fig:va-pattern-figure}
presents an example, showing how one of van Andel's patterns of
serendipity can be rewritten as a design pattern using the template
suggested by our model.  In future work, we would aim to build a more
complete pattern language along similar lines, and show how 
this language can be used to transform raw data into ``strategic data.''
%
The example pattern describes a scenario that is quite close to Pease et al.'s \citeyear{pease2013discussion} description of an online
system that gathers new modules over time, and for which,
periodically, new combinations of modules can yield new and
interesting results.
%
Developing experiments along these lines may help prepare the
groundwork for the more involved development projects discussed in the
current paper.
%
Patterns of serendipity, like the one in Figure \ref{fig:va-pattern-figure},
offer useful heuristic guidelines for human programmers and convey a sense of our long-term
plans for serendipitous computing systems.

\begin{figure}[!ht]
\begin{mdframed}
\paragraph{\textbf{Successful error}}~
\vskip -1\baselineskip
\begin{flushright}\emph{Van Andel's example} -- Post-it\texttrademark\ Notes
\end{flushright}
\vspace{-.15cm}
\begin{description}[itemsep=2pt]
\item[{\tt context}] -- You run a creative organisation with several different divisions and many contributors with different expertise.  
\item[{\tt problem}] -- One of the members of your organisation
  discovers something with interesting properties, but no one
  knows how to turn it into a product with industrial or commercial application.
\item[{\tt solution}] -- You create a space for sharing and discussing
  interesting ideas on an ongoing basis (perhaps a Writers Workshop).
\item[{\tt rationale}] -- You suspect it's possible that one of the
  other members of the firm will come up with an idea about an
  application; you know that if a potential application is found, it
  may not be directly marketable, but at least there will be a
  prototype that can be concretely discussed.
\item[{\tt resolution}] -- The \emph{Successful error} pattern
  rewritten using this template is an example of a similar
  prototype, showing that serendipity can be talked about in
  terms of design patterns.
\end{description}
\end{mdframed}
\caption{Our design pattern template applied to van Andel's \emph{Successful error} pattern\label{fig:va-pattern-figure}}
\end{figure}





% Is ``having a stretch goal'' an example of a serendipity pattern?  I think so!


\subsection{Challenges for future research} \label{sec:recommendations}

Viewing the concepts in Section \ref{sec:by-example} through the
practice scenarios we have discussed, we can describe the following
challenges for research in computational serendipity.

\begin{itemize}
\item \textbf{Autonomy}: Our case studies in Section
  \ref{sec:computational-serendipity} highlight the potential value of
  increased autonomy on the system side.
The search for connections that make raw data into ``strategic data''
is an appropriate theme for research in computational intelligence and
machine learning to grapple with.  In the standard cybernetic model,
we control computers, and we also control the computer's operating
context.  There is little room for serendipity if there is nothing
outside of our direct control.  In contrast to this mainstream model,
von Foerster \citeyear[p. 286]{von2003cybernetics} advocated a
second-order cybernetics in which ``the observer who enters the system
shall be allowed to stipulate his own purpose.''  \emph{A
  primary challenge for the serendipitous operation of computers is
  developing computational agents that specify their own problems.}
\end{itemize}

\begin{itemize}
\item \textbf{Learning}: Each of the case studies considered in
  Section \ref{sec:computational-serendipity} describes a system that
  is able, in one way or another, to learn from experience.  As we
  considered ways to enhance measures of serendipity in these
  examples, we were led to consider computational agents that
  participate meaningfully in ``our world'' rather than in a
  circumscribed microdomain.  \emph{A second challenge is for
    computational agents to learn more and more about the world we
    live in.}
\end{itemize}

\begin{itemize}
\item \textbf{Sociality}: We may be aided in our pursuit of the
  ``smart mind'' required for serendipity by recalling Turing's
  proposal that computers should ``be able to converse with each other
  to sharpen their wits'' \cite{turing-intelligent}.  Turing
  recognised that computers would have to be coached in the direction
  of social learning, but that once they attain that standard they
  will learn much more quickly.  The four supportive factors for
  serendipity described in this paper resemble nothing more than our
  experience of social reality.  \emph{A third challenge is for
    computational agents to interact in a recognisably social way with
    us and with each other, resulting in emergent effects.}
\end{itemize}

\begin{itemize}
\item \textbf{Embedded evaluation}:
  \citeA{stakeholder-groups-bookchapter} outline a general programme
  for computational creativity, and examined perceptions of creativity
  in computational systems found among members of the general public,
  computational creativity researchers, and existing creative
  communities.  We should now add a fourth important ``stakeholder''
  group in computational creativity research: computer systems
  themselves.  System designers need to teach their systems how to
  make evaluations in way that is both reasonable and ethical.  This
  condition is exemplified by the preference for a ``non-zero sum''
  criterion for value introduced in Section \ref{sec:by-example}.
  \emph{A fourth challenge is for computational agents to evaluate
    their own creative process and products.}
\end{itemize}


\subsection{Future Work} \label{sec:futurework} \label{sec:hatching}

In looking for ways to manage and encourage serendipity, we are drawn
to the approach taken by the \emph{design pattern} community
\cite{alexander1999origins}. 
\citeA{meszaros1998pattern} describe the typical scenario for authors of design patterns:
\begin{quote}
\noindent ``You are an experienced practitioner in your field.  You
have noticed that you keep using a certain solution to a commonly
occurring problem.  You would like to share your experience with
others.''
\end{quote}
There are many ways to describe a solution.
Meszaros and Doble remark,
\begin{quote}
\noindent ``What sets patterns apart is their ability to explain the
rationale for using the solution (the `why') in addition to describing
the solution (the `how').''
\end{quote}
Regarding the criteria that pattern writers seek to address: 
\begin{quote}
\noindent ``The most appropriate solution to a problem in a context is
the one that best resolves the highest priority forces as determined
by the particular context.''
\end{quote}

%
%% Their article describes a number of criteria relevant to writing
%% good design patterns, e.g. \emph{Clear target audience},
%% \emph{Visible forces}, and \emph{Relationship to other patterns}.
%
Applying the solution achieves this resolution of forces, and a design
pattern shows how this works.  The design pattern itself achieves
something further: it encapsulates knowledge in a brief, shareable
form, often with meaningful relationships to other patterns.  Tracing
the steps involved, we see that the creation of a new design pattern
is always somewhat serendipitous (Figure \ref{fig:pattern-schematic};
compare Figure \ref{fig:1b}).

\begin{figure}
\begin{center}
\begingroup
\tikzset{
block/.style = {draw, fill=white, rectangle, minimum height=3em, minimum width=3em},
tmp/.style  = {coordinate}, 
sum/.style= {draw, fill=white, circle, node distance=1cm},
input/.style = {coordinate},
output/.style= {coordinate},
pinstyle/.style = {pin edge={to-,thin,black}}
}

\begin{tikzpicture}[auto, node distance=2cm,>=latex']
    \node [sum] (sum1) {};
    \node [input, name=pinput, above left=.7cm and .7cm of sum1] (pinput) {};
    \node [input, name=tinput, left=2cm of sum1] (tinput) {};
    \node [input, name=minput, below left of=sum1] (minput) {};
    \node [input, name=minput, right of=sum1] (moutput) {};
    \draw [->] (tinput) -- node{\vphantom{{\tiny g}}{\tiny context}} (sum1);
    \draw [->] (pinput) -- node{{\tiny problem}} (sum1);
    \draw [->] (sum1) -- node{\vphantom{{\tiny g}}{\tiny solution}}  (moutput);
\end{tikzpicture}
\hspace{1cm}
\begin{tikzpicture}[auto, node distance=2cm,>=latex']
    \node [sum] (sum1) {};
    \node [input, name=pinput, above left=.7cm and .7cm of sum1] (pinput) {};
    \node [input, name=tinput, left of=sum1] (tinput) {};
    \node [input, name=minput, below left of=sum1] (minput) {};
    \node [sum, right=1.5cm of sum1] (sum2) {};
    \node [input, name=minput, right of=sum2] (moutput) {};
    \draw [->] (tinput) -- node{\vphantom{{\tiny g}}{\tiny solution}} (sum1);
    \draw [->] (pinput) -- node{{\tiny rationale}} (sum1);
    \draw [->] (sum1) -- node{\vphantom{{\tiny g}}{\tiny pattern}} (sum2);
    \draw [->] (sum2) -- node[text width=1.5cm,execute at begin node=\setlength{\baselineskip}{.3ex}]{\tiny \emph{resolution\\~of forces}}  (moutput);
\end{tikzpicture}
\endgroup
\end{center}

\caption{The components of design patterns mapped to our process schematic\label{fig:pattern-schematic}}
\end{figure}

\begin{figure}[!h]
\setlist[description]{font=\normalfont\itshape}
\begin{mdframed}
\vspace{2mm}
\textbf{\emph{Successful error}}~
\begin{description}[leftmargin=0\parindent,labelindent=0em,itemsep=2pt]
\item[{Context.}] You run an organisation with {\sl different
  divisions} and contributors with {\sl varied expertise}.  One of
  them discovers something interesting, but no one knows how to {\sl
    turn the discovery into a product}.
\item[{Problem.}]  How can you get value from the discovery?
\item[{Solution.}] Allow people to work on pet projects, and encourage
  interaction between people in different divisions.
\item[{Rationale.}] Prototypes can be discussed, even if they are not
  directly marketable.  Following the interests of contributors
  preserves their autonomy; contact with different points of view
  brings additional knowledge to bear.
\item[{Resolution.}] 
Open discussion can
  expose flaws at any stage, and help guide work in the direction of
  real innovation.  Participants in these conversations
  will learn something, and will help each other maximise the value of
  the discovery.  
\item[{Example.}] Low-tack restickable glue, discovered by a 3M
  engineer in 1968, ultimately proved useful for making
  Post-it\texttrademark\ Notes, which were launched in 1980 after
  several rounds of in-house prototyping.
\end{description}
\vspace{-1mm}
\end{mdframed}
\caption{Standard design pattern template applied to van Andel's \em{Successful error}\label{fig:va-pattern-figure}}
\end{figure}



 To van Andel's assertion that ``The very moment I can
plan or programme `serendipity' it cannot be called serendipity
anymore,'' we reply that we can certainly describe patterns -- and
programs -- with built-in indeterminacy.  We can foster circumstances
that may make an unexpected happy outcome more likely.  Figure
\ref{fig:va-pattern-figure} illustrates this with one van Andel's
patterns of serendipity, rewritten using the standard design pattern
template.  In future work, we intend to build a more complete
serendipity pattern language -- and put it to use within autonomous
programming systems.


% Is ``having a stretch goal'' an example of a serendipity pattern?  I think so!
