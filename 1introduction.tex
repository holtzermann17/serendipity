\section{Introduction}

Serendipity is centred on reevaluation.  For example, a
non-sticky ``superglue'' that no one was quite sure how to use turned
out to be just the right ingredient for 3M's
Post-it\texttrademark\ notes.
%
Serendipity is related, firstly, to deviations from expected or
familiar patterns, and secondly, to new insight.
%
When we consider the practical uses for weak glue, the possibility
that a life-saving antibiotic might be found growing on contaminated
petri dishes, and or the idea that burdock burrs could be anything but
annoying, we encounter radical changes in the evaluation of what's
interesting.  In the \emph{d\'enouement}, what was initially
unexpected is found to be both explicable and useful.  Importantly,
serendipity is not the same as luck.  It involves making sense of
the unexpected.

Serendipity is increasingly seen as relevant in the arts
\cite{mckay-serendipity}, in the tech industry \cite{rao2015breaking}, and elsewhere.  Serendipity in the workplace may be encouraged with methods drawn from architecture and data science \cite{kakko2009homo,engineering-serendipity}.
An interdisciplinary perspective on the phenomenon of serendipity promises further illumination.
This paper reassesses and updates \citeA{pease2013discussion}, developing a robust computational characterisation of serendipity for computer modelling and system evaluation.

Pek van Andel \citeyear{van1994anatomy} -- echoing Poincar\'e's
\citeyear{poincare1910creation} (negative) reflections on the potential
for a purely computational approach to mathematics -- claimed that:
\begin{quote}
``\emph{Like all intuitive operating, pure serendipity is not amenable
    to generation by a computer.  The very moment I can plan or
    programme `serendipity' it cannot be called serendipity
    anymore}.'' \cite{van1994anatomy}
\end{quote}
We believe that serendipity is not so mystical as such statements
might seem to imply.  In Section \ref{sec:discussion} we will
show that ``patterns of serendipity'' like those collected by van Andel
can be applied in the design of computational systems.

First, in
Section \ref{sec:literature-review}, we survey the broad literature on
serendipity including the etymology of the term itself, and examine prior applications of the concept of serendipity in a computing context.  Then in Section \ref{sec:our-model} we present our own
definition of serendipity, which synthesises the understanding gained from these historical examples, and prepares the way for evaluation of serendipity in computational systems.  Section
\ref{sec:computational-serendipity} examines
three such case studies.  Section
\ref{sec:discussion} offers recommendations for researchers working in the computational modelling of serendipity and related areas such as computational creativity, and describes our own plans for future
work.  Section \ref{sec:conclusion} reviews the contributions of this paper towards computational modelling and evaluation of serendipity.  


