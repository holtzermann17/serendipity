\section{Introduction}

Serendipity is centred on re-evaluation.  For example, a failed
attempt to develop an ultra-strong superglue resulted in a
re-stickable adhesive that no one was quite sure how to use.  After
considerable trial and error, this turned out to be just the right
ingredient for making the now ubiquitous Post-it\texttrademark\ notes.
%
% In this way, serendipity is related to deviations from expected or
% familiar patterns, and to new insight.
%
When we consider the practical uses for weak glue, the possibility
that a life-saving antibiotic might be found growing on contaminated
petri dishes, or the idea that burdock burrs could be anything but
annoying, we encounter radical changes in the evaluation of what's
interesting.  Importantly, serendipity is not the same as luck. 

The notion of serendipity is increasingly seen as relevant in the arts
\cite{mckay-serendipity}, in the tech industry \cite{rao2015breaking},
and elsewhere.  In the workplace, people have sought to encourage
serendipity with methods drawn from architecture, data science, and
cultural engineering
\cite{kakko2009homo,engineering-serendipity,who-moved-cube}.
Continued development of an interdisciplinary perspective on the
phenomenon of serendipity promises further illumination.  This paper
reassesses and updates an earlier discussion of serendipity in a
computational context.  \cite{pease2013discussion}.  Beginning with a
survey of historical definitions, perceptions, and examples of
serendipity in order to surface its several facets and features, we
then develop a process-based model of serendipity that can be applied
to system design and evaluation.

\citeA{van1994anatomy} -- echoing the
(negative) reflections on the potential
for a purely computational approach to mathematics from \citeA{poincare1910creation} -- claimed that:
\begin{quote}
``\emph{Like all intuitive operating, pure serendipity is not amenable
    to generation by a computer.  The very moment I can plan or
    programme `serendipity' it cannot be called serendipity
    anymore}.'' \cite[p.~646]{van1994anatomy}
\end{quote}
We believe that serendipity is not so mystical as such statements
might seem to imply.  In Section \ref{sec:discussion} we will show
that ``patterns of serendipity'' like those collected by van Andel can
be applied in the design of computational systems.  Purposive acts can
have unintended consequences \cite{merton1936unanticipated}.
Similarly, even if we cannot plan or program serendipity, we can
prepare for it.

If serendipity was ruled out as a matter of principle, computing would
be restricted to happy or unhappy \emph{unsurprises} -- preprogrammed,
preunderstood behaviour -- that would be interspersed periodically,
perhaps, with an \emph{unhappy} surprise.  (Perhaps this latter case
would ultimately reduce to ``programmer oversight''.)  Venkatesh Rao
\citeyearpar{rao2015breaking} uses the term \emph{zemblanity} -- after
William Boyd \citeyearpar{boyd2010armadillo}: ``zemblanity, the
opposite of serendipity, the faculty of making unhappy, unlucky and
expected discoveries by design'' -- to describe systems that are
doomed to produce only unhappy unsurprises.  According to Rao, this is
the implied fate of systems that are tied inextricably to a fixed
vision, from which any deviation constitutes a mistake.  This condition
stands at a sharp contrast with the ``second-order cybernetics''
introduced by \citeA{von2003cybernetics}, which envisions systems that
are able to specify their own purpose, and adapt it with respect to a
wider environment.  It also contrasts with Taleb's
\citeyearpar{taleb2012antifragile} notion of ``antifragility'' in
which disturbances within a certain range strengthen the system.
\citeA{minsky1967programming} argues that any sufficiently complex
computational system is bound to make decisions that its creators
could not foresee, and may not fully understand.  Demonstrably
gracefully behaviour in response to unexpected circumstances, and a
preference for ``happy'' as opposed to ``unhappy'' outcomes may be
prerequisites for the development of autonomous systems that are
worthy of our trust.  While not the same as ``Serendipity as a
Service'', such systems should at least be able to recognise
serendipity when it is happening.  Over time a system might also come
to recognise previous missed opportunities and false realisations --
what van Andel \citeyearpar[p.~639]{van1994anatomy} terms \emph{negative
  serendipity} -- and learn from them.

Less controversial than ``programmed serendipity'', but no less worthy
of study, is serendipity that arises in the course of user
interaction.  Indeed, it could be argued that everyday social media
already offers something approaching ``Serendipity as a Service''.
The user logs on hoping, but without any guarantee, that they will
find something interesting, charming, or entertaining, and potentially
relevant to whatever is going on in their life at the moment.
Of course, it should not be assumed that any system that can accommodate
user interaction can generate serendipity; take for example the use of
a calculator, where the potential for serendipity through user
interaction is minimal.  The frameworks introduced in this paper are
broad enough to be used in the design and evaluation of sociotechnical
systems, and we will touch on some examples, however we focus on
modelling serendipity in a computational context.

Section \ref{sec:literature-review} surveys the broad literature on
serendipity including the etymology of the term itself.  Section
\ref{sec:by-example} assembles a novel catalog of historical examples
of serendipity that we will use to scaffold our model.  In Section
\ref{sec:our-model} we present our own definition of serendipity,
which synthesises the understanding gained from these historical
examples, and prepares the way for evaluation of serendipity in
computational systems.  Section \ref{sec:computational-serendipity}
examines three such case studies.  Section \ref{sec:discussion}
examines prior applications of the concept of serendipity in
computing, and offers recommendations for researchers working in the
computational modelling of serendipity and related areas such as
computational creativity.  It also describes our own plans for future
work.  Section \ref{sec:conclusion} reviews the contributions of this
paper towards computational modelling and evaluation of serendipity.


