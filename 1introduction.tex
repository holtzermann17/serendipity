\section{Introduction}

Serendipity is centred on reevaluation.  For example, a
non-sticky ``superglue'' that no one was quite sure how to use turned
out to be just the right ingredient for 3M's
Post-it\texttrademark\ notes.
%
Serendipity is related, firstly, to deviations from expected or
familiar patterns, and secondly, to new insight.
%
When we consider the practical uses for weak glue, the possibility
that a life-saving antibiotic might be found growing on contaminated
petri dishes, or the idea that burdock burrs could be anything but
annoying, we encounter radical changes in the evaluation of what's
interesting.  Importantly, serendipity is not the same as luck.
In the \emph{d\'enouement}, through effort, what was initially unexpected is found to be both explicable and useful. 

Serendipity is increasingly seen as relevant in the arts
\cite{mckay-serendipity}, in the tech industry \cite{rao2015breaking},
and elsewhere. 
Serendipity in the workplace may be encouraged with methods drawn from
architecture, data science, and cultural engineering
\cite{kakko2009homo,engineering-serendipity,who-moved-cube}.  Continued development
of an interdisciplinary perspective on the phenomenon of serendipity
promises further illumination.  This paper reassesses and updates an
earlier discussion of serendipity in a computational context
\cite{pease2013discussion}, and develops robust criteria for computer
modelling and system evaluation.

\citeA{van1994anatomy} -- echoing the
(negative) reflections on the potential
for a purely computational approach to mathematics from \citeA{poincare1910creation} -- claimed that:
\begin{quote}
``\emph{Like all intuitive operating, pure serendipity is not amenable
    to generation by a computer.  The very moment I can plan or
    programme `serendipity' it cannot be called serendipity
    anymore}.'' \citep{van1994anatomy}
\end{quote}
We believe that serendipity is not so mystical as such statements
might seem to imply.  In Section \ref{sec:discussion} we will
show that ``patterns of serendipity'' like those collected by van Andel
can be applied in the design of computational systems.

Serendipity is particularly relevant for thinking about
\emph{autonomous systems}.  There is a certain amount of fear,
apprehension, and concern surrounding such systems, which
\citeA{machine-ethics-status} suggest is largely focused around the
question: will these systems behave in an ethical manner?  The more we
constrain the system's operation, the less chance there is of it
``running off the rails.''  However, constraints come with a serious
downside.  Highly constained systems will not be able to \emph{learn}
anything very new while they operate.  And yet, if this means that the
system's ethical judgement is fixed once and for all, we cannot trust
it to behave ethically if circumstances change
\cite{powers2005deontological}.  Highly constrained systems are
unlikely to be convincingly \emph{social}, if emergent behaviour is
ruled out in advance.  Systems that only act normatively (that is,
pursuing purposes for which they have been pre-programmed) serve as
proxies for their creator's judgements, and do not make
\emph{evaluations} that are meaningfully ``their own.''  We return to these themes in Section \ref{sec:discussion}.

If serendipity was ruled out as a matter of principle, computing would
be restricted to happy or unhappy (as the case may be)
\emph{unsurprises}, interspersed with unhappy surprises.
\citeA{rao2015breaking} uses the term \emph{zemblanity} -- after
William Boyd \citeyearpar{boyd2010armadillo}: ``zemblanity, the
opposite of serendipity, the faculty of making unhappy, unlucky and
expected discoveries by design'' -- to describe systems that are
doomed to produce only unhappy unsurprises.  According to Rao,
this is the implied fate of
systems that are tied inextricably to a fixed vision, from which any
deviation constitues a mistake.  This condition stands at a sharp
contrast with the ``second-order cybernetics'' introduced by
\citeA{von2003cybernetics}, which envisions systems that are able to
specify, and adapt, their own purpose.  It also contrasts with Taleb's
\citeyearpar{taleb2012antifragile} notion of ``antifragility'' in
which disturbances within a certain range strengthen the system.  In
fact, any sufficiently complex computational system is bound to make
decisions that its creators could not foresee, and may not fully
understand \cite{minsky1967programming}.  Demonstrably gracefully
behaviour in response to surprises and a preference for ``happy'' as
opposed to ``unhappy'' outcomes may be prerequisites for the
development of autonomous systems that are worthy of our trust.

Less controversial than ``programmed serendipity'', but no less worthy
of study, is serendipity that arises in the course of user
interaction.  However, it should not be assumed that a system that can
accommodate user interaction will directly lead to serendipity; take
for example the use of a calculator, where the potential for
serendipity through user interaction is minimal.  The frameworks
introduced in this paper are broad enough to be used in the design and evaluation of sociotechnical systems, and we touch on
some examples, although we focus on modelling serendipity in
computational systems.

In Section \ref{sec:literature-review}, we survey the broad literature
on serendipity including the etymology of the term itself, and examine
prior applications of the concept of serendipity in a computing
context.  Then in Section \ref{sec:our-model} we present our own
definition of serendipity, which synthesises the understanding gained
from these historical examples, and prepares the way for evaluation of
serendipity in computational systems.  Section
\ref{sec:computational-serendipity} examines three such case studies.
Section \ref{sec:discussion} offers recommendations for researchers
working in the computational modelling of serendipity and related
areas such as computational creativity, and describes our own plans
for future work.  Section \ref{sec:conclusion} reviews the
contributions of this paper towards computational modelling and
evaluation of serendipity.


