\section{Patterns of Serendipity} \label{sec:patterns-of-serendipity}

\begin{figure}[p]
\centering
%\input{grid-input}
\resizebox{1.0\textwidth}{!}{%
\begin{tikzpicture}[framed]
\draw[step=1cm,black,thin] (0,0) grid (12,13);
% Zeroth column
\node (bottom) at (-0.5,12.5)[draw,fill=green!60,minimum width=1cm,minimum height=1cm] {};  
\node[draw=none,left=.2cm of bottom.west,anchor=east] {\emph{Analogy}};
\node (bottom) at (-0.5,11.5)[draw,fill=green!60,minimum width=1cm,minimum height=1cm] {};  
\node[draw=none,left=.2cm of bottom.west,anchor=east] {\emph{One surprising obs.}};
\node (bottom) at (-0.5,10.5)[draw,fill=green!60,minimum width=1cm,minimum height=1cm] {};  
\node[draw=none,left=.2cm of bottom.west,anchor=east] {\emph{Rep. of surprise}};
\node (bottom) at (-0.5,9.5)[draw,fill=green!60,minimum width=1cm,minimum height=1cm] {};  
\node[draw=none,left=.2cm of bottom.west,anchor=east] {\emph{Successful error}}; 
\node (bottom) at (-0.5,8.5)[draw,fill=green!60,minimum width=1cm,minimum height=1cm] {};  
\node[draw=none,left=.2cm of bottom.west,anchor=east] {\emph{Side effect}};
\node (bottom) at (-0.5,7.5)[draw,fill=green!60,minimum width=1cm,minimum height=1cm] {};  
\node[draw=none,left=.2cm of bottom.west,anchor=east] {\emph{Spin off}};
\node (bottom) at (-0.5,6.5)[draw,fill=green!60,minimum width=1cm,minimum height=1cm] {};  
\node[draw=none,left=.2cm of bottom.west,anchor=east] {\emph{Wrong hypothesis}};
\node (bottom) at (-0.5,5.5)[draw,fill=green!60,minimum width=1cm,minimum height=1cm] {};  
\node[draw=none,left=.2cm of bottom.west,anchor=east] {\emph{No hypothesis}};
\node (bottom) at (-0.5,4.5)[draw,fill=green!60,minimum width=1cm,minimum height=1cm] {};  
\node[draw=none,left=.2cm of bottom.west,anchor=east] {\emph{Inversion}};
\node (bottom) at (-0.5,3.5)[draw,fill=green!60,minimum width=1cm,minimum height=1cm] {};  
\node[draw=none,left=.2cm of bottom.west,anchor=east] {\emph{Testing popular belief}};
\node (bottom) at (-0.5,2.5)[draw,fill=green!60,minimum width=1cm,minimum height=1cm] {};  
\node[draw=none,left=.2cm of bottom.west,anchor=east] {\emph{Outsider}};
\node (bottom) at (-0.5,1.5)[draw,fill=green!60,minimum width=1cm,minimum height=1cm] {};  
\node[draw=none,left=.2cm of bottom.west,anchor=east] {\emph{Disturbance}};
\node (bottom) at (-0.5,0.5)[draw,fill=green!60,minimum width=1cm,minimum height=1cm] {};  
\node[draw=none,left=.2cm of bottom.west,anchor=east] {\emph{Scarcity}}; 
\node (bottom) at (-0.5,-0.5)[draw,fill=green!60,minimum width=1cm,minimum height=1cm] {};  
\node[draw=none,left=.2cm of bottom.west,anchor=east] {\emph{Interruption}};
\node[draw=none,rotate=90,below=.2cm of bottom.south,yshift=-.1cm,anchor=east] {Focus shift};
% First column
\node (bottom) at (0.5,12.5)[draw,fill=green!60,minimum width=1cm,minimum height=1cm] {};  
\node (bottom) at (0.5,11.5)[draw,fill=green!60,minimum width=1cm,minimum height=1cm] {};  
\node (bottom) at (0.5,10.5)[draw,fill=green!60,minimum width=1cm,minimum height=1cm] {};  
\node (bottom) at (0.5,9.5)[draw,fill=green!60,minimum width=1cm,minimum height=1cm] {};  
\node (bottom) at (0.5,8.5)[draw,fill=green!60,minimum width=1cm,minimum height=1cm] {};  
\node (bottom) at (0.5,7.5)[draw,fill=green!60,minimum width=1cm,minimum height=1cm] {};  
\node (bottom) at (0.5,6.5)[draw,fill=green!60,minimum width=1cm,minimum height=1cm] {};  
\node (bottom) at (0.5,5.5)[draw,fill=green!60,minimum width=1cm,minimum height=1cm] {};  
\node (bottom) at (0.5,4.5)[draw,fill=green!60,minimum width=1cm,minimum height=1cm] {};  
\node (bottom) at (0.5,3.5)[draw,fill=green!60,minimum width=1cm,minimum height=1cm] {};  
\node (bottom) at (0.5,2.5)[draw,fill=green!60,minimum width=1cm,minimum height=1cm] {};  
\node (bottom) at (0.5,1.5)[draw,fill=green!60,minimum width=1cm,minimum height=1cm] {};  
\node (bottom) at (0.5,0.5)[draw,fill=green!60,minimum width=1cm,minimum height=1cm] {};  
\node (bottom) at (0.5,-0.5)[draw,fill=green!60,minimum width=1cm,minimum height=1cm] {};  
\node[draw=none,rotate=90,below=.2cm of bottom.south,yshift=-.1cm,anchor=east] {Prepared mind};
 % Interruption               
% Second column
\node at (1.5,12.5) [draw,fill=green!60,minimum width=1cm,minimum height=1cm]{};
\node at (1.5,11.5) [draw,fill=green!60,minimum width=1cm,minimum height=1cm]{};
\node at (1.5,10.5) [draw,fill=green!60,minimum width=1cm,minimum height=1cm]{};
\node at (1.5,9.5)  [draw,fill=green!60,minimum width=1cm,minimum height=1cm]{};
\node at (1.5,8.5)  [draw,fill=green!60,minimum width=1cm,minimum height=1cm]{};
\node at (1.5,7.5)  [draw,fill=green!60,minimum width=1cm,minimum height=1cm]{};
\node at (1.5,6.5)  [draw,fill=green!60,minimum width=1cm,minimum height=1cm]{};
\node at (1.5,5.5)  [draw,fill=green!60,minimum width=1cm,minimum height=1cm]{};
\node at (1.5,4.5)  [draw,fill=green!60,minimum width=1cm,minimum height=1cm]{};
\node at (1.5,3.5)  [draw,fill=red!80,minimum width=1cm,minimum height=1cm]{};
\node at (1.5,2.5)  [draw,fill=green!60,minimum width=1cm,minimum height=1cm]{};
\node at (1.5,1.5)  [draw,fill=green!60,minimum width=1cm,minimum height=1cm]{};
\node at (1.5,0.5)  [draw,fill=green!60,minimum width=1cm,minimum height=1cm]{};
\node (bottom) at (1.5,-0.5)[draw,fill=green!60,minimum width=1cm,minimum height=1cm] {};  
\node[draw=none,rotate=90,below=.2cm of bottom.south,yshift=-.1cm,anchor=east] {Serendipity trigger};
% Third column
\node at (2.5,12.5) [draw,fill=green!60,minimum width=1cm,minimum height=1cm]{}; % Analogy                    
\node at (2.5,11.5) [draw,fill=green!60,minimum width=1cm,minimum height=1cm]{}; % One surprising observation 
\node at (2.5,10.5) [draw,fill=green!60,minimum width=1cm,minimum height=1cm]{}; % Repetition of surprise     
\node at (2.5,9.5)  [draw,fill=green!60,minimum width=1cm,minimum height=1cm]{}; % Successful error           
\node at (2.5,8.5)  [draw,fill=green!60,minimum width=1cm,minimum height=1cm]{}; % Side effect                
\node at (2.5,7.5)  [draw,fill=green!60,minimum width=1cm,minimum height=1cm]{}; % Spin off                   
\node at (2.5,6.5)  [draw,fill=green!60,minimum width=1cm,minimum height=1cm]{}; % Wrong hypothesis           
\node at (2.5,5.5)  [draw,fill=green!60,minimum width=1cm,minimum height=1cm]{}; % No hypothesis              
\node at (2.5,4.5)  [draw,fill=green!60,minimum width=1cm,minimum height=1cm]{}; % Inversion                  
\node at (2.5,3.5)  [draw,fill=green!60,minimum width=1cm,minimum height=1cm]{}; % Testing popular belief     
\node at (2.5,2.5)  [draw,fill=green!60,minimum width=1cm,minimum height=1cm]{}; % outsider    
\node at (2.5,1.5)  [draw,fill=green!60,minimum width=1cm,minimum height=1cm]{}; % Disturbance                
\node at (2.5,0.5)  [draw,fill=green!60,minimum width=1cm,minimum height=1cm]{}; % Scarcity                   
\node (bottom) at (2.5,-0.5)[draw,fill=green!60,minimum width=1cm,minimum height=1cm] {};  
\node[draw=none,rotate=90,below=.2cm of bottom.south,yshift=-.1cm,anchor=east] {Bridge};
% Fourth column
\node at (3.5,12.5) [draw,fill=green!60,minimum width=1cm,minimum height=1cm]{}; % Analogy                    
\node at (3.5,11.5) [draw,fill=green!60,minimum width=1cm,minimum height=1cm]{}; % One surprising observation 
\node at (3.5,10.5) [draw,fill=green!60,minimum width=1cm,minimum height=1cm]{}; % Repetition of surprise     
\node at (3.5,9.5)  [draw,fill=green!60,minimum width=1cm,minimum height=1cm]{}; % Successful error           
\node at (3.5,8.5)  [draw,fill=green!60,minimum width=1cm,minimum height=1cm]{}; % Side effect                
\node at (3.5,7.5)  [draw,fill=green!60,minimum width=1cm,minimum height=1cm]{}; % Spin off                   
\node at (3.5,6.5)  [draw,fill=green!60,minimum width=1cm,minimum height=1cm]{}; % Wrong hypothesis           
\node at (3.5,5.5)  [draw,fill=green!60,minimum width=1cm,minimum height=1cm]{}; % No hypothesis              
\node at (3.5,4.5)  [draw,fill=green!60,minimum width=1cm,minimum height=1cm]{}; % Inversion
\draw [fill=red!80] (3,4) -- (3,5) -- (4,4);  
\node at (3.5,3.5)  [draw,fill=green!60,minimum width=1cm,minimum height=1cm]{}; % Testing popular belief     
\node at (3.5,2.5)  [draw,fill=green!60,minimum width=1cm,minimum height=1cm]{}; % Outsider
\node at (3.5,1.5)  [draw,fill=green!60,minimum width=1cm,minimum height=1cm]{}; % Disturbance                
\node at (3.5,0.5)  [draw,fill=green!60,minimum width=1cm,minimum height=1cm]{}; % Scarcity                   
%\node at (3,0)  [draw,fill=green!60,minimum width=1cm,minimum height=1cm]{}; % Interruption               
\node (bottom) at (3.5,-0.5)[draw,fill=green!60,minimum width=1cm,minimum height=1cm] {};  
\node[draw=none,rotate=90,below=.2cm of bottom.south,yshift=-.1cm,anchor=east] {Result};
% Fifth column
\node at (4.5,12.5) [draw,fill=green!60,minimum width=1cm,minimum height=1cm]{}; % Analogy                    
\node at (4.5,11.5) [draw,fill=green!60,minimum width=1cm,minimum height=1cm]{}; % One surprising observation 
\node at (4.5,10.5) [draw,fill=green!60,minimum width=1cm,minimum height=1cm]{}; % Repetition of surprise     
\node at (4.5,9.5)  [draw,fill=green!60,minimum width=1cm,minimum height=1cm]{}; % Successful error           
\node at (4.5,8.5)  [draw,fill=green!60,minimum width=1cm,minimum height=1cm]{}; % Side effect                
\node at (4.5,7.5)  [draw,fill=green!60,minimum width=1cm,minimum height=1cm]{}; % Spin off                   
\node at (4.5,6.5)  [draw,fill=green!60,minimum width=1cm,minimum height=1cm]{}; % Wrong hypothesis           
\node at (4.5,5.5)  [draw,fill=green!60,minimum width=1cm,minimum height=1cm]{}; % No hypothesis              
\node at (4.5,4.5)  [draw,fill=green!60,minimum width=1cm,minimum height=1cm]{}; % Inversion
\node at (4.5,3.5)  [draw,fill=green!60,minimum width=1cm,minimum height=1cm]{}; % Testing popular belief     
\node at (4.5,2.5)  [draw,fill=green!60,minimum width=1cm,minimum height=1cm]{}; % Testing popular belief     
\node at (4.5,1.5)  [draw,fill=red!80,minimum width=1cm,minimum height=1cm]{}; % Disturbance                
\node at (4.5,0.5)  [draw,fill=green!60,minimum width=1cm,minimum height=1cm]{}; % Scarcity                   
% \node at (4,0)  [draw,fill=green!60,minimum width=1cm,minimum height=1cm]{}; % Interruption               
\node (bottom) at (4.5,-0.5)[draw,fill=green!60,minimum width=1cm,minimum height=1cm] {};  
\node[draw=none,rotate=90,below=.2cm of bottom.south,yshift=-.1cm,anchor=east] {Chance};
% Sixth column
\node at (5.5,12.5) [draw,fill=green!60,minimum width=1cm,minimum height=1cm]{}; % Analogy                    
\node at (5.5,11.5) [draw,fill=green!60,minimum width=1cm,minimum height=1cm]{}; % One surprising observation 
\node at (5.5,10.5) [draw,fill=green!60,minimum width=1cm,minimum height=1cm]{}; % Repetition of surprise     
\node at (5.5,9.5)  [draw,fill=green!60,minimum width=1cm,minimum height=1cm]{}; % Successful error           
\node at (5.5,8.5)  [draw,fill=green!60,minimum width=1cm,minimum height=1cm]{}; % Side effect                
\node at (5.5,7.5)  [draw,fill=green!60,minimum width=1cm,minimum height=1cm]{}; % Spin off                   
\node at (5.5,6.5)  [draw,fill=green!60,minimum width=1cm,minimum height=1cm]{}; % Wrong hypothesis           
\node at (5.5,5.5)  [draw,fill=green!60,minimum width=1cm,minimum height=1cm]{}; % No hypothesis              
\node at (5.5,4.5)  [draw,fill=green!60,minimum width=1cm,minimum height=1cm]{}; % Inversion                  
\draw [fill=red!80] (5,4) -- (5,5) -- (6,4);  
\node at (5.5,3.5)  [draw,fill=green!60,minimum width=1cm,minimum height=1cm]{}; % Testing popular belief     
\node at (5.5,2.5)  [draw,fill=green!60,minimum width=1cm,minimum height=1cm]{}; % Testing popular belief     
\node at (5.5,1.5)  [draw,fill=green!60,minimum width=1cm,minimum height=1cm]{}; % Disturbance                
\node at (5.5,0.5)  [draw,fill=red!80,minimum width=1cm,minimum height=1cm]{}; % Scarcity                   
% \node at (5,0)  [draw,fill=green!60,minimum width=1cm,minimum height=1cm]{}; % Interruption               
\node (bottom) at (5.5,-0.5)[draw,fill=green!60,minimum width=1cm,minimum height=1cm] {};  
\node[draw=none,rotate=90,below=.2cm of bottom.south,yshift=-.1cm,anchor=east] {Curiosity};
% Seventh column
\node at (6.5,12.5) [draw,fill=green!60,minimum width=1cm,minimum height=1cm]{}; % Analogy                    
\node at (6.5,11.5) [draw,fill=green!60,minimum width=1cm,minimum height=1cm]{}; % One surprising observation 
\node at (6.5,10.5) [draw,fill=green!60,minimum width=1cm,minimum height=1cm]{}; % Repetition of surprise     
\node at (6.5,9.5)  [draw,fill=green!60,minimum width=1cm,minimum height=1cm]{}; % Successful error           
\node at (6.5,8.5)  [draw,fill=green!60,minimum width=1cm,minimum height=1cm]{}; % Side effect                
\node at (6.5,7.5)  [draw,fill=green!60,minimum width=1cm,minimum height=1cm]{}; % Spin off                   
\node at (6.5,6.5)  [draw,fill=green!60,minimum width=1cm,minimum height=1cm]{}; % Wrong hypothesis           
\node at (6.5,5.5)  [draw,fill=green!60,minimum width=1cm,minimum height=1cm]{}; % No hypothesis              
\node at (6.5,4.5)  [draw,fill=red!80,minimum width=1cm,minimum height=1cm]{}; % Inversion                  
\node at (6.5,3.5)  [draw,fill=green!60,minimum width=1cm,minimum height=1cm]{}; % Testing popular belief     
\node at (6.5,2.5)  [draw,fill=green!60,minimum width=1cm,minimum height=1cm]{}; % Testing popular belief     
\node at (6.5,1.5)  [draw,fill=green!60,minimum width=1cm,minimum height=1cm]{}; % Disturbance                
\node at (6.5,0.5)  [draw,fill=green!60,minimum width=1cm,minimum height=1cm]{}; % Scarcity                   
% \node at (6,0)  [draw,fill=green!60,minimum width=1cm,minimum height=1cm]{}; % Interruption               
\node (bottom) at (6.5,-0.5)[draw,fill=green!60,minimum width=1cm,minimum height=1cm] {};  
\node[draw=none,rotate=90,below=.2cm of bottom.south,yshift=-.1cm,anchor=east] {Sagacity};
% Eighth column
\node at (7.5,12.5) [draw,fill=green!60,minimum width=1cm,minimum height=1cm]{}; % Analogy                    
\node at (7.5,11.5) [draw,fill=green!60,minimum width=1cm,minimum height=1cm]{}; % One surprising observation 
\node at (7.5,10.5) [draw,fill=green!60,minimum width=1cm,minimum height=1cm]{}; % Repetition of surprise     
\node at (7.5,9.5)  [draw,fill=green!60,minimum width=1cm,minimum height=1cm]{}; % Successful error           
\node at (7.5,8.5)  [draw,fill=green!60,minimum width=1cm,minimum height=1cm]{}; % Side effect                
\node at (7.5,7.5)  [draw,fill=green!60,minimum width=1cm,minimum height=1cm]{}; % Spin off                   
\node at (7.5,6.5)  [draw,fill=green!60,minimum width=1cm,minimum height=1cm]{}; % Wrong hypothesis           
\node at (7.5,5.5)  [draw,fill=green!60,minimum width=1cm,minimum height=1cm]{}; % No hypothesis              
%\node at (7.5,4.5)  [draw,shading=axis,bottom color=green!60,top color=red,minimum width=1cm,minimum height=1cm,shading angle=45]{}; % Inversion
\node at (7.5,4.5)  [draw,fill=green!60,minimum width=1cm,minimum height=1cm]{}; % Inversion
%% half off
\draw [fill=red!80] (7,4) -- (7,5) -- (8,4);  % Interruption
%\draw (7,4) -- (8,5);  % Interruption
\node at (7.5,3.5)  [draw,fill=green!60,minimum width=1cm,minimum height=1cm]{}; % Testing popular belief     
\node at (7.5,2.5)  [draw,fill=green!60,minimum width=1cm,minimum height=1cm]{}; % outsider
\node at (7.5,1.5)  [draw,fill=green!60,minimum width=1cm,minimum height=1cm]{}; % Disturbance                
\node at (7.5,0.5)  [draw,fill=green!60,minimum width=1cm,minimum height=1cm]{}; % Scarcity                   
%\node at (7,0)  [draw,fill=green!60,minimum width=1cm,minimum height=1cm]{}; % Interruption               
\node (bottom) at (7.5,-0.5)[draw,fill=green!60,minimum width=1cm,minimum height=1cm] {};  
\node[draw=none,rotate=90,below=.2cm of bottom.south,yshift=-.1cm,anchor=east] {Value};
%%%%%%%%%%%%%%%
% Ninth column
\node at (8.5,12.5) [draw,fill=green!60,minimum width=1cm,minimum height=1cm]{}; % Analogy                    
\node at (8.5,11.5) [draw,fill=green!60,minimum width=1cm,minimum height=1cm]{}; % One surprising observation 
\node at (8.5,10.5) [draw,fill=green!60,minimum width=1cm,minimum height=1cm]{}; % Repetition of surprise     
\node at (8.5,9.5)  [draw,fill=green!60,minimum width=1cm,minimum height=1cm]{}; % Successful error           
\node at (8.5,8.5)  [draw,fill=green!60,minimum width=1cm,minimum height=1cm]{}; % Side effect                
\node at (8.5,7.5)  [draw,fill=green!60,minimum width=1cm,minimum height=1cm]{}; % Spin off                   
\node at (8.5,6.5)  [draw,fill=green!60,minimum width=1cm,minimum height=1cm]{}; % Wrong hypothesis           
\node at (8.5,5.5)  [draw,fill=green!60,minimum width=1cm,minimum height=1cm]{}; % No hypothesis              
\node at (8.5,4.5)  [draw,fill=green!60,minimum width=1cm,minimum height=1cm]{}; % Inversion                  
\node at (8.5,3.5)  [draw,fill=green!60,minimum width=1cm,minimum height=1cm]{}; % Testing popular belief     
\node at (8.5,2.5)  [draw,fill=green!60,minimum width=1cm,minimum height=1cm]{}; % Testing popular belief     
\node at (8.5,1.5)  [draw,fill=green!60,minimum width=1cm,minimum height=1cm]{}; % Disturbance                
\node at (8.5,0.5)  [draw,fill=green!60,minimum width=1cm,minimum height=1cm]{}; % Scarcity                   
% \node at (8,0)  [draw,fill=green!60,minimum width=1cm,minimum height=1cm]{}; % Interruption               
\node (bottom) at (8.5,-0.5)[draw,fill=green!60,minimum width=1cm,minimum height=1cm] {};  
\node[draw=none,rotate=90,below=.2cm of bottom.south,yshift=-.1cm,anchor=east] {Dynamic world};
% Tenth column
\node at (9.5,12.5) [draw,fill=green!60,minimum width=1cm,minimum height=1cm]{}; % Analogy                    
\node at (9.5,11.5) [draw,fill=green!60,minimum width=1cm,minimum height=1cm]{}; % One surprising observation 
\node at (9.5,10.5) [draw,fill=green!60,minimum width=1cm,minimum height=1cm]{}; % Repetition of surprise     
\node at (9.5,9.5)  [draw,fill=green!60,minimum width=1cm,minimum height=1cm]{}; % Successful error           
\node at (9.5,8.5)  [draw,fill=green!60,minimum width=1cm,minimum height=1cm]{}; % Side effect                
\node at (9.5,7.5)  [draw,fill=green!60,minimum width=1cm,minimum height=1cm]{}; % Spin off
\draw [fill=red!80] (9,7) -- (9,8) -- (10,7);  
\node at (9.5,6.5)  [draw,fill=green!60,minimum width=1cm,minimum height=1cm]{}; % Wrong hypothesis           
\node at (9.5,5.5)  [draw,fill=green!60,minimum width=1cm,minimum height=1cm]{}; % No hypothesis              
\node at (9.5,4.5)  [draw,fill=green!60,minimum width=1cm,minimum height=1cm]{}; % Inversion                  
\node at (9.5,3.5)  [draw,fill=green!60,minimum width=1cm,minimum height=1cm]{}; % Testing popular belief     
\node at (9.5,2.5)  [draw,fill=green!60,minimum width=1cm,minimum height=1cm]{}; % Testing popular belief     
\node at (9.5,1.5)  [draw,fill=green!60,minimum width=1cm,minimum height=1cm]{}; % Disturbance                
\node at (9.5,0.5)  [draw,fill=green!60,minimum width=1cm,minimum height=1cm]{}; % Scarcity                   
% \node at (9,0)  [draw,fill=green!60,minimum width=1cm,minimum height=1cm]{}; % Interruption               
\node (bottom) at (9.5,-0.5)[draw,fill=green!60,minimum width=1cm,minimum height=1cm] {};  
\node[draw=none,rotate=90,below=.2cm of bottom.south,yshift=-.1cm,anchor=east] {Multiple contexts};
% Eleventh column
\node at (10.5,12.5) [draw,fill=red!80,minimum width=1cm,minimum height=1cm]{}; % Analogy                    
\node at (10.5,11.5) [draw,fill=red!80,minimum width=1cm,minimum height=1cm]{}; % One surprising observation 
\node at (10.5,10.5) [draw,fill=red!80,minimum width=1cm,minimum height=1cm]{}; % Repetition of surprise     
\node at (10.5,9.5)  [draw,fill=green!60,minimum width=1cm,minimum height=1cm]{}; % Successful error           
\node at (10.5,8.5)  [draw,fill=green!60,minimum width=1cm,minimum height=1cm]{}; % Side effect                
\node at (10.5,7.5)  [draw,fill=red!80,minimum width=1cm,minimum height=1cm]{}; % Spin off                   
\node at (10.5,6.5)  [draw,fill=green!60,minimum width=1cm,minimum height=1cm]{}; % Wrong hypothesis           
\node at (10.5,5.5)  [draw,fill=red!80,minimum width=1cm,minimum height=1cm]{}; % No hypothesis              
\node at (10.5,4.5)  [draw,fill=green!60,minimum width=1cm,minimum height=1cm]{}; % Inversion                  
\node at (10.5,3.5)  [draw,fill=red!80,minimum width=1cm,minimum height=1cm]{}; % Testing popular belief     
\node at (10.5,2.5)  [draw,fill=green!60,minimum width=1cm,minimum height=1cm]{}; % outsider
\node at (10.5,1.5)  [draw,fill=green!60,minimum width=1cm,minimum height=1cm]{}; % Disturbance                
\node at (10.5,0.5)  [draw,fill=red!80,minimum width=1cm,minimum height=1cm]{}; % Scarcity                   
% \node at (10,0)  [draw,fill=green!60,minimum width=1cm,minimum height=1cm]{}; % Interruption               
\node (bottom) at (10.5,-0.5)[draw,fill=green!60,minimum width=1cm,minimum height=1cm] {};  
\node[draw=none,rotate=90,below=.2cm of bottom.south,yshift=-.1cm,anchor=east] {Multiple tasks};
% Twelfth column
\node at (11.5,12.5) [draw,fill=green!60,minimum width=1cm,minimum height=1cm]{}; % Analogy                    
\node at (11.5,11.5) [draw,fill=red!80,minimum width=1cm,minimum height=1cm]{}; % One surprising observation 
\node at (11.5,10.5) [draw,fill=red!80,minimum width=1cm,minimum height=1cm]{}; % Repetition of surprise     
\node at (11.5,9.5)  [draw,fill=green!60,minimum width=1cm,minimum height=1cm]{}; % Successful error           
\node at (11.5,8.5)  [draw,fill=green!60,minimum width=1cm,minimum height=1cm]{}; % Side effect                
\node at (11.5,7.5)  [draw,fill=red!80,minimum width=1cm,minimum height=1cm]{}; % Spin off                   
\node at (11.5,6.5)  [draw,fill=green!60,minimum width=1cm,minimum height=1cm]{}; % Wrong hypothesis           
\node at (11.5,5.5)  [draw,fill=green!60,minimum width=1cm,minimum height=1cm]{}; % No hypothesis              
\node at (11.5,4.5)  [draw,fill=green!60,minimum width=1cm,minimum height=1cm]{}; % Inversion                  
\node at (11.5,3.5)  [draw,fill=green!60,minimum width=1cm,minimum height=1cm]{}; % Testing popular belief     
\node at (11.5,2.5)  [draw,fill=green!60,minimum width=1cm,minimum height=1cm]{}; % Testing popular belief     
\node at (11.5,1.5)  [draw,fill=green!60,minimum width=1cm,minimum height=1cm]{}; % Disturbance                
\node at (11.5,0.5)  [draw,fill=red!80,minimum width=1cm,minimum height=1cm]{}; % Scarcity                   
\node (bottom) at (11.5,-0.5)[draw,fill=red!80,minimum width=1cm,minimum height=1cm] {};  
\node[draw=none,rotate=90,below=.2cm of bottom.south,yshift=-.1cm,anchor=east] {Multiple influences};

\begin{scope}[xshift=.5cm,yshift=0cm]
\node (A) at (-5.75,-2.0)  [draw,fill=green!60,minimum width=.7cm,minimum height=.7cm]{}; %
\node[draw=none,right=.2cm of A.east,text width=3.10cm] {{\footnotesize Pattern includes\par feature}}; 
\node (B) at (-5.75,-3.0)  [draw,fill=red!80,minimum width=.7cm,minimum height=.7cm]{}; %
\node[draw=none,right=.2cm of B.east,text width=3.10cm] {{\footnotesize $\ldots$ does not include $\ldots$}}; 

\node (C) at (-5.75,-4)  [draw,fill=green!60,minimum width=.7cm,minimum height=.7cm]{}; %
\node[draw=none,right=.2cm of C.east,text width=3.10cm] (Ct) {{\footnotesize $\ldots$ only eventually}};
%% half off
\draw [fill=red!80] (-6.1,-4.35) -- (-6.1,-3.65) -- (-5.4,-4.35);  

\node[draw=black!80, fit=(A) (B) (C) (Ct)](FIt1) {};
\end{scope}
\end{tikzpicture}
}
\caption{Characteristics of Pek van Andel's patterns of serendipity\label{fig:grid}}
\end{figure}

% ***AJ what is a 'situational pattern of serendipity'? Can we add a definition e.g. from van Andel***
Figure \ref{fig:grid} examines 14 situational patterns of serendipity
collected by van Andel \cite{van1994anatomy} through the lens of the evaluation
criteria described in Section \ref{sec:literature-review}.
%
As required by our theory, a ``focus shift'' appears in each instance,
although it has a different flavour in the different examples.  In
this analysis, only three of the other criteria mentioned above are
clearly present in \emph{all} of the patterns: ``a prepared mind'', a
``bridge'', and a ``dynamic world.''  Similarly, only four of van
Andel's patterns exhibit all of the characteristics we identified:
\emph{Successful error}, \emph{Side effect}, \emph{Wrong hypothesis},
and \emph{Outsider}.

``Near misses'' are also of interest, and help to illustrate the role
of the various factors from Section \ref{sec:literature-review}.
%
For example, the \emph{Inversion} pattern is somewhat closer to what is called an \emph{antipattern} in the design pattern literature \cite{brown1998antipatterns}.  Van Andel describes the story of a researcher observing an effect (the anticoagulant heparine) which was precisely the opposite of the one sought (factors that \emph{cause} blood clotting) -- and failing to acknowledge that this observation was important for over 40 years.  The result was eventually seen to be of value: however, in this instance, we may have an example of a mind that is \emph{over-prepared}, and focused on a particular sort of result, rather than a truly ``sagacious'' mind that is both prepared and open to serendipitous findings.

In the case of \emph{Testing popular belief}, van Andel gives an
account of a medical practise that originated in a folk claim, namely
cowpox-derived immunity to smallpox.  This effect, for milkmaids,
might indeed be called serendipitous.  Indeed, the medical use of
cowpox has been described as ``widely know'' \cite{riedel2005edward}
prior to its popularisation by Edward Jenner.  Nevertheless, 
Jenner's ``relentless promotion and devoted research of vaccination
\ldots changed the way medicine was practised'' \cite{riedel2005edward}.
This again might be called serendipity, but most clearly at the social
rather than personal level.  These comments should not be seen to
disparage Jenner's contribution, or diminish the role of a curious
chain of events in his personal history that tied his fate to that of
the smallpox vaccine.  Many of these had the air of serendipity about
them -- but even so, it is hard to find one specific ``serendipity
trigger.''
 
In describing \emph{Disturbance}, van Andel's exemplar is the
creation of radio telescopy from noise in transatlantic telephone calls
(paralleling the subsequent discovery by Penzias and Wilson).  Here it
is hard to see an overt role for ``chance,'' since as machinery at
various scales is created, disturbance is somewhat inevitable, even if
a specific disturbance in a specific machine is unexpected.
Similarly, in cases of \emph{Scarcity}, ``curiosity'' may not play a
significant role, and may instead be replaced by the drive of desire
and corresponding ingenuity.

Multiple contexts, tasks, and influences should be seen to be
conducive to serendipitous discovery, but not strictly necessary.  For
example, in addition to the context of a research laboratory, there
may be the context of subsequent industrial application.  However,
within the laboratory itself (where a \emph{Spin off} discovery might
be made) the future context is not typically in force.

There are a number of additional reoccurring themes, which are worthy
of further comment, and which could form the basis of further
(meta-)patterns.

\begin{description}
\item[\emph{It's all part of a day's work.}] Often the discoverer had
  a problem to solve or job to do, and made the serendipitous
  discovery in the course of doing their job.  This sort of
  serendipity is often ``social.''  For example, in the
  \emph{Outsider} pattern, the ophthalmologist Gregg was simply
  listening to his patient and taking what she said seriously; in
  other words, he was doing his job.  But this led to a new
  hypothesis.
\item[\emph{Factorisation is useful.}] Variability, and in the case of
  scientific work, factorisation (e.g. via control studies) often
  plays a key role in establishing ``multiple contexts.''
  Serendipitous discovery often happens in the context of ``natural
  experiments,'' for example, in the case of \emph{One surprising
    observation}, where van Andel's example dealt with the observation
  that one tree in a row was taller and healthier than its
  neighbours.\footnote{Concerning the broader issues associated with
    the ``design'' of such experiments, see
    \cite{imbens1994identification}.}
\item[\emph{A good story is liable to change.}] Comparing
  \emph{Inversion} and \emph{Spin off} suggests the value of being
  able to change the story.  If Perkin had suppressed his discovery of
  mauvine because he hadn't successfully synthesised quinine, there
  would have been no spin off, and it would be hard to call the
  discovery ``serendipitous'' -- or, indeed, to consider it to be a
  discovery at all.  Whatever its value, an event may only be
  \emph{described} as serendipitous at the narrative level.
\item[\emph{Watch out for hidden symmetries.}] The \emph{Wrong
  hypothesis} pattern involves several of the points above.  In van
  Andel's anecdote about John Cade's discovery of lithium as a
  \emph{treatment} for mania, the issues under investigation were,
  rather, the \emph{causes} of the illness.  This was initially
  conceptualised in terms of \emph{surfeit} and \emph{deficiency}.  A
  more general interpretation is that the factors influencing the
  course of an illness have hidden interactions between them.
  Serendipitous discovery may be able to find and capitalise on this
  type of (unexpected) invariant.
\end{description}

Van Andel describes three additional patterns that seem to be
connected with personal qualities of the investigator rather than with
situational features.  These are \emph{Playing}, \emph{Joke}, and
\emph{Dream}.  The theme of personal qualities and skills that support
serendipitous discovery will be taken up below, as part of a general
approach to modelling serendipity.

\subsection{Modelling serendipity with design patterns} \label{sec:unified-approach}

As illustrated above, serendipity can take place on multiple scales.
Something can be personally surprising while being socially mundane
(Boden's \emph{P-creativity} \cite{boden}), or vice versa, as in
the case of personally mundane discoveries that take on surprising
social value.

In the case of serendipitous discoveries at the personal level, the
qualities of the investigator are understood to be important features.
Thus, for example, van Andel writes that a ``sense of humour and
  sense of serendipity have a lot in common.'' 
% 
Van Andel relates the \emph{Dream} pattern -- exemplified, for him by
Descartes, but Kekul\'e's ouroborus provides another instance -- to
Poincar\'e's \cite{poincare1910creation,poincare2013science} model of
``preparation, incubation, illumination, and verification''
(cf. \cite{wallas1926art}).  Poincar\'e \cite{poincare1910creation}
clarifies that
\begin{quote}
``\emph{unconscious work}~\ldots~\emph{is possible, and of a
    certainty, it is only fruitful, if it is on the one hand preceded
    and on the other hand followed by a period of conscious work.}''
\end{quote}

What might conceptions like this mean for serendipity that takes place
on a social or indeed computational level?  In order to understand
this, we will refer van Andel's patterns and the serendipity factors
introduced above to the heterodox theory of patterns coming from the
field of design, mentioned briefly above.  First introduced by the
architect Christopher Alexander
\cite{alexander1979timeless,alexander1977pattern}, the design pattern
methodology spread from architecture to software
\cite{gabriel1996patterns}, and later, to other fields, including
public affairs \cite{schuler2008liberating} and education
\cite{bergin2012pedagogical}.

Alexander's patterns are presented in a tree-like structure called a
\emph{pattern language}, ordered in a top-down manner from large-scale
to small-scale levels of application, with each pattern presented in
terms of a \emph{picture}, a \emph{context} (including links to
relevant larger patterns), the \emph{problem} that the pattern
addresses, the \emph{solution}, a \emph{diagram}, and \emph{links to
  smaller patterns} \cite[pp. x-xi]{alexander1977pattern}.
%
A relatively convincing implementation of
Alexander's idea of patterns as a ``living
  language''
\cite[p. xvii]{alexander1977pattern} was realised
with one of the earliest applications of wiki
software developed by Ward Cunningham: the
Portland Pattern
Repository.\footnote{\url{http://c2.com/ppr/}}
The notion of pattern-finding as a process
related to, but distinct from abstraction, is
described by Richard Gabriel, who emphasises that
the ``patterns and the social process for
  applying them are designed to produce organic
  order through piecemeal growth''
\cite[p. 31]{gabriel1996patterns}.
%
In its original form, this statement describes the generative use of
patterns to create artefacts (buildings, object oriented programs,
etc.).  However, this criterion can also be applied to the growth and
development the pattern language itself, and this is the key idea
underlying our application.

Christian Kohls \cite{kohls2010structure,kohls2011structure}, deploys
a ``path'' or ``journey'' metaphor to describe design patterns in the
language of constrained optimisation problems, considering in
particular the \emph{initial state}, \emph{end state}, and
\emph{forces acting}.  This is useful because of its general nature:
it suggests that any time there are predictable dynamics observed in
the world, there is a corresponding design pattern waiting to be seen
and recorded.  This perspective can be usefully combined with the
proposal advanced by Manual DeLanda \cite{delanda2011philosophy},
among others, to give the system a simulated embodiment, putting it in
contact with a virtual world in which it does not need to, and indeed
cannot, have everything worked out in advance.  DeLanda uses the term
\emph{gradient} to describe the forces acting in a way that focuses on
the relevant features.  Like Kohls, Peter Andersen
\cite{andersen2002dynamic} considers one-dimensional paths through a
two-dimensional space with a gradient, and writes that the basic
metaphor for thought is travel.  A more general metaphor suggested
by DeLanda would take into account
%
``a population of interacting physical entities, such as the molecules
in a thin layer of soap'' \cite{delanda2005deleuze} exhibiting more
complex non-linear interactions over higher-dimensional gradients.

This discussion makes a distinction between an agential system of
interest and its broader context, which could also be described as a
physical ``system,'' or a simulation of one.  While such distinctions
tend to be leaky, to avoid undo confusion about terminology, when we
refer to ``the system'' without further qualification, we mean the
agential sub-system -- the part that behaves -- and the context will
be referred to as ``the environment.''

Modelling serendipitous behaviour requires us, as designers, to engage
in \emph{meta-modelling}: we need to build systems which are capable
of modelling their environments.  Terence Deacon
\cite{deacon2006emergence} refers to such systems as
\emph{teleodynamic}, that is, organised with respect to what they are
not.
%
However, most typical computational scenarios that simply involve reasoning
about representations will not yield the twin features of discovery
and invention that are central to our understanding of serendipity.
Such reasoning considers
%
``identity with regard to concepts, opposition with regard to the determination of concepts, analogy with regard to judgement, resemblance with regard to objects''
%
and Gilles Deleuze \cite[p. 174]{deleuze1994difference} cautions that
this activity relies on an assumed ``common sense'' that is not the
same as thought.  For Deleuze, when thought arises, it is as a matter
of necessity: ``the contingency of an encounter \ldots\ forces us to
think'' \cite[p. 176]{deleuze1994difference}.

Cast in the terms we introduced earlier: a ``prepared mind'' will have
available to it certain patterns as designs for action.  It is
understood to have an interactive dimension that makes it capable of
enacting some of these designs in the context of a ``dynamic world.''
An encounter between the system and some other aspect or occupant of
this environment forms a ``trigger'' that composes with preexisting
patterns, leading to a ``bridge'' that makes sense of the stimuli and
that leads to new designs for action as a ``result,'' which may
fundamentally change the system's subsequent behaviour.

Representational forms will certainly play a role in such systems, but
this role is secondary.  For example, actions are selected, delected,
or deplored depending on their relationship to the gradient, by way of
a model.  Nevertheless, the gradient is its own ``best model'' and it
contributes the final evaluation of systems.
%
Design patterns may be communicable between agents, but in the manner
of blueprints or genes, whereas it is the actualised building, body,
or manifest pattern of behaviour forms the crux of the
encounter.\footnote{DeLanda \cite{delanda1993virtual} emphasises the
  role of population thinking on several scales, for example, at the
  personal level relative to society, or at the neuronal level
  relative to the person.  Design patterns are strictly lower level
  than agents, and agents are lower level than interactions, but we
  cannot reduce the trajectory of an evironment's evolution to its
  representation by the agents that inhabit it: cf.
  \url{http://c2.com/cgi/wiki?OfMiceAndMen}.}

Jonathan Rowe \cite{rowe1994creativity} is one of the researchers who
argue for ``the generation of structure and regularity as emergent
phenomena arising from the interaction of low level structures,
without any central control'' (cf. \cite{pearce-boden-and-beyond}).
He favourably compares Hofstadter and Mitchell's {\sf Copycat}, in
which ``[a]nalogies are generated through the interactions of
low-level structures without any central control'' to Lenat's {\sf
  EURISKO}, in which metarules provide ``templates for expressing a
number of rules in a concise from'' and
(cf. \cite{hofstadter1994copycat,mitchell1993analogy}).
%
Low-level explorations that take place before high-level structures
have emerged can afford to be more random than changes in the
high-level structures \cite[pp. 232--233]{hofstadter1994copycat}.
\begin{quote}
``\emph{In the early stages of a run, almost all discoveries are on a
    very small, local scale: a primitive object acquires a
    description, a bond is built, and so on.  Gradually, the scale of
    actions increases: small groups begin to appear, acquire their own
    descriptions, and so on.  In the later stages of a run, actions
    take place on an even larger scale, often involving complex,
    hierarchically structured objects.}''
  \cite[p. 228]{hofstadter1994copycat}
\end{quote}

For {\sf Copycat}, a serendipitous discovery might take
the form of an especially clever or unexpected solution to an analogy
problem.  More broadly, it concerns observations that do not match a
system's preprogrammed understanding or capabilities, but which it
must nevertheless make sense of, learn from, and adapt to.  The
successor system {\sf Metacat} explicitly aims to:
\begin{quote}
``\emph{perceive patterns in its own behavior in much the same way
    that Copycat perceives patterns in letter-strings: via codelets
    looking for relationships among perceptual
    structures.}''~\cite{DBLP:journals/jetai/Marshall06}
\end{quote}

These patterns ``serve as a `medium' through which the program is able
to wield control over its own behavior''
\cite{DBLP:journals/jetai/Marshall06}.  It can also use thematic
patterns to evaluate and explain examples supplied by the user.

Our perspective is that computer programs in general can be described
as collections of ``design patterns,'' understood to encode the
dynamics of response to events which take place in the system's
environment.  We are particularly interested in the process whereby
\emph{new} patterns form, and we expect that this will typically
progress through a process of progressive skill refinement.  We will
develop the investigation of this theme further in the following
section.
