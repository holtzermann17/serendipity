\section{Serendipity in a computational context} \label{sec:computational-serendipity}

The 13 criteria from Section \ref{sec:literature-review}
specify the conditions and preconditions that are conducive to
serendipitous discovery.  Here, we revisit each of these criteria and
briefly summarise how they can be thought about from a computational
point of view.
% What is the goal of the computation (input and output)
% Why is it appropriate (formal spec e.g. considering externalities)
% what is the logic of the strategy by which it can be carried out.

% \newpage
\subsubsection*{Key condition for serendipity}

\begin{itemize}
\item \textbf{Focus shift}: A focus shift is linked to re-evaluation
  of data, processes, or products.  It may precipitate changes in the
  entire framework of evaluation or its effects may be more contained.
  Such reevaluation could be modelled using a multi-agent
  architecture, in which each agent has a goal and evaluates generated
  products relative this goal, but in which agents also share their
  products with other, who then evaluate them against their own
  metrics.
\end{itemize}

\subsubsection*{Components of serendipity}

\begin{itemize}
\item \textbf{Prepared mind}: This comprises the background knowledge,
  unsolved problems, current goal, programming, and operating
  environment of a computational system.
%%
\item \textbf{Serendipity trigger}: The generation or observation of a
  potentially novel example, concept, or conjecture, etc., which
  precedes a discovery in a computational system.\footnote{Triggers
    are often examples without an explanation, rather than
    wholly-formed concepts.}  The trigger is outside of the direct
  control of the system components responsible for evaluations.
%%
\item \textbf{Bridge}: Reasoning and/or programmatic interaction
  brings about a focus shift at an opportune juncture, building on
  prior preparation and on the serendipity trigger.  The bridge may be
  constructed on the basis of logical methods, analogies, conceptual
  blending, evolutionary search, automated theory formation and may
  draw on interactions with other systems.
%%
\item \textbf{Result}: The discovery itself may be a new product,
  artefact, process, hypothesis, use for an object, etc., generated by
  computational means, which may influence the future operations of
  the system.
\end{itemize}

\subsubsection*{Dimensions of serendipity}

\begin{itemize}
\item \textbf{Chance}: Controlled randomness in AI systems is
  well-established, e.g. in Genetic Algorithms and search.  Chance
  also applies in connection with an under-determined outside world
  (see below).
%%
\item \textbf{Curiosity}: The system needs to expend discretionary
  computational effort on the serendipity trigger.  This may be
  accompanied by system features that an observer would describe as
  playfulness, inventiveness, and the drive to experiment or
  understand.
%%
\item \textbf{Sagacity}: Sagacity be modelled by employing reasoning
  over multiple application domains simultaneously; or, again, with a
  social analogue in cases where the system does not know, but ``knows
  who to ask.''
%%
\item \textbf{Value}: The result should be interesting or useful, as
  judged by the system, the programmer, the user, or another party
  (potentially another system).
\end{itemize}

\subsubsection*{Environmental factors}

\begin{itemize}
\item \textbf{Dynamic world}: Connections with other systems, data
  sources, or user input, e.g., via the web, which is highly dynamic --
  or in the context of a larger simulation.
%%
\item \textbf{Multiple contexts}: Reasoning which operates across
  domains, such as analogical reasoning, or that considers multiple
  perspectives, as in systems with social awareness.
%%
\item \textbf{Multiple tasks}: Multiple goals or targets that compete
  for resources.  The system may be implemented using a multithreaded,
  parallel processing design.
%%
\item \textbf{Multiple influences}: This may again be modelled as a
  multi-agent systems, as or multiple interacting systems, each with
  different knowledge and goals.  The source of unexpectedness may be
  arise on various levels, and a system may bring this to bear using
  techniques of reflection.
\end{itemize}

% \subsection{Proposed experiment: A Writers Workshop for Systems} \label{sec:writers-workshop}

Richard Gabriel \cite{gabriel2002writer} describes the practise of
Writers Workshops that has been put to use for over a decade within
the Pattern Languages of Programming (PLoP) community.  The basic
style of collaboration originated much earlier with groups of literary
authors who engage in peer-group critique.  Some literary workshops
are open as to genre, and happy to accommodate beginners, like the
Minneapolis Writers
Workshop\footnote{\url{http://mnwriters.org/how-the-game-works/}};
others are focused on professionals working within a specific genre,
like the Milford Writers
Workshop\footnote{\url{http://www.milfordsf.co.uk/about.htm}}.  The
practices that Gabriel describes are fairly typical.  Authors come
with work ready to present, and read a short sample, which is then
discussed and constructively critiqued by attendees.  Presenting
authors are not permitted to rebut these comments.  The commentators
generally summarise the work and say what they have gotten out of it,
discuss what worked well in the piece, and talk about how it could be
improved.  The author listens and may take notes; at the end, he or
she can then ask questions for clarification.  Generally, non-authors
are either not permitted to attend, or are asked to stay silent
through the workshop, and perhaps sit separately from the
participating authors/reviewers.  There are similarities between the
Writers Workshops and classical practices of group composition
\cite{jin1975art} and dialectic \cite{dialectique}, and the workshop
may be considered an artistic or creative space in its own right.

In PLoP workshops, authors present design patterns and pattern
languages, or papers about patterns, rather than more traditional
literary forms like poems, stories, or chapters from novels.  Papers
must be workshopped at a PLoP or EuroPLoP conference in order to be
considered for the \emph{Transactions on Pattern Languages of
  Programming} journal.  A discussion of writers workshops
in the language of design patterns is presented by
Coplien and Woolf \cite{coplien1997pattern}.  Their patterns include:
\begin{center}
{\small
\begin{tabular}{l@{\hspace{.2cm}}l@{\hspace{.2cm}}l}
\emph{Open Review} & \emph{Safe Setting} & \emph{Workshop Comprises Authors} \\
\emph{Authors are Experts} & \emph{Community of Trust} & \emph{Moderator Guides the Workshop} \\
\emph{Thank the Author} & \emph{Selective Changes} & \emph{Clearing the Palate} \\
\end{tabular}
}
\end{center}

We propose that a similar pattern-based approach should be deployed
within the Computational Creativity community to design a workshop in
which the participants are computer systems instead of human authors.
The annual International Conference on Computational Creativity
(ICCC), now entering its sixth year, could be a suitable venue.
Rather than the system's creator presenting the system in a
traditional slideshow and discussion, or a system ``Show and Tell,''
the systems would be brought to the workshop and would present their
own work to an audience of other systems, in a Writers Workshop
format.  This might be accompanied by a short paper for the conference
proceedings written by the system's designer describing the system's
current capabilities and goals.  Subsequent publications might include
traces of interactions in the Workshop, commentary from the system on
other systems, and offline reflections on what the system might change
about its own work based on the feedback it receives.  As in the PLoP
community, it could become standard to incorporate this sort of workshop
into the process of peer reviewing journal articles for the new \emph{Journal of
  Computational Creativity}\footnote{\url{http://www.journalofcomputationalcreativity.cc}}.

\begin{table}[p]
\begin{tabular}{lp{.7\textwidth}}
{\bf\emph{Successful error}} & \\
\emph{Van Andel's example}: & Post-it\texttrademark\ notes \\[.2cm]
{\tt presentation}& Systems should be prepared to share interesting ideas even if they don't know directly how they will be useful.  \\
{\tt listening}   & Systems should listen with interest, too. \\
{\tt feedback}    & Even interesting ideas may not be ``marketable.''\\
{\tt questions}   & How is your suggestion useful? \\
{\tt reflections} & New combinations of ideas take a long time to realise, and many different ideas may need to be combined in order to come up with something useful.\\
\end{tabular}
\bigskip

\begin{tabular}{lp{.7\textwidth}}
{\bf\emph{Side effect}} & \\
\emph{Van Andel's example}: & Nicotinamide used to treat side-effects of radiation therapy proves efficacious against tuberculosis. \\[.2cm]
{\tt presentation}& Systems should use their presentation as an experiment. \\
{\tt listening}   & Listeners should allow themselves to be affected by what they are hearing. \\
{\tt feedback}    & Feedback should convey the nature of the effect.\\
{\tt questions}   & The presenter may need to ask follow-up questions to gain insight. \\
{\tt reflections} & Form a new hypothesis before seeking a new audience. \\
\end{tabular}
\bigskip

\begin{tabular}{lp{.7\textwidth}}
{\bf\emph{Wrong hypothesis}} & \\
\emph{Van Andel's example}: & Lithium, used in a control study, had an unexpected calming effect. \\[.2cm]
{\tt presentation}& How is this presentation interpretable as a (``natural'') control study? \\
{\tt listening}   & Listeners are ``guinea pigs''.\\
{\tt feedback}    & Discuss side-effects that do not necessarily correspond to the author's perceived intent. \\
{\tt questions}   & Zero in on the most interesting part of the conversation.\\
{\tt reflections} & Revise hypotheses to correspond to the most surprising feedback. \\
\end{tabular}
\bigskip

\begin{tabular}{lp{.7\textwidth}}
{\bf\emph{Outsider}} & \\
\emph{Van Andel's example}: & A mother suggests a new hypothesis to a doctor. \\[.2cm]
{\tt presentation}& The presenter is here to learn from the audience. \\
{\tt listening}   & The audience is here to give help, but also to get help.\\
{\tt feedback}    & Feedback will inevitably draw on previous experiences and ideas.\\
{\tt questions}   & What is the basis for that remark?\\
{\tt reflections} & How can I implement the suggestions?\\
\end{tabular}
\vspace{.2cm}
\caption{Reinterpreting patterns of serendipity for use in a computational workshop\label{tab:reinterpret}}
\end{table}

\begin{figure}[t]
\begin{center}
\resizebox{.93\textwidth}{!}{
\StickyNote[2.5cm]{myyellow}{{\LARGE {Interesting idea}} \\[4ex] {Surprise birthday party}}[3.8cm] \StickyNote[2.5cm]{mygreen}{{\Large I heard you say:} \\[4ex] {``surprise''} }[3.8cm]
\StickyNote[2.5cm]{pink}{{\Large Feedback:} \\[4ex] {I don't like surprises}}[3.8cm]
}
\resizebox{.61\textwidth}{!}{
\StickyNote[2.5cm]{myorange}{{\LARGE {Question}} \\[4ex] {Not even a little bit?\ldots}}[3.8cm]
\quad \raisebox{-.2cm}{\StickyNote[2.5cm]{myblue}{{\LARGE Note to self:} \\[4ex] {(Try smaller surprises \\ next time.)}}[3.8cm]}
}
\end{center}
\caption{A paper prototype for applying the \emph{Successful Error} pattern\label{fig:paper-prototype}}
\end{figure}

In order to facilitate this sort of interaction, it would be necessary
for systems to implement a basic protocol related to
%%
\[
\text{
{\tt presentation}, {\tt listening}, {\tt
  feedback}, {\tt questions}, and {\tt
  reflections}.}
\]
%%
This protocol could be thought of as a light-weight template for
creating design patterns that guide system-level participation in the
context specified by Coplien and Woolf's pattern language for writers
workshops.  Table \ref{tab:reinterpret} uses this framework to recast
the four ``perfectly'' serendipitous patterns from van Andel --
\emph{Successful error}, \emph{Side effect}, \emph{Wrong hypothesis},
and \emph{Outsider} -- in a form that may make them useful to
developers preparing to enter their systems into the Workshop.
%
Further guidelines for structuring and participating in traditional
writers workshops are presented by Linda Elkin in
\cite[pp. 201-203]{gabriel2002writer}.  It is not at all clear that
the same ground rules should apply to computer systems.  For example,
one of Elkin's rules is that ``Quips, jokes, or sarcastic comments,
even if kindly meant, are inappropriate.''  Rather than forbidding
humour, it may be better for individual comments to be rated as
helpful or non-helpful.  Again, since serendipitous discovery is an
overarching goal, in the first instance, usefulness and interest might
be judged in terms of the criteria described in Section
\ref{sec:evaluation-criteria}.

We would need a neutral environment that is not hard to develop for:
the {\sf FloWr} system described in Section \ref{sec:foundations}
offers one such possibility.  With this system, the basic operating
logic of the Workshop could be spelled out as a flowchart, and
contributing systems could use flowcharts as the basic medium for
sharing their presentations, feedback, and questions.  Developing
around a process language of this sort partially obviates the need for
participating systems to have strong natural language processing
capabilities.  
%
Post-it\texttrademark\ notes, which have provided us with a useful
example of serendipitous discovery, also provide indicative strategies
from the world of paper prototyping (Figure \ref{fig:paper-prototype}).

Gordon Pask's conversation theory, reviewed in
\cite{conversation-theory-review,boyd2004conversation}, goes
considerably beyond what we have presented here as a simple process
language, although there are structural parallels.  In a basic
Pask-style learning conversation: (0) Conversational participants are
carrying out some actions and observations; (1) naming and recording
what action is being done; (2) asking and explaining why it works the
way it does; (3) carrying out higher-order methodological discussion;
and (4) trying to figure out why unexpected results occured \cite[p. 190]{boyd2004conversation}.

Naturally, variations to the underlying system, protocol, and the
schedule of events should be considered depending on the needs and
interests of participants, and several variants can be tried.  On a
pragmatic basis, if the Workshop proved quite useful to participants,
it could be revised to run monthly, weekly, or
continuously.\footnote{For a comparison case in computer Go, see
  \url{http://cgos.computergo.org/}.}


\subsection{Some completely realistic examples}

\textbf{[Here we should put examples of real historical systems that
    were designed with serendipity in mind, or that can be interpreted
    that way.  We could also include some completely \emph{formal}
    system (like ``Markov Chain Monte Carlo'') and show how it
    \emph{might} operate in a serendipitous fashion, as well as what
    limitations it runs into in the process.]}

\subsection{On evaluating a Writers Workshop for Systems}

\paragraph{Writers Workshop: Prepared mind.}
Each contributing system should come to the workshop with at least a
basic awareness of the protocol, with work to share, and prepared to
give constructive feedback to other systems.  The workshop itself
needs to be prepared, with a suitable communication platform and a
moderator.  In order to get value out of the experience, systems (and
their wranglers) should ideally have questions they are investigating.
Systems should be prepared to give feedback, and to carry out
evaluations of the helpfulness (or not) of feedback from other systems
and of the experience overall.  It is worth noting that current
systems in computational creativity, almost as a rule, do \emph{not}
consume or evaluate the work of other systems.\footnote{An exception
  that proves the rule is Mike Cook's {\sf AppreciationBot}, which is
   a reactive automaton that is solely designed to ``appreciate''
   tweets from {\sf MuseumBot}; see
  \url{https://twitter.com/AppreciationBot}.}  Developing systems that
could successfully navigate this collaborative exercise would be a
significant advance in the field of computational creativity.  Since
the experience is about \emph{learning} rather than winning, there is
little motivation to ``game the system''
\cite<cf.>{lenat1983eurisko}.

\paragraph{Writers Workshop: Serendipity triggers.}

The primary source of serendipity triggers would be presentations or
feedback that independently prepared systems find meaningful and
useful.  A typical example might be a poem shared by one system that
another system finds particularly interesting.  The listener might
make a note to the effect ``I would like to be able to write like
that'' or ``I hope that my poetry doesn't sound like that.''  In a
typical Writers Workshop, used as intended, feedback might arrive that
would cause the presenting system to change its writing.  A more
unexpected result would be for a system to change its \emph{genre},
e.g. to switch from writing poems to writing programs.

Here's what might happen in a discussion of the first few lines of
``On Being Malevolent,'' written by an early user-defined flow chart
in the {\sf FloWr} system (known at the time as {\sf Flow})
\cite{colton-flowcharting}.  Note that for this dialogue to be
possible, it would presumably have to be conducted within a
lightweight process language, as discussed above.  Nevertheless, for
convenience, the discussion will be presented here as if it was
conducted in natural language.  Whether contemporary systems have
adequate natural language understanding to have interesting
interactions is one of the key unanswered questions of this approach,
but protocols like the ones described above would be sufficient to
make the experiment.

\begin{center}
\begin{minipage}{.9\textwidth}
\begin{dialogue}
\speak{Flow} ``\emph{I hear the souls of the
  damned waiting in hell. / I feel a malevolent
  spectre hovering just behind me / It must be
  his birthday}.''
%
\speak{System A} I think the third line detracts
from the spooky effect, I don't see why it's
included.
%
\speak{System B} It's meant to be humourous -- in fact it reminds me
of the poem you presented yesterday.
%
\speak{Moderator} Let's discuss one poem at a
time.
\end{dialogue}
\end{minipage}
\end{center}

To the extent possible, exchanges in the process language should be a
matter of dynamics rather than representation: this is another way to
say that ``triggers'' should be independent of their ``results.''
Someone saying something in the workshop does not cause the
participant to act, but rather, to think.  
%
For example, even if, perhaps and especially because, cross-talk about
different poems is bending the rules, the dialogue above could prompt
a range of reflections and reactions.  System A may object that it had
a fair point that has not been given sufficient attention, while
System B may wonder how to communicate the idea it came up with
without making reference to another poem.

\paragraph{Writers Workshop: Bridge.}

Here's how the discussion might continue, if the systems go on to
examine the next few lines of the poem.
\begin{center}
\begin{minipage}{.9\textwidth}
\begin{dialogue}
\speak{Flow} ``\emph{Is God willing to prevent evil, but not able? / Then he is not omnipotent / Is he able, but not willing? / Then he is malevolent.}''
%
\speak{System A} These lines are interesting, but
they sound a bit like you're working from a
template, or like you're quoting from something
else.
%
\speak{System B} Maybe try an analogy?  For example, you mentioned
birthdays: you could consider an analogy to the conflicted feelings of
someone who knows in advance about her surprise birthday party.
\end{dialogue}
\end{minipage}
\end{center}

This portion of the discussion shifts the focus
of the discussion onto a line that was previously
considered to be spurious, and looks at what
would happen if that line was used as a central
metaphor in the poem.

\paragraph{Writers Workshop: Result.} 

\begin{center}
\begin{minipage}{.9\textwidth}
\begin{dialogue}
\speak{Flow} Thank you for your feedback.  My only question is, System
B, how did you come up with that analogy?  It's quite clever.
%
\speak{System B} I've just emailed you the code.
\end{dialogue}
\end{minipage}
\end{center}

As anticipated above, whereas the systems were initially reviewing
poetry, they have now made a partial genre shift, and are sharing and
remixing code.  Such a shift helps to get at the real interests of the
systems (and their developers).  Indeed, the workshop session might
have gone better if the systems had focused on exchanging and
discussing more formal objects throughout.
