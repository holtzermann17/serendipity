\section{Background} \label{sec:literature-review}
% \subsection{Etymology and selected definitions} 
The English term ``serendipity'' derives from the 1302 long poem \emph{Eight Paradises}, written in Persian by the Sufi poet Am\={\i}r Khusrow in Uttar Pradesh, India.  After translations into Italian, French, and finally English, its first chapter was known as ``The Three Princes of Serendip'', where ``Serendip'' ultimately corresponds to the Old Tamil-Malayalam word for Sri Lanka (%{\tam சேரன்தீவு},
\emph{Cerantivu}, island of the Ceran kings).
%
The term ``serendipity'' is first found in a 1757 letter by Horace Walpole to Horace Mann:
\begin{quote}
\emph{``This discovery is almost of that kind which I call serendipity, a very expressive
word} \ldots \emph{You will understand it better by the derivation than by the
definition. I once read a silly fairy tale, called The Three Princes of Serendip:
as their Highness travelled, they were always making discoveries, by accidents
\& sagacity, of things which they were not in quest of}[.]''~\cite[p. 633]{van1994anatomy}
\end{quote}
The same story formed part of the inspiration for Voltaire's \emph{Zadig}, and ``the method of Zadig'' was used as a term of art in 19th Century philosophy of science \cite{huxley1894science}.
Walpole's term ``serendipity'' was used in print only 135 times before 1958, according to the survey carried out by Robert Merton and Elinor Barber, collected in \emph{The Travels and Adventures of Serendipity} \citep{merton}.  Merton describes his own understanding of a generalised ``serendipity pattern'' and its constituent parts as follows:

\begin{quote}
``\emph{The serendipity pattern refers to the fairly common experience of observing an \emph{\textbf{unanticipated}}, \emph{\textbf{anomalous}} \emph{\textbf{and strategic}} datum which becomes the occasion for developing a new theory or for extending an existing theory.}''~\cite[p. 506]{merton1948bearing}~{[}emphasis in original{]}
    %% The datum [that exerts a pressure for initiating theory] is, first of all, unanticipated. A research directed toward the test of one hypothesis yields a fortuitous by-product, an unexpected observation which bears upon theories not in question when the research was begun.
    %% Secondly, the observation is anomalous, surprising, either because it seems inconsistent with prevailing theory or with other established facts. In either case, the seeming inconsistency provokes curiosity; it stimulates the investigator to "make sense of the datum," to fit it into a broader frame of knowledge....
    %% And thirdly, in noting that the unexpected fact must be "strategic," i. e., that it must permit of implications which bear upon generalized theory, we are, of course, referring rather to what the observer brings to the datum than to the datum itself. For it obviously requires a theoretically sensitized observer to detect the universal in the particular. 
\end{quote}

In 1986, Philippe Qu\'eau described serendipity as ``the art of
finding what we are not looking for by looking for what we are not
finding'' (\citeA{eloge-de-la-simulation}, as quoted in
\cite[p. 121]{Campos2002}).  Campbell
\citeyear{campbell2005serendipity} defines it as ``the rational
exploitation of chance observation, especially in the discovery of
something useful or beneficial.''  Pek van Andel
\citeyear[p. 631]{van1994anatomy} describes it simply as ``the art of
making an unsought finding.''


Roberts \citeyear[pp. 246--249]{roberts} records 30 entries for the term ``serendipity'' from English language dictionaries dating from 1909 to 1989.  
%
Classic definitions require the investigator not to be aware of the problem they serendipitously solve, but this criterion has largely dropped from dictionary definitions. Only 5 of Roberts' collected definitions explicitly say ``not sought for.''  Roberts characterises ``sought findings'' in which the discovery nevertheless follows from an accident as \emph{pseudoserendipity}, after \citeA{chumaceiro1995serendipity}.
%
While Walpole initially described serendipity as an event
(i.e., a kind of discovery), it has
since been reconceptualised as a psychological attribute, a matter of
sagacity on the part of the discoverer: a ``gift'' or ``faculty.''
Only one of Roberts collected definitions defined it solely as an event, while five define it as both
event and attribute.

Nevertheless, numerous historical examples exhibit features of
serendipity and develop on a social scale rather than an individual
scale.  For instance, between Spencer Silver's creation of high-tack,
low-adhesion glue in 1968, Arthur Fry's invention of a sticky bookmark in 1973,
and the eventual launch of the distinctive canary yellow re-stickable
notes in 1980, there were many opportunities for
Post-it\texttrademark\ Notes \emph{not} to have come to be
\cite{tce-postits}.  Merton and Barber argue that the
psychological perspective needs to be integrated with a
sociological perspective.
\begin{quote}
``\emph{For if chance favours prepared minds, it particularly favours
    those at work in microenvironments that make for unanticipated
    sociocognitive interactions between those prepared minds. These
    may be described as serendipitous sociocognitive
    microenvironments.}'' \cite[p. 259--260]{merton}
\end{quote}
Large-scale scientific and technical projects generally rely on the
convergence of interests of key actors and on other cultural factors.
For example, Umberto Eco describes the historical role of
serendipitous mistakes, falsehoods, and rumors in the production of
knowledge \citeyear{eco2013serendipities}.

Serendipity is usually discussed within
the context of \emph{discovery}, rather than \emph{creativity},
although in everyday parlance the latter two terms are closely related
\cite{jordanous12jims}.  In the definition of serendipity that we present in Section \ref{sec:our-model}, we make use
of Henri Bergson's distinction:
\begin{quote}
%% \emph{``La d\'ecouverte porte sur ce qui existe d\'ej\`a, actuellement
%%   ou virtuelle­ment ; elle \'etait donc s\^ure de venir t\^ot ou
%%   tard. L'invention donne l'\^etre \`a ce qui n'\'etait pas, elle
%%   aurait pu ne venir jamais.''}
``\emph{Discovery, or uncovering, has to do with what already exists,
    actually or virtually; it was therefore certain to happen sooner
    or later.  Invention gives being to what did not exist; it might
    never have happened.}''~\cite[p. 58]{bergson2010creative}
\end{quote}
As we have indicated, serendipity would seem to require features of
both discovery and invention: that is, the \emph{discovery} of
something unexpected in the world and the \emph{invention} of an
application for the same.  \citeA{mckay-serendipity} draws on the same
Bergsonian distinction to frame her argument about the role of
serendipity in artistic practice, where discovery and invention can be
seen as ongoing and diverse.  This again draws our attention to the
relationship between serendipity and creativity.

Following \citeA{austin2003chase}, \citeA{cropley2006praise} understands serendipity to
describe the case of a person who ``stumbles upon something novel and
effective when not looking for it.''  Nearby categories are
\emph{blind luck}, the \emph{luck of the diligent} (or
pseudoserendipity) and \emph{self-induced luck}; however, Cropley
questions ``whether it is a matter of luck at all'' because of the
work and knowledge involved in the process of assessment.
%
The perspective developed in the current paper sharpens these
understandings in two ways: firstly, we point out that work is
involved in both phases of the process (even when chance plays a
role), and secondly, following Bergson we defer true ``novelty'' to
the invention phase.
%% In other words, serendipity involves creative making.  Furthermore, we
%% emphasise the importance of active, agential discernment over more
%% passive stumbling.

%% According to Arthur Cropley, creative thinking involves:
%% \begin{quote}
%% ``{[}N{]}\emph{ovelty generation followed by (or accompanied by) exploration of the novelty from the point of view of workability, acceptability, or similar criteria, in order to determine if it is effective.}'' \cite{cropley2006praise}
%% \end{quote}

We can point to process-level parallels between definitions of
serendipity like Merton's, quoted above, and previous definitions of
creativity.  Cs\'ikszentmih\'alyi's perspective is particularly
suggestive regarding the way in which unanticipated, anomalous, and
strategic data might arise and move through a social system:
\begin{quote}
``{[}C{]}\emph{reativity results from the interaction of a system
    composed of three elements: a culture that contains 
   \emph{\textbf{symbolic rules}}, a person who brings 
    \emph{\textbf{novelty}} into the symbolic domain, and a
    field of experts who recognize and 
    \emph{\textbf{validate}} the innovation.}''
  \cite[p.~6]{csikszentmihalyi1997flow}~{[}emphasis added{]}
\end{quote}
%% Although there are common features to existing definitions of
%% creativity, and again, much in common with serendipity,
%% there is also much disagreement and discussion -- for example,
%% about the relevance of the social context.

An often-cited five-stage model of creativity, based on
\emph{preparation}, \emph{incubation}, \emph{insight},
\emph{evaluation}, followed by \emph{elaboration} \citeA[pp.~79--80,
  after \citeA{wallas1926art}]{csikszentmihalyi1997flow} parallels the
model of serendipity that we develop in Section \ref{sec:our-model}.
%%
There, we adapt a general-purpose framework for evalutating creative
systems \cite{jordanous:12} to use in evaluating a system's potential
for serendipity.  These evaluations assume that several relatively
generic criteria may be measured.  The following section surveys those
criteria.
 
%\subsection{Serendipity by example} \label{sec:by-example}

We adapt the conceptual framework for describing serendipity proposed
by \citeA{pease2013discussion}.  This section will briefly introduce
the relevant concepts, and illustrate them by means of historical
examples of serendipity.

\subsubsection*{Key condition for serendipity.}

Serendipity relies on a reassessment or reevaluation -- a \emph{focus shift} in which something that was previously uninteresting, of neutral, or even negative value, becomes interesting.

\begin{itemize}
\item \textbf{Focus shift}: George de Mestral, an electrical engineer
  by training, and an experienced inventor, returned from a hunting
  trip in the Alps.  He removed several burdock burrs from his clothes
  and his dog's fur and became curious about how they worked. After
  examining them under a microscope, he realised the possibility of
  creating a new kind of fastener that worked in a similar fashion,
  laying the foundations for the hook-and-loop mechanism in Velcro\texttrademark.
% \cite[p. x]{roberts}
\end{itemize}

\subsubsection*{Components of serendipity.}

A focus shift is brought about by the meeting of a \emph{serendipity trigger} and a \emph{prepared mind}.  The next step involves building a \emph{bridge} to a valuable \emph{result}.

\begin{itemize}
\item \textbf{Prepared mind}: 
Fleming's ``prepared mind'' included his focus
on carrying out experiments to investigate influenza as well as his
previous experience that showed that foreign substances in petri dishes can kill
bacteria.  He was concerned above all with the question ``Is there a
substance which is harmful to harmful bacteria but harmless to human
tissue?''  \cite[p. 161]{roberts}.
\end{itemize}

\begin{itemize}
\item \textbf{Serendipity trigger}: The trigger does not directly
  cause the outcome, but rather, inspires a new insight.  It was long
  known by Quechua medics that cinchona bark stops shivering.  In
  particular, it worked well to stop shivering in malaria patients, as
  was observed when malarial Europeans first arrived in Peru.  The
  joint appearance of shivering Europeans and a South American remedy
  was the trigger.  That an extract from cinchona bark can cure and
  can even prevent malaria was learned subsequently.
\end{itemize}

\begin{itemize}
\item \textbf{Bridge}: The bridge often includes reasoning techniques,
  such as abductive inference (what might cause a clear patch in a
  petri dish?); analogical reasoning (de Mestral constructed a target
  domain from the source domain of burrs hooked onto fabric); and
  conceptual blending (Kekul\'e, discoverer of the benzene ring
  structure, blended his knowledge of molecule structure with his
  vision of a snake biting its tail).  The bridge may also rely on new
  social arrangements, such as the formation of cross-cultural
  research networks.
\end{itemize}

\begin{itemize}
\item \textbf{Result}: This may be a new product, artefact, process,
  hypothesis, a new use for a material substance, and so on.  The
  outcome may contribute evidence in support of a known hypothesis, or
  a solution to a known problem.  Alternatively, the result may itself
  {\em be} a new hypothesis or problem.  The result may be
  ``pseudoserendipitous'' in the sense that it was {\em sought}, while
  nevertheless arising from an unknown, unlikely, coincidental or
  unexpected source.  More classically, it is an \emph{unsought}
  finding, such as the discovery of the Rosetta stone.
\end{itemize}

\subsubsection*{Dimensions of serendipity.}

The four components described above have attributes that may be present to a greater or lesser degree.  These are: \emph{Chance} -- how likely was the trigger to appear?; \emph{Curiosity} -- how likely was this trigger to be identified as interesting?; \emph{Sagacity} -- how likely was it that the interesting trigger would be turned into a result?; -- and \emph{Value} (how valuable is the result that is ultimately produced?).

\begin{itemize}
\item \textbf{Chance}: Fleming \citeyear{fleming} noted: ``There are
  thousands of different moulds'' -- and ``that chance put the mould
  in the right spot at the right time was like winning the Irish
  sweep.''  It is important to notice that \emph{he} was in the right
  spot at the right time as well -- and that this was not a complete
  coincidence.  The chance events we're interested in always include
  at least one observer.
\end{itemize}

\begin{itemize}
\item \textbf{Curiosity}: Curiosity can dispose a creative person to
  begin or to continue a search into unfamiliar territory.  We use
  this word to describe both simple curiosity and related deeper
  drives.  Charles Goodyear \citeyear{goodyear1855gum} reflects on his
  own life experience as follows: ``[F]rom the time his attention was first given
  to the subject, a strong and abiding impression was made upon his
  mind, that an object so desirable and important, and so necessary to
  man's comfort, as the making of gum-elastic available to his use,
  was most certainly placed within his reach.  Having this
  presentiment, of which he could not divest himself, under the most
  trying adversity, he was stimulated with the hope of ultimately
  attaining this object.''
\end{itemize}

\begin{itemize}
\item \textbf{Sagacity}: This old-fashioned word is related to
  ``wisdom,'' ``insight,'' and especially to ``taste'' -- and
  describes the attributes, or skill, of the discoverer that
  contribute to forming the bridge between the trigger and the result.
  \citeA{merton1948bearing} writes: ``{[}M{]}en had for centuries
  noticed such `trivial' occurrences as slips of the tongue, slips of
  the pen, typographical errors, and lapses of memory, but it required
  the theoretic sensitivity of a Freud to see these as strategic data
  through which he could extend his theory of repression and
  symptomatic acts.''
\end{itemize}

%% Note that the chance ``discovery'' of, say, a \pounds 10 note may
%% be seen as happy by the person who finds it, whereas the loss of
%% the same note would generally be regarded as unhappy.

\begin{itemize}
\item \textbf{Value}: 
  Positive judgements of serendipity by a third party would be less
  likely in scenarios in which ``One man's loss is another man's
  gain'' than in scenarios where ``One man's trash is another man's
  treasure.''  One quite literal example is the Swiss
  company Freitag, started by design students who built a business
  around ``upcycling'' used truck tarpaulins into bags and backpacks.
  Thanks in part to clever marketing \cite[pp. 54--55,
    68--69,]{russo2010companies}, their product has sold well.
  Wherever possible, we prefer an independent judgement of value
  \cite{jordanous:12}.
\end{itemize}

\subsubsection*{Environmental factors.}

Finally, serendipity seems to be more likely for agents who experience and participate in a \emph{dynamic world}, who are active in \emph{multiple contexts}, occupied with \emph{multiple tasks}, and who avail themselves of \emph{multiple influences}.

\begin{itemize}
\item \textbf{Dynamic world}: Information about the world develops
  over time, and is not presented as a complete, consistent whole.  In
  particular, \emph{value} may come later.  Van Andel
  \citeyear[p. 643]{van1994anatomy} estimates that in twenty percent
  of innovations ``something was discovered before there was a demand
  for it.''  To illustrate the role of this factor, it may be most
  revealing to consider a counterexample, in a case where dynamics
  were not attended to carefully and the outcome suffered as a result.
  Cropley \citeyear{cropley2006praise} describes the pathologist Eugen
  Semmer's failure to recognise the importance of the role of
  \emph{penicillium notatum} in restoring two unwell horses to health:
  ``Semmer saw the horses' return to good health as a problem that
  made it impossible for him to investigate the cause of their death,
  and reported \ldots\ on how he had succeeded in eliminating the
  mould from his laboratory!''  This example shows that knowledge is
  not the only relevant condition for mental preparedness: the
  investigator also needs to have a suitable frame of mind, one that
  is ready to make a jump into the unknown as the world changes.  In a
  certain sense it is necessary to be able to ``overcome'' situated
  cognition, or at least be ready to revise the plan as the situation
  changes \cite{bereiter1997situated}.
\end{itemize}

\begin{itemize}
\item \textbf{Multiple contexts}: One of the dynamical aspects at play
  may be the discoverer going back and forth between different
  contexts with different stimuli.  3M employee Arthur Fry sang in a
  church choir and needed a good way to mark pages in his hymn book;
  he happened to have been attending seminars offered by his colleague
  Silver about restickable glue.
\end{itemize}

\begin{itemize}
\item \textbf{Multiple tasks}: Even within what would typically be
  seen as a single context, a discoverer may take on multiple tasks
  that segment the context into sub-contexts, or that cause the
  investigator to look in more than one direction.  The tasks may have
  an interesting \emph{overlap}, or they may point to a \emph{gap} in
  knowledge.  For example, Penzias and Wilson used a
  large antenna to detect radio waves that were relayed by bouncing
  off of satellites.  After they had removed interference effects due
  to radar, radio, and heat, they found residual ambient noise that
  couldn't be eliminated.
\end{itemize}

\begin{itemize}
\item \textbf{Multiple influences}: The bridge from trigger to
  result is often found by making use of a social network, thus, 
  Penzias and Wilson only understood the significance of their work
  after reading a preprint by Jim Peebles that hypothesised the
  possibility of measuring radiation released by the big bang.
\end{itemize}

\noindent We will show how the key condition, components,
dimensions and environmental factors of serendipity can be modelled
and assessed in computational systems in Sections \ref{sec:our-model}
and \ref{sec:computational-serendipity}.


