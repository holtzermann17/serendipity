\section{Literature review} \label{sec:literature-review}

\subsection{Etymology and selected definitions} \label{sec:overview-serendipity}  
The English term ``serendipity'' derives from the 1302 long poem \emph{Eight Paradises}, written in Persian by the Sufi poet Am\={\i}r Khusrow in Uttar Pradesh, India.  After translations into Italian, French, and finally English, its first chapter was known as ``The Three Princes of Serendip'', where ``Serendip'' ultimately corresponds to the Old Tamil-Malayalam word for Sri Lanka (%{\tam சேரன்தீவு},
\emph{Cerantivu}, island of the Ceran kings).
%
The term ``serendipity'' is first found in a 1757 letter by Horace Walpole to Horace Mann:
\begin{quote}
\emph{``This discovery is almost of that kind which I call serendipity, a very expressive
word} \ldots \emph{You will understand it better by the derivation than by the
definition. I once read a silly fairy tale, called The Three Princes of Serendip:
as their Highness travelled, they were always making discoveries, by accidents
\& sagacity, of things which they were not in quest of}[.]''~\cite[p. 633]{van1994anatomy}
\end{quote}
The same story formed part of the inspiration for Voltaire's \emph{Zadig}, and ``the method of Zadig'' was used as a term of art in 19th Century philosophy of science \cite{huxley1894science}.
Walpole's term ``serendipity'' became more widely known in the 1940s through studies on the sociology of science by Robert Merton and Elinor Barber, collected in \emph{The Travels and Adventures of Serendipity} \citep{merton}.  Merton describes a generalised ``serendipity pattern'' and its constituent parts:

\begin{quote}
``\emph{The serendipity pattern refers to the fairly common experience of observing an \emph{\textbf{unanticipated}}, \emph{\textbf{anomalous}} \emph{\textbf{and strategic}} datum which becomes the occasion for developing a new theory or for extending an existing theory.}''~\cite[p. 506]{merton1948bearing}~{[}emphasis in original{]}
    %% The datum [that exerts a pressure for initiating theory] is, first of all, unanticipated. A research directed toward the test of one hypothesis yields a fortuitous by-product, an unexpected observation which bears upon theories not in question when the research was begun.
    %% Secondly, the observation is anomalous, surprising, either because it seems inconsistent with prevailing theory or with other established facts. In either case, the seeming inconsistency provokes curiosity; it stimulates the investigator to "make sense of the datum," to fit it into a broader frame of knowledge....
    %% And thirdly, in noting that the unexpected fact must be "strategic," i. e., that it must permit of implications which bear upon generalized theory, we are, of course, referring rather to what the observer brings to the datum than to the datum itself. For it obviously requires a theoretically sensitized observer to detect the universal in the particular. 
\end{quote}

In 1986, Philippe Qu\'eau described serendipity as ``the art of
finding what we are not looking for by looking for what we are not
finding'' (\citeA{eloge-de-la-simulation}, as quoted in
\cite[p. 121]{Campos2002}).  Campbell
\citeyear{campbell2005serendipity} defines it as ``the rational
exploitation of chance observation, especially in the discovery of
something useful or beneficial.''  Pek van Andel
\citeyear[p. 631]{van1994anatomy} describes it simply as ``the art of
making an unsought finding.''


Roberts \citeyear[pp. 246--249]{roberts} records 30 entries for the term ``serendipity'' from English language dictionaries dating from 1909 to 1989.  
%
Classic definitions require the investigator not to be aware of the problem they serendipitously solve, but this criterion has largely dropped from dictionary definitions. Only 5 of Roberts' collected definitions explicitly say ``not sought for.''  Roberts characterises ``sought findings'' in which the discovery nevertheless follows from an accident as \emph{pseudoserendipity}, after \citeA{chumaceiro1995serendipity}.
%
While Walpole initially described serendipity as an event
(i.e., a kind of discovery), it has
since been reconceptualised as a psychological attribute, a matter of
sagacity on the part of the discoverer: a ``gift'' or ``faculty.''
Only one of Roberts collected definitions defined it solely as an event, while five define it as both
event and attribute.

Nevertheless, numerous historical examples exhibit features of
serendipity and develop on a social scale rather than an individual
scale.  For instance, between Spencer Silver's creation of high-tack,
low-adhesion glue in 1968, Arthur Fry's invention of a sticky bookmark in 1973,
and the eventual launch of the distinctive canary yellow re-stickable
notes in 1980, there were many opportunities for
Post-it\texttrademark\ Notes \emph{not} to have come to be
\cite{tce-postits}.  Merton and Barber argue that the
psychological perspective needs to be integrated with a
sociological perspective.\footnote{ ``For if chance favours prepared
  minds, it particularly favours those at work in microenvironments
  that make for unanticipated sociocognitive interactions between
  those prepared minds. These may be described as serendipitous
  sociocognitive microenvironments'' \cite[p. 259--260]{merton}.}
Large-scale scientific and technical projects generally rely on the
convergence of interests of key actors and on other cultural factors.
For example, Umberto Eco describes the
historical role of serendipitous mistakes and falsehoods in the
production of knowledge \citeyear{eco2013serendipities}.

It is important to note that serendipity is usually discussed within
the context of \emph{discovery}, rather than \emph{creativity},
although in everyday parlance these terms are closely related
\cite{jordanous12jims}.  In the definition of serendipity that we present in Section \ref{sec:our-model}, we make use
of Henri Bergson's distinction:
\begin{quote}
%% \emph{``La d\'ecouverte porte sur ce qui existe d\'ej\`a, actuellement
%%   ou virtuelle­ment ; elle \'etait donc s\^ure de venir t\^ot ou
%%   tard. L'invention donne l'\^etre \`a ce qui n'\'etait pas, elle
%%   aurait pu ne venir jamais.''}
``\emph{Discovery, or uncovering, has to do with what already exists,
    actually or virtually; it was therefore certain to happen sooner
    or later.  Invention gives being to what did not exist; it might
    never have happened.}''~\cite[p. 58]{bergson2010creative}
\end{quote}
As we have indicated, serendipity would seem to require features of
both discovery and invention: that is, the discovery of something
unexpected and the invention of an application for the same.  Both
processes can be seen as ongoing and diverse, which underscores the
relationship between serendipity and creativity.  According to Arthur
Cropley, creative thinking involves:
\begin{quote}
``{[}N{]}\emph{ovelty generation followed by (or accompanied by) exploration of the novelty from the point of view of workability, acceptability, or similar criteria, in order to determine if it is effective.}'' \cite{cropley2006praise}
\end{quote}
Following \cite{austin2003chase}, Cropley understands serendipity to
describe the case of a person who ``stumbles upon something novel and
effective when not looking for it.''  Nearby categories are
\emph{blind luck}, the \emph{luck of the diligent} (or
pseudoserendipity) and \emph{self-induced luck}; however, Cropley
questions ``whether it is a matter of luck at all'' because of the
work and knowledge involved in the process of assessment.
%
The perspective developed here sharpens these understandings in two ways:
firstly, we point out that work is involved in both phases of the process (even when chance plays a role), and secondly, following Bergson we defer true ``novelty'' to the invention phase.
%% In other words, serendipity involves creative making.  Furthermore, we
%% emphasise the importance of active, agential discernment over more
%% passive stumbling.

We can point to several process-level parallels between
established definitions of serendipity, like Merton's, and previous
definitions of creativity, like Cropley's, or Cs\'ikszentmih\'alyi's:
\begin{quote}
``{[}C{]}\emph{reativity results from the interaction of a system
    composed of three elements: a culture that contains symbolic
    rules, a person who brings novelty into the symbolic domain, and a
    field of experts who recognize and validate the innovation.}''
  \cite[p.~6]{csikszentmihalyi1997flow}
\end{quote}
Although there are common features to existing definitions of
creativity, and again, much in common with serendipity,
there is also much disagreement and discussion -- for example,
about the relevance of the social context. 
%
Without proposing an equivalence between serendipity and creativity,
in Section \ref{sec:our-model} we adapt a general-purpose framework
for evalutating creative systems to use when evaluating serendipity.  These
evaluations assume that several relatively generic criteria may be
measured.  This approach may help to side-step some of the contention that
exists around the concept of creativity.
The following section surveys the relevant criteria.
 
%\subsection{Serendipity by example} \label{sec:by-example}

We adapt the conceptual framework for describing serendipity proposed
by \citeA{pease2013discussion}.  This section will briefly introduce
the relevant concepts, and illustrate them by means of historical
examples of serendipity.

\subsubsection*{Key condition for serendipity.}

Serendipity relies on a reassessment or reevaluation -- a \emph{focus shift} in which something that was previously uninteresting, of neutral, or even negative value, becomes interesting.

\begin{itemize}
\item \textbf{Focus shift}: George de Mestral, an electrical engineer
  by training, and an experienced inventor, returned from a hunting
  trip in the Alps.  He removed several burdock burrs from his clothes
  and his dog's fur and became curious about how they worked. After
  examining them under a microscope, he realised the possibility of
  creating a new kind of fastener that worked in a similar fashion,
  laying the foundations for the hook-and-loop mechanism in Velcro\texttrademark.
% \cite[p. x]{roberts}
\end{itemize}

\subsubsection*{Components of serendipity.}

A focus shift is brought about by the meeting of a \emph{serendipity trigger} and a \emph{prepared mind}.  The next step involves building a \emph{bridge} to a valuable \emph{result}.

\begin{itemize}
\item \textbf{Prepared mind}: 
Fleming's ``prepared mind'' included his focus
on carrying out experiments to investigate influenza as well as his
previous experience that showed that foreign substances in petri dishes can kill
bacteria.  He was concerned above all with the question ``Is there a
substance which is harmful to harmful bacteria but harmless to human
tissue?''  \cite[p. 161]{roberts}.
\end{itemize}

\begin{itemize}
\item \textbf{Serendipity trigger}: The trigger does not directly
  cause the outcome, but rather, inspires a new insight.  It was long
  known by Quechua medics that cinchona bark stops shivering.  In
  particular, it worked well to stop shivering in malaria patients, as
  was observed when malarial Europeans first arrived in Peru.  The
  joint appearance of shivering Europeans and a South American remedy
  was the trigger.  That an extract from cinchona bark can cure and
  can even prevent malaria was learned subsequently.
\end{itemize}

\begin{itemize}
\item \textbf{Bridge}: The bridge often includes reasoning techniques,
  such as abductive inference (what might cause a clear patch in a
  petri dish?); analogical reasoning (de Mestral constructed a target
  domain from the source domain of burrs hooked onto fabric); and
  conceptual blending (Kekul\'e, discoverer of the benzene ring
  structure, blended his knowledge of molecule structure with his
  vision of a snake biting its tail).  The bridge may also rely on new
  social arrangements, such as the formation of cross-cultural
  research networks.
\end{itemize}

\begin{itemize}
\item \textbf{Result}: This may be a new product, artefact, process,
  hypothesis, a new use for a material substance, and so on.  The
  outcome may contribute evidence in support of a known hypothesis, or
  a solution to a known problem.  Alternatively, the result may itself
  {\em be} a new hypothesis or problem.  The result may be
  ``pseudoserendipitous'' in the sense that it was {\em sought}, while
  nevertheless arising from an unknown, unlikely, coincidental or
  unexpected source.  More classically, it is an \emph{unsought}
  finding, such as the discovery of the Rosetta stone.
\end{itemize}

\subsubsection*{Dimensions of serendipity.}

The four components described above have attributes that may be present to a greater or lesser degree.  These are: \emph{Chance} -- how likely was the trigger to appear?; \emph{Curiosity} -- how likely was this trigger to be identified as interesting?; \emph{Sagacity} -- how likely was it that the interesting trigger would be turned into a result?; -- and \emph{Value} (how valuable is the result that is ultimately produced?).

\begin{itemize}
\item \textbf{Chance}: Fleming \citeyear{fleming} noted: ``There are
  thousands of different moulds'' -- and ``that chance put the mould
  in the right spot at the right time was like winning the Irish
  sweep.''  It is important to notice that \emph{he} was in the right
  spot at the right time as well -- and that this was not a complete
  coincidence.  The chance events we're interested in always include
  at least one observer.
\end{itemize}

\begin{itemize}
\item \textbf{Curiosity}: Curiosity can dispose a creative person to
  begin or to continue a search into unfamiliar territory.  We use
  this word to describe both simple curiosity and related deeper
  drives.  Charles Goodyear \citeyear{goodyear1855gum} reflects on his
  own life experience as follows: ``[F]rom the time his attention was first given
  to the subject, a strong and abiding impression was made upon his
  mind, that an object so desirable and important, and so necessary to
  man's comfort, as the making of gum-elastic available to his use,
  was most certainly placed within his reach.  Having this
  presentiment, of which he could not divest himself, under the most
  trying adversity, he was stimulated with the hope of ultimately
  attaining this object.''
\end{itemize}

\begin{itemize}
\item \textbf{Sagacity}: This old-fashioned word is related to
  ``wisdom,'' ``insight,'' and especially to ``taste'' -- and
  describes the attributes, or skill, of the discoverer that
  contribute to forming the bridge between the trigger and the result.
  \citeA{merton1948bearing} writes: ``{[}M{]}en had for centuries
  noticed such `trivial' occurrences as slips of the tongue, slips of
  the pen, typographical errors, and lapses of memory, but it required
  the theoretic sensitivity of a Freud to see these as strategic data
  through which he could extend his theory of repression and
  symptomatic acts.''
\end{itemize}

%% Note that the chance ``discovery'' of, say, a \pounds 10 note may
%% be seen as happy by the person who finds it, whereas the loss of
%% the same note would generally be regarded as unhappy.

\begin{itemize}
\item \textbf{Value}: 
  Positive judgements of serendipity by a third party would be less
  likely in scenarios in which ``One man's loss is another man's
  gain'' than in scenarios where ``One man's trash is another man's
  treasure.''  One quite literal example is the Swiss
  company Freitag, started by design students who built a business
  around ``upcycling'' used truck tarpaulins into bags and backpacks.
  Thanks in part to clever marketing \cite[pp. 54--55,
    68--69,]{russo2010companies}, their product has sold well.
  Wherever possible, we prefer an independent judgement of value
  \cite{jordanous:12}.
\end{itemize}

\subsubsection*{Environmental factors.}

Finally, serendipity seems to be more likely for agents who experience and participate in a \emph{dynamic world}, who are active in \emph{multiple contexts}, occupied with \emph{multiple tasks}, and who avail themselves of \emph{multiple influences}.

\begin{itemize}
\item \textbf{Dynamic world}: Information about the world develops
  over time, and is not presented as a complete, consistent whole.  In
  particular, \emph{value} may come later.  Van Andel
  \citeyear[p. 643]{van1994anatomy} estimates that in twenty percent
  of innovations ``something was discovered before there was a demand
  for it.''  To illustrate the role of this factor, it may be most
  revealing to consider a counterexample, in a case where dynamics
  were not attended to carefully and the outcome suffered as a result.
  Cropley \citeyear{cropley2006praise} describes the pathologist Eugen
  Semmer's failure to recognise the importance of the role of
  \emph{penicillium notatum} in restoring two unwell horses to health:
  ``Semmer saw the horses' return to good health as a problem that
  made it impossible for him to investigate the cause of their death,
  and reported \ldots\ on how he had succeeded in eliminating the
  mould from his laboratory!''  This example shows that knowledge is
  not the only relevant condition for mental preparedness: the
  investigator also needs to have a suitable frame of mind, one that
  is ready to make a jump into the unknown as the world changes.  In a
  certain sense it is necessary to be able to ``overcome'' situated
  cognition, or at least be ready to revise the plan as the situation
  changes \cite{bereiter1997situated}.
\end{itemize}

\begin{itemize}
\item \textbf{Multiple contexts}: One of the dynamical aspects at play
  may be the discoverer going back and forth between different
  contexts with different stimuli.  3M employee Arthur Fry sang in a
  church choir and needed a good way to mark pages in his hymn book;
  he happened to have been attending seminars offered by his colleague
  Silver about restickable glue.
\end{itemize}

\begin{itemize}
\item \textbf{Multiple tasks}: Even within what would typically be
  seen as a single context, a discoverer may take on multiple tasks
  that segment the context into sub-contexts, or that cause the
  investigator to look in more than one direction.  The tasks may have
  an interesting \emph{overlap}, or they may point to a \emph{gap} in
  knowledge.  For example, Penzias and Wilson used a
  large antenna to detect radio waves that were relayed by bouncing
  off of satellites.  After they had removed interference effects due
  to radar, radio, and heat, they found residual ambient noise that
  couldn't be eliminated.
\end{itemize}

\begin{itemize}
\item \textbf{Multiple influences}: The bridge from trigger to
  result is often found by making use of a social network, thus, 
  Penzias and Wilson only understood the significance of their work
  after reading a preprint by Jim Peebles that hypothesised the
  possibility of measuring radiation released by the big bang.
\end{itemize}

\noindent We will show how the key condition, components,
dimensions and environmental factors of serendipity can be modelled
and assessed in computational systems in Sections \ref{sec:our-model}
and \ref{sec:computational-serendipity}.


\subsection{Serendipity by example: the condition, components, dimensions, and factors of serendiptious occurrences} \label{sec:by-example}

% This section introduces key concepts for understanding serendipitous occurrences, and illustrates them by means of historical examples.
%% We adapt the conceptual framework proposed by
%% \citeA{pease2013discussion}.

\subsubsection{Key condition for serendipity.}

Serendipity relies on a reassessment or reevaluation -- a \emph{focus shift} in which something that was previously uninteresting, of neutral, or even negative value, becomes interesting.

\begin{itemize}
\item \textbf{Focus shift}: George de Mestral, an electrical engineer
  by training, and an experienced inventor, returned from a hunting
  trip in the Alps.  He removed several burdock burrs from his clothes
  and his dog's fur and became curious about how they worked. After
  examining them under a microscope, he realised the possibility of
  creating a new kind of fastener that worked in a similar fashion,
  laying the foundations for the hook-and-loop mechanism in Velcro\texttrademark.
% \cite[p. x]{roberts}
\end{itemize}

\subsubsection{Components of serendipity.}

A focus shift is brought about by the meeting of a \emph{serendipity trigger} and a \emph{prepared mind}.  The next step involves building a \emph{bridge} to a valuable \emph{result}.

\begin{itemize}
\item \textbf{Prepared mind}: 
Fleming's ``prepared mind'' included his focus
on carrying out experiments to investigate influenza as well as his
previous experience that showed that foreign substances in petri dishes can kill
bacteria.  He was concerned above all with the question ``Is there a
substance which is harmful to harmful bacteria but harmless to human
tissue?''  \cite[p. 161]{roberts}.
\end{itemize}

\begin{itemize}
\item \textbf{Serendipity trigger}: The trigger does not directly
  cause the outcome, but rather, inspires a new insight.  It was long
  known by Quechua medics that cinchona bark stops shivering.  In
  particular, it worked well to stop shivering in malaria patients, as
  was observed when malarial Europeans first arrived in Peru.  The
  joint appearance of shivering Europeans and a South American remedy
  was the trigger.  That an extract from cinchona bark can cure and
  can even prevent malaria was learned subsequently.
\end{itemize}

\begin{itemize}
\item \textbf{Bridge}: The bridge often includes reasoning techniques,
  such as abductive inference (what might cause a clear patch in a
  petri dish?); analogical reasoning (de Mestral constructed a target
  domain from the source domain of burrs hooked onto fabric); and
  conceptual blending (Kekul\'e, discoverer of the benzene ring
  structure, blended his knowledge of molecule structure with his
  vision of a snake biting its tail).  The bridge may be
  non-conceptual, relying on new social arrangements, or physical
  prototypes.  It may have many steps, and, like the trigger, it may
  feature chance elements.  Several serendipitous episodes may be
  chained together in sequence, on the way to an unprecedented result.
  C\'edric Villani \citeyear[p.~16]{birth-of-a-theorem} describes a hallway
  conversation with his colleague \'Etienne Ghys, who said ``I didn't
  really want to say anything, C\'edric, but those figures there on
  the board -- I've seen them before.''
\end{itemize}

\begin{itemize}
\item \textbf{Result}: This is the new product, artefact, process,
  theory, use for a material substance, or other outcome.  The
  outcome may contribute evidence in support of a known hypothesis, or
  a solution to a known problem.  Alternatively, the result may itself
  {\em be} a new hypothesis or problem.  The result may be
  ``pseudoserendipitous'' in the sense that it was {\em sought}, while
  nevertheless arising from an unknown, unlikely, coincidental or
  unexpected source.  More classically, it is an \emph{unsought}
  finding, such as the discovery of the Rosetta stone.
\end{itemize}

\subsubsection{Dimensions of serendipity.}

The four components described above have attributes that may be present to a greater or lesser degree.  These are: \emph{Chance} -- how likely was the trigger to appear?; \emph{Curiosity} -- how likely was this trigger to be identified as interesting?; \emph{Sagacity} -- how likely was it that the interesting trigger would be turned into a result?; -- and \emph{Value} (how valuable is the result that is ultimately produced?).

\begin{itemize}
\item \textbf{Chance}: Fleming \citeyear{fleming} noted: ``There are
  thousands of different moulds'' -- and ``that chance put the mould
  in the right spot at the right time was like winning the Irish
  sweep.''  It is important to notice that \emph{he} was in the right
  spot at the right time as well -- and that this was not a complete
  coincidence.  The chance events we're interested in always include
  at least one observer.
\end{itemize}

\begin{itemize}
\item \textbf{Curiosity}: Curiosity can dispose a creative person to
  begin or to continue a search into unfamiliar territory.  We use
  this word to describe both simple curiosity and related deeper
  drives.  Charles Goodyear \citeyear{goodyear1855gum} reflects on his
  own life experience as follows: ``[F]rom the time his attention was first given
  to the subject, a strong and abiding impression was made upon his
  mind, that an object so desirable and important, and so necessary to
  man's comfort, as the making of gum-elastic available to his use,
  was most certainly placed within his reach.  Having this
  presentiment, of which he could not divest himself, under the most
  trying adversity, he was stimulated with the hope of ultimately
  attaining this object.''
\end{itemize}

\begin{itemize}
\item \textbf{Sagacity}: This old-fashioned word is related to
  ``wisdom,'' ``insight,'' and especially to ``taste'' -- and
  describes the attributes, or skill, of the discoverer that
  contribute to forming the bridge between the trigger and the result.
  \citeA{merton1948bearing} writes: ``{[}M{]}en had for centuries
  noticed such `trivial' occurrences as slips of the tongue, slips of
  the pen, typographical errors, and lapses of memory, but it required
  the theoretic sensitivity of a Freud to see these as strategic data
  through which he could extend his theory of repression and
  symptomatic acts.''
\end{itemize}

%% Note that the chance ``discovery'' of, say, a \pounds 10 note may
%% be seen as happy by the person who finds it, whereas the loss of
%% the same note would generally be regarded as unhappy.

\begin{itemize}
\item \textbf{Value}: Serendipity concerns happy surprises, but
  readings of a given situation as ``happy'' or ``surprising'' may be
  different for different parties.  A third party judgement of value
  can help to discriminate between mere luck and actual value
  creation.  Consider the difference between the two sayings ``One
  man's loss is another man's gain'' and ``One man's trash is another
  man's treasure.''  In the first case, it is unlikely that new value
  has been created, whereas the second case evokes a non-zero sum.
  A literal example of this second scenario is provided by the
  Swiss company Freitag, which was started by design students who
  built a business around ``upcycling'' used truck tarpaulins into
  bags and backpacks.  Thanks in part to clever marketing
  \cite[pp. 54--55, 68--69,]{russo2010companies}, their product is now
  a global brand.  Wherever possible, we prefer to make use of an
  independent judgement of value \cite{jordanous:12}.
\end{itemize}

\subsubsection{Environmental factors.}

Finally, serendipity seems to be more likely for agents who experience and participate in a \emph{dynamic world}, who are active in \emph{multiple contexts}, occupied with \emph{multiple tasks}, and who avail themselves of \emph{multiple influences}.

\begin{itemize}
\item \textbf{Dynamic world}: Information about the world develops
  over time, and is not presented as a complete, consistent whole.  In
  particular, \emph{value} may come later.  Van Andel
  \citeyear[p. 643]{van1994anatomy} estimates that in twenty percent
  of innovations ``something was discovered before there was a demand
  for it.''  To illustrate the role of this factor, it may be most
  revealing to consider a counterexample, in a case where dynamics
  were not attended to carefully and the outcome suffered as a result.
  Cropley \citeyear{cropley2006praise} describes the pathologist Eugen
  Semmer's failure to recognise the importance of the role of
  \emph{penicillium notatum} in restoring two unwell horses to health:
  ``Semmer saw the horses' return to good health as a problem that
  made it impossible for him to investigate the cause of their death,
  and reported \ldots\ on how he had succeeded in eliminating the
  mould from his laboratory!''  This example shows that knowledge is
  not the only relevant condition for mental preparedness: the
  investigator also needs to have a suitable frame of mind, one that
  is ready to make a jump into the unknown as the world changes.  In a
  certain sense it is necessary to be able to ``overcome'' situated
  cognition, or at least be ready to revise the approach as the
  situation changes \cite{bereiter1997situated}.
\end{itemize}

\begin{itemize}
\item \textbf{Multiple contexts}: One of the dynamical aspects at play
  may be the discoverer going back and forth between different
  contexts with different stimuli.  3M employee Arthur Fry sang in a
  church choir and needed a good way to mark pages in his hymn book;
  he happened to have been attending seminars offered by his colleague
  Silver about restickable glue.
\end{itemize}

\begin{itemize}
\item \textbf{Multiple tasks}: Even within what would typically be
  seen as a single context, a discoverer may take on multiple tasks
  that segment the context into sub-contexts, or that cause the
  investigator to look in more than one direction.  The tasks may have
  an interesting \emph{overlap}, or they may point to a \emph{gap} in
  knowledge.  For example, Penzias and Wilson used a
  large antenna to detect radio waves that were relayed by bouncing
  off of satellites.  After they had removed interference effects due
  to radar, radio, and heat, they found residual ambient noise that
  couldn't be eliminated.
\end{itemize}

\begin{itemize}
\item \textbf{Multiple influences}: The bridge from trigger to
  result is often found by making use of a social network, thus, 
  Penzias and Wilson only understood the significance of their work
  after reading a preprint by Jim Peebles that hypothesised the
  possibility of measuring radiation released by the big bang.
\end{itemize}

\noindent We will show how the key condition, components,
dimensions and environmental factors of serendipity can be modelled
and assessed in computational systems in Sections \ref{sec:our-model}
and \ref{sec:computational-serendipity}.

% \subsection{Related work} \label{sec:related}

An active research community investigating computational models of serendipity exists in the field of information retrieval, and specifically, in recommender systems \cite{Toms2000}. In this domain, \citeA{Herlocker2004} and \citeA{McNee2006} view serendipity as an important factor for user satisfaction, alongside accuracy and diversity.  Serendipity in recommendations is
understood to imply that the system suggests \emph{unexpected} items, which the user considers to be \emph{useful}, \emph{interesting}, \emph{attractive} or \emph{relevant}. 
% \cite{Herlocker2004} \cite{Lu2012},\cite{Ge2010}.  
Definitions differ as to the requirement of \emph{novelty}; \citeA{Adamopoulos2011}, for example, describe systems that suggest items that may already be known, but are still unexpected in the current context.  While standardised measures such as the $F_1$-score or the (R)MSE are used to determine the \emph{accuracy} of a recommendation (i.e.~whether the recommended item is very close to what the user is already known to prefer), there is no common agreement on a measure for serendipity yet, although there are several proposals \cite{Murakami2008, Adamopoulos2011, McCay-Peet2011,iaquinta2010can}.
In terms of our model, these systems focus mainly on producing a \emph{serendipity trigger} and predicting the potential for serendipitous \emph{discovery} on the side of the user.  Intelligent user modeling could bring other components of serendipity into play, as we will discuss in Section \ref{sec:computational-serendipity}.

Recent work has examined related topics of \emph{curiosity}
\cite{wu2013curiosity} and \emph{surprise} \cite{grace2014using} in
computing.  This latter work seeks to ``adopt methods from the field
of computational creativity [$\ldots$] to the generation of scientific
hypotheses.''  This is an example of an effort focused on
computational \emph{invention}.

Paul Andr{\'e} et al.~\citeyear{andre2009discovery} have examined
serendipity from a design perspective.  Like us, these authors
proposed a two-part model encompassing ``the chance encountering of
information, and the sagacity to derive insight from the encounter.''
According to Andr\'e et al., the first phase is the one that has most
frequently been automated -- but they suggest that computational
systems should be developed that support both aspects.  They
specifically suggest to focus on representational features:
\emph{domain expertise} and a \emph{common language model}.

These features seem to exemplify
aspects of the \emph{prepared mind}.  However, as we mentioned above,
the \emph{bridge} is a distinct process that mental preparation can
support, but that it does not necessarily fully determine.  For example, participants in
a poetry workshop may possess a very limited understanding of each
other's aims or of the work they are critiquing, and may as a
consequence talk past one another to a greater or lesser degree --
while nevertheless finding the overall process of participating in the
workshop illuminating and rewarding (often precisely because
such misunderstandings elucidate poor communication choices!).
Various social strategies, ranging from Writers Workshops to open
source software, pair programming, and design charettes
\cite[p. 11]{gabriel2002writer} have been developed to exploit similar
emergent effects and to develop \emph{new} shared language.  We have
recently investigated the feasibility of using
designs of this sort in multi-agent systems that learn by sharing and
discussing partial understandings \cite{corneli2015computational,corneli2015feedback}.

\citeA{robot-rendezvous} develop a discussion of serendipitous
rendezvous in a multi-agent system for a graph exploration problem, in
which ``[h]aving more data about their colleagues, better decisions
are made about the potential serendipity path.''  This has some
similarity to the discursive scenario described above, and shows that
\emph{asymmetric partial knowledge} can support serendipitious
findings.  These examples suggest that a distinction between emergent
knowledge of other actors and knowledge about an underlying domain may
be useful -- although the distinction may be less relevant if
the underlying domain itself has dynamic and emergent features.
\emph{Social coordination} among human users of information systems is
a current research topic. \citeA{rubin2010everyday} point out that
naive end users often \emph{talk about} serendiptious occurrences,
which presents another route for research and evaluation.

The {\sf SerenA} system developed by Deborah Maxwell et al.~\citeyear{maxwell2012designing} offers a case study of several of the points discussed above.
This system is designed to support serendipitous discovery for its (human) users
\cite{forth2013serena}.  The authors rely on a process-based model of
serendipity \cite{Makri2012,Makri2012a} that is derived from user
studies, including interviews with 28 researchers, looking for
instances of serendipity from both their personal and professional
lives.  This material was coded along three dimensions:
\emph{unexpectedness}, \emph{insightfulness}, and \emph{value}.  This
research aims to support the process of forming bridging connections
from an unexpected encounter to a previously unanticipated but valuable
outcome.  The theory focuses on the acts of \emph{reflection}
that foment both the creation of a bridge and estimates of the
potential value of the result.
%
While this description touches on all of the features of our model, {\sf
  SerenA} largely matches the description offered by Andr{\'e} et
al.~\citeyear{andre2009discovery} of discovery-focused systems, in which
the user experiences an ``aha'' moment and takes the
creative steps to realise the result.  {\sf SerenA}'s primary computational method is to
search outside of the normal search parameters in order to engineer
potentially serendipitous (or at least pseudo-serendipitous)
encounters.
%% Another
%% earlier related example of this sort of system is {\sf Max}, created
%% by Figueiredo and Campos \citeyear{Campos2002}.  The user emailed {\sf
%%   Max} with an existing list of interests and {\sf Max} would return a
%% web page that might also be of interest.  Other systems with similar
%% support for serendipitous discovery involve searching for analogies
%% \cite{Donoghue2002,Donoghue2012}) as well as content \cite{Iaquinta2008}.

In recent joint work \cite{colton-assessingprogress}, we presented a
diagrammatic formalism for evaluating progress in computational
creativity.  It is useful to ask what serendipity would add to this
formalism, and how the result compares with other attempts to
formalise serendipity, notably Figueiredo and Campos's
\citeyear{Figueiredo2001} `Serendipity Equations'.  
Figueiredo and Campos describe serendipitous ``moves'' from one
problem to another, which transform a problem that cannot be solved
into one that can.  In our diagrammatic formalism, we spoke about
progress with \emph{systems} rather than with \emph{problems}.  It
would be a useful generalisation of the formalism -- and not just a
simple relabelling -- for it to be able to tackle problems as well.
However, it is important to notice that progress with problems does not always mean transforming a
problem that cannot be solved into one that can.  Progress may also
apply to growth in the ability to \emph{posit} problems.  In keeping
track of progress, it would be useful for system designers to record
(or get their systems to record) what problem a given system solves,
and the degree to which the computer was responsible for coming up
with this problem.

As \citeA[p. 69]{pease2013discussion} remark, anomaly detection and
outlier analysis are part of the standard machine learning toolkit --
but recognising \emph{new} patterns and defining \emph{new} problems
is more ambitious.  Establishing complex analogies between evolving problems and
solutions is one of the key strategies used by teams of human designers
\cite{Analogical-problem-evolution-DCC}.  Kazjon Grace
\citeyear{kaz-thesis} presents a computational model of the creation
of new concepts and interpretations, but this work did include the
ability to create new higher order relationships necessary for complex
analogies.  New patterns and higher-order analogies were considered in
Hofstadter and Mitchell's {\sf Copycat} and the subsequent {\sf
  Metacat}, but these systems operated in a simple and fairly abstract
``microdomain''
\cite{hofstadter1994copycat,DBLP:journals/jetai/Marshall06}.  %% More
%% recent work in this tradition is surveyed in
%% \cite{eric-nichols-thesis}.

The relationship between serendipity and novel problems receives
considerable attention in the current work, since we want to
increasingly turn over responsibility for creating and maintaining a
prepared mind to the machine.


\subsection{Related work} \label{sec:related}

Models of serendipity have appeared in the field of information retrieval \cite{foster2003serendipity} and recommender systems \cite{Toms2000}.  \citeA{Herlocker2004} and \citeA{McNee2006} view serendipity as an important component of recommendation quality, alongside accuracy and diversity.  Serendipity in recommendations is
understood to imply that the system suggests \emph{unexpected} items, which the user considers to be \emph{useful}, \emph{interesting}, \emph{attractive} or \emph{relevant}. 
% \cite{Herlocker2004} \cite{Lu2012},\cite{Ge2010}.  
Definitions differ as to the requirement of \emph{novelty}; \citeA{Adamopoulos2011}, for example, describe systems that suggest items that may already be known, but are still unexpected in the current context.  While standardised measures such as the $F_1$-score or the (R)MSE are used to determine the \emph{accuracy} of a recommendation (i.e.~whether the recommended item is very close to what the user is already known to prefer), there is no common agreement on a measure for serendipity yet, although there are several proposals
\cite{Murakami2008,Adamopoulos2011,McCay-Peet2011,iaquinta2010can}.
In terms of the framework from Section \ref{sec:by-example}, these systems focus mainly on generating a \emph{serendipity trigger} for the user, and preparing the ground for serendipitous \emph{discovery}.  Intelligent modelling approaches could bring other components of serendipity into play in future systems, as we discuss in Section \ref{sec:computational-serendipity}.

Recent work has examined related topics of \emph{curiosity}
\cite{wu2013curiosity} and \emph{surprise} \cite{grace2014using} in
computing.  The latter work seeks to ``adopt methods from the field
of computational creativity [$\ldots$] to the generation of scientific hypotheses.''  This is a useful example of an effort focused on computational \emph{invention}.  
%
As we indicated earlier, creativity and serendipity are often discussed in similar ways.
A further terminological clarification is warranted.
The word \emph{creative} can be used to describe a ``creative output'',
a ``creative person'', or even a ``creative method.''
On the understanding developed here, serendipity is only meaningfully attributed to a particular kind of process.
It is not a property of a generated artefact (like novelty or usefulness), nor is it a system trait (like skill or autonomy).
% This is why we speak of potential for serendipity, and instances of serendipity.

Paul Andr{\'e} et al.~\citeyear{andre2009discovery} have previously proposed a
two-part model of serendipity encompassing ``the chance encountering
of information, and the sagacity to derive insight from the
encounter.''  The first phase has been automated more frequently --
but these authors suggest that computational systems should be
developed that support both aspects.  They specifically suggest to
pursue this work by developing systems with better representational
features: \emph{domain expertise} and a \emph{common language model}.

These features seem to exemplify aspects of the \emph{prepared mind}.
However, as we mentioned above, the \emph{bridge} is a distinct step
in the process that preparation can support, but not always fully
determine.  Domain understanding is not always a precondition; it can
be emergent.  For instance, persons involved in a dialogue may
understand each other quite poorly, while nevertheless finding the
conversation interesting and rewarding.  Misunderstandings can present
learning opportunities.  Various social strategies, ranging from
Writers Workshops, to open source software, to group therapy have been
developed to exploit similar emergent effects and to develop
\emph{new} shared language \cite{gabriel2002writer,seikkula2014open}.
Inspired by these examples, we have investigated the feasibility of
building multi-agent systems that learn by sharing and discussing
partial understandings
\cite{corneli2015computational,corneli2015feedback}.

Figueiredo and Campos \citeyear{Figueiredo2001} describe serendipitous ``moves'' from one
problem to another, which transform a problem that cannot be solved
into one that can.  
However, it is important to notice that progress with problems does not always mean transforming a
problem that cannot be solved into one that can.  Progress may also
apply to growth in the ability to \emph{posit} problems.  In keeping
track of progress, it would be useful for system designers to record
(or get their systems to record) what problem a given system solves,
and the degree to which the computer was responsible for coming up
with this problem.
%
As Pease et al. \citeyearpar[p. 69]{pease2013discussion} remark, anomaly detection and
outlier analysis are part of the standard machine learning toolkit --
but recognising \emph{new} patterns and defining \emph{new} problems
is more ambitious \cite{von2003cybernetics}.  Establishing complex
analogies between evolving problems and solutions is one of the key
strategies used by teams of human designers
\cite{Analogical-problem-evolution-DCC}.  The creation of new patterns
and higher-order analogies were considered in Hofstadter and
Mitchell's work on {\sf Copycat} and the subsequent {\sf Metacat}, but
these systems operate in a simple and fairly abstract ``microdomain''
\cite{hofstadter1994copycat,DBLP:journals/jetai/Marshall06}.
%
Turning over increased responsibility to the machine will be important
if we want to foster the possibility of genuine surprises.

The {\sf SerenA} system developed by Deborah Maxwell et
al.~\citeyear{maxwell2012designing} offers a case study in some
of these concepts.  This system is designed to support
serendipitous discovery for its (human) users
\cite{forth2013serena}.  The authors rely on a process-based
model of serendipity \cite{Makri2012,Makri2012a} that is derived
from user studies that draws on interviews with 28 researchers,
who were asked to look for instances of serendipity from both
their personal and professional lives.  The research aims to
support the formation of bridging connections from an unexpected
encounter to a previously unanticipated but valuable outcome.
The theory focuses on the acts of reflection that support both
the creation of a bridge, and the estimation of the potential
value of the result.
%
While this description touches on all of the features of our model, {\sf
  SerenA} largely matches the description offered by Andr{\'e} et
al.~\citeyear{andre2009discovery} of discovery-focused systems, in which
the user experiences an ``aha'' moment and takes the
creative steps to realise the result.  {\sf SerenA}'s primary computational method is to
search outside of the normal search parameters in order to engineer
potentially serendipitous (or at least pseudo-serendipitous)
encounters.

In sum, computer-supported serendipity has been well-studied, but
purely computational serendipity, much less so.  This may partly be
due to the absence of clear criteria for serendipity, which we address
in the current paper.  However, there are other underlying factors.
Existing standards for assessing computational creativity have
historically focused on product evaluations.
\citeA{ritchie07} uses metrics that depend on observable properties of artifacts.  He suggests ``typicality'', i.e., the extent to which an artifact belongs to a certain genre, and ``quality'' as atomic measures for more complex metrics, including ``novelty.'' Ritchie initially bases these metrics on human judgment, but points out that it may also be possible to compute them automatically. For instance, quality could be computed using a fitness score of the assessed artifacts, which should correlate highly with human-perceived quality. The typicality of produced artifacts can be calculated according to their similarity to the artifacts inspiring the generative process. Both fitness functions and distance metrics are subject to an ongoing debate in computational aesthetics.  Section \ref{sec:evomusic} will return to these issues.

In recent years, artefact-centred evaluations are increasingly complemented by methods
that consider process \cite{colton2008creativity,colton-assessingprogress} or a combination
of product and process \cite{jordanous:12}.  However, processes that
arise outside of the control of the system (and ultimately, the
researcher) are generally seen as out of scope for computational
creativity \emph{per se}.  External effects may even be seen to
``invalidate'' research into computational creativity.
%

We would argue that the concept of serendipity brings autonomous
creative systems into clearer focus, where ``autonomous creativity''
does not mean creativity \emph{sui generis}, but creative interaction
with the world.  This often requires a different mindset, and a
different approach to system building and evaluation.
\begin{quote}
``\emph{Tinkering is a process of serendipity-seeking that does not
    just tolerate uncertainty and ambiguity, it requires it.  When
    conditions for it are right, the result is a snowballing effect
    where pleasant surprises lead to more pleasant surprises.}''
  \cite[``Tinkering versus Goals'']{rao2015breaking}
%% What makes this a problem-solving mechanism is diversity of individual perspectives coupled with the law of large numbers (the statistical idea that rare events can become highly probable if there are enough trials going on). If an increasing number of highly diverse individuals operate this way, the chances of any given problem getting solved via a serendipitous new idea slowly rises. This is the luck of networks.

%% Serendipitous solutions are not just cheaper than goal-directed ones. They are typically more creative and elegant, and require much less conflict. Sometimes they are so creative, the fact that they even solve a particular problem becomes hard to recognize. For example, telecommuting and video-conferencing do more to “solve” the problem of fossil-fuel dependence than many alternative energy technologies, but are usually understood as technologies for flex-work rather than energy savings.

%% Ideas born of tinkering are not targeted solutions aimed at specific problems, such as “climate change” or “save the middle class,” so they can be applied more broadly. As a result, not only do current problems get solved in unexpected ways, but new value is created through surplus and spillover. The clearest early sign of such serendipity at work is unexpectedly rapid growth in the adoption of a new capability. This indicates that it is being used in many unanticipated ways, solving both seen and unseen problems, by both design and “luck”.
\end{quote}
%% If we control the system, at bottom the best we can hope for is
%% ``pleasant unsurprises.''  At the same time, understanding serendipity
%% may help build autonomous systems that produce fewer ``unpleasant surprises,'' a
%% serious contemporary concern
%% \cite{philosophy-machine-morality,machine-ethics-status}.


%
% Ritchie initially bases his metrics on human judgment, but points out different ways to compute them automatically, arising from practical study.  For instance, quality could be computed using a fitness score of the assessed artifacts, which should highly correlate with human-perceived quality.  The typicality of produced artifacts was calculated as their similarity to the artifacts inspiring the generative process.  Nevertheless, this requires a good distant metric.  Both fitness functions and distance metrics are subject to an ongoing debate in computational aesthetics.

Although the notion of serendipity is process-focused, value is a crucial dimension of serendipity, and evaluations of an outcome (often an artefact) continue to be relevant.  Furthermore, an ``embedded'' evaluation is required to effect the critical focus shift, that is, to notice that the trigger is interesting.    Adapting qualitative artefact-oriented measures (like novelty) may be necessary in order to build systems that are capable of carrying out the necessary formative evaluation steps for serendipitous processing, as well as a final summative evaluation of the result.  
