\subsection{Recommendations} \label{sec:recommendations}

%% Von Foerster \citeyear[p. 286]{von2003cybernetics} advocated a \emph{second-order cybernetics} in which ``the observer who enters the system shall be allowed to stipulate his own purpose.''  Deleuze \citeyear[p. 26]{deleuze1994difference} wrote:
%% \begin{quote}
%%  ``\emph{We learn nothing from those who say: `Do as I do'. Our only
%%     teachers are those who tell us to `do with me', and are able to
%%     emit signs to be developed in heterogeneity rather than propose
%%     gestures for us to
%%     reproduce.}''~\cite[p. 26]{deleuze1994difference}
%% \end{quote}
%% These perspectives present a number of challenges to the typical
%% programming paradigm, which is linked far more closely to a
%% \emph{first-order cybernetics} that specifies system goals and
%% operations at the outset, and provides external evaluation.


% Dewey, Whitehead similar too.
Our thought experiment in Section \ref{sec:ww} develops a design
illustrating the relationship between creativity at the level of
artefacts (e.g. new poems) and creativity at the level of
\emph{problem specification}.  The search for connections that make
raw data into ``strategic data'' is an appropriate theme for
researchers in computational creativity to grapple with.

%% Problem
%% evolution (by analogy to existing design solutions) is discussed in
%% \cite{Analogical-problem-evolution-DCC}, focusing on the case of human
%% designers.  As \cite[p. 69]{pease2013discussion} remark, anomaly
%% detection and outlier analysis are part of the standard machine
%% learning toolkit, but it seems 

In \cite{stakeholder-groups-bookchapter}, we outlined a general
programme for computational creativity, and examined perceptions of
creativity in computational systems found among members of the general
public, Computational Creativity researchers, and creative communities
-- understood as human communities.  We should now add a fourth
important ``stakeholder'' group in computational creativity research:
computer systems themselves.  Creativity may look very different to
this fourth stakeholder group than it looks to us.  We should help
computers evaluate their own results and creative process.

%% These ideas set a relatively high bar, if only because computational
%% creativity has often been focused on generative rather than reflective
%% acts.  
As Campbell \citeyear{campbell} writes: ``serendipity
presupposes a smart mind.''  We may be aided in our pursuit by
recalling Turing's proposal that computers should ``be able to
converse with each other to sharpen their wits''
\cite{turing-intelligent}.  Other fields, including computer Chess,
Go, and argumentation have achieved this, and to good effect.

The Writers Workshop described in Section \ref{sec:ww} is an example
of one such social model, but more fundamentally, it is an example of
\emph{learning from feedback}.  The Workshop model ``personifies'' the
wider world as one or several critics.  It is clearly also possible
for a lone creative agent to take its own critical approach in relationship
to the world at large, using an experimental approach to generate
feedback, and then looking for models to fit this feedback.

%% While the pursuit of serendipitous findings may not enhance,
%% and may even diminish, results from a computationally creative system
%% and the evaluation of such a system's process, we believe that
%% serendipity is both possible and useful to model in future systems.
