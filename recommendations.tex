\subsection{Recommendations} \label{sec:recommendations}

Our thought experiment in Section \ref{sec:ww} develops a design
illustrating the relationship between creativity at the level of
artefacts (e.g.~new poems) and creativity at the level of
\emph{problem specification}.  The search for connections that make
raw data into ``strategic data'' is an appropriate theme for
researchers in computational creativity to grapple with.

\citeA{stakeholder-groups-bookchapter} outlined a general programme
for computational creativity, and examined perceptions of creativity
in computational systems found among members of the general public,
Computational Creativity researchers, and creative communities --
understood as human communities.  We should now add a fourth important
``stakeholder'' group in computational creativity research: computer
systems themselves.  Creativity may look very different to this fourth
stakeholder group than it looks to us.  We should help computers
evaluate their own results and creative process.

As Campbell \citeyear{campbell} writes: ``serendipity
presupposes a smart mind.''  We may be aided in our pursuit by
recalling Turing's proposal that computers should ``be able to
converse with each other to sharpen their wits''
\cite{turing-intelligent}.  Other fields, including computer Chess,
Go, and argumentation have achieved this, and to good effect.

The Writers Workshop described in Section \ref{sec:ww} is an example
of one such social model, but more fundamentally, it is an example of
\emph{learning from experience}.  The Workshop model ``personifies''
the wider world in the form of one or several critics.  It is clearly
also possible for a lone creative agent to take its own critical
approach in relationship to the world at large, using an experimental
approach to generate feedback, and then looking for models to fit this
feedback.

To summarise: we advance the following further criteria for research
in computational serendipity, viewing the concepts in Section
\ref{sec:by-example} through the practice scenarios we have discussed.
% \subsubsection*{The ``serendipity programme'' for computational creativity}

\begin{itemize}
\item \textbf{Autonomy}: In the standard cybernetic model, we control computers, and we control the computer's context.  There is little room for serendipity because there is nothing outside of our direct control. Von Foerster \citeyear[p. 286]{von2003cybernetics} advocated a \emph{second-order cybernetics} in which ``the observer who enters the system shall be allowed to stipulate his own purpose.'' An eventual corollary of serendipitous operation of computers will be that \emph{Computational agents can specify their own problems to work on.}
\end{itemize}

\begin{itemize}
\item \textbf{Learning}: If we admit the possibility of computational agents that operate our world rather than a circumscribed microdomain, and that are curious about this world, then another corollary is that \emph{Computational agents will learn more and more about the world we live in.}
\end{itemize}

\begin{itemize}
\item \textbf{Sociality}:  Deleuze \citeyear[p. 26]{deleuze1994difference} wrote: ``We learn nothing from those who say: `Do as I do'. Our only teachers are those who tell us to `do with me'[.]''  Turing recognised that computers would have to be coached in the direction of social learning, but that once they attain that standard they will learn much more quickly.  A third corollary of serendipitous computing is that \emph{Computational agents will interact in a recognisably social way with us and with each other, resulting in emergent effects.}
\end{itemize}

\begin{itemize}
\item \textbf{Embedded evaluation and ethics}: Finally, a fourth
  corollary is that \emph{Computational agents will evaluate their own
    creative process and products.}  It is our responsibility as
  system designers to teach them how to make evaluations in an ethical
  manner.  This is exemplified by the preference for a ``non-zero
  sum'' criterion for value suggested in our discussion of the
  dimensions of serendipity in Section \ref{sec:by-example}.
\end{itemize}

A quick survey of word occurrences from the last three years of
proceedings from the International Conference on Computational
Creativity shows that related themes are broadly active in the
research community.

\medskip

{\centering \setlength{\tabcolsep}{7pt} \small
\hspace{-7pt}\begin{tabular}{r|l|l|l|l|l|l|}
\multicolumn{1}{r}{} & \multicolumn{1}{c}{autonom*} & \multicolumn{1}{c}{learn*} & \multicolumn{1}{c}{social*} & \multicolumn{1}{c}{evaluat*} & \multicolumn{1}{c}{ethic*} & \multicolumn{1}{c}{\textbf{total}}\\
\cline{2-7}
2012&36 (.1\%) & 124 (1\%) & 74 (.6\%) & 478 (3.8\%) & 3 (<.1\%) & 12472 \\
\cline{2-7}
2013&69 (.6\%) & 218 (1.7\%) & 116 (.9\%) & 360 (2.9\%) & 0 (0\%) & 12531\\
\cline{2-7}
2014&77 (.4\%) & 292 (1.6\%) & 360 (2\%) & 993 (5.4\%) & 2 (<.1\%) & 18192\\
\cline{2-7}
\end{tabular}

\par
}

\bigskip

\noindent Given the community's active interest in evaluation, we
believe that the proposed standards we have outlined and piloted for
evaluating computational serendipity will be of use.  We hope that our
clarifications to the often subtle concept of serendipity will help to
encourage future computer-aided (and computer-driven) investigations
of the constituent processes of discovery and invention.
