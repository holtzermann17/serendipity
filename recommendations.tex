\subsection{Recommendations} \label{sec:recommendations}

Deleuze writes: ``True freedom lies in the power to decide, to
constitute problems themselves'' \cite[p. 15]{deleuze1991bergsonism};
and, elsewhere, rephrasing this sentiment in a social way:
\begin{quote}
``\emph{We learn nothing from those who say: `Do as I do'. Our only teachers
  are those who tell us to `do with me', and are able to emit signs to
  be developed in heterogeneity rather than propose gestures for us to
  reproduce.}''~\cite[p. 26]{deleuze1994difference}
\end{quote}
Dewey emphasised a child's training must
deal  with objects which ``arise out of their interests and their
own problems'' \cite[p. 73]{dewey-by-mead}.  Von Foerster advocated a form of cybernetics in which ``the observer who enters the system
shall be allowed to stipulate his own purpose''
\cite[p. 286]{von2003essays}.
% Whitehead is similar too.

The thought experiment presented in Section \ref{sec:ww} illustrated
the relationship between problem creation and serendipity.  Looking
for the connections that make raw data into ``strategic data'' is a
core pattern of problem creation.  This is an appropriate theme for
researchers in computational creativity to grapple with.

In \cite{stakeholder-groups-bookchapter}, we outlined a general
programme for computational creativity, and examined perceptions of
creativity in computational systems found among members of the general
public, Computational Creativity researchers, and creative communities
-- understood as human communities.  We should now add a fourth
important ``stakeholder'' group in computational creativity research:
computer systems themselves.  Creativity may look very different to
this fourth stakeholder group than it looks to us.  When computers are
required to evaluate their own results, we are also implicitly
requiring them to evaluate their creative process.  We should give
them the tools to do that effectively.
%
These ideas set a relatively high bar, if only because computational
creativity has often been focused on generative rather than reflective
acts.  As Campbell \citeyear{campbell} writes: ``serendipity
presupposes a smart mind.''  We may be aided in our pursuit by
recalling Turing's proposal that computers should ``be able to
converse with each other to sharpen their wits''
\cite{turing-intelligent}.  Other fields, including computer Chess,
Go, and argumentation have achieved such standards, and to good effect.

The Writers Workshop described in Section \ref{sec:ww} is an example
of one such social model, but more fundamentally, it is an example of
\emph{learning from feedback}.  The Workshop model ``personifies'' the
wider world as one or several critics.  It is clearly also possible
for a lone creative agent to take its own critical approach in relationship
to the world at large, using an experimental approach to generate
feedback, and then looking for models to fit this feedback.

%% While the pursuit of serendipitous findings may not enhance,
%% and may even diminish, results from a computationally creative system
%% and the evaluation of such a system's process, we believe that
%% serendipity is both possible and useful to model in future systems.
