\subsection{Recommendations} \label{sec:recommendations}


%
However, progress with problems does not always mean transforming a
problem that cannot be solved into one that can.  Progress may also
apply to growth in the ability to posit problems.  As Deleuze writes:
``True freedom lies in the power to decide, to constitute problems
themselves'' \cite[p. 15]{deleuze1991bergsonism}; or, rephrase this in
a social way:
\begin{quote}
``\emph{We learn nothing from those who say: `Do as I do'. Our only teachers
  are those who tell us to `do with me', and are able to emit signs to
  be developed in heterogeneity rather than propose gestures for us to
  reproduce.}''~\cite[p. 26]{deleuze1994difference}
\end{quote}
Although disagreeing with certain aspects of the Bergsonian
foundations for Deleuze's assertion, Dewey's perspective on the matter
of learning was in fact similar \cite[p. 73]{dewey-by-mead}.

Bearing these thoughts in mind we would recommend that in applying a
formalism like that of \cite{colton-assessingprogress}, system
designers clearly record what problem a given system solves, and the
degree to which the computer was responsible for coming up with this
problem.

In \cite{stakeholder-groups-bookchapter}, we advanced a broader
programme for computational creativity.  In this work we examined
perceptions of creativity in computational systems found among members
of the general public, Computational Creativity researchers, and
creative communities -- understood as human communities.  We should
now add a fourth important ``stakeholder'' group in computational
creativity research: computer systems themselves.  Creativity may look
very different to this fourth stakeholder group than it looks to us.

One possible criticism of the model presented here is that it sets a
high bar.  As Campbell \citeyear{campbell} writes: ``serendipity
presupposes a smart mind.''  We should keep in mind Turing's proposal
that computers should ``be able to converse with each other to sharpen
their wits'' \cite{turing-intelligent}.  Other fields, including
computer Chess, Go, and argumentation have achieved this standard, and
to good effect.  We should move in that direction too.

 %% While the pursuit of serendipitous findings may not enhance,
%% and may even diminish, results from a computationally creative system
%% and the evaluation of such a system's process, we believe that
%% serendipity is both possible and useful to model in future systems.
