\section{Recommendations} \label{sec:recommendations}

In the diagrammatic formalism advanced in
\cite{colton-assessingprogress}, we spoke about progress with
\emph{systems} rather than with \emph{problems}.  It would be a useful
generalisation of the formalism -- and not just a simple relabelling
-- to tackle problems as well.
%
Figueiredo and Campos \cite{Figueiredo2001}, for example, describe
serendipitous ``moves'' from one problem to another.
%
However, progress with problems does not always mean transforming a
problem that cannot be solved into one that can.  Progress may also
apply to growth in the ability to posit problems.  As Deleuze writes:
``True freedom lies in the power to decide, to constitute problems
themselves'' \cite[p. 15]{deleuze1991bergsonism}.  Indeed, against any
education by means of ready-made problems, Dewey's perspective was
that
\begin{quote}
``\emph{the child's mind can be trained only in so far as the objects
    with which they are occupied arise out of their interests and
    their own problems.}''~\cite{dewey-by-mead}
\end{quote}

This was our emphasis in Section \ref{sec:unified-approach}:
developing new design patterns is closely connected with -- and in the
dynamical interpretation we prefer, effectively synonymous with --
positing new problems.  Although \cite{colton-assessingprogress}
presented a way to model creative progress at various levels of
granularity, it dealt primarily with \emph{solutions}; and although it
exhibited progress in a way that would be recognised by impartial
observers, the formalism did not focus on expositing the features that
would permit a system to actually \emph{make} creative progress.
Accordingly, we would recommend that in applying our earlier
formalism, system designers clearly record what problem a given system
solves, and the degree to which the computer was responsible for
coming up with this problem.

In \cite{stakeholder-groups-bookchapter}, we advanced a broader
programme for computational creativity, in which we argue in favour of
studying the \emph{perceptions} of creativity by various parties.  The
criteria developed in the current paper -- including the focus shift,
which we regard as fundamental -- can be used in the same way, as we
will describe below.

%% MC> Angelina is a able to read Twitter to find out what people think of
%% MC> people like Hamid Karzai, and then change the sorts of images that
%% MC> it finds as a result.  So you're going to see a happy picture of
%% MC> President Obama later next to a very angry picture of Hamid Karzai.
%% MC> While some of this might look creative and intelligent, a lot of it
%% MC> comes down to serendipity as well.  So the image you're about to see
%% MC> comes up for a Google search for terrorism that doesn't really have
%% MC> much relevance to the news article, and the sound that you're
%% MC> hearing now, the electronic drone, sounds like it's a good choice
%% MC> for a game that's about war and about feeling unsettling.  But in
%% MC> actuality I have no idea how Angelina came up with that choice.

Our proposed Writers Workshop is very different from the Turing-style
imitation game, but nevertheless may prove to be a useful aptitude
test for computer systems, and as a context in which computationally
creative programs may become aware of each other, and participate
actively in advancing the field of research.  We previously examined
perceptions of creativity in computational systems found among members
of the general public, Computational Creativity researchers, and
creative communities -- understood as human communities.  We should
now add a fourth important ``stakeholder'' group in computational
creativity research: computer systems themselves.

To make the point emphatically: the writers workshop proposed above is
very different from a traditional system ``Show and Tell'' presented
by system developers, for system developers.  Traditional academic
practices associated with presenting finished work, or even
work-in-progress, are not entirely suitable for the field of
computational creativity, where engagement between systems may exhibit
manifestly serendipitous results.  If the community does not implement
a suggestion like the one presented here, it will be missing out on a
key idea for enhancing computational creativity that has been
circulating since Turing suggested that computers should ``be able to
converse with each other to sharpen their wits''
\cite{turing-intelligent}.  Other fields, including computer Go
\cite{bouzy2001computer} and argumentation \cite{yuan2008towards} have
their own dedicated servers and protocols for exchange.  We should
move in that direction too.

There is ample room for unpredictability in such pursuits.  Creativity
may look very different to this fourth stakeholder group than it looks
to us.  In time to come, computer systems will increasingly take
leadership in matters of genre, interaction design, and their own
artistic and scientific training.  For now, our job is not at all to
get out of the way, like the parents of young adults, but rather to
participate in creating the ``play schools'' in which systems that are
quite frankly in early development can begin to socialise with each
other.
%
In \cite{stakeholder-groups-bookchapter}, we introduced nine
hypotheses related to the perception of creativity in computational
systems. 
The last of these hypotheses stated that:
\begin{quote}
``\emph{The perception of creativity in software which produces
  artefacts within a creative community will be increased if the
  software can exhibit subjective judgements about its own work and
  that of others, and defend those judgements in an accountable
  way.}''~\cite{stakeholder-groups-bookchapter}
\end{quote}
If the framework described in this paper is developed further, we may
be able to test this hypothesis in computer simulations.

Our proposed template for design patterns for participation in writers
workshops is different from, but complementary to Alexander's
framework.  Whereas Alexander focused on solutions to common
architectural problems (\emph{A place to wait}, etc.), our framework
is primarily designed to elicit and engage with new and unexpected
problems.  We presented four examples using the template, but our
intention is for the template to be used in a reflective mode by
systems to generate new patterns, in a manner appropriate to
second-order cybernetics.  Many practical issues remain to be settled
for a future computational enterprise that seeks to combine existing
design patterns and new stimuli in order to generate new, useful
design patterns.  One thing that becomes clear from this discussion is
that \emph{problem-setting} is a fundamental issue for the field of
computational creativity that will only be given due attention when
the research culture is ready to fully embrace serendipity.

\begin{quote}
``[S]\emph{ocial cybernetics must be a second-order cybernetics--a
    cybernetics of cybernetics--in order that the observer who enters
    the system shall be allowed to stipulate his own purpose: he is
    autonomous.}'' \cite[p. 286]{von2003essays}
\end{quote}
