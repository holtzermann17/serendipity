\subsection{Challenges for future research} \label{sec:recommendations}

To summarise: we advance the following further criteria for research
in computational serendipity, viewing the concepts in Section
\ref{sec:by-example} through the practice scenarios we have discussed.

\begin{itemize}
\item \textbf{Autonomy}: Our thought experiment in Section
  \ref{sec:ww} develops a design illustrating the relationship between
  creativity at the level of artefacts (e.g.~new poems) and creativity
  at the level of \emph{problem specification} (capturing new poetic
  concepts).  The search for connections that make raw data into
  ``strategic data'' is an appropriate theme for researchers in
  computational creativity to grapple with.  In the standard
  cybernetic model, we control computers, and we also control the
  computer's operating context.  There is little room for serendipity
  because there is nothing outside of our direct control. Von Foerster
  \citeyear[p. 286]{von2003cybernetics} advocated a \emph{second-order
    cybernetics} in which ``the observer who enters the system shall
  be allowed to stipulate his own purpose.''  \emph{A primary
    challenge in the serendipitous operation of computers will be for
    computational agents to specify their own problems to work on.}
\end{itemize}

\begin{itemize}
\item \textbf{Learning}: The Writers Workshop described in Section
  \ref{sec:ww} is fundamentally an example of a design for a system
  that can \emph{learn from experience}.  The Workshop model
  ``personifies'' the wider world in the form of one or several
  critics.  It is clearly also possible for a lone creative agent to
  take its own critical approach in relationship to the world at
  large, using an experimental approach to generate feedback, and then
  looking for models to fit this feedback.   We are led to consider 
  computational agents that operate our world rather
  than a circumscribed microdomain, and that are curious about this
  world.  \emph{A second challenge is for computational agents to
    learn more and more about the world we live in.}
\end{itemize}

\begin{itemize}
\item \textbf{Sociality}: As Campbell \citeyear{campbell} writes:
  ``serendipity presupposes a smart mind.''  We may be aided in our
  pursuit by recalling Turing's proposal that computers should ``be
  able to converse with each other to sharpen their wits''
  \cite{turing-intelligent}.  Other fields, including computer Chess,
  Go, and argumentation have achieved this, and to good effect.
  Deleuze \citeyear[p. 26]{deleuze1994difference} wrote: ``We learn
  nothing from those who say: `Do as I do'. Our only teachers are
  those who tell us to `do with me'[.]''  Turing recognised that
  computers would have to be coached in the direction of social
  learning, but that once they attain that standard they will learn
  much more quickly.  \emph{A third challenge is for computational
    agents to interact in a recognisably social way with us and with
    each other, resulting in emergent effects.}
\end{itemize}

\begin{itemize}
\item \textbf{Embedded evaluation}:
  \citeA{stakeholder-groups-bookchapter} outlined a general programme
  for computational creativity, and examined perceptions of creativity
  in computational systems found among members of the general public,
  Computational Creativity researchers, and creative communities --
  understood as human communities.  We should now add a fourth
  important ``stakeholder'' group in computational creativity
  research: computer systems themselves.  Creativity may look very
  different to this fourth stakeholder group than it looks to us.  We
  should help computers evaluate their own results and creative
  process.  \emph{A fourth challenge is for computational agents to
    evaluate their own creative process and products.}
\end{itemize}

It is our responsibility as system designers to teach them how to make
evaluations in both a reasonable and an ethical manner.  This is
exemplified by the preference for a ``non-zero sum'' criterion for
value suggested in our discussion of the dimensions of serendipity in
Section \ref{sec:by-example}.

A quick survey of word occurrences from a recent special issue of
\emph{Cognitive Computation} shows that related themes are broadly
active in the research community.\footnote{Articles converted to text
  via {\tt pdftotext -layout}, individual counts found via {\tt tr \textquotesingle~\textquotesingle~\textquotesingle\textbackslash n\textquotesingle~< \$i | grep -c "stem*"}, and total word counts via {\tt wc -w}.  The
corresponding counts for the \emph{current} paper are 9, \emph{25}, \emph{15}, \emph{43} and 11.9K.}  Here
\emph{italics} indicates that the word stem accounted for .1\% of the
article or more, and \textbf{\emph{bold italics}} indicates that it
accounted for 1\% or more.

\medskip

{\centering \setlength{\tabcolsep}{5pt} \tiny
\begin{tabular}{ccccccccccccccc}
paper \#
&1
&2
&3
&4
&5
&6
&7
&8
&9
&10
&11
&12
&13
&14
\\
\cline{2-15}
"autonom.*"
&0
&\textbf{\emph{32}}
&\emph{12}
&\emph{41}
&0
&1
&\emph{31}
&2
&1
&\emph{92}
&11
&2
&5
&\textbf{\emph{22}}
\\
"learn.*"
&6
&2
&2
&\emph{14}
&\emph{9}
&\textbf{\emph{118}}
&\emph{14}
&\emph{18}
&\emph{44}
&\emph{12}
&11
&\emph{42}
&\emph{44}
&2
\\
"social.*"
&0
&0
&\emph{23}
&\emph{25}
&0
&1
&2
&\emph{10}
&\emph{19}
&\emph{19}
&8
&\emph{21}
&13
&2
\\
"evaluat.*"
&0
&1
&\emph{11}
&\emph{20}
&0
&1
&3
&6
&4
&9
&8
&2
&\textbf{\emph{304}}
&0
\\
\cline{2-15}
\textbf{total(K)}
& 8.3  % &8337 (/ 6 8337.0)
& 2.2  % &2221 (/ 32 2221.0)  0.0135074290859973
& 7.5  % &7507  (/ 12 7507.0) 0.0015985080591447982 (/ 23 7507.0) 0.0026641800985746636 (/ 11 7507.0)0.001465299054216065
& 7.4  % &7453 (/ 41 7453.0) 0.004964443848114853 (/ 14 7453.0) 0.0009392191064001073 (/ 16 7453.0) 0.002146786528914531 (/ 19 7453.0) 0.0025493090030860054
& 8.6  % &8675 (/ 9 8675.0)
& 5.8  % & 5816 (/ 89 5816.0) 0.015302613480055021
&10.3 % &10341 (/ 30 10341.0) 0.002901073397156948
& 9.6  % &9632  (/ 18 9632.0) 0.0018687707641196014  (/ 10 9632.0)0.0010382059800664453
&10.8 % &10851 (/ 36 10851.0) 0.0033176665745092617
&11.6 % &11693 (/ 92 11693.0)0.007867955186863935 (/ 12 11693.0)
&14.4 % &14407 (/ 11 14407.0) 0.0007635177344346498
&10.8 % &10840 (/ 31 10840.0) 0.0028597785977859777
&25.3 % &25326 (/ 13  25326.0)  (/ 44  25326.0) 0.0008291873963515755 (/ 304 25326.0) 0.011174287293690278
& 1.6  % &1673 (/ 21 1673.0) 0.012552301255230125
\\
\end{tabular}
}

\bigskip

Paper 4, Rob Saunder's \citeyear{saunders2012towards} ``Towards
Autonomous Creative Systems: A Computational Approach'' was the only
one to emphasise all four of our themes.  Saunders asks: ``What would
it mean to produce an autonomous creative system? How might we
approach this task? And, how would we know if we had succeeded?''  He
argues for an approach ``that models personal motivations, social
interactions and the evolution of domains.''  Paper 10, d'Inverno and
Luck's \citeyear{d2012creativity} ``Creativity Through Autonomy and
Interaction'', also contains a theoretical engagement with these
themes, and presents a formalism for multi-agent systems that could
usefully be adapted to model serendipitous encounters.

We believe that our clarifications to the multifaceted concept of
serendipity will help to encourage future computer-aided (and
computer-driven) investigations of the above themes and their
interrelationships.  We are particularly interested in the
relationship between discovery and invention, and we discuss some of
our own plans in this direction below.
