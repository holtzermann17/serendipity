\section{Revision plan}\label{plan-of-action}

\begin{enumerate}
\def\labelenumi{\arabic{enumi}.}
\itemsep1pt\parskip0pt\parsep0pt
\item
  \textbf{(Joe, Alison): Significantly clarify the argument and
  summarise it in the introduction.}
\end{enumerate}

\begin{itemize}
\itemsep1pt\parskip0pt\parsep0pt
\item
  We are offering one possible computational definition of serendipity
\item
  Serendipity is not the same as luck. ~It's a matter of learning
  something, in a way that's unanticipated. ~Looking for something and
  finding something else.
\item
  Explain the aspects of the model better, e.g.~why is it essential that
  the trigger is not under the control of the system.
\item
  Clearly summarise the offering of the paper.
\end{itemize}

\begin{enumerate}
\def\labelenumi{\arabic{enumi}.}
\setcounter{enumi}{1}
\itemsep1pt\parskip0pt\parsep0pt
\item
  \textbf{(Joe): Move our formal definition of serendipity (e.g.~the
  diagram) up to meet the literature review, as a new section `Formal
  definition of Serendipity'. (It's a key contribution of the paper.)}
\end{enumerate}

\begin{itemize}
\itemsep1pt\parskip0pt\parsep0pt
\item
  We will clearly connect the heuristic criteria from Alison with the
  figure.
\item
  In addition a quick graphical summary of the 13 criteria
\end{itemize}

\begin{enumerate}
\def\labelenumi{\arabic{enumi}.}
\setcounter{enumi}{2}
\itemsep1pt\parskip0pt\parsep0pt
\item
  \textbf{(Joe): Drop sections 3 and 4, and move key concepts to
  ``future work''}
\end{enumerate}

\begin{itemize}
\itemsep1pt\parskip0pt\parsep0pt
\item
  Section 3 (FloWr) - heavily condense and put into future work (some
  overview of Joe's concrete implementation plans). ~Explain with
  minimal references.
\item
  Section 4 (Design patterns) - heavily condensed - ``Just So Stories''
  paragraph in Section 5.3 as a potential application. ~Explain some
  history about design patterns and say that, for serendipity, the
  question is where do new ``design'' ideas come from. ~(I.e. discovery
  of a new approach.) ~But make this future work.
\item
  ``We are highlighting how design patterns and the other ideas in this
  paper could be used to build a context where serendipity will take
  place.''
\end{itemize}

\begin{enumerate}
\def\labelenumi{\arabic{enumi}.}
\setcounter{enumi}{3}
\itemsep1pt\parskip0pt\parsep0pt
\item
  \textbf{(Anna): Remove Section 5.3 (save it for another paper about
  Writers Workshops). It's relevant for ``embedded creativity'' but
  ``Writers Workshops'' themselves can be a footnote. The actual idea
  here is more general.}
\end{enumerate}

\begin{itemize}
\itemsep1pt\parskip0pt\parsep0pt
\item
  Anna can add more about evaluation in the creative process
\item
  The idea of the WW (or just social revision) is an example of a place
  where serendipity can occur.
\end{itemize}

\begin{enumerate}
\def\labelenumi{\arabic{enumi}.}
\setcounter{enumi}{4}
\itemsep1pt\parskip0pt\parsep0pt
\item
  \textbf{(Anna): Leading into our thought experiment: ``An emerging
  theme in computing is exploitation of social creativity and feedback.
  Our computational model contributes to theorising this work.''}
\end{enumerate}

\begin{itemize}
\itemsep1pt\parskip0pt\parsep0pt
\item
  Include another example with computational serendipity? Maybe the
  example from Kaz's thesis
\item
  It would not be hard to find an example of a music system noticing
  that a note was wrong and playing. Make sure we include at least one
  example that is not ``technically improbable'' -- better to include
  several that have been realised (e.g.~Copycat)
\end{itemize}

\begin{enumerate}
\def\labelenumi{\arabic{enumi}.}
\setcounter{enumi}{5}
\itemsep1pt\parskip0pt\parsep0pt
\item
  \textbf{(Christian): Copycat or any other historical examples of
  serendipity in computing, or explanation of why there are none (argue
  for or against, in the background section, as a new §§, and perhaps
  again later in the document as a further analysis to accompany our
  thought experiment).}
\end{enumerate}

\begin{itemize}
\itemsep1pt\parskip0pt\parsep0pt
\item
  Concrete lower bound examples and counterexamples, e.g.~would it be
  possible for ``merely generative'' systems to exhibit serendipity? --
  case of genetic algorithms
\item
  What is the difference between serendipity and good luck? (E.g. a
  random act that leads to an outcome that is evaluated positively.)
\item
  What are the strict requirements and what are only the supportive
  factors that make serendipity ``likely''? Or is it a matter of degree?
\item
  Is it the case that serendipitous systems would be more `sagacious' in
  recognizing interesting triggers? - explain, especially in connection
  with computational search.
\item
  What about `regular' systems that work by applying inference
  procedures on symbolic representations to yield new representations?

  \begin{itemize}
  \itemsep1pt\parskip0pt\parsep0pt
  \item
    e.g.~theorem provers
  \end{itemize}
\item
  Evaluate existing approaches to ``computational learning'' - are they
  serendipitous?
\end{itemize}

\begin{enumerate}
\def\labelenumi{\arabic{enumi}.}
\setcounter{enumi}{6}
\itemsep1pt\parskip0pt\parsep0pt
\item
  \textbf{(Simon, Alison): Clarify the extent to which serendipity is
  something that ``actually exists'' or is something that is only
  perceived to exist.}
\end{enumerate}

\begin{itemize}
\itemsep1pt\parskip0pt\parsep0pt
\item
  It does not seem to be an ``essentially contested concept'', just a
  potentially confusing one. One contribution of the paper is to clarify
  this.
\item
  Clarify the relationship to other key concepts in computational
  creativity / creative computing
\end{itemize}

\begin{enumerate}
\def\labelenumi{\arabic{enumi}.}
\setcounter{enumi}{7}
\itemsep1pt\parskip0pt\parsep0pt
\item
  \textbf{(Alison): Include a section early on that defines any other
  keywords that we refer to later, like the word ``dynamic''.}
\item
  \textbf{(Alison): Improve exposition of the analysis of Pek van
  Andel's patterns.}
\end{enumerate}

\begin{itemize}
\item
  \begin{enumerate}
  \def\labelenumi{(\arabic{enumi})}
  \itemsep1pt\parskip0pt\parsep0pt
  \item
    What do we hope to achieve with this analysis, and our diagram?
  \end{enumerate}
\item
  \begin{enumerate}
  \def\labelenumi{(\arabic{enumi})}
  \setcounter{enumi}{1}
  \itemsep1pt\parskip0pt\parsep0pt
  \item
    Have we done the analysis in some verifiable way, i.e. ``where does
    the analysis come from (i.e.~which aspect occurs in which pattern)?
    Is there clear consensus on this?''
  \end{enumerate}
\end{itemize}

\begin{enumerate}
\def\labelenumi{\arabic{enumi}.}
\setcounter{enumi}{9}
\itemsep1pt\parskip0pt\parsep0pt
\item
  \textbf{(Joe, all): Make referencing less intensive for the reader.}
\end{enumerate}

\begin{itemize}
\itemsep1pt\parskip0pt\parsep0pt
\item
  Use APA style referencing and cut down on number of references.
\item
  Clearly explain in narrative form what sort of literature we will draw
  on.
\item
  Perhaps the historical examples of serendipity should be confined to a
  separate ``recommended reading'' section and not referenced directly
  in the text.
\end{itemize}

\begin{enumerate}
\def\labelenumi{\arabic{enumi}.}
\setcounter{enumi}{10}
\itemsep1pt\parskip0pt\parsep0pt
\item
  \textbf{(Christian, Anna): Shorten and improve the literature review.}
\end{enumerate}

\begin{itemize}
\itemsep1pt\parskip0pt\parsep0pt
\item
  Preserve key features of the general survey, but include a more
  thorough review of recent related work in computing, including work in
  the Cognitive Computation journal.
\item
  There has been prior work on surprise (Mary Lou Maher + Kazjon Grace -
  \href{http\%20s://www.google.com/url?q=https\%3A\%2F\%2Fwww.aaai.org\%2Focs\%2Findex.php\%2F\%20WS\%2FAAAIW14\%2Fpaper\%2Fview\%2F8779\&sa=D\&sntz=1\&usg=AFQjCNGFIWctyzoi4ZSfD\%20oIrAznrL4Be0g}{https://www.aaai.org/ocs/index.php/WS/AAAIW14/paper/view/8779}
  and also their paper at ICCC 2013 or 2014) and discovery (Kaz's AAAI
  paper)
\end{itemize}

\begin{enumerate}
\def\labelenumi{\arabic{enumi}.}
\setcounter{enumi}{11}
\itemsep1pt\parskip0pt\parsep0pt
\item
  \textbf{(Joe): Confine philosophy references (e.g.~Bergson, Deleuze)
  to the background section so that it doesn't confuse anyone about what
  we're actually offering in the paper.}
\end{enumerate}

\begin{itemize}
\itemsep1pt\parskip0pt\parsep0pt
\item
  Don't refer to them in the conclusion, but do summarise the
  contribution of this paper again in the conclusion (hint: it should be
  what we say in the title).
\item
  Re-summarise again in the abstract.
\end{itemize}
