
\noindent \textbf{Prepared mind.}
Participating systems need to be able to follow the Workshop protocol.
The {\tt listening} and {\tt questions} stages of the protocol
correspond to $p$ and $p^{\prime}$ our model of serendipity.  The
corresponding ``comment generator'' and ``feedback integrator''
modules in the architectural sketch represent the primary points of
interface between author and critic. 
In principle these modules need to be prepared to deal, more or less
thoughtfully, with \emph{any} text, and in turn, with \emph{any}
comment on that text.  Certain limits may be agreed in advance,
e.g.~as to genre or length in the case of texts; ground rules may
constrain the type of comments that may be made.
%%   The loop for learning by asking questions as they arise is
%% reminiscent of the operating strategy of {\sf SHRDLU}
%% \cite{winograd1972understanding}.
A participating system -- particularly one with prior experience in
the Workshop -- will have a catalogue of outstanding unresolved, or
partially resolved, problems (denoted ``X'' in Figure \ref{fig:generative-diagram}).
Embodied in code, these drive comments, questions, and other
behaviour -- and they may be addressed in unexpected ways.\par\medskip

\noindent \textbf{Serendipity triggers.}
Although the poem is under the control of the initial generative
subsystem, it is \emph{not} under control of the listening subsystem.
The listening subsystem expects some poem, but it does not know what
poem to expect.  In this sense, the poem constitutes a serendipity
trigger $T$, not only for the listening subsystem, but for the
Workshop as a whole.
%
To expand this point, note that there may be several listeners, each
sharing their own feedback and listening to the feedback presented by
others (which, again, is outside of their direct control).  This
creates further potential for serendipity, since each listener can
learn what others see in the poem.  More formally, in this case
$T^\star$ may be seen as an evolving vector with shared state, but viewed
and handled from different perspectives.  With multiple agents
involved in the discussion, the ``comment generator'' component would
expand to contain its own feedback loops.\par\medskip

\noindent \textbf{Bridge.}
Feedback on portions of the poem may lead the system to identify new
problems and possibly new \emph{types} of problems that it hadn't
considered before.  %% The most immediately feasible case is one in which
%% the critic is a programmer who can directly program new concepts into
%% the computer \cite<cf.>{winograd1972understanding}.  However, it would
%% be hard to call that ``serendipity.''
This sort of system extension is quite typical when a human programmer
is involved.  However, here we are interested in the possibility of
agents building new poetic concepts \emph{without} outside
intervention, starting with some basic concepts and abilities related
to poetry (e.g.~definitions of words, valence of sentiments, metre,
repetition, density, etc.) and code (e.g.~the data, functions, and
macros in which the poetic concepts and workshop protocols are
embodied).  Some notable early experiments with concept invention have
been fraught with questions about autonomy
\cite{ritchie1984case,lenat1984and}. \citeA{colton2002automated}
presented a system that was convincingly autonomous: it was able to
generate interesting novel conjectures that surprised its author.
However, \citeA{pease2013discussion} note that this system was not
convincingly serendipitous: ``we had to willingly make the system less
effective to encourage incidents onto which we might project the word
serendipity.''

One cognitively inspired hypothesis is that 
the development of new concepts is closely related to development of new
sensory experiences \cite{milan2013kiki}.  
%% If the workshop
%% participants have the capacity to identify the distinctive features of
%% a given poem, then training via a machine learning or genetic
%% algorithm approach could be used to assemble a battery of existing
%% low-level tools that can approximate the effect.  Relatedly, a
%% compression process could seek to produce a given complex poetic
%% effect with a maximally-succinct
%% algorithm \cite<cf.>{schmidhuber2007simple}.
%
Feedback on the poem -- simply describing what
is in the poem from several different points of view -- can be used to
define new problems for the system to solve.
%%   This is not simply a
%% matter of decomposing the poem into pieces, but also of reconstructing
%% the way in which the pieces work together.  This is
One of the functions of the {\tt questions} step, corresponding to $p^{\prime}$ in
our formalism, is to give the poet the opportunity to enquire about
how different pieces of feedback fit together, and learn more about
where they come from.  The reconstructive process may steadily approach the ideal case --
familiar to humans -- of relating to the sentiment expressed by the
poem as a whole \cite[p. 209]{bergson1983creative}.\par\medskip

%% Several of us are involved with a contemporary project
%% \cite{coinvent14} to develop a formal theory of concept invention,
%% focusing on \emph{concept blending}.  The additive or subtractive
%% blending of existing poetry profiles may be another way to create new
%% concepts.
%% should be possible Modifer Grammar
%% Counting Breathing Position Distribution Phonics Rhythm Repetition
%% Thematic Narrative Entropy

\noindent \textbf{Result.}
In the most straightforward case, the poet would simply make changes
to the draft poem that seem to improve it in some way.  For example,
the poet might remove or alter material that elicited a negative
response from a critic.  The system may then proceed to update its modules
related to poetry generation.  It may also update its own
feedback modules, after reflecting on questions like: ``How might the
critic have noticed that feature in my poem?''\par\medskip

\noindent \textbf{Likelihood scores and potential value.}
Assuming the poems presented to the system are not too repetitive, the
chance of encountering a given serendipity trigger would be small.  It
should be straightforward for a critic to detect some known feature,
like metre or rhyme, but at least moderately difficult to notice a
novel poetic idea.  There is some nuance here, since whenever the
system learns a new concept, the low-hanging fruit from the pool of
new concepts is used up, and the system's perceptiveness
simultaneously increases.  The chance that a newly-observed feature
will result in usable code seems relatively high, but only some of
these new ideas will prove to have lasting value.  Our likelihood
score would be $\mathit{low}\times\mathit{medium}\times\mathit{high}$,
or fairly low overall, and value would be varied, with at least some
high-valued cases meriting the description ``highly
serendipitous.''\par\medskip

\noindent \textbf{Environmental factors.}
The system would set up its own internal dynamics, but it could also
provide an interface for human poets to share their poetry and
critical remarks.  There is one primary context, the Workshop, shared
by all participants.  The primary tasks envisaged in the system design
are \emph{poetry generation}, \emph{comment generation},
and \emph{code generation}.  Although these are different tasks, they
may have similar features (i.e.~they all may present opportunities to
learn from feedback).  Influences could be highly multiple, including
many very different kinds of poetry and various approaches from NLP.
