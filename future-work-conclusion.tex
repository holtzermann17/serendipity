%% Here we do not mean to suggest that every instance of ``a solution to a
%% problem in a context'' is due to serendipity at work -- on the
%% contrary, that is just the discovery step.  Inventing a viable design pattern
%% only happens when the solution is found to be explicable and useful.

To van Andel's assertion that ``The very moment I can plan or
programme `serendipity' it cannot be called serendipity anymore,'' we
would reply that we can certainly describe patterns (and programs)
with built-in indeterminacy.  Figure \ref{fig:va-pattern-figure}
presents an example, showing how one of van Andel's patterns of
serendipity can be rewritten as a design pattern using the template
suggested by our model; in future work, we would aim to build a more
complete pattern language along similar lines.
%
The example pattern describes a scenario that is quite close to Pease et al.'s \citeyear{pease2013discussion} description of an online
system that gathers new modules over time, and for which,
periodically, new combinations of modules can yield new and
interesting results.
%
Developing experiments along these lines may help prepare the
groundwork for the more involved development projects discussed in the
current paper.
%
Patterns of serendipity, like the one above, offer useful heuristic
guidelines for human programmers and convey a sense of our long-term
plans for serendipitous computing systems.





% Is ``having a stretch goal'' an example of a serendipity pattern?  I think so!
