Here we do not mean to suggest that every instance of ``a solution to a
problem in a context'' is due to serendipity at work -- on the
contrary, that is just the discovery step.  Inventing a viable design pattern
only happens when the solution is found to be explicable and useful.

To van Andel's assertion that ``The very moment I can plan or
programme `serendipity' it cannot be called serendipity anymore,'' we
would reply that we can certainly describe patterns (and programs)
with built-in indeterminacy.  As an example, we exhibit one of van
Andel's patterns of serendipity rewritten as a design pattern using
the template suggested by our model.

\begin{mdframed}
\vspace{-.35cm}
\paragraph{\textbf{Successful error}}~
\vskip -1\baselineskip
\begin{flushright}\emph{Van Andel's example} -- Post-it\texttrademark\ Notes
\end{flushright}
\vspace{-.15cm}
\begin{description}[itemsep=2pt]
\item[{\tt context}] -- You run a creative organisation with several different divisions and many contributors with different expertise.  
\item[{\tt problem}] -- One of the members of your organisation
  discovers something with interesting properties, but no one
  knows how to turn it into a product with industrial or commercial application.
\item[{\tt solution}] -- You create a space for sharing and discussing
  interesting ideas on an ongoing basis (perhaps a Writers Workshop).
\item[{\tt rationale}] -- You suspect it's possible that one of the
  other members of the firm will come up with an idea about an
  application; you know that if a potential application is found, it
  may not be directly marketable, but at least there will be a
  prototype that can be concretely discussed.
\item[{\tt resolution}] -- The \emph{Successful error} pattern
  rewritten using this template is an example of a similar
  prototype, showing that serendipity can be talked about in
  terms of design patterns.
\end{description}
\end{mdframed}

\bigskip

Although at first glance this may seem too high-level to be
computationally relevant, it is quite close to the example of an
online system that gathers new modules over time, and for which,
periodically, new combinations of modules can yield new and
interesting results -- already described in outline form in
\cite{pease2013discussion}.  
%
Developing experiments along these lines may help prepare the
groundwork for the more involved designs discussed in the current
paper.
%
Patterns of serendipity, like the one above, will offer useful
heuristic guidelines for human programmers
and convey a sense of our long-term plans.

% Is ``having a stretch goal'' an example of a serendipity pattern?  I think so!
