This diagram does not suggest that every instance of ``a solution to a
problem in a context'' is serendipity -- on the contrary, that is just
the discovery step.  To van Andel's assertion that ``The very moment I
can plan or programme `serendipity' it cannot be called serendipity
anymore'' -- of course, if the context is fully determined in advance,
and if the solution is completely replicable, then some of the
fundamental conditions for serendipity are not met.  However, we can
also describe patterns with built-in indeterminacy:

\begin{mdframed}
\vspace{-.35cm}
\paragraph{Successful error}
\emph{Van Andel's example} -- Post-it\texttrademark\ Notes\\[.05cm]
\begin{description}
\item[{\tt context}] -- You run a creative organisation with several different divisions and many contributors with different expertise.  Unexpected discoveries are often made.
\item[{\tt problem}] -- One of the members of your organisation
  discovers something with interesting properties, but that no one
  knows how to turn into a product with industrial or commercial application.
\item[{\tt solution}] -- You create a space for sharing and discussing
  interesting ideas on an ongoing basis (perhaps a Writers Workshop).
\item[{\tt rationale}] -- You suspect it's possible that one of the
  other members of the firm will come up with an idea about an
  application; you know that if a potential application is found, it
  may not be directly marketable, but at least there will be a
  prototype that can be concretely discussed.
\item[{\tt resolution}] -- Writing down and promulgating the
  \emph{Successful error} pattern using this template gives one such
  prototype.
\end{description}
\end{mdframed}
