\section{Serendipity in computational systems} \label{sec:computational-serendipity}

The 13 criteria from Section \ref{sec:literature-review} specify the
conditions and preconditions that are conducive to serendipitous
discovery.  These criteria have been further formalised
in Section \ref{specs-overview}.
% 
\citeA{pease2013discussion} used a slightly different version of the
SPECS criteria to discuss three examples of serendipitous behaviour:
in dynamic investigation problems, model generation, and poetry
flowcharts.  Two additional examples using the revised criteria are
described below.  These example serve the purpose of illustrating our
revised criteria, and also show forays of computational intelligence
into domains known for serendipity in their everyday cultural context.
We then turn to a more elaborated thought experiment that evaluates
these ideas in the course of developing a new system design.  First, a
bracketing remark.

\paragraph{Minimally serendipitous systems.}
According to our standards, there are various ways to achieve a result
with \emph{low} serendipity: if the observation was likely, if further
developments happened with little skill, and if the the value of the
result was low, then we would not say the outcome was serendipitous.
We would be prepared to attribute ``minimal serendipity'' to cases
where the observation was \emph{moderately} likely, \emph{some} skill
or effort was involved, and the result was only \emph{fairly good}.
For computational systems, if most of the skill involved lies with the
user, then there is little reason to call the system's operation
serendipitous -- even if it consistently does its job very well.  For
example, machines can learn to recognise instances of a given pattern
quite consistently, but it is an interesting surprise if a
computational system independently finds an entirely new kind of
pattern.  Furthermore, the position of the evaluator is important: a
spell-checking system might suggest a particularly fortuitous
substitution, but we would not expect the spell-checker to know when
it was being clever.  In such a case, we may say serendipity has
occurred, but not that we have a serendipitous system.

%% If the system learns an $N$th fact or
%% If applied to a system which could be described as minimally
%% serendipitous at best, and perhaps not at all serendipitous, does our
%% model identify the lack or presence of serendipity?  
%% %% As example, a spellchecker
%% %% program identifies spelling errors in text input and optionally can
%% %% correct spelling automatically. The only situation we can conceive of
%% %% where serendipity could possibly occur is tenuous; perhaps a suggested
%% %% correction may be incorrect, but may lead the user to interpret the
%% %% correction in an unexpected way. In all other aspects that we have
%% %% considered, spellchecker software would be a decidedly unlikely
%% %% candidate for harbouring serendipitous opportunities.  
%% Traditional spellchecker programs could be said to have a
%% \textbf{prepared mind}, in that they are constructed with internal
%% dictionaries with which to check spelling and ways of deciding what a
%% misspelled word might be.  Given our above discussion of how the
%% system might be serendipitous, the \textbf{serendipity trigger} could
%% be seen as the user misspelling a word and the system suggesting
%% alternative possibilities that the user had not previously conceived.
%% However, the \textbf{bridge} from trigger to serendipitous result (if
%% any) would have been built by the user, not by the system.  With
%% adaptive context-aware text completion tools, we can imagine a
%% ``Cyrano de Bergero'' effect in which the machine finds a
%% serendipitous bridge and offers the \textbf{result} to the user.
%% However, the current generation of text completion tools are known
%% more for infelicities than for exceptional wit.





\subsection{Case Studies: Prior art}

\paragraph{Evolutionary music improvisation systems.}

\citeA{jordanous10} reported a computational jazz improvisation system using genetic algorithms. Genetic algorithms, and evolutionary computing more generally, could encourage computational serendipity. We examine Jordanous's system (later given the name {\em GAmprovising} \cite{jordanous:12}) as a case study for evolutionary computing in the context of our model of computational serendipity: to what extent does GAmprovising model serendipity?

GAmprovising uses genetic algorithms to evolve a population of Improvisors. Each Improvisor is able to randomly generate music based on various parameters such as range of notes to be used, preferred notes to be used, rhythmic implications around note lengths and other musical parameters \cite<see>{jordanous10}. These parameters are what defines the Improvisor at any point in evolution. After a cycle of evolution, each Improvisor is evaluated via a fitness function based on Ritchie's \citeyear{ritchie07} criteria of how creative the Improvisor is. Ritchie's criteria use user-supplied ratings of how novel and how appropriate the music produced by the Improvisor is, to calculate 18 criteria that collectively evaluate how creative a system is. The most successful Improvisors (as deemed by the fitness function) are used to seed a new generation of Improvisors, through crossover and mutation operations.

The GAmprovising system can be said to have a \textbf{prepared mind} through its background knowledge of what musical knowledge to embed in the Improvisors and the evolutionary abilities to evolve Improvisors. A \textbf{serendipity trigger} comes from the combination of the mutation and crossover operations employed in the genetic algorithm, and the user input feeding into the fitness function to evaluate produced music. A \textbf{bridge}, from the genetic algorithm operations and user input, to the result is built by the creation of new Improvisors. The \textbf{results} are the various musical improvisations produced by the fittest Improvisors (as well as, perhaps, the parameters that have been considered fittest).
%
The likelihood of serendipitous evolution is greatly enhanced by the use of mutation and crossover operations within the genetic algorithm, to increase the diversity of search space covered by the system during evolution. However the \textbf{chance} of any particular Improvisor being discovered is low, given the massive dimensions of the search space.  Interesting developments in evolution would be guided by \textbf{curiosity} through the particular human user identifying Improvisors as interesting at that time. \textbf{Sagacity} is determined by how likely the user is to appreciate the same Improvisor's music (or similar music) over time, as tastes of the user may change. The \textbf{value} of the results are maximised through employing a fitness function.

Evolutionary systems such as GAmprovising necessarily operate in a \textbf{dynamic world} which is evolving continuously and may also be affected by changes in user tastes as they evaluate musical output from Improvisors. The \textbf{multiple contexts} arise from the possibility of having multiple users evaluate the musical output (though this is as yet not implemented formally) or through the user changing their preferences over time. \textbf{Multiple tasks} are carried out by the system including evolution of Improvisors, generation of music by individual Improvisors, capturing of user ratings of a sample of the Improvisors' output, and fitness calculations. \textbf{Multiple influences} are captured through the various combinations of parameters that could be set and the potential range of values for each parameter.


\paragraph{Recommender systems.} 

As discussed in Section \ref{sec:related}, recommender systems are one
of the primary contexts in computing where serendipity is seen to play
a role.  As we noted, these systems mostly focus on discovery.
Although this describes the mainstream of recommender system
development, it seems that there are some architectures that also take
account of invention.  We have in mind Bayesian methods (surveyed in
Chapter 3 of \citeNP{shengbo-guo-thesis}).  The current discussion
focuses on possibilities for serendipity on the system side, drawing
on the observation that recommender systems not \emph{stimulate}
serendipitous discovery: they also have the task of \emph{simulating}
when this is likely to occur.

A recommendation is typically provided if the system suspects that the
item will be likely to introduce ideas that are close to what the user
knows, but that will be unexpected.  Typical discussions of
serendipity in recommender systems focus on this.  However, user
behaviour (e.g.~following up on these recommendations) may also serve
as a \textbf{serendipity trigger} for the system, and change the way
it makes recommendations in the future.  Note that it is typically the
system's \emph{developers} who adapt the system; even in the Bayesian
case, the system has limited responsibilities.  Nevertheless, the
impetus to develop increasingly autonomous systems is present,
especially in complex domains where hand-tuning reaches its limits.
Current systems have at least the makings of a \textbf{prepared mind},
including both a \emph{user model} and a \emph{domain model}, both of
which can be updated dynamically.  A \textbf{bridge} to a new kind of
recommendation may be found by pattern matching, and especially by
looking for exceptional cases: new elements are introduced into the
domain which do not cluster well, or different clusters appear in the
user model that do not have obvious connections between them.  The
intended outcome of recommendations depends on the organisational
mission: to make money, to provide a good user experience, etc.  The
serendiptious \textbf{result} would be learning a new approach that
helps to address these goals.

\textbf{Chance} will only have a significant role in the system if it
has the capacity to learn from user behaviour.
%% The typical commercial perspective on recommendations is related to
%% the process of ``conversion'' -- turning recommendations into
%% clicks and clicks into purchases.
Combined with the ability to learn, \textbf{curiosity} could be
described as the urge to make
``outside-the-box''\footnote{\citeA{abbassi2009getting}.}
%%%
\begin{table}[ht!]
{\centering \renewcommand{\arraystretch}{1.5}
\footnotesize
\begin{tabular}{p{.7in}@{\hspace{.1in}}p{1.9in}@{\hspace{.1in}}p{1.9in}}
\multicolumn{1}{c}{} & \multicolumn{1}{c}{\textbf{Evolutionary music systems}} & \multicolumn{1}{c}{\textbf{Recommender systems}} \\[-.1in]
\multicolumn{1}{l}{\em Components} & \multicolumn{1}{c}{} & \multicolumn{1}{c}{} \\
\cline{2-3}
\textbf{Serendipity trigger} & Evolutionary operations and user input & Input from user behaviour \\
% \cline{2-3}
\textbf{Prepared mind} & Musical knowledge, evolution mechanisms & Through user/domain model \\
% \cline{2-3}
\textbf{Bridge}  & Creation of newly-evolved Improvisors & Elements identified outside clusters \\
% \cline{2-3}
\textbf{Result} & Music generated by fittest Improvisors& Dependent on organisation goals \\ \cline{2-3}
\multicolumn{1}{l}{\em Dimensions} & \multicolumn{1}{c}{} & \multicolumn{1}{c}{} \\
\cline{2-3}
\textbf{Chance} & If discovered in huge search space & If learning from user behaviour \\
% \cline{2-3}
\textbf{Curiosity} & If a particular user notes an Improvisor & Making unusual recommendations \\
% \cline{2-3}
\textbf{Sagacity} & User appreciation of Improvisor over time & Updating models after user behaviour \\
% \cline{2-3}
\textbf{Value} & Via fitness function (proxy of creativity) & As per business metrics/objectives \\
\cline{2-3}
\multicolumn{1}{l}{\em Factors} & \multicolumn{1}{c}{} & \multicolumn{1}{c}{} \\
\cline{2-3}
\textbf{Dynamic world}  & Continuous computational evolution and changes in user tastes& As precondition for testing system's influences on user behaviour\\
%\cline{2-3}
\textbf{Multiple contexts} & Multiple users opinions? & User model and domain model\\
% \cline{2-3}
\textbf{Multiple tasks} & Evolving Improvisors, generating music, collecting user input, fitness calculations & Making recommendations, learning from users, updating models \\
% \cline{2-3}
\textbf{Multiple influences} & Through various musical parameter combinations& Experimental design, psychology, domain understanding\\
\cline{2-3}
\end{tabular}
\par}
\bigskip
\caption{Summary: applying computational serendipity model to positive case studies\label{caseStudies}}
\end{table}%
\normalsize
%%%
recommendations specifically for the purposes of learning more about
users, possibly to the detriment of other metrics over the short term.
Measures of \textbf{sagacity} would relate to the system's ability to
draw inferences from user behaviour.  For example, the system might do
A/B testing to decide how novel recommendation strategies influence
conversion.  Again, currently this would typically be organised by the
system's developers.  The \textbf{value} of recommendation strategies
can be measured in terms of traditional business metrics or other
organisational objectives.

A \textbf{dynamic world} which nevertheless exhibits some regularity
is a precondition for useful A/B testing.  As mentioned above the
primary \textbf{(multiple) contexts} are the user model and the domain
model.  A system matching the description here would have
\textbf{multiple tasks}: making useful recommendations, generating new
experiments to learn about users, and building new models.  Such a
system could avail itself of \textbf{multiple influences} related to
experimental design, psychology, and domain understanding.

% As a general comment, we would say that this is largely how
% \emph{research and development} of recommender systems works, but
% without the same levels of system automony envisioned here.




