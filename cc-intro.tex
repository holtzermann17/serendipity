\section{Serendipity in a computational context} \label{sec:computational-serendipity}

The 13 criteria from Section \ref{sec:literature-review} specify the
conditions and preconditions that are conducive to serendipitous
discovery.  Here, we revisit each of these criteria and briefly
summarise how they can be thought about from a computational point of
view, again focusing on examples.  We then present a thought
experiment that evaluates the ideas described above in the course of
developing a new system design.

% \subsection{Proposed experiment: A Writers Workshop for Systems} \label{sec:writers-workshop}

Richard Gabriel \cite{gabriel2002writer} describes the practise of
Writers Workshops that has been put to use for over a decade within
the Pattern Languages of Programming (PLoP) community.  The basic
style of collaboration originated much earlier with groups of literary
authors who engage in peer-group critique.  Some literary workshops
are open as to genre, and happy to accommodate beginners, like the
Minneapolis Writers
Workshop\footnote{\url{http://mnwriters.org/how-the-game-works/}};
others are focused on professionals working within a specific genre,
like the Milford Writers
Workshop\footnote{\url{http://www.milfordsf.co.uk/about.htm}}.  The
practices that Gabriel describes are fairly typical.  Authors come
with work ready to present, and read a short sample, which is then
discussed and constructively critiqued by attendees.  Presenting
authors are not permitted to rebut these comments.  The commentators
generally summarise the work and say what they have gotten out of it,
discuss what worked well in the piece, and talk about how it could be
improved.  The author listens and may take notes; at the end, he or
she can then ask questions for clarification.  Generally, non-authors
are either not permitted to attend, or are asked to stay silent
through the workshop, and perhaps sit separately from the
participating authors/reviewers.  There are similarities between the
Writers Workshops and classical practices of group composition
\cite{jin1975art} and dialectic \cite{dialectique}, and the workshop
may be considered an artistic or creative space in its own right.

In PLoP workshops, authors present design patterns and pattern
languages, or papers about patterns, rather than more traditional
literary forms like poems, stories, or chapters from novels.  Papers
must be workshopped at a PLoP or EuroPLoP conference in order to be
considered for the \emph{Transactions on Pattern Languages of
  Programming} journal.  A discussion of writers workshops
in the language of design patterns is presented by
Coplien and Woolf \cite{coplien1997pattern}.  Their patterns include:
\begin{center}
{\small
\begin{tabular}{l@{\hspace{.2cm}}l@{\hspace{.2cm}}l}
\emph{Open Review} & \emph{Safe Setting} & \emph{Workshop Comprises Authors} \\
\emph{Authors are Experts} & \emph{Community of Trust} & \emph{Moderator Guides the Workshop} \\
\emph{Thank the Author} & \emph{Selective Changes} & \emph{Clearing the Palate} \\
\end{tabular}
}
\end{center}

We propose that a similar pattern-based approach should be deployed
within the Computational Creativity community to design a workshop in
which the participants are computer systems instead of human authors.
The annual International Conference on Computational Creativity
(ICCC), now entering its sixth year, could be a suitable venue.
Rather than the system's creator presenting the system in a
traditional slideshow and discussion, or a system ``Show and Tell,''
the systems would be brought to the workshop and would present their
own work to an audience of other systems, in a Writers Workshop
format.  This might be accompanied by a short paper for the conference
proceedings written by the system's designer describing the system's
current capabilities and goals.  Subsequent publications might include
traces of interactions in the Workshop, commentary from the system on
other systems, and offline reflections on what the system might change
about its own work based on the feedback it receives.  As in the PLoP
community, it could become standard to incorporate this sort of workshop
into the process of peer reviewing journal articles for the new \emph{Journal of
  Computational Creativity}\footnote{\url{http://www.journalofcomputationalcreativity.cc}}.

\begin{table}[p]
\begin{tabular}{lp{.7\textwidth}}
{\bf\emph{Successful error}} & \\
\emph{Van Andel's example}: & Post-it\texttrademark\ notes \\[.2cm]
{\tt presentation}& Systems should be prepared to share interesting ideas even if they don't know directly how they will be useful.  \\
{\tt listening}   & Systems should listen with interest, too. \\
{\tt feedback}    & Even interesting ideas may not be ``marketable.''\\
{\tt questions}   & How is your suggestion useful? \\
{\tt reflections} & New combinations of ideas take a long time to realise, and many different ideas may need to be combined in order to come up with something useful.\\
\end{tabular}
\bigskip

\begin{tabular}{lp{.7\textwidth}}
{\bf\emph{Side effect}} & \\
\emph{Van Andel's example}: & Nicotinamide used to treat side-effects of radiation therapy proves efficacious against tuberculosis. \\[.2cm]
{\tt presentation}& Systems should use their presentation as an experiment. \\
{\tt listening}   & Listeners should allow themselves to be affected by what they are hearing. \\
{\tt feedback}    & Feedback should convey the nature of the effect.\\
{\tt questions}   & The presenter may need to ask follow-up questions to gain insight. \\
{\tt reflections} & Form a new hypothesis before seeking a new audience. \\
\end{tabular}
\bigskip

\begin{tabular}{lp{.7\textwidth}}
{\bf\emph{Wrong hypothesis}} & \\
\emph{Van Andel's example}: & Lithium, used in a control study, had an unexpected calming effect. \\[.2cm]
{\tt presentation}& How is this presentation interpretable as a (``natural'') control study? \\
{\tt listening}   & Listeners are ``guinea pigs''.\\
{\tt feedback}    & Discuss side-effects that do not necessarily correspond to the author's perceived intent. \\
{\tt questions}   & Zero in on the most interesting part of the conversation.\\
{\tt reflections} & Revise hypotheses to correspond to the most surprising feedback. \\
\end{tabular}
\bigskip

\begin{tabular}{lp{.7\textwidth}}
{\bf\emph{Outsider}} & \\
\emph{Van Andel's example}: & A mother suggests a new hypothesis to a doctor. \\[.2cm]
{\tt presentation}& The presenter is here to learn from the audience. \\
{\tt listening}   & The audience is here to give help, but also to get help.\\
{\tt feedback}    & Feedback will inevitably draw on previous experiences and ideas.\\
{\tt questions}   & What is the basis for that remark?\\
{\tt reflections} & How can I implement the suggestions?\\
\end{tabular}
\vspace{.2cm}
\caption{Reinterpreting patterns of serendipity for use in a computational workshop\label{tab:reinterpret}}
\end{table}

\begin{figure}[t]
\begin{center}
\resizebox{.93\textwidth}{!}{
\StickyNote[2.5cm]{myyellow}{{\LARGE {Interesting idea}} \\[4ex] {Surprise birthday party}}[3.8cm] \StickyNote[2.5cm]{mygreen}{{\Large I heard you say:} \\[4ex] {``surprise''} }[3.8cm]
\StickyNote[2.5cm]{pink}{{\Large Feedback:} \\[4ex] {I don't like surprises}}[3.8cm]
}
\resizebox{.61\textwidth}{!}{
\StickyNote[2.5cm]{myorange}{{\LARGE {Question}} \\[4ex] {Not even a little bit?\ldots}}[3.8cm]
\quad \raisebox{-.2cm}{\StickyNote[2.5cm]{myblue}{{\LARGE Note to self:} \\[4ex] {(Try smaller surprises \\ next time.)}}[3.8cm]}
}
\end{center}
\caption{A paper prototype for applying the \emph{Successful Error} pattern\label{fig:paper-prototype}}
\end{figure}

In order to facilitate this sort of interaction, it would be necessary
for systems to implement a basic protocol related to
%%
\[
\text{
{\tt presentation}, {\tt listening}, {\tt
  feedback}, {\tt questions}, and {\tt
  reflections}.}
\]
%%
This protocol could be thought of as a light-weight template for
creating design patterns that guide system-level participation in the
context specified by Coplien and Woolf's pattern language for writers
workshops.  Table \ref{tab:reinterpret} uses this framework to recast
the four ``perfectly'' serendipitous patterns from van Andel --
\emph{Successful error}, \emph{Side effect}, \emph{Wrong hypothesis},
and \emph{Outsider} -- in a form that may make them useful to
developers preparing to enter their systems into the Workshop.
%
Further guidelines for structuring and participating in traditional
writers workshops are presented by Linda Elkin in
\cite[pp. 201-203]{gabriel2002writer}.  It is not at all clear that
the same ground rules should apply to computer systems.  For example,
one of Elkin's rules is that ``Quips, jokes, or sarcastic comments,
even if kindly meant, are inappropriate.''  Rather than forbidding
humour, it may be better for individual comments to be rated as
helpful or non-helpful.  Again, since serendipitous discovery is an
overarching goal, in the first instance, usefulness and interest might
be judged in terms of the criteria described in Section
\ref{sec:evaluation-criteria}.

We would need a neutral environment that is not hard to develop for:
the {\sf FloWr} system described in Section \ref{sec:foundations}
offers one such possibility.  With this system, the basic operating
logic of the Workshop could be spelled out as a flowchart, and
contributing systems could use flowcharts as the basic medium for
sharing their presentations, feedback, and questions.  Developing
around a process language of this sort partially obviates the need for
participating systems to have strong natural language processing
capabilities.  
%
Post-it\texttrademark\ notes, which have provided us with a useful
example of serendipitous discovery, also provide indicative strategies
from the world of paper prototyping (Figure \ref{fig:paper-prototype}).

Gordon Pask's conversation theory, reviewed in
\cite{conversation-theory-review,boyd2004conversation}, goes
considerably beyond what we have presented here as a simple process
language, although there are structural parallels.  In a basic
Pask-style learning conversation: (0) Conversational participants are
carrying out some actions and observations; (1) naming and recording
what action is being done; (2) asking and explaining why it works the
way it does; (3) carrying out higher-order methodological discussion;
and (4) trying to figure out why unexpected results occured \cite[p. 190]{boyd2004conversation}.

Naturally, variations to the underlying system, protocol, and the
schedule of events should be considered depending on the needs and
interests of participants, and several variants can be tried.  On a
pragmatic basis, if the Workshop proved quite useful to participants,
it could be revised to run monthly, weekly, or
continuously.\footnote{For a comparison case in computer Go, see
  \url{http://cgos.computergo.org/}.}


\subsection{Prior partial examples}

\paragraph{{[}To add: Jazz.{]}}

\paragraph{{[}To add: HR.{]}}

\paragraph{Recommender systems.} Research in recommender systems focusses on stimulating serendipity on the user side, by suggesting items that might more be unexpected, novel, and valuable to the user. 
% Several components of the proposed model were implemented in related work, and art and music recommendation connects to the area of computational creativity.

A recommender system must come with a \textbf{prepared mind} in terms of full knowledge of the items in the search space, and of the user's partial knowledge of their existence and properties. Predictions are based on existing knowledge of the items known to the user, and his or her preferences. The system's goal is to recommend an item as \textbf{result} which is unexpected, novel, and valuable to a specific user. % while the user's goal is to find items that are valuable in different respects. 

Related work tries to structure the search space and exploit patterns as \textbf{serendipity triggers}.  For example, \cite{Herlocker2004} as well as \cite{Lu2012} associate less popular items with a higher unexpectedness. Clustering was also frequently used to discover latent structures in the search space. For example, \cite{Kamahara2005} partition users into clusters of common interest, while \cite{Onuma2009} as well as \cite{Zhang2011} perform clustering on both users and items. In the work by \cite{Oku2011}, the user is allowed to select two items in order to mix their features in some sort of conceptual blending. 

If a pattern is found, it is used to \textbf{bridge} between items that are known and valuable to the user, and those that are potentially unexpected.  As an example,  \cite{Sugiyama2011} connects users with divergent interests, while \cite{Onuma2009} weight items stronger that bridge between topical clusters. 

Recommender systems have to cope with a \textbf{dynamic world} of user preferences and new items that are introduced to the system. The imperfect knowledge about the user's preferences and interests represents perhaps the strongest dimension of \textbf{chance}.  Determining the \textbf{value} of an item, both in terms of value for the user and unexpectedness, is of paramount importance.  
