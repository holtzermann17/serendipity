\section{Serendipity in computational systems} \label{sec:computational-serendipity}

The 13 criteria from Section \ref{sec:literature-review} specify the
conditions and preconditions that are conducive to serendipitous
discovery.  These criteria have been further formalised
in Section \ref{specs-overview}.
% 
\citeA{pease2013discussion} used a slightly different version of the
SPECS criteria to discuss three examples of serendipitous behaviour:
in dynamic investigation problems, model generation, and poetry
flowcharts.  Two additional examples using the revised criteria are
described below.  These example serve the purpose of illustrating our
revised criteria, and also show forays of computational intelligence
into domains known for serendipity in their everyday cultural context.
We then turn to a more elaborated thought experiment that evaluates
these ideas in the course of developing a new system design.

% \subsection{Proposed experiment: A Writers Workshop for Systems} \label{sec:writers-workshop}

Richard Gabriel \cite{gabriel2002writer} describes the practise of
Writers Workshops that has been put to use for over a decade within
the Pattern Languages of Programming (PLoP) community.  The basic
style of collaboration originated much earlier with groups of literary
authors who engage in peer-group critique.  Some literary workshops
are open as to genre, and happy to accommodate beginners, like the
Minneapolis Writers
Workshop\footnote{\url{http://mnwriters.org/how-the-game-works/}};
others are focused on professionals working within a specific genre,
like the Milford Writers
Workshop\footnote{\url{http://www.milfordsf.co.uk/about.htm}}.  The
practices that Gabriel describes are fairly typical.  Authors come
with work ready to present, and read a short sample, which is then
discussed and constructively critiqued by attendees.  Presenting
authors are not permitted to rebut these comments.  The commentators
generally summarise the work and say what they have gotten out of it,
discuss what worked well in the piece, and talk about how it could be
improved.  The author listens and may take notes; at the end, he or
she can then ask questions for clarification.  Generally, non-authors
are either not permitted to attend, or are asked to stay silent
through the workshop, and perhaps sit separately from the
participating authors/reviewers.  There are similarities between the
Writers Workshops and classical practices of group composition
\cite{jin1975art} and dialectic \cite{dialectique}, and the workshop
may be considered an artistic or creative space in its own right.

In PLoP workshops, authors present design patterns and pattern
languages, or papers about patterns, rather than more traditional
literary forms like poems, stories, or chapters from novels.  Papers
must be workshopped at a PLoP or EuroPLoP conference in order to be
considered for the \emph{Transactions on Pattern Languages of
  Programming} journal.  A discussion of writers workshops
in the language of design patterns is presented by
Coplien and Woolf \cite{coplien1997pattern}.  Their patterns include:
\begin{center}
{\small
\begin{tabular}{l@{\hspace{.2cm}}l@{\hspace{.2cm}}l}
\emph{Open Review} & \emph{Safe Setting} & \emph{Workshop Comprises Authors} \\
\emph{Authors are Experts} & \emph{Community of Trust} & \emph{Moderator Guides the Workshop} \\
\emph{Thank the Author} & \emph{Selective Changes} & \emph{Clearing the Palate} \\
\end{tabular}
}
\end{center}

We propose that a similar pattern-based approach should be deployed
within the Computational Creativity community to design a workshop in
which the participants are computer systems instead of human authors.
The annual International Conference on Computational Creativity
(ICCC), now entering its sixth year, could be a suitable venue.
Rather than the system's creator presenting the system in a
traditional slideshow and discussion, or a system ``Show and Tell,''
the systems would be brought to the workshop and would present their
own work to an audience of other systems, in a Writers Workshop
format.  This might be accompanied by a short paper for the conference
proceedings written by the system's designer describing the system's
current capabilities and goals.  Subsequent publications might include
traces of interactions in the Workshop, commentary from the system on
other systems, and offline reflections on what the system might change
about its own work based on the feedback it receives.  As in the PLoP
community, it could become standard to incorporate this sort of workshop
into the process of peer reviewing journal articles for the new \emph{Journal of
  Computational Creativity}\footnote{\url{http://www.journalofcomputationalcreativity.cc}}.

\begin{table}[p]
\begin{tabular}{lp{.7\textwidth}}
{\bf\emph{Successful error}} & \\
\emph{Van Andel's example}: & Post-it\texttrademark\ notes \\[.2cm]
{\tt presentation}& Systems should be prepared to share interesting ideas even if they don't know directly how they will be useful.  \\
{\tt listening}   & Systems should listen with interest, too. \\
{\tt feedback}    & Even interesting ideas may not be ``marketable.''\\
{\tt questions}   & How is your suggestion useful? \\
{\tt reflections} & New combinations of ideas take a long time to realise, and many different ideas may need to be combined in order to come up with something useful.\\
\end{tabular}
\bigskip

\begin{tabular}{lp{.7\textwidth}}
{\bf\emph{Side effect}} & \\
\emph{Van Andel's example}: & Nicotinamide used to treat side-effects of radiation therapy proves efficacious against tuberculosis. \\[.2cm]
{\tt presentation}& Systems should use their presentation as an experiment. \\
{\tt listening}   & Listeners should allow themselves to be affected by what they are hearing. \\
{\tt feedback}    & Feedback should convey the nature of the effect.\\
{\tt questions}   & The presenter may need to ask follow-up questions to gain insight. \\
{\tt reflections} & Form a new hypothesis before seeking a new audience. \\
\end{tabular}
\bigskip

\begin{tabular}{lp{.7\textwidth}}
{\bf\emph{Wrong hypothesis}} & \\
\emph{Van Andel's example}: & Lithium, used in a control study, had an unexpected calming effect. \\[.2cm]
{\tt presentation}& How is this presentation interpretable as a (``natural'') control study? \\
{\tt listening}   & Listeners are ``guinea pigs''.\\
{\tt feedback}    & Discuss side-effects that do not necessarily correspond to the author's perceived intent. \\
{\tt questions}   & Zero in on the most interesting part of the conversation.\\
{\tt reflections} & Revise hypotheses to correspond to the most surprising feedback. \\
\end{tabular}
\bigskip

\begin{tabular}{lp{.7\textwidth}}
{\bf\emph{Outsider}} & \\
\emph{Van Andel's example}: & A mother suggests a new hypothesis to a doctor. \\[.2cm]
{\tt presentation}& The presenter is here to learn from the audience. \\
{\tt listening}   & The audience is here to give help, but also to get help.\\
{\tt feedback}    & Feedback will inevitably draw on previous experiences and ideas.\\
{\tt questions}   & What is the basis for that remark?\\
{\tt reflections} & How can I implement the suggestions?\\
\end{tabular}
\vspace{.2cm}
\caption{Reinterpreting patterns of serendipity for use in a computational workshop\label{tab:reinterpret}}
\end{table}

\begin{figure}[t]
\begin{center}
\resizebox{.93\textwidth}{!}{
\StickyNote[2.5cm]{myyellow}{{\LARGE {Interesting idea}} \\[4ex] {Surprise birthday party}}[3.8cm] \StickyNote[2.5cm]{mygreen}{{\Large I heard you say:} \\[4ex] {``surprise''} }[3.8cm]
\StickyNote[2.5cm]{pink}{{\Large Feedback:} \\[4ex] {I don't like surprises}}[3.8cm]
}
\resizebox{.61\textwidth}{!}{
\StickyNote[2.5cm]{myorange}{{\LARGE {Question}} \\[4ex] {Not even a little bit?\ldots}}[3.8cm]
\quad \raisebox{-.2cm}{\StickyNote[2.5cm]{myblue}{{\LARGE Note to self:} \\[4ex] {(Try smaller surprises \\ next time.)}}[3.8cm]}
}
\end{center}
\caption{A paper prototype for applying the \emph{Successful Error} pattern\label{fig:paper-prototype}}
\end{figure}

In order to facilitate this sort of interaction, it would be necessary
for systems to implement a basic protocol related to
%%
\[
\text{
{\tt presentation}, {\tt listening}, {\tt
  feedback}, {\tt questions}, and {\tt
  reflections}.}
\]
%%
This protocol could be thought of as a light-weight template for
creating design patterns that guide system-level participation in the
context specified by Coplien and Woolf's pattern language for writers
workshops.  Table \ref{tab:reinterpret} uses this framework to recast
the four ``perfectly'' serendipitous patterns from van Andel --
\emph{Successful error}, \emph{Side effect}, \emph{Wrong hypothesis},
and \emph{Outsider} -- in a form that may make them useful to
developers preparing to enter their systems into the Workshop.
%
Further guidelines for structuring and participating in traditional
writers workshops are presented by Linda Elkin in
\cite[pp. 201-203]{gabriel2002writer}.  It is not at all clear that
the same ground rules should apply to computer systems.  For example,
one of Elkin's rules is that ``Quips, jokes, or sarcastic comments,
even if kindly meant, are inappropriate.''  Rather than forbidding
humour, it may be better for individual comments to be rated as
helpful or non-helpful.  Again, since serendipitous discovery is an
overarching goal, in the first instance, usefulness and interest might
be judged in terms of the criteria described in Section
\ref{sec:evaluation-criteria}.

We would need a neutral environment that is not hard to develop for:
the {\sf FloWr} system described in Section \ref{sec:foundations}
offers one such possibility.  With this system, the basic operating
logic of the Workshop could be spelled out as a flowchart, and
contributing systems could use flowcharts as the basic medium for
sharing their presentations, feedback, and questions.  Developing
around a process language of this sort partially obviates the need for
participating systems to have strong natural language processing
capabilities.  
%
Post-it\texttrademark\ notes, which have provided us with a useful
example of serendipitous discovery, also provide indicative strategies
from the world of paper prototyping (Figure \ref{fig:paper-prototype}).

Gordon Pask's conversation theory, reviewed in
\cite{conversation-theory-review,boyd2004conversation}, goes
considerably beyond what we have presented here as a simple process
language, although there are structural parallels.  In a basic
Pask-style learning conversation: (0) Conversational participants are
carrying out some actions and observations; (1) naming and recording
what action is being done; (2) asking and explaining why it works the
way it does; (3) carrying out higher-order methodological discussion;
and (4) trying to figure out why unexpected results occured \cite[p. 190]{boyd2004conversation}.

Naturally, variations to the underlying system, protocol, and the
schedule of events should be considered depending on the needs and
interests of participants, and several variants can be tried.  On a
pragmatic basis, if the Workshop proved quite useful to participants,
it could be revised to run monthly, weekly, or
continuously.\footnote{For a comparison case in computer Go, see
  \url{http://cgos.computergo.org/}.}


\subsection{Case Studies: Prior art}

\paragraph{Evolutionary music improvisation systems.}

\citeA{jordanous10} reported a computational jazz improvisation system using genetic algorithms. Genetic algorithms, and evolutionary computing more generally, could encourage computational serendipity. We examine Jordanous's system (later given the name {\em GAmprovising} \cite{jordanous:12}) as a case study for evolutionary computing in the context of our model of computational serendipity: to what extent does GAmprovising model serendipity?

GAmprovising uses genetic algorithms to evolve a population of Improvisors. Each Improvisor is able to randomly generate music based on various parameters such as range of notes to be used, preferred notes to be used, rhythmic implications around note lengths and other musical parameters (see \cite{jordanous10}. These parameters are what defines the Improvisor at any point in evolution. After a cycle of evolution, each Improvisor is evaluated via a fitness function based on Ritchie's criteria \cite{ritchie07} of how creative the Improvisor is. Ritchie's criteria use user-supplied ratings of how novel and how appropriate the music produced by the Improvisor is, to calculate 18 criteria that collectively evaluate how creative a system is. The most successful Improvisors (as deemed by the fitness function) are used to seed a new generation of Improvisors, through crossover and mutation operations.

The GAmprovising system can be said to have a \textbf{prepared mind} through its background knowledge of what musical knowledge to embed in the Improvisors and the evolutionary abilities to evolve Improvisors. A \textbf{serendipity trigger} comes from the combination of the mutation and crossover operations employed in the genetic algorithm, and the user input feeding into the fitness function to evaluate produced music. A \textbf{bridge}, from the genetic algorithm operations and user input, to the result is built by the creation of new Improvisors. The \textbf{results} are the various musical improvisations produced by the fittest Improvisors (as well as, perhaps, the parameters that have been considered fittest).

The likelihood of serendipitous evolution is greatly enhanced by the use of mutation and crossover operations within the genetic algorithm, to increase the diversity of search space covered by the system during evolution. However the \textbf{chance} of any particular Improvisor being discovered is low, given the massive dimensions of the search space.  Interesting developments in evolution would be guided by \textbf{curiosity} through the particular human user identifying Improvisors as interesting at that time. \textbf{Sagacity} is determined by how likely the user is to appreciate the same Improvisor's music (or similar music) over time, as tastes of the user may change. The \textbf{value} of the results are maximised through employing a fitness function.

Evolutionary systems such as GAmprovising necessarily operate in a \textbf{dynamic world} which is evolving continuously and may also be affected by changes in user tastes as they evaluate musical output from Improvisors. The \textbf{multiple contexts} arise from the possibility of having multiple users evaluate the musical output (though this is as yet not implemented formally) or through the user changing their preferences over time. \textbf{Multiple tasks} are carried out by the system including evolution of Improvisors, generation of music by individual Improvisors, capturing of user ratings of a sample of the Improvisors' output, and fitness calculations. \textbf{Multiple influences} are captured through the various combinations of parameters that could be set and the potential range of values for each parameter.


\paragraph{Recommender systems.} 

As discussed in Section \ref{sec:related}, recommender systems are one
of the primary contexts in computing where serendipity is seen to play
a role.  As we noted, these systems mostly focus on discovery.
Nevertheless, certain architectures that also take account of
invention would match all of criteria described by our model.  Here we
draw on the observation that recommender systems not \emph{stimulate}
serendipitous discovery for the user: they also have the task of
\emph{simulating} when this is likely to occur.

A recommendation is typically provided if the system suspects that the
item will be likely to introduce ideas that are close to what the user
knows, but that will be unexpected.  In other words, the system aims
to stimulate serendipity for the user. For example, a museum
recommender service might suggest a colourful medieval painting to a
user who seems to like colourful paintings by the modern artist Keith
Haring.  User behaviour (e.g.~following up on these recommendations)
is outside of the direct control of the system and may serve as a
\textbf{serendipity trigger}, and change the way it makes
recommendations in the future.  The system has a \textbf{prepared
  mind}, including both a \emph{user model} and a \emph{domain model},
both of which can be updated dynamically.  The connections through
which recommendations are made usually happen when the system notices
that elements of the domain have something in common via clustering or
faceting.  A \textbf{bridge} to a new kind of recommendation may be
found if new elements are introduced into the domain which do not
cluster well, or if the user appears to know about different clusters
that do not have obvious connections between them.  The intended
outcome of recommendations depend on the organisational mission
e.g.~to make money, to provide a good user experience, etc.; at the
system level, the serendiptious \textbf{result} would be learning a
new approach that helps to address these goals better.

From the perspective of our model, \textbf{chance} will only have a
significant role when the system has the capacity to learn from user
behaviour.  In fact, Bayesian methods are used in contemporary
recommender systems (surveyed in Chapter 3 of
\citeNP{shengbo-guo-thesis}).  The typical commercial perspective on
recommendations is related to the process of ``conversion'' -- turning
recommendations into clicks and clicks into purchases.  Combined with
the ability to learn, \textbf{curiosity} could be described as the
urge to make ``outside-the-box''\footnote{\citeA{abbassi2009getting}.}
recommendations specifically for the purposes of learning more about
users, possibly to the detriment of other goals over the short term.
Measures of \textbf{sagacity} would relate to the system's ability to
draw inferences from user behaviour that would update the
recommendation model.  For example, the system might do A/B testing to
decide how novel recommendation strategies influence conversion.  The
\textbf{value} of recommendation strategies can be measured in terms
of traditional business metrics or other organisational objectives.

A \textbf{dynamic world} which nevertheless exhibits some regularity
is a precondition for useful A/B testing.  As mentioned above the
primary \textbf{(multiple) contexts} are the user model and the domain
model.  A system matching the description here would have
\textbf{multiple tasks}: making useful recommendations, generating new
experiments to learn about users, and building new models based on the
results of these experiments.  Such a system could avail itself of
\textbf{multiple influences} related to experimental design,
psychology, and domain understanding.

% As a general comment, we would say that this is largely how
% \emph{research and development} of recommender systems works, but
% without the same levels of system automony envisioned here.
\begin{table}[ht]%dp]
\caption{Summary: applying computational serendipity model to positive case studies}
\begin{center}
\footnotesize
\begin{tabular}{|c|l|l|}
\hline
  & Evolutionary music systems & Recommender systems \\
\hline
\hline
%{\em Key Condition}  && \\
%Focus shift && \\
%\hline
{\em Components} && \\
\hline
\hline
Serendipity trigger & Evolutionary operations and user input & Input from user behaviour \\
\hline
Prepared mind  & Musical knowledge, evolution mechanisms & Through user model/domain model \\
\hline
Bridge  & Creation of newly-evolved Improvisors & Elements identified outside clusters \\
\hline
Result & Music generated by fittest Improvisors& Dependent on organisation goals \\
\hline
\hline
{\em Dimensions} && \\
\hline
\hline
Chance & If discovered in huge search space & If learning from user behaviour \\
\hline
Curiosity & If a particular user notes an Improvisor & Making unusual recommendations \\
\hline
Sagacity & User appreciation of Improvisor over time & Updating models after user behaviour \\
\hline
Value & Via fitness function (proxy of creativity) & As per business metrics/objectives \\
\hline
\hline
{\em Environmental} && \\
{\em Factors} && \\
\hline
\hline
Dynamic & Continuous computational evolution& As precondition for testing system's \\
 world & \hspace{3mm}and changes in user tastes & \hspace{3mm} influences on user behaviour\\
\hline
Multiple & Multiple & User model and domain model\\
contexts & \hspace{3mm}users opinions? & \\
\hline
Multiple & Evolving Improvisors, generating music,  & Making recommendations, learning\\
 tasks & \hspace{3mm}collecting user input, fitness calculations& \hspace{3mm}from users, updating models \\
\hline
Multiple & Through various musical& Experimental design, psychology, \\
influences &\hspace{3mm}parameter combinations& \hspace{3mm} domain understanding\\
\hline
\end{tabular}
\end{center}
\label{caseStudies}
\end{table}%
\normalsize


\paragraph{Minimally serendipitous systems}

If applied to a system which could be described as minimally serendipitous at best, and perhaps not at all serendipitous, does our model identify the lack of serendipity?

As example, a spellchecker program identifies spelling errors in text input and optionally can correct spelling automatically. The only situation we can conceive of where serendipity could possibly occur is tenuous; perhaps a suggested correction may be incorrect, but may lead the user to interpret the correction in an unexpected way. In all other aspects that we have considered, spellchecker software would be a decidedly unlikely candidate for harbouring serendipitous opportunities.

Spellchecker programs could indeed be said to have a \textbf{prepared mind}, in that they are constructed with internal dictionaries with which to check spelling and ways of calculating what a misspelled word might be. Given our above discussion of how the system might be serendipitous, the \textbf{serendipity trigger} could be seen as the user misspelling a word and the system suggesting a correction (which might cause the user to think of alternative possibilities for their text that they had not previously conceived). The \textbf{result} of this serendipity would be new text typed in by the user in response to the serendipity trigger, but the \textbf{bridge} from trigger to result would have been built by the user, not the system. So all components in our model are not present in spellchecker software, as the ability to create a bridge from trigger to result is absent in the system. At this point we do not need to look further at the dimensions, or attributes of these components, because not all components are present in spellchecker software. 

% \paragraph{{[}To add: HR.{]}} *** AJ PERHAPS WE DON'T NEED TO?