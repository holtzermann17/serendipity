\subsubsection*{Step 2: Evaluation standards for computational serendipity}
\begin{quote} {\em Using Step 1, clearly state what standards you use to evaluate the serendipity of your
    system. }\end{quote}

\noindent With our definition and other features of the model in mind, we propose the following standards for evaluating serendipity in computational systems. These criteria allow the evaluator to assess the degree of seredipity that is present in a given system's operation.

%% Serendipity relies on a reassessment or reevaluation -- a \emph{focus shift} in which something that was previously uninteresting, of neutral, or even negative value, becomes interesting.

\begin{description}[itemsep=4pt]
\item[\emph{(\textbf{A - Definitional characteristics})}] \emph{The
  system can be said to have a \emph{\textbf{prepared mind}},
  consisting of previous experiences, background knowledge, a store of
  unsolved problems, skills, expectations, and (optionally) a current
  focus or goal.  It then processes a \emph{\textbf{serendipity
  trigger}} that is at least partially the result of factors outside
  of its control, including randomness or unexpected events.  The
  system then uses reasoning techniques and/or social or otherwise
  externally enacted alternatives to create a \emph{\textbf{bridge}}
  from the trigger to a result.  The \emph{\textbf{result}} is
  evaluated as useful, by the system and/or by an external source.}
\item[\emph{(\textbf{B - Dimensions})}] \emph{Serendipity, and its
  various dimensions, can be present to a greater or lesser degree.
  If the criteria above have been met, we consider the system (and optionally, generate ratings as
  estimated probabilities) along several dimensions:
%
\emph{($\mathbf{a}$: \textbf{chance})} how likely was this trigger to appear to
  the system?
%
\emph{($\mathbf{b}$: \textbf{curiosity})} On a population basis, comparing
  similar circumstances, how likely was the trigger to be identified
  as interesting?
%
\emph{($\mathbf{c}$: \textbf{sagacity})} On a population basis, comparing
  similar circumstances, how likely was it that the trigger
  would be turned into a result?
%
Finally, we ask, again, comparing similar results where possible:
\emph{($\mathbf{d}$: \textbf{value})} How valuable is the result that
is ultimately produced?}
%
%Then combining $\mathbf{a}\times\mathbf{b}\times\mathbf{c}$ gives a
 % likelihood score: 
\emph{Low likelihood $\mathbf{a}\times\mathbf{b}\times\mathbf{c}$ 
 and high value $\mathbf{d}$ are the criteria we use to say that the event was ``highly serendipitous.''}

\item[\emph{(\textbf{C - Factors})}] \emph{Finally, if the criteria
  from Part A are met, and if the event is deemed ``highly
  serendipitous'' according to the criteria in Part B, then in order
  to deepen our qualitative understanding of the serendipitous
  behaviour, we ask: To what extent does the system exist in a
  \emph{\textbf{dynamic world}}, spanning \emph{\textbf{multiple
      contexts}}, featuring \emph{\textbf{multiple tasks}}, and
  incorporating \emph{\textbf{multiple influences}}?}
\end{description}

\subsubsection*{Step 3: Testing our serendipitous system}

\begin{quote} {\em Test your serendipitous system against the standards stated in Step 2 and report the
results.}\end{quote}

\noindent In Section \ref{sec:computational-serendipity} we pilot our framework by examining the degree of serendipity of existing computational systems and looking for ways that their serendipity could be enhanced.  We will also use the framework to guide the high-level design of a novel system. 

