
\begin{abstract}
There is relatively little prior work dealing with serendipity in a computing context, and the majority of it focuses on support for human discovery via recommender systems.  We argue that serendipity also includes an important invention aspect. Building on a survey of serendipitous discovery and invention in science and technology, we advance a definition of serendipity and an accompanying model that can be used evaluate the potential for serendipitous discovery and invention within autonomous or interactive computational systems.   Practitioners can use this model to  evaluate a system's potential for unexpected online behaviour that may have a beneficial outcome.  In addition to a quantitative rating of serendipity potential -- computed in terms of population-based estimates of chance and curiosity (in the discovery phase) and sagacity and value (in the invention phase) -- the model also suggests qualitative features that can guide development work.   We show how the model is used in three case studies of existing and hypothetical systems, in the context of evolutionary computing, automated programming, and (next-generation) recommender systems.  From this analysis, we extract recommendations for practitioners working with computational serendipity, and outline future directions for research.  
\\[.3cm]

\keywords{serendipity, evaluation, computational creativity, autonomous systems}
\end{abstract}
\newpage
