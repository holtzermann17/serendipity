\begin{abstract}
Most prior work that deals with serendipity in a computing context focuses on computational ``discovery''; we argue that serendipity also includes an important ``invention'' aspect.
Building on a survey that describes serendipitous discovery and invention in science and technology, we advance a definition of serendipity and an accompanying model that can be used evaluate the potential for serendipity in computational systems.  The model adapts existing recommendations for evaluating computational creativity.
It is applied in three case studies that evaluate the serendipity of existing and hypothetical systems in the context of
evolutionary computing, recommender systems, and automated programming.
From this analysis, we extract recommendations for practitioners working with computational serendipity, and outline future directions for research.  We argue there is much to be gained by creating systems with the potential for serendipity, and that serendipity is particularly critical for autonomous systems.
\\[.3cm]

\keywords{serendipity, evaluation, computational creativity, autonomous systems}
\end{abstract}
