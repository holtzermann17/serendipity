\begin{abstract}
Most prior work that deals with serendipity in a computing context focuses on computational ``discovery''; we argue that serendipity also includes an important ``invention'' aspect.
Building on a survey of describing serendipitous discovery and invention in science and technology, we advance a model that can be used evaluate the potential for serendipity in computational systems.  The model adapts existing recommendations for evaluating computational creativity.
It is applied in three case studies that evaluate the serendipity of existing and hypothetical systems in the context of
evolutionary computing, recommender systems, and automated programming.
From this analysis, we extract recommendations for practitioners working with computational serendipity, and outline future directions for research.  We argue that patterns of serendipity can be used in the design of computational systems, and that there is much to be gained by building an awareness of serendipity into computational systems, particularly from the perspective of machine ethics.
\\[.3cm]

\keywords{serendipity, evaluation, computational creativity, machine ethics}
\end{abstract}
