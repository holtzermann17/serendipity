\begin{abstract}
%
Most prior work that deals with serendipity in a computing context focuses on computational ``discovery''; we argue that serendipity also includes an important ``invention'' aspect.
% 
We survey literature describing serendipitous discovery and invention in science and technology, as well as the etymology and definitions of the term ``serendipity''. Building upon and refining previous work, we propose a model of computational serendipity that can be used to evaluate computational systems.  To this end we adapt existing recommendations for evaluating computational \emph{creativity}.
%
We develop case studies that evaluate the serendipity of existing systems, and develop a thought experiment that applies our model to design a multi-agent environment for computer poetry.
%
From our analyses, we extract recommendations for practitioners working with computational serendipity, and outline future directions for research.
\\[.5cm]
%
%% \keywords{serendipity,
%% design patterns,
%% intelligent machinery,
%% Writers Workshops}
\end{abstract}
