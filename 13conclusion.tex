\section{Conclusion} \label{sec:conclusion}

%
We began the paper by surveying the historical meaning of
``serendipity'', drawing on previous theories and collecting
historical examples from various domains.  We used this survey to
develop a collection of criteria which we propose to be
computationally salient.
%
Adapting the ``Standardised Procedure for Evaluating Creative
Systems'' (SPECS) model from \citeA{jordanous:12}, we proposed a
detailed set of evaluation standards for serendipity.
%
We used this model to analyse the potential for serendipity in case
studies of evolutionary computing, recommender systems, and automated
programming.  We saw that the proposed framework can be used both
retrospectively, as an evaluation tool, and prospectively, as a design
tool.  In every case, the model surfaced themes that can help to guide
implementation.
%
We then reviewed related work: like \citeA{andre2009discovery}, we
propose a two-part definition of serendipity: \emph{discovery}
followed by \emph{invention}.  Our process-focused model of
serendipity elaborates both stages, and our case studies show how to
use this model to evaluate existing and hypothetical computer systems.
In contrast to most prior work in \emph{computational creativity}, for
\emph{computational serendipity}, evaluation of various forms needs to
be embedded inside of computational systems.

In our discussion, we reflected back over the proposed model and case
studies, and outlined a programme of research into computational
serendipity.  We have also drawn attention to broader considerations
in system design.  Our examples show that serendipity is not foreign
to computing practice.  There are further gains to be had for research
in computing by planning -- and programming -- for serendipity.


