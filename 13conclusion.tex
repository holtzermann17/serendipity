\section{Conclusion} \label{sec:conclusion}

%
We began the paper by surveying the historical meaning of
``serendipity'', drawing on previous theories and collecting
historical examples from various domains.  We used this survey to
develop a collection of criteria which we propose to be
computationally salient.
%
Adapting the ``Standardised Procedure for Evaluating Creative
Systems'' (SPECS) model from \citeA{jordanous:12}, we proposed a
detailed set of evaluation standards for serendipity.
%
We used this model to analyse the potential for serendipity in case
studies of evolutionary computing, recommender systems, and automated
programming.  We saw that the proposed framework can be used
retrospectively, as an evaluation tool, and prospectively, as a design
tool.
%
We then reviewed related work: like
\citeA{andre2009discovery}, we propose a two-part definition of
serendipity: \emph{discovery} followed by \emph{invention}.
%
We then reflected back over our definition and analyses, and outlined
a programme for serendipitous computing in the pursuit of
\emph{autonomy}, \emph{learning}, \emph{sociality}, and \emph{embedded
  evaluation} that would tackle the following challenges:
%
\begin{itemize}
\item \emph{A primary challenge for the serendipitous operation of
  computers is developing computational agents that specify their own
  problems.}
\item \emph{A second challenge is for computational agents to learn
  more and more about the world we live in.}
\item \emph{A third challenge is for computational agents to interact
  in a recognisably social way with us and with each other, resulting
  in emergent effects.}
\item \emph{A fourth challenge is for computational agents to evaluate
  their own creative process and products.}
\end{itemize}
%
In the current work, we have limited ourselves to clarifying
conceptual issues surrounding serendipty, and examining their
implications for computational systems.
% 
We indicate several possible further directions for implementation
work in each of our case studies.  We have also drawn attention to
broader considerations in system design.  Our examples show that
serendipity is not foreign to computing practice.  There are further
gains to be had for research in computing by planning -- and
programming -- for serendipity.
%

