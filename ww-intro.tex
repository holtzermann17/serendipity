\subsection{A Writers Workshop for Systems} \label{sec:ww}

%% \textbf{[It would be good to go back over our other paper and make
%%     sure we make good on the idea in the Related Work section of the
%%     current paper that ``This earlier paper remains broadly
%%     indicative, however, and the ideas it describes can see
%%     considerable benefit from the more formal thinking we develop in
%%     the current work.''}

%% \textbf{In particular: at least one of the reviewers found the Writers
%% Workshop ``technologically unrealistic'' or similar, so let's try to
%% make sure we're not overpromising.  I think the other paper makes it
%% all fairly realistic.]}

%% In \cite{poetry-workshop}, we investigate the feasibility of using
%% designs of this sort in multi-agent systems that learn by sharing and
%% discussing partial understandings.  This earlier paper remains broadly
%% indicative, however, and the ideas it describes can see considerable
%% benefit from the more formal thinking we develop in the current work.

% \citeA{poetry-workshop} describes a Writers Workshop for poetry
%systems. 

Following \citeA{gabriel2002writer}
% we described a template for a pattern
% language for interactions in a computational poetry workshop, closely
we define a \emph{Workshop} to be an activity for two or more agents
consisting of the following steps:
%itemize?
{\tt presentation}, {\tt listening}, {\tt feedback}, {\tt questions},
and {\tt reflections}.  In general, the first and most important
feature of {\tt feedback} is for the listener to say what they heard;
in other words, what they find in the presented work.  In some
settings this is augmented with {\tt suggestions}.  After any {\tt
  questions} from the author, the commentators may make {\tt replies}
to offer clarification.  This is how these steps map into the diagram
we introduced in Section \ref{sec:background}:

\begin{center}
\begingroup
\tikzset{
block/.style = {draw, fill=white, rectangle, minimum height=3em, minimum width=3em},
tmp/.style  = {coordinate}, 
sum/.style= {draw, fill=white, circle, node distance=1cm},
input/.style = {coordinate},
output/.style= {coordinate},
pinstyle/.style = {pin edge={to-,thin,black}}
}

\begin{tikzpicture}[auto, node distance=2cm,>=latex']
    \node [sum] (sum1) {};
    \node [input, name=pinput, above left=.9cm and .9cm of sum1] (pinput) {};
    \node [input, name=tinput, left=2.2cm of sum1] (tinput) {};
    \node [input, name=minput, below left of=sum1] (minput) {};
    \node [input, name=minput, right of=sum1] (moutput) {};
    \draw [->] (tinput) -- node{\vphantom{{\footnotesize g}}{\footnotesize \emph{presentation}~~}} (sum1);
    \draw [->] (pinput) -- node{{\footnotesize listening}} (sum1);
    \draw [->] (sum1) -- node{\vphantom{{\footnotesize g}}{\footnotesize feedback}}  (moutput);
\end{tikzpicture}
\hspace{1cm}
\begin{tikzpicture}[auto, node distance=2cm,>=latex']
    \node [sum] (sum1) {};
    \node [input, name=pinput, above left=.9cm and .9cm of sum1] (pinput) {};
    \node [input, name=tinput, left of=sum1] (tinput) {};
    \node [input, name=minput, below left of=sum1] (minput) {};
    \node [sum, right=1.5cm of sum1] (sum2) {};
    \node [input, name=minput, right of=sum2] (moutput) {};
    \draw [->] (tinput) -- node{\vphantom{{\footnotesize g}}{\footnotesize feedback~~}} (sum1);
    \draw [->] (pinput) -- node{{\footnotesize \emph{questions}}} (sum1);
    \draw [->] (sum1) -- node{\vphantom{{\footnotesize g}}{\footnotesize answers}} (sum2);
    \draw [->] (sum2) -- node{{\footnotesize \emph{reflections}}}  (moutput);
\end{tikzpicture}
\endgroup
\end{center}

