\subsection{Thought experiment evaluating our model of serendipity} \label{sec:ww}

To evaluate our computational framework in usage, we apply a thought
experiment based around a scenario where there is high potential for
serendipity.  As discussed above, sociological factors can influence
serendipitous discoveries on a social scale.  The exploitation of
social creativity and feedback can create scenarios where serendipity
could occur.

In \cite{poetry-workshop}, we considered multi-agent systems that
learn by sharing work in progress, and discussing partial
understandings.  The thought experiment we apply here explores
serendipity in such scenarios, and is influenced by the ideas of
\citeA{gabriel2002writer} on Writers Workshops.

Following \citeA{gabriel2002writer}
% we described a template for a pattern
% language for interactions in a computational poetry workshop, closely
we define a \emph{Workshop} to be an activity for two or more agents
consisting of the following steps:
%itemize?
{\tt presentation}, {\tt listening}, {\tt feedback}, {\tt questions},
and {\tt reflections}.  In general, the first and most important
feature of {\tt feedback} is for the listener to say what they heard;
in other words, what they find in the presented work.  In some
settings this is augmented with {\tt suggestions}.  After any {\tt
  questions} from the author, the commentators may make {\tt replies}
to offer clarification.\footnote{We return to discuss further work with Writers Workshops and serendipity in Section \ref{sec:futurework}.}
This is how these steps map into the diagram we introduced in Section \ref{sec:background}:

\begin{center}
\begingroup
\tikzset{
block/.style = {draw, fill=white, rectangle, minimum height=3em, minimum width=3em},
tmp/.style  = {coordinate}, 
sum/.style= {draw, fill=white, circle, node distance=1cm},
input/.style = {coordinate},
output/.style= {coordinate},
pinstyle/.style = {pin edge={to-,thin,black}}
}

\begin{tikzpicture}[auto, node distance=2cm,>=latex']
    \node [sum] (sum1) {};
    \node [input, name=pinput, above left=.9cm and .9cm of sum1] (pinput) {};
    \node [input, name=tinput, left=2.2cm of sum1] (tinput) {};
    \node [input, name=minput, below left of=sum1] (minput) {};
    \node [input, name=minput, right of=sum1] (moutput) {};
    \draw [->] (tinput) -- node{\vphantom{{\footnotesize g}}{\footnotesize \emph{presentation}~~}} (sum1);
    \draw [->] (pinput) -- node{{\footnotesize listening}} (sum1);
    \draw [->] (sum1) -- node{\vphantom{{\footnotesize g}}{\footnotesize feedback}}  (moutput);
\end{tikzpicture}
\hspace{1cm}
\begin{tikzpicture}[auto, node distance=2cm,>=latex']
    \node [sum] (sum1) {};
    \node [input, name=pinput, above left=.9cm and .9cm of sum1] (pinput) {};
    \node [input, name=tinput, left of=sum1] (tinput) {};
    \node [input, name=minput, below left of=sum1] (minput) {};
    \node [sum, right=1.5cm of sum1] (sum2) {};
    \node [input, name=minput, right of=sum2] (moutput) {};
    \draw [->] (tinput) -- node{\vphantom{{\footnotesize g}}{\footnotesize feedback~~}} (sum1);
    \draw [->] (pinput) -- node{{\footnotesize \emph{questions}}} (sum1);
    \draw [->] (sum1) -- node{\vphantom{{\footnotesize g}}{\footnotesize answers}} (sum2);
    \draw [->] (sum2) -- node{{\footnotesize \emph{reflections}}}  (moutput);
\end{tikzpicture}
\endgroup
\end{center}

