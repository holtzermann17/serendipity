\section{Literature review} \label{sec:literature-review}

In this section, we give a short overview covering the etymology of
the term ``serendipity'' and trace its development in order to pin
down the key commonalities from many definitions and instances.  In
particular, we point out key conditions of serendipity, their
components and general characteristics, including environmental
factors.  The structure of this section follows and updates an earlier
survey from \citeA{pease2013discussion}.

\subsection{Etymology and selected definitions} \label{sec:overview-serendipity}
The English term ``serendipity'' derives from the 1302 long poem ``Eight Paradises'', written in Persian by the Sufi poet Am\={\i}r Khusrow in Uttar Pradesh.\footnote{\url{http://en.wikipedia.org/wiki/Hasht-Bihisht}}  In the English-speaking world, its first chapter became known as ``The Three Princes of Serendip'', where ``Serendip'' represents the Old Tamil-Malayalam word for Sri Lanka (%{\tam சேரன்தீவு},
\emph{Cerantivu}), ``island of the Ceran kings.''

The term ``serendipity'' is first found in a 1557 letter by Horace Walpole to Horace Mann:
\begin{quote}
\emph{``This discovery is almost of that kind which I call serendipity, a very expressive
word} \ldots \emph{You will understand it better by the derivation than by the
definition. I once read a silly fairy tale, called The Three Princes of Serendip:
as their Highness travelled, they were always making discoveries, by accidents
\& sagacity, of things which they were not in quest of}[.]''~\cite[p. 633]{van1994anatomy}
\end{quote}
The term became more widely known in the 1940s through studies of serendipity as a factor in scientific discovery, surveyed by Robert Merton and Elinor Barben \citeyear{merton} in their 1957 analyis ``The Travels and Adventures of Serendipity, A Study in Historical Semantics and the Sociology of Sciences''.  Merton and Barben define the term as follows:
\begin{quote}
\emph{``The serendipity pattern refers to the fairly common experience of observing
an unanticipated, anomalous and strategic datum which becomes the occasion
for developing a new theory or for extending an existing theory.''} \cite[p. 635]{van1994anatomy}
\end{quote}
In 1986, Philippe Qu\'eau described serendipity as ``the art of
finding what we are not looking for by looking for what we are not
finding'' \cite{eloge-de-la-simulation}, as quoted in
\cite[p. 121]{Campos2002}.  Pek van Andel
\citeyear[p. 631]{van1994anatomy} describes it simply as ``the art of
making an unsought finding''.

Roberts \citeyear[pp. 246--249]{roberts} records 30 entries for the term ``serendipity'' from English language dictionaries dating from 1909 to 1989.  
%
Classic definitions require the investigator not to be aware of the problem they serendipitously solve, but this criterion has largely dropped from dictionary definitions. Only 5 of Roberts' collected definitions explicitly say ``not sought for.''  Roberts characterises ``sought findings'' in which an accident leads to a discovery with the term \emph{pseudoserendipity} \cite{chumaceiro1995serendipity}.
%
While Walpole initially described serendipity as an event (a discovery), it has since been reconceptualised as a psychological attribute, a matter of sagacity on the part of the discoverer: a ``gift'' or ``faculty'' more than a ``state of mind.''  Only one of the collected definitions, from 1952, defined it solely as an event, while five define it as both event and attribute.

However, there are numerous examples that exhibit features of
serendipity which develop on a social scale rather than an individual
scale.  For instance, between Spencer Silver's creation of high-tack,
low-adhesion glue in 1968, the invention of a sticky bookmark in 1973,
and the eventual launch of the distinctive canary yellow re-stickable
notes in 1980, there were many opportunities for
Post-its\texttrademark\ \emph{not} to have come to be
\cite{tce-postits}. Accordingly, Merton and Barber argue that the
psychological perspective needs to be integrated with a
\emph{sociological} one.\footnote{ ``For if chance favours prepared
  minds, it particularly favours those at work in microenvironments
  that make for unanticipated sociocognitive interactions between
  those prepared minds. These may be described as serendipitous
  sociocognitive microenvironments'' \cite[p. 259--260]{merton}.}
Large-scale scientific and technical projects generally rely on the
``convergence of interests of several key actors''
\cite{companions-in-geography}, along with other supporting cultural
factors.  Umberto Eco \citeyear{eco2013serendipities} focuses on the
historical role of serendipitous mistakes and falsehoods in the
production of knowledge.

It is important to note that serendipity is usually discussed within
the context of \emph{discovery}, rather than \emph{creativity},
although in typical parlance these terms are closely related
\cite{jordanous12jims}.  Henri Bergson's distinction will be useful in
what follows:
\begin{quote}
``\emph{Discovery, or uncovering, has to do with what already exists,
    actually or virtually; it was therefore certain to happen sooner
    or later.  Invention gives being to what did not exist; it might
    never have happened.}''~\cite{bergson2010creative}
\end{quote}
Serendipity, as we understand the term, would seem to require features
of both; that is, the discovery of something unexpected and the
invention of an application for the same.  We must complement analysis
with synthesis \cite{delanda1993virtual}.  The balance between these
two features will differ from case to case.  In the following section,
we will elaborate on the characteristics of serendipity with
particular reference to classic examples.

\subsection{Characteristics of serendipity}\label{sec:characteristics}

Here we will describe our key condition for serendipity, the presence
of a Focus Shift, together with four key components that implement
this (Prepared Mind, Serendipity Trigger, Bridge, Result), four
dimensions that are generally present to some degree in instances of
serendipitous discovery or invention (Chance, Curiosity, Sagacity,
Value) and four supporting environmental factors that, if not strictly
required, are at least conducive to serendipity (Dynamic world,
Multiple contexts, Multiple tasks, Multiple influences). We shall relate these descriptions to some of the most famous examples of serendipity.
With the characteristics in mind it is not hard to spot further examples.
%% with similar characteristics: e.g. the invention of dry cleaning by
%% a professional dye-maker after his maid spilled kerosene on the
%% tablecloth, or the discovery of a marketable use for sildenafil
%% citrate (better known as {\em Viagra}\texttrademark) which had been
%% trialled as a heart medicine.

\begin{itemize}
\item The 17\textsuperscript{th} Century discovery that \emph{quinine} extracted from
  the bark of South American cinchona trees could be used to treat and
  prevent malaria -- building on a much earlier indigenous Quechua
  discovery that the extract stops shivering.
\item Fleming's discovery of {\em penicillin}.\footnote{Merton and
  Barber \citeyear{merton} state that the description of this discovery
  was the first time that the word \emph{serendipity} was used without
  inverted commas or accompanying definition.}
\item de Mestral's invention of {\em Velcro}\texttrademark\ following
  the model presented by cockle-burs that stuck to his jacket while
  out walking \cite[pp 220-222]{roberts}.
\item Arthur Fry's invention of sticky bookmarks (the prototype for
  {\em Post-it}\texttrademark\ notes), using a weak glue developed by
  his colleague, Spencer Silver \cite[p. 224]{roberts}.
\item Penzias and Wilson's discovery of the {\em echoes of the Big
  Bang} \cite{singh2004big}.
\item Kekul\'e's dream-inspired discovery of the {\em structure of the
  benzine ring} \cite[p. 21]{benfey}, cf. \cite[p. 77]{roberts}.
\item Charles Goodyear's invention of {\em vulcanised rubber}
  \cite{goodyear1855gum}.
\item The {\em Rosetta Stone} was found by a soldier who was
  demolishing a wall in order to clear ground for what was to be Fort
  St. Julien \cite[pp. 109 - 111]{roberts}.
\end{itemize}

\subsubsection*{Key condition for serendipity}

\paragraph{Focus shift.}
The most extreme cases show focus establishing itself as if from
nowhere: de Mestral was walking through the Alps when he encountered
the ``seeds'' of his discovery.
\begin{quote}
``\emph{After removing several of the burdock burrs (seeds) that kept sticking to his clothes and his dog's fur, he became curious as to how it worked. He examined them under a microscope, and noted hundreds of `hooks' that caught on anything with a loop, such as clothing, animal fur, or hair. He saw the possibility of binding two materials reversibly in a simple fashion, if he could figure out how to duplicate the hooks and loops.}''~\cite{wiki:velcro}
\end{quote}
In some cases, the focus shift takes place within a social context:
Arthur Fry and Spencer Silver had different ideas about what could be
done with weak glue.
%
In all of the discoveries listed above, there was a radical change in
the discoverer's evaluation of what is interesting.  We can think of
this as a reclassification of ``noise'' to ``signal.''

\subsubsection*{Components of serendipity}

\paragraph{Prepared Mind.}

Kekul\'e's ``prepared mind'' included his focus on the problem of
finding the structure of the benzine molecule and his knowledge and
skill as a scientist.  Fleming's ``prepared mind'' included his focus
on carrying out experiments to investigate influenza as well as his
previous experience that foreign substances in petri dishes can kill
bacteria.  He was concerned above all with the question ``Is there a
substance which is harmful to harmful bacteria but harmless to human
tissue?''  \cite[p. 161]{roberts}.  The social analogues are clear:
for example, 3M not only had a talented staff, but ran internal
technical forums where staff members could exchange ideas.
 
\paragraph{Serendipity Trigger.}

The trigger does not directly cause the outcome, but rather, inspires
thought.  Indeed, the trigger may bear very little resemblance to the
eventual result.  On its own, the trigger would not typically be seen
as an important discovery.  Examples include a dream, a petri dish
with a clear area, and cockle-burs attached to a jacket.  In a social
context, the trigger may have several parallel or sequential
components, and may rely on the circumstantial alignment of interest
between different parties.  For example, it was long known that
cinchona bark stops shivering; in particular, it stops shivering in
malaria patients, as was observed when malarial Europeans arrived for
the first time in Peru.  That it additionally can cure and can even
prevent malaria was subsequently revealed.

\paragraph{Bridge.}

The bridge is what affords movement from the trigger to the result.
These include reasoning techniques, such as abductive inference (what
might cause a clear patch in a petri dish?); analogical reasoning (de
Mestral constructed a target domain from the source domain of burs
hooked onto fabric); and conceptual blending (Kekul\'e blended his
knowledge of molecule structure with his vision of a snake biting its
tail).  The bridge may also rely on new social arrangements, such as
the formation of cross-cultural research networks
\cite{companions-in-geography}.

\paragraph{Result.}

This is the outcome itself. This may be a new product, artefact,
process, hypothesis, a new use for a material substance, and so on.
The outcome may contribute evidence in support of a known hypothesis,
or a solution to a known problem.  Alternatively, the result may
itself be a {\em new} hypothesis or problem.  The result may be a
``pseudoserendipitous'' in the sense that it was {\em sought}, while
nevertheless arising from an unknown, unlikely, coincidental or
unexpected source.  More classically, it is an \emph{unsought}
finding, such as the discovery of the Rosetta stone.


\subsubsection*{Dimensions of serendipity}


\paragraph{Chance.}

The {\em serendipity trigger} tends to be unlikely, unexpected,
unsought, accidental, random, or coincidental.  The trigger has
features that arise independently of the result, and even
independently of any search for a result.  The relevant features may
be ``hidden in plain view,'' and chance may apply to the conditions
that eventuate in their discovery, as when malarial Europeans chanced
upon a remedy found only in South America.  Fleming \citeyear{fleming}
noted: ``There are thousands of different moulds'' -- and ``that
chance put the mould in the right spot at the right time was like
winning the Irish sweep.''

\paragraph{Curiosity.}

The capacity for keeping an \emph{open mind}, and the corresponding
ability to take advantage of the unpredictable, is necessary for a
focus shift to take place.  Many of the investigators described above
went beyond simply keeping an open mind in order to actively exercise
their curiosity about the way things work.  Importantly, a preliminary
evaluation of interestingness often takes place well before a final
evaluation of the outcome.  Venkatesh Rao \citeyear{rao2011tempo} refers
to a \emph{cheap trick} that takes place early on in many narratives
in order to establish preliminary conditions of order, and curiosity
with respect to unexpected stimuli can play this role.

\paragraph{Sagacity.}

This old-fashioned word is related to ``wisdom,'' ``insight,'' and
especially to ``taste'' -- and describes the attributes, or skill, of
the discoverer that contribute to forming the bridge between the
trigger and the result.  In many cases, such as an entanglement with
cockle-burs, many others will have already been in a similar position
and not obtained an interesting result.  Relevant skills include the
ability to keep an open mind, to perform a focus shift, to see the
value in a discovery, and to build a suitable bridge.

\paragraph{Value.}

It is generally agreed that a serendipitous result is one that is seen
to be happy or useful.
%
This judgement may be made independently (and, in the computational
creativity context \cite{jordanous:12} argues that this is
preferable) or by the discoverer/creator.  Note that the chance
``discovery'' of, say, a \pounds 10 note may be seen as happy by the
person who finds it, whereas the loss of the same note would generally
be regarded as unhappy.  Positive judgements of serendipity by a third
party would be less likely in scenarios in which ``One man's loss is
another man's gain'' than in scenarios where ``One man's trash is
another man's treasure.''

\newpage
\subsubsection*{Environmental factors}

\paragraph{Dynamic world.}

Firstly, in the settings we are interested in, information about the
world develops over time, and is not presented as a complete,
consistent whole.
%
Secondly, the components of serendipity as described above have an
order of operations: the prepared mind takes the stage first, then the
serendipity trigger takes place, a bridge is found, and after that the
result.  Value may come later.  Van Andel
\citeyear[p. 643]{van1994anatomy} estimates that in twenty percent of
innovations ``something was discovered before there was a demand for
it.''

\paragraph{Multiple contexts.}

One of the dynamical aspects at play may be the discoverer going back
and forth between different contexts, with different stimuli.
%
For exmple, 3M employee Arthur Fry sang in a church choir and needed a
good way to mark pages in his hymn book.  Malaria was not indigenous
to Peru, where cinchona trees grow.  Some contexts may play the role
of a training ground for a subsequent discovery: for example, Goodyear
had spent years experimenting with rubber using different processes
before he hit upon the process of vulcanisation.

\paragraph{Multiple tasks.}

Even within what would typically be seen as a single context, a
discoverer may take on multiple tasks that segment the context into
sub-contexts, or that cause the investigator to look in more than one
direction.
%
Fleming happened to be doing the washing up after a holiday when he
made his discovery.  He might have overlooked the critical details had
he not also been chatting with a former lab assistant who had stopped
by.  Penzias and Wilson used a large antenna to detect radio waves
that were relayed by bouncing off of satellites.  After they had
removed interference effects due to radar, radio, and heat, they found
residual ambient noise that couldn't be eliminated
\cite{wiki:cosmic-radiation}.

\paragraph{Multiple influences.}

A prepared mind, or its distributed analogue, may draw on a range of
different skills and experiences.  The ``bridge'' from trigger to
result is often found through a social network, thus, for instance
Penzias and Wilson only understood the significance of their work
after reading a preprint by Jim Peebles that hypothesised the
possibility of measuring radiation released by the big bang
\cite{wiki:cosmic-radiation}.
%
The process of discovery and invention may involve more than one
``aha!'' moment and skill set: Post-it\texttrademark\ notes again make
a good example.
%%%%%%%%%%%%%%%%%%%%%%%%%%%%%%%%%%%%%%%%%%%%%%%%%%%%%%%%%%%%%%%%%%%%%%%%%%%%%%%%%%%%%%%%%%%%%%%%%%%%

% \newpage 
