\section{Literature review} \label{sec:literature-review}

In this section, we give a short overview covering the etymology of
the term ``serendipity'' and trace its development in order to pin
down the key commonalities from many definitions and instances.  In
particular, we point out key conditions of serendipity, their
components and general characteristics, including environmental
factors.  The structure of this section follows and updates an earlier
survey from \citeA{pease2013discussion}.

\subsection{Etymology and selected definitions} \label{sec:overview-serendipity}
The English term ``serendipity'' derives from the 1302 long poem ``Eight Paradises'', written in Persian by the Sufi poet Am\={\i}r Khusrow in Uttar Pradesh.\footnote{\url{http://en.wikipedia.org/wiki/Hasht-Bihisht}}  In the English-speaking world, its first chapter became known as ``The Three Princes of Serendip'', where ``Serendip'' represents the Old Tamil-Malayalam word for Sri Lanka (%{\tam சேரன்தீவு},
\emph{Cerantivu}), ``island of the Ceran kings.''

The term ``serendipity'' is first found in a 1557 letter by Horace Walpole to Horace Mann:
\begin{quote}
\emph{``This discovery is almost of that kind which I call serendipity, a very expressive
word} \ldots \emph{You will understand it better by the derivation than by the
definition. I once read a silly fairy tale, called The Three Princes of Serendip:
as their Highness travelled, they were always making discoveries, by accidents
\& sagacity, of things which they were not in quest of}[.]''~\cite[p. 633]{van1994anatomy}
\end{quote}
The term became more widely known in the 1940s through studies of serendipity as a factor in scientific discovery, surveyed by Robert Merton and Elinor Barben \citeyear{merton} in their 1957 analyis ``The Travels and Adventures of Serendipity, A Study in Historical Semantics and the Sociology of Sciences''.  Merton and Barben define the term as follows:
\begin{quote}
\emph{``The serendipity pattern refers to the fairly common experience of observing
an unanticipated, anomalous and strategic datum which becomes the occasion
for developing a new theory or for extending an existing theory.''} \cite[p. 635]{van1994anatomy}
\end{quote}
In 1986, Philippe Qu\'eau described serendipity as ``the art of
finding what we are not looking for by looking for what we are not
finding'' \cite{eloge-de-la-simulation}, as quoted in
\cite[p. 121]{Campos2002}.  Pek van Andel
\citeyear[p. 631]{van1994anatomy} describes it simply as ``the art of
making an unsought finding''.


Roberts \citeyear[pp. 246--249]{roberts} records 30 entries for the term ``serendipity'' from English language dictionaries dating from 1909 to 1989.  
%
Classic definitions require the investigator not to be aware of the problem they serendipitously solve, but this criterion has largely dropped from dictionary definitions. Only 5 of Roberts' collected definitions explicitly say ``not sought for.''  Roberts characterises ``sought findings'' in which an accident leads to a discovery with the term \emph{pseudoserendipity} \cite{chumaceiro1995serendipity}.
%
While Walpole initially described serendipity as an event (a discovery), it has since been reconceptualised as a psychological attribute, a matter of sagacity on the part of the discoverer: a ``gift'' or ``faculty'' more than a ``state of mind.''  Only one of the collected definitions, from 1952, defined it solely as an event, while five define it as both event and attribute.

However, there are numerous examples that exhibit features of
serendipity which develop on a social scale rather than an individual
scale.  For instance, between Spencer Silver's creation of high-tack,
low-adhesion glue in 1968, the invention of a sticky bookmark in 1973,
and the eventual launch of the distinctive canary yellow re-stickable
notes in 1980, there were many opportunities for
Post-its\texttrademark\ \emph{not} to have come to be
\cite{tce-postits}. Accordingly, Merton and Barber argue that the
psychological perspective needs to be integrated with a
\emph{sociological} one.\footnote{ ``For if chance favours prepared
  minds, it particularly favours those at work in microenvironments
  that make for unanticipated sociocognitive interactions between
  those prepared minds. These may be described as serendipitous
  sociocognitive microenvironments'' \cite[p. 259--260]{merton}.}
Large-scale scientific and technical projects generally rely on the
``convergence of interests of several key actors''
\cite{companions-in-geography}, along with other supporting cultural
factors.  Umberto Eco \citeyear{eco2013serendipities} focuses on the
historical role of serendipitous mistakes and falsehoods in the
production of knowledge.

It is important to note that serendipity is usually discussed within
the context of \emph{discovery}, rather than \emph{creativity},
although in typical parlance these terms are closely related
\cite{jordanous12jims}.  Henri Bergson's distinction will be useful in
what follows:
\begin{quote}
``\emph{Discovery, or uncovering, has to do with what already exists,
    actually or virtually; it was therefore certain to happen sooner
    or later.  Invention gives being to what did not exist; it might
    never have happened.}''~\cite{bergson2010creative}
\end{quote}
Serendipity, as we understand the term, would seem to require features
of both; that is, the discovery of something unexpected and the
invention of an application for the same.  We must complement analysis
with synthesis \cite{delanda1993virtual}.  The balance between these
two features will differ from case to case.  In the following section,
we will elaborate on the characteristics of serendipity with
particular reference to classic examples.


The story of ``Eight Paradises'' was also adapted into an early
chapter of Voltaire's Zadig, and in turn ``the method of Zadig''
informed subsequent approaches both to fiction writing and natural
science.  This method is, to be sure, rooted in discovery:

\begin{quote}
``[H]\emph{e pry’d into the Nature and Properties of Animals and
    Plants, and soon, by his strict and repeated Enquiries, he was
    capable of discerning a Thousand Variations in visible Objects,
    that others, less curious, imagin’d were all
    alike.}''~\cite[pp. 21--22]{zadig}
\end{quote}

\noindent However the essential thing is that from these various
disparate observations, Zadig is able to assemble a coherent picture:

\begin{quote}
\emph{It was his peculiar Talent to render Truth as obvious as
  possible: Whereas most Men study to render it intricate and
  obscure.}~\cite[p. 54]{zadig}
\end{quote}
Similarly, but in reverse, a coherent picture may be reduced to
fragmented pieces each of which tell a different story from the whole.
In enumerating the various features of serendipity below, we will draw
connections with the schematic diagram presented in Section
\ref{specs-overview}, in order to best present the multifaceted but
coherent notion of serendipity.

\subsection{Connections between prior literature on serendipity and our formal definition} \label{sec:connections-to-formal-definition}

Each of the features described below, using an example drawn from the
literature on serendipity, with connections to one part of our diagram.

\subsubsection*{Key condition for serendipity}

\begin{itemize}
\item \textbf{Focus shift}: ``\emph{After removing several of the
  burdock burrs (seeds) that kept sticking to his clothes and his
  dog's fur,}~[de Mestral]~\emph{became curious as to how it
  worked. He examined them under a microscope, and noted hundreds of
  `hooks' that caught on anything with a loop, such as clothing,
  animal fur, or hair. He saw the possibility of binding two materials
  reversibly in a simple fashion, if he could figure out how to
  duplicate the hooks and loops.}''~\cite{wiki:velcro}
%
\inlineitem{This corresponds to the identification of $T^\star$, which
  is common to both sides of the diagram.  \citeA{creativity-crisis}
  write that: ``To be creative requires divergent thinking (generating
  many unique ideas) and then convergent thinking (combining those
  ideas into the best result).''  Accordingly $T^\star$ may be thought
  of as an evolving vector of interesting possibilities.  In de
  Mestral's case, the initial idea of a hook-and-loop fastener
  occurred in 1941, followed by a decade of experimentation before he
  was ready to file a patent claim.}
\end{itemize}

\subsubsection*{Components of serendipity}

\begin{itemize}
\item \textbf{Prepared mind}: 
Fleming's ``prepared mind'' included his focus
on carrying out experiments to investigate influenza as well as his
previous experience that foreign substances in petri dishes can kill
bacteria.  He was concerned above all with the question ``Is there a
substance which is harmful to harmful bacteria but harmless to human
tissue?''  \cite[p. 161]{roberts}.
%%
%
\inlineitem{This corresponds to the prior
  training $p$ and $p^{\prime}$ in our diagram.}
\item \textbf{Serendipity trigger}: The trigger does not directly
  cause the outcome, but rather, inspires a new insight.  It was long
  known by Quechua medics that cinchona bark stops shivering.  In
  particular, it worked well to stop shivering in malaria patients, as
  was observed when malarial Europeans first arrived in Peru.  The
  joint appearance of shivering Europeans and a South American remedy
  was the trigger.  That an extract from cinchona bark can cure and
  can even prevent malaria was subsequently revealed.
%
\inlineitem{This corresponds to the stimulus $T$ in our diagram.}
%%
\item \textbf{Bridge}: These include reasoning techniques, such as
  abductive inference (what might cause a clear patch in a petri
  dish?); analogical reasoning (de Mestral constructed a target domain
  from the source domain of burs hooked onto fabric); and conceptual
  blending (Kekul\'e blended his knowledge of molecule structure with
  his vision of a snake biting its tail).  The bridge may also rely on
  new social arrangements, such as the formation of cross-cultural
  research networks.
%
\inlineitem{This corresponds to the actions based on $p^{\prime}$
  taken on $T^\star$ leading to $R$.}
%%
\item \textbf{Result}: This may be a new product, artefact, process,
  hypothesis, a new use for a material substance, and so on.  The
  outcome may contribute evidence in support of a known hypothesis, or
  a solution to a known problem.  Alternatively, the result may itself
  be a {\em new} hypothesis or problem.  The result may be a
  ``pseudoserendipitous'' in the sense that it was {\em sought}, while
  nevertheless arising from an unknown, unlikely, coincidental or
  unexpected source.  More classically, it is an \emph{unsought}
  finding, such as the discovery of the Rosetta stone.
%
\inlineitem{This corresponds to our $R$.  Note that $R$ may imply
  updates to $p$ or $p^{\prime}$ in further phases of research.}
\end{itemize}

\subsubsection*{Dimensions of serendipity}

Whereas the foregoing items are the central features of the
definition, the following further characterise the circumstances under
which serendipity occurs in practice.

\begin{itemize}
\item \textbf{Chance}: Fleming \citeyear{fleming} noted: ``There are
  thousands of different moulds'' -- and ``that chance put the mould
  in the right spot at the right time was like winning the Irish
  sweep.''
%
\inlineitem{One must assume that relatively few triggers $T^\star$
  that are identified as interesting actually lead to useful results;
  in other words, the process is fallible.}
%%
\item \textbf{Curiosity}: Venkatesh Rao \citeyear{rao2011tempo} refers
  to a \emph{cheap trick} that takes place early on in a narrative in
  order to establish the preliminary conditions of order.  Curiosity
  with can play this role, and can dispose a creative person to begin,
  or to continue, a search into unfamiliar territory.
%
\inlineitem{The prior training $p$ causes interesting features to be
  extracted, even if they are not necessarily useful; $p^{\prime}$
  asks how these features \emph{might} be useful.  }
%%
\item \textbf{Sagacity}: This old-fashioned word is related to
  ``wisdom,'' ``insight,'' and especially to ``taste'' -- and
  describes the attributes, or skill, of the discoverer that
  contribute to forming the bridge between the trigger and the result.
  In many cases, such as an entanglement with cockle-burs, many others
  will have already been in a similar position and not obtained an
  interesting result.  Once a phenomenon has been identified as
  interesting, the disposition of the investigator may lead to a
  dogged pursuit of a useful application or improvement.
%
\inlineitem{Rather than a simple look-up
  rule, $p^{\prime}$ involves creating new knowledge.}
%%
\item \textbf{Value}: Note that the chance ``discovery'' of, say, a
  \pounds 10 note may be seen as happy by the person who finds it,
  whereas the loss of the same note would generally be regarded as
  unhappy.  Positive judgements of serendipity by a third party would
  be less likely in scenarios in which ``One man's loss is another
  man's gain'' than in scenarios where ``One man's trash is another
  man's treasure.''  If possible we prefer this sort of independent
  judgement \cite{jordanous:12}.
%
\inlineitem{The evaluation $|R|>0$ may be carried out ``locally'' (as
  an embedded part of the process of invention of $R$) or ``globally''
  (i.e.~as an external process).  }
\end{itemize}

\subsubsection*{Environmental factors}

\begin{itemize}
\item \textbf{Dynamic world}: Information about the world develops
  over time, and is not presented as a complete, consistent whole.  In
  particular, value may come later.  Van Andel
  \citeyear[p. 643]{van1994anatomy} estimates that in twenty percent
  of innovations ``something was discovered before there was a demand
  for it.''
%
\inlineitem{$T$ (and $T^\star$) appears within a stream of data with
  indeterminacy.  There is a further feedback loop, insofar as
  products $R$ influence the future state.}
%%
\item \textbf{Multiple contexts}: One of the dynamical aspects at play
  may be the discoverer going back and forth between different
  contexts, with different stimuli.  3M employee Arthur Fry sang in a
  church choir and needed a good way to mark pages in his hymn book;
  he happened to have been attending seminars offered by his colleague
  Silver about restickable glue.
%
\inlineitem{This is reflected directly in our model by the difference
  between the ``context of discovery'' involving prior preparations
  $p$, and the ``context of invention'' involving prior preparations
  $p^{\prime}$.  Both of these may be subdivided further.}
%%
\item \textbf{Multiple tasks}: Even within what would typically be
  seen as a single context, a discoverer may take on multiple tasks
  that segment the context into sub-contexts, or that cause the
  investigator to look in more than one direction.  The tasks may have
  an interesting \emph{overlap}, or they may point to a \emph{gap} in
  knowledge.  As an example of the latter, Penzias and Wilson used a
  large antenna to detect radio waves that were relayed by bouncing
  off of satellites.  After they had removed interference effects due
  to radar, radio, and heat, they found residual ambient noise that
  couldn't be eliminated \cite{wiki:cosmic-radiation}.
%
\inlineitem{Both $T$ and $T^\star$ may be multiple, causing an
  individual process to fork into communicating sub-processes that
  involve different skills sets.}
%%
\item \textbf{Multiple influences}: The ``bridge'' from trigger to
  result is often found through a social network, thus, for instance
  Penzias and Wilson only understood the significance of their work
  after reading a preprint by Jim Peebles that hypothesised the
  possibility of measuring radiation released by the big bang
  \cite{wiki:cosmic-radiation}.
%
\inlineitem{The process as a whole may be multiplied out among
  different communicating investigators.}
\end{itemize}
