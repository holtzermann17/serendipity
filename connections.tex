% \section{Connections} \label{sec:connections-to-formal-definition}

The features of our model match Merton's \citeyear{merton1948bearing} earlier description quite
well: $T$ is the unexpected observation; $T^\star$ highlights its
interesting or anomalous features and casts it as ``strategic data''; and the result $R$ may include updates to $p$ or $p^{\prime}$ that inform further phases of research.  

The connection to the key condition and components of serendipity
introduced in our literature survey are as follows:
%
The \textbf{focus shift} corresponds to the identification of
$T^\star$, which is common to both the discovery and the invention
phase.  If the process operates in an ``online'' manner, $T^\star$ may
be an evolving vector of interesting possibilities.
%
The \textbf{prepared mind} corresponds to the prior training $p$ and
$p^{\prime}$ in our diagram.
%
The \textbf{serendipity trigger} is denoted by $T$ in our diagram.
%
The \textbf{bridge} is comprised of the actions based on $p^{\prime}$
that are taken on $T^\star$ leading to the \textbf{result} $R$.


Although they do not directly figure in our definition, the supportive
dimensions and factors can be interpreted using this schematic, as
follows:
%
From the point of view of this model, $T$ is indeterminate.
Furthermore, one must assume that relatively few of triggers $T^\star$
that are identified as interesting actually lead to useful results; in
other words, the process is fallible and \textbf{chance} is likely to
play a role.
%
The prior training $p$ causes interesting features
to be extracted, even if they are not necessarily useful; $p^{\prime}$
asks how these features \emph{might} be useful.  These routines 
suggest the relevance of a computational model of \textbf{curiousity}.  One existing algorithmic approach is developed by \citeA{schmidhuber2007simple}.
%
Rather than a simple look-up rule, $p^{\prime}$ involves creating new knowledge.  A simple example is found in clustering systems, which generate new categories on the fly.  A more complicated example, necessary in the case of updating $p$ or $p$ is automatic programming.  There is ample room for \textbf{sagacity} in this affair.
%
Judging the \textbf{value} of the result $R$ may be carried out
``locally'' (as an embedded part of the process of invention of $R$)
or ``globally'' (i.e.~as an external process).
%
As noted above, $T$ (and $T^\star$) appears within a stream of data
with indeterminacy.  There is a further feedback loop, insofar as
products $R$ influence the future state and behaviour of the system.
Thus, the model exists in a \textbf{dynamic world}.
%
Our model separates the
``context of discovery'', involving prior preparations $p$, from the
``context of invention'' involving prior preparations $p^{\prime}$.
Both of these may be subdivided further into \textbf{multiple contexts}. 
%
Both $T$ and $T^\star$ may be multiple, causing an individual process
to fork into sub-processes dealing with \textbf{multiple tasks} that
involve different skills sets.
%
The process as a whole may be multiplied out among different
communicating investigators, so that the final result bears the mark
of \textbf{multiple influences}.




