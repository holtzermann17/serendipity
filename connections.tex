% \section{Connections} \label{sec:connections-to-formal-definition}

The features of our model match Merton's \citeyear{merton1948bearing} earlier description quite
well: $T$ is the unexpected observation; $T^\star$ highlights its
interesting or anomalous features and casts it as ``strategic data''; and the result $R$ may include
updates to $p$ or $p^{\prime}$ that inform further phases of research.  

Connections to the core components of serendipity introduced in our literature survey are as follows:
%
\textbf{Focus shift}. This corresponds to the identification of
$T^\star$, which is common to both sides of the diagram.
$T^\star$ may be thought of as an evolving vector of interesting
possibilities.
%
\textbf{Prepared mind}. This corresponds to the prior training $p$ and $p^{\prime}$ in our diagram.
%
\textbf{Serendipity trigger}. This corresponds to the stimulus $T$ in our diagram.
%
\textbf{Bridge}. This corresponds to the actions based on $p^{\prime}$ taken on
$T^\star$ leading to $R$.
%
\textbf{Result}. This corresponds to our $R$.  Note that $R$ may imply
  updates to $p$ or $p^{\prime}$ in further phases of research.

In addition, the supportive dimensions and factors can be interpreted using this schematic, as follows:
%
\textbf{Chance}. One must assume that relatively few triggers $T^\star$ that are
identified as interesting actually lead to useful results; in other
words, the process is fallible.
%
\textbf{Curiosity}. The prior training $p$ causes interesting features to be
  extracted, even if they are not necessarily useful; $p^{\prime}$
  asks how these features \emph{might} be useful.  
%
\textbf{Sagacity}. Rather than a simple look-up rule, $p^{\prime}$ involves creating new knowledge.
%
\textbf{Value}. The evaluation $|R|>0$ may be carried out ``locally'' (as
  an embedded part of the process of invention of $R$) or ``globally''
  (i.e.~as an external process).  
%
\textbf{Dynamic world}. $T$ (and $T^\star$) appears within a stream of data with
  indeterminacy.  There is a further feedback loop, insofar as
  products $R$ influence the future state.
%
\textbf{Multiple contexts}. This is reflected directly in our model by the difference
  between the ``context of discovery'' involving prior preparations
  $p$, and the ``context of invention'' involving prior preparations
  $p^{\prime}$.  Both of these may be subdivided further.
%
\textbf{Multiple tasks}. Both $T$ and $T^\star$ may be multiple, causing an
  individual process to fork into communicating sub-processes that
  involve different skills sets.
%
\textbf{Multiple influences}. The process as a whole may be multiplied out among
different communicating investigators.




