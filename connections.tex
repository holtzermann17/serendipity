The features of our model match and expand upon Merton's \citeyear{merton1948bearing} description of the ``serendipity pattern.'' $T$ is an unexpected observation; $T^\star$ highlights its interesting or anomalous features and recasts them as ``strategic data''; and, finally, the result $R$ may include updates to $p$ or $p^{\prime}$ that inform further phases of research.  

From the point of view of the system under consideration, $T$ is
indeterminate.  Furthermore, one must assume that relatively few of
triggers $T^\star$ that are identified as interesting actually lead to
useful results; in other words, the process is fallible and
\textbf{chance} is likely to play a role.
%
The prior training $p$ causes interesting features
to be extracted, even if they are not necessarily useful; $p^{\prime}$
asks how these features \emph{might} be useful.  These routines 
suggest the relevance of a computational model of \textbf{curiosity}.
%
Far from being a simple look-up rule, $p^{\prime}$ involves creating new knowledge.  A simple example is found in clustering systems, which generate new categories on the fly.  A more complicated example, necessary in the case of updating $p$ or $p^{\prime}$, is automatic programming.  There is a need for \textbf{sagacity} in this sort of affair.
%
Judgment of the \textbf{value} of the result $R$ may be carried out
``locally'' (as an embedded part of the process of invention of $R$)
or ``globally'' (i.e.~as an external process).

As noted, $T$ (and $T^\star$) appears within a stream of data with
indeterminacy.  There is an additional feedback loop, insofar as
products $R$ influence the future state and behaviour of the system.
Thus, the system exists in a \textbf{dynamic world}.
%
Our model separates the
``context of discovery'', involving prior preparations $p$, from the
``context of invention'' involving prior preparations $p^{\prime}$.
Both of these, and the data they deal with, may be subdivided further into \textbf{multiple contexts}. 
%
And correspondingly, since both $T$ and $T^\star$ may be complex, they
may be processed using multiple sub-processes that deal with
\textbf{multiple tasks} using different skills sets.
%
The process as a whole may be multiplied out across different
communicating investigators, so that the final result bears the mark
of \textbf{multiple influences}.




